\documentclass{article}
\usepackage[utf8]{inputenc}
\usepackage{mathrsfs}
\usepackage{stmaryrd}

\usepackage{mathe}
\usepackage{enumerate}

\title{7. Topologie-Übung}
\author{Joachim Breitner}
\date{5. Dezember 2007}

\begin{document}
\maketitle
\section*{Aufgabe 1}

$X_k$ seien für $k\in\mathbb N_0$ endliche Mengen. Auf $\prod_{k=0}^\infty X_k$ sei die Norm 
\[
D( (x_k), (y_k)) \da 
\begin{cases}
0, &x_k =y_k \text{ für alle $k$} \\
2^{-m}, & \text{sonst}
\end{cases}
\]
definiert, wobei $m\da \min \{k\in \mathbb N_0 \mid x_k \ne y_k\}$

\paragraph{Frage:} Wie sieht $B_r( (x_k))$ für $r>0$ aus?

Sei $m\in \mathbb N$ so gewählt, dass $\frac{1}{2^{m+1}} \le r\le \frac1{2^m}$. Dann ist $B_r( (x_k))$ die Menge aller Folgen in $\prod_{k=0}^\infty X_k$, die mindestens in den ersten $m$ Folgengliedern mit $(k_k)$ übereinstimmen.

\paragraph{Behauptung} $\prod_{k=0}^\infty X_k$ ist kompakt.

$\prod_{k=0}^\infty X_k$ ist ein metrischer Raum, also ist  $\prod_{k=0}^\infty X_k$ genau dann kompakt, wenn  $\prod_{k=0}^\infty X_k$ folgenkompakt ist.

Wir zeigen: $\prod_{k=0}^\infty X_k$ ist folgenkompakt. Sei also $(A_k)$ eine Folge in  $\prod_{k=0}^\infty X_k$:
\begin{align*}
A_1: a_{11}, a_{12}, a_{13}, \ldots \\
A_2: a_{21}, a_{22}, a_{23}, \ldots\\
A_3: a_{31}, a_{32}, a_{33}, \ldots
\end{align*}
Es ist $X_0$ endlich, also gibt es $a_o\in X_0$, so dass $a_{l1}=a_0$ für unendlich viele $l$ gilt. Betrachte die Teilfolge $(\tilde A_k)$ von $(A_k)$, für die gilt: $\tilde a_{k1} = a_0$ für alle $k\in\mathbb N$.

Es ist auch $X_1$ enlich, also gibt es $a_1\in X_1$, so dass $\tilde a_{l2} = a_1$ für unendlich viele $l$.  Betrachte die Teilfolge $(\tilde{\tilde A}_k)$ von $(\tilde A_k)$, für die gilt: $\tilde {\tilde a}_{k1} = a_0$ für alle $k\in\mathbb N$.

Setzt man dieses Verfahren fort, so erhält man eine Teilfolge von $(A_k)$, die gegen $(a_0,a_1,\ldots)$ konvergiert.

\section*{Aufgabe 2}

Sei $U\subseteq \mathbb R^n$ offen, $f:U\to \MdR$ stetig differenzierbar und es gelte 
\[
f(X)= 0 \implies \exists i\in \{1,\ldots,n\}: \frac{\partial f(x)}{\partial x_i} \ne 0
\]

\paragraph{Behauptung:} $X\da \{x\in U\mid f(x)=0\}$ ist eine $(n-1)$-dimensionale topologische Mannigfaltigkeit.

Dazu konstruieren wir einen Atlas auf $X$, mit Hilfe des Satzes über implizit definierte Funktionen. Sei $x=(x_1,\ldots,x_n)\in X$. Nach Voraussetzung existiert ein $i\in\{1,\ldots,n\}$ mit $\frac{\partial f}{\partial u_i}(x) \ne 0$. Nach dem Satz über implizit definierte Funktionen existiert daher eine Umgebung $U_x\subseteq \mathbb R^{n-1}$ von $(x_1,\ldots,x_{i-1},x_{i+1},\ldots,x_n)$ und eine Umgebung $V_x\subseteq \mathbb R$ von $x_i$ sowie eine stetige Abbildung $g:U_x\to V_x$ mit $g(x_1,\ldots,x_{i-1},x_{i+1},\ldots,x_n) = x_i$, so dass (nach geeigneter Variablenumsortierung) $f(u,g(u)) = 0$ für alle $u\in U_x$.

Setze $O\da (U_x\times V_x)\cap X$. Das ist eine offene Menge in $X$, da $U_x$ und $V_x$ offen sind. Definiere $\varphi : \mathbb R^{n-1}\supseteq U_x \to O$ mit $\varphi(u) \da (u,g(u))$. Klar: $\varphi$ ist stetig und injektiv.

\emph{Der Rest fehlt mangels Akkulaufzeit.}


\section*{Aufgabe 4}

Es sei $K$ eine kompakter topologischer RAum, der Gruppenstruktur hat, $\Phi: K\to GL(n,\mathbb C)$ sei stetig und Gruppenhomomorphismus.

\paragraph{Behauptung:} Für $k\in K$ sind haben alle Eigenwerte von $\Phi(k)$ den Betrag 1.

Es ist $K$ kompakt und $\Phi$ stetig, also ist $\Phi(K)$ ebenfalls kompakt und als Teilmenge eines metrischen Raumes damit beschränkt. Für $k\in K$ gilt $\Phi(k^n)=\Phi(k)^n\in\Phi(K)$ und $\Phi(k^{-1})=\Phi(k)^{-1}\in \Phi(K)$.

Sei $A\da \Phi(k)\in GL_(n,\mathbb C)$. Aus der linearen Algebra wissen wir, dass es ein $U\in GL(n,\mathbb C)$ gibt, so dass $\tilde A \da UAU^{-1}$ in Jordan-Normalform vorliegt. Auf der Diagonalen von $\tilde A^n$ stehen die $n$-ten Potenzen der Eigenwerte von $A$.

Wäre also $\lambda$ ein Eigenwert von $A$ mit $|\lambda|>1$, so würde für $n\to\infty$ gelten: $\|U\Phi(k)^nU^{-1}\|\to \infty$. Weil die Konjugation mit $U\in GL(n,\mathbb C)$ eine stetige Abbildung ist, gilt dann auch $\|\Phi(k)\|\to\infty$, also wäre $\Phi(K)$ nicht beschränkt, was ein Widerspruch wäre.

Wäre dagagen $\lambda$ ein Eigenwert von $A$ mit $|\lambda|<1$, so ist $\frac1\lambda$ Eingenwert von $\Phi(k^{-1})$, was wie eben gezeigt ein Widerspruch ist.

\paragraph{Behauptung:} $\Phi(k)$ ist diagonalisierbar.

Angenommen, $A\da \Phi(k)$ wäre nicht diagonalisierbar. Hat das erste Jordankästchen in $\tilde A$ die Form 
\[
\begin{pmatrix}
\lambda &  & & 0\\
1 &\ddots & & \\
  & \ddots &\ddots &\\
0 & & 1 &\lambda
\end{pmatrix}
\]
und sei $b_{21}$ der Eintrag an der Stelle $(2,1)$ in der Matrix $\tilde A^n$, dann gilt:
$b_{21} = n\cdot \lambda^{n-1}$, das heißt für $n\to\infty$ ist $\|\Phi(k)^n\|\to \infty$, also $\Phi(K)$ nicht beschränkt, was ein Widerspruch ist.

Damit ist die Jordan-Normalform von $\Phi(k)$ diagonalisierbar, also ist $\Phi(k)$ diagonalisierbar.


\end{document}
