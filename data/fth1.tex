\documentclass[a4paper,twoside,DIV15,BCOR12mm]{scrbook}
\usepackage{ana}

\lecturer{Dr. C. Schmoeger}
\semester{Sommersemester 2006}
\scriptstate{complete}

\def\gdw{\equizu}
\def\Arg{\text{Arg}}
\def\MdD{\mathbb{D}}
\def\Log{\text{Log}}
\def\Tr{\text{Tr}}
\def\Ext{\text{Ext}(\ensuremath{\gamma)}}
\def\abnC{\ensuremath{[a,b]\to\MdC}}
\def\wegint{\ensuremath{\int\limits_\gamma}}
\def\gdw{\equizu}
\def\ie{\rm i}
\def\Rand{\partial}
\def\Aut{\text{Aut}}
\def\Gs{\ensuremath{\widetilde{G}}}
\def\phis{\ensuremath{\widetilde{\varphi}}}
\newtheorem{beachte}{Beachte}



%iso-int
\def\iint{\ensuremath{\int\limits}}

\pdfinfo{
	/Author (Die Mitarbeiter von http://mitschriebwiki.nomeata.de/)
	/Title   (Funktionentheorie I)
	/Subject (Funktionentheorie I)
	/Keywords (Analysis)
}

\author{Die Mitarbeiter von \url{http://mitschriebwiki.nomeata.de/}}
\title{Funktionentheorie I}
\makeindex

\begin{document}
\maketitle

\renewcommand{\thechapter}{\Roman{chapter}}
%\chapter{Inhaltsverzeichnis}
\addcontentsline{toc}{chapter}{Inhaltsverzeichnis}
\tableofcontents

\chapter{Vorwort}

\section{Über dieses Skriptum}
Dies ist ein erweiterter Mitschrieb der Vorlesung \glqq Funktionentheorie I\grqq\ von Herrn Schmoeger im
Sommersemester 2006 an der Universität Karlsruhe (TH). Die Mitschriebe der Vorlesung werden mit
ausdrücklicher Genehmigung von Herrn Schmoeger hier veröffentlicht, Herr Schmoeger ist für den
Inhalt nicht verantwortlich.

\section{Wer}
Beteiligt am Mitschrieb sind Ferdinand Szekeresch, Christian Schulz  und andere.

\section{Wo}
Alle Kapitel inklusive \LaTeX-Quellen können unter \url{http://mitschriebwiki.nomeata.de} abgerufen werden.
Dort ist ein \emph{Wiki} eingerichtet und von Joachim Breitner um die \LaTeX-Funktionen erweitert.
Das heißt, jeder kann Fehler nachbessern und sich an der Entwicklung
beteiligen. Auf Wunsch ist auch ein Zugang über \emph{Subversion} möglich.



\renewcommand{\thechapter}{\arabic{chapter}}
\renewcommand{\chaptername}{§}
\setcounter{chapter}{0}

\chapter{Komplexe Zahlen}

$\MdR ^2 = \{(a,b) : a,b \in \MdR\}$ Für $(a,b),(c,d) \in \MdR ^2$ definieren wir : \\
$(a,b) + (c,d) := (a+c , b+d); (a,b)\cdot (c,d) := (ac-bd , ad+bc)$ \\
Wir setzen abkürzend: $i := (0,1)$ \begriff{(imaginäre Einheit)}. Dann: $i^2 = (-1,0)$ \\

%Satz 1.1
\begin{satz}
$\MdR ^2$ ist mit obiger Addition und Multiplikation ein Körper. Dieser wird mit $\MdC$ bezeichnet und heißt \begriff{Körper der Komplexen Zahlen}.

\begin{liste}
\item $(0,0)$ ist das neutrale Element bzgl. der Addition.
$(1,0)$ ist das neutrale Element bzgl. der Multiplikation.
\item Für $(a,b) \in \MdC$ ist $(-a, -b)$ das inverse Element bzgl. der Addition
Für $(a,b) \in \MdC \backslash \{(0,0)\}$ ist $(\frac{a}{a^2+b^2},\frac{-b}{a^2+b^2})$ das inverse Element bzgl. der Multiplikation
\end{liste}
\end{satz}

\begin{beweis}
Nachrechnen!
\end{beweis}

Definiere $\varphi : \MdR \rightarrow \MdC$ durch $\varphi (a) := (a,0) \quad (a \in \MdR)$. Dann gilt: \\
$\varphi (a+b) = \varphi (a) + \varphi (b), \varphi (ab)= \varphi(a) \cdot \varphi(b), \varphi (0) = (0,0), \varphi(1) = (1,0). \, \varphi$ ist also ein injektiver Körperhomomorphismus. Also: $\MdR \subseteq \MdC$. \\
Wir schreiben $a$ statt $(a,0)$ für $a \in \MdR$. Insbesondere: $i^2 = -1$.

%Satz 1.2
\begin{satz}
Jedes $z \in \MdC$ hat eine eindeutige Darstellung $z = a + ib$ mit $a,b \in \MdR$ \\
$\Re z := a$ \begriff{(Realteil von $z$)}, \begriff{$\Im z:= b$ (Imaginärteil von $z$)}
\end{satz}

\begin{beweis}
Sei $z = (a,b) \in \MdC \quad (a,b \in \MdR); z = (a,0) + (0,b) = (a,0) + (0,1)\cdot (b,0) = a + ib$ \\
Eindeutigkeit: klar
\end{beweis}

\begin{definition}
Sei $z = a + ib \in \MdC \quad (a,b \in \MdR)$\\
\begin{liste}
\item $\bar z := a - ib$ heißt die zu $z$ \begriff{konjugiert komplexe Zahl} \\
\item $|z| := (a^2 + b^2)^{\frac{1}{2}} (=\|(a,b)\| =$ eukl. Norm von $(a,b) \in \MdR ^2$) heißt \begriff{Betrag von $z$}; $|z|\geq 0$
\end{liste}
\textbf{Geometrische Veranschaulichung von \MdC :} Komplexe Ebene \\
$|z| =$ Abstand von $z$ und $0$
\end{definition}

%Satz 1.3
\begin{satz}
Seien $z,w \in \MdC$\\
\begin{liste}
\item $\Re z = \frac{1}{2}(z + \overline{z}); \Im z = \frac{1}{2 i}(z-\overline{z}); z \in \MdR \equizu z = \overline{z}; \overline{\bar z}=z; z=w \equizu \Re z = \Re w, \Im z = \Im w; |z|=0 \equizu z=0$
\item $\overline{z+w} = \overline{z} + \overline{w}; \overline{zw} = \overline{z} \cdot \overline{w}; \overline{\frac{1}{w}} = \frac{1}{\bar w}, $ falls $w \neq 0$
\item $| \Re z| \leq |z|; |\Im z| \leq |z|$
\item $|\bar z| = |z|; |z|^2 = z\cdot \bar z = \bar z \cdot z;$ für $z \neq 0 : \frac{1}{z} = \frac{\overline{z}}{z \cdot \overline{z}} = \frac{\overline{z} }{|z|^2}$
\item $|zw| = |z|\cdot |w|; |\frac{1}{w}| = \frac{1}{|w|}$ falls $w \neq 0$
\item $|z + w| \leq |z| + |w|$ \quad (Dreiecksungleichung)
\item $\big ||z|-|w|\big| \leq |z - w|$
\end{liste}
\end{satz}

\begin{beweis}
(1) - (5): nachrechnen! \\
(7) folgt aus (6) wörtlich wie in $\MdR$ \\
(6) $|z + w|^2 \gleichwegen{(3)} (z + w)(\bar {z+w}) \gleichwegen{(2)} (z+w)(\bar z + \bar w) = z \bar z + z \bar w + \bar z w + w \bar w \\
\gleichwegen{(1),(3)} |z| ^2 + 2\Re(z \bar w) + |w|^2 \leq |z|^2 + 2|\Re (z\bar w)| + |w|^2 \\
\stackrel{(3)}{\leq} |z|^2 + 2|z\bar w| + |w|^2 = |z|^2 + 2|z||w| + |w|^2 = (|z| + |w|)^2$
\end{beweis}

\textbf{Polarkoordinaten}\\
Sei $z = x + iy \in \MdC \backslash \{0\} \quad (x,y \in \MdR). \quad r:=|z|$ \\
Bekannt: $\exists \varphi \in \MdR : x = r\cos\varphi, y = r\sin\varphi$ \\
Dann: $z = x + iy = r(\cos \varphi + i\sin \varphi) = |z|(\cos\varphi + i\sin\varphi)$ \\
Die Zahl $\varphi$ heißt \textbf{ein} Argument von $z$ und wird mit $\arg z$ bezeichnet. Mit $\varphi$ ist auch $\varphi + 2k\pi \quad (k \in \MdZ)$ ein Argument von z.

\textbf{Aber:} es gibt genau ein $\varphi \in (-\pi, \pi]$ mit $z = |z|(\cos\varphi + i\sin\varphi)$. Dieses $\varphi$ heißt der \begriff{Hauptwert des Arguments} und wird mit $\text{Arg } z$ bezeichnet. \\
Seien $z_1 = |z|(\cos\varphi_1 + i\sin\varphi_1), z_2 = |z|(\cos\varphi_2 + i\sin\varphi_2) \in \MdC \backslash \{0\} (\varphi_1, \varphi_2 \in \MdR)$.\\\\
Aus Additionstheoremen von Sinus und Cosinus folgt: \\
$(*)\quad z_1\cdot z_2 = |z_1||z_2|\big(\cos(\varphi_1+\varphi_2) + i\sin(\varphi_1+\varphi_2)\big)$ \\
Aus $(*)$ folgt induktiv:

%Satz 1.4
\begin{satz}[Formel von de Moivre]
$(\cos\varphi + i\sin\varphi)^n = \cos(n\varphi) + i\sin(n\varphi) \quad \forall n \in \MdN_0 \, \forall \varphi \in \MdR$
\end{satz}

\textbf{Wurzeln:}\\
Beachte: $z^0 := 1 \quad \forall z \in \MdC$ \\

\begin{definition}
Sei $a \in \MdC \backslash \{0\}$ und $n \in \MdN$. Jedes $z \in \MdC$ mit $z^n = a$ heißt eine \begriff{$n$-te Wurzel aus $a$.}
\end{definition}

%Satz 1.5
\begin{satz}
Sei $a \in \MdC \backslash \{0\}, n \in \MdN$ und $a = |a|(\cos\varphi + i\sin\varphi)\quad (\varphi \in \MdR)$ \\
Für $k=0,1,\ldots ,n-1$ setze $z_k = \sqrt[n]{|a|}\big(\cos(\frac{\varphi}{n} + \frac{2k\pi}{n}) + i\sin(\frac{\varphi}{n} + \frac{2k\pi}{n})\big)$ \\
Dann: \begin{liste}
\item $z_j \neq z_k$ für $j \neq k$ \\
\item für $z \in \MdC : z^n = a \equizu z \in \{z_0,z_1,\ldots , z_{n-1}\}$
\end{liste}
\end{satz}

\textbf{Spezialfall:} $a = 1$ \\
$z_k = \cos(\frac{2k\pi}{n}) + i\sin(\frac{2k\pi}{n}) \quad (k=0,\ldots ,n-1) n$-te Einheitswurzeln

\begin{beispiel}
$a=1, n=4, z_k = \cos(\frac{k\pi}{2}) + i\sin(\frac{k\pi}{2}) \quad (k=0,\ldots , 3) \\
z_0 = 1, z_1 = i, z_2 = -1, z_3 = -i$
\end{beispiel}

\begin{beweis}[von 1.5]
\begin{liste}
\item Übung \\
\item $"\Leftarrow":\; z_k^{\phantom kn} \gleichwegen{1.4} |a|\big(\cos(\varphi + 2k\pi) + i\sin(\varphi + 2k\pi)\big) = |a|(\cos\varphi + i\sin\varphi) = a$ \\
$"\Rightarrow":$ Sei $z^n = a \folgt |z| = \sqrt[n]{|a|}, z \neq 0;$ \\
Sei $z = |z|(\cos\alpha + i\sin\alpha) \quad (\alpha \in \MdR) \\
a = |a|(\cos\varphi + i\sin\varphi ) = z^n \gleichwegen{1.4} \underbrace{|z|^n}_{=|a|}\big(\cos (n\alpha) + i\sin(n\alpha )\big) \\
\folgt \cos\varphi = \cos(n\alpha), \sin\varphi = \sin(n\alpha) \\
\folgt \exists j \in \MdZ : n\alpha = \varphi + 2\pi j \folgt \alpha = \frac{\varphi}{n} + \frac{2\pi j}{n} \\
\exists l \in \MdZ , k \in \{0,\ldots , n-1\} : j = ln + k \\
\folgt \frac{j}{n} = l + \frac{k}{n} = \alpha = \frac{\varphi}{n} + 2\pi (l + \frac{k}{n}) = \frac{\varphi}{n} + \frac{2\pi k}{n} + 2\pi l \\
\folgt \cos\alpha = \cos\frac{\varphi}{n} + \frac{2\pi k}{n}, \sin\alpha = \sin\frac{\varphi}{n} + \frac{2\pi k}{n}\\
\folgt z = z_k$
\end{liste}
\end{beweis}



\chapter{Topologische Begriffe}

\begin{definition}
$(a_n)$ sei eine Folge in $\MdC$.
\begin{liste}
	\item $(a_n)$ heißt \begriff{beschränkt} $\gdw \exists c \geq 0 : |a_n| \leq c$ $\forall n \in \MdN$
	\item $(a_n)$ heißt eine \begriff{Cauchy-Folge} (CF) $:\gdw$ $\forall \epsilon > 0 \exists n_0 \in \MdN : |a_n-a_m|< \epsilon $ $\forall n,m \geq n_0$
	\item $(a_n)$ heißt \begriff{konvergent} $:\gdw$ $\exists a \in \MdC : |a_n -a | \to 0$ $ ( \gdw \forall \epsilon > 0 \exists n_0 \in \MdN : |a_n - a| < \epsilon$ $\forall n \geq n_0)$ \\
	      In diesem Fall ist a eindeutig bestimmt (Übung) und heißt der \begriff{Grenzwert} (GW) oder \begriff{Limes} von $(a_n)$. Man schreibt : $\lim_{n \to \infty} a_n = a$ oder $a_n \to a (n \to \infty)$ 
	\item $(a_n)$ heißt \begriff{divergent} $:\gdw (a_n)$ konvergiert nicht.
\end{liste}
\end{definition}

\begin{beispiel}
$a_n = \frac{1}{n} + i(1+\frac{1}{n})$; $|a_n -i| = |\frac{1}{n}-i\frac{1}{n}| = \frac{|1-i|}{n} \to 0 (n \to \infty)$
$\Rightarrow a_n \to i$
\end{beispiel}
Wie in $\MdR$ bzw. mit 1.3, zeigt man:

%Satz 2.1 
\begin{satz}
$(a_n),(b_n)$ seien Folgen in $\MdC$; $a,b \in \MdC$
\begin{liste}
	\item $(a_n)$ konvergent $\Rightarrow$ $a_n$ ist beschränkt.
	\item $(a_n)$ konvergent $:\gdw$ $($Re $ a_n), ($Im $a_n)$ sind konvergent. In diesem Fall gilt $\lim a_n = \lim$ Re $a_n + i \lim$ Im $a_n$
	\item Es gelte $(a_n) \to a, (b_n) \to b$. Dann: \\ $a_n+b_n \to a+b$, $a_nb_n \to ab$, $\bar{a_n} \to \bar{a}$, $|a_n| \to |a|$ \\
		  Ist $a \neq 0$ $\Rightarrow \exists m \in \MdN : a_n \neq 0$ $\forall n \geq m$ und $\frac{1}{a_n} \to \frac{1}{a}$ \\
		  Ist $a_{n_k}$ eine Teilfolge (TF) von $(a_n)$ $\Rightarrow$ $a_{n_k} \to a (k \to \infty)$
	\item Ist $(a_n)$ beschränkt $\Rightarrow$ $(a_n)$ enthält eine konvergente TF (\begriff{Bolzano-Weierstraß})
	\item $(a_n)$ ist eine CF $\gdw$ $(a_n)$ ist konvergent (\begriff{Cauchykriterium})
\end{liste}
\end{satz}

\begin{definition}
$(a_n)$ sei eine Folge in $\MdC$ und $s_n := \sum_{i=1}^{n} a_i$ $n\in \MdN$. $(s_n)$ heißt eine \begriff{unendliche Reihe} und wird mit 
$\sum_{n=1}^{\infty} a_n$ bezeichnet. $\sum_{n=1}^{\infty} a_n$ heißt konvergent/divergent $\gdw$ $(s_n)$ konvergent/divergent. \\
Ist $\sum_{n=1}^{\infty} a_n$ konvergent, so schreibt man $\sum_{n=1}^{\infty} a_n$ $:= \lim_{n \to \infty} s_n$ 
\end{definition}

\begin{beispiel}[Geometrische Reihe]
$\sum_{n=0}^{\infty} z^n = 1+z+\cdots$ $(z \in \MdC)$. Wie in $\MdR$ zeigt man:
\begin{liste}
\item $1+z+\cdots+z^n =\begin{cases}
		\frac{1-z^{n+1}}{1-z} &, \text{falls } z \neq 1 \\
		n+1 &, \text{falls } z = 1
		\end{cases}
		$
\item $\sum_{n=0}^{\infty} z^n$ konvergent $\gdw$ $|z| < 1$. In diesem Fall $\sum_{n=0}^{\infty} z^n = \frac{1}{1-z}$ 
\end{liste}
\end{beispiel}

\begin{definition}
$\sum_{n=1}^{\infty} a_n $ heißt \begriff{absolut konvergent} $:\gdw$ $\sum_{n=1}^{\infty} |a_n|$ konvergent.
\end{definition}

Wörtlich wie in $\MdR$ beweist, bzw. formuliert man:
%Satz 2.2 
\begin{satz}
$(a_n)$ sei eine Folge in $\MdC$
\begin{liste}
\item Ist $\sum_{n=1}^{\infty} a_n$ konvergent $\Rightarrow$ $a_n \to 0$
\item Ist $\sum_{n=1}^{\infty} a_n$ absolut konvergent $\Rightarrow$ $\sum_{n=1}^{\infty} a_n$ konvergent und $|\sum_{n=1}^{\infty} a_n| \leq \sum_{n=1}^{\infty} |a_n|$
\item Es gelten Cauchykriterium, Majorantenkriterium, Minorantenkriterium, Wurzelkriterium, Quotientenkriterium und der Satz über das Cauchyprodukt.
\end{liste}
\end{satz}

\begin{definition}
Sei $A \subseteq \MdC, z_0 \in \MdC$ und $\epsilon > 0$
\begin{liste}
\item $U_{\epsilon}(z_0) := \{ z \in \MdC : |z-z_0| < \epsilon \}$ \begriff{$\epsilon$-Umgebung von $z_0$} oder \begriff{offene Kreisscheibe} von $z_0$ mit Radius $\epsilon$ \\
      $\overline{U_{\epsilon}(z_0)} := \{ z \in \MdC |$ $|z-z_0| \leq \epsilon \}$ (\begriff{abgeschlossene Kreisscheibe} von $z_0$ mit Radius $\epsilon$) \\
      $\dot{U}_{\epsilon}(z_0) := U_{\epsilon}(z_0) \backslash \{z_0\}$ (\begriff{punktierte Kreisschreibe})
\item $z_0 \in A$ heißt \begriff{innerer Punkt von A} $:\gdw$ $\exists \epsilon > 0 : U_{\epsilon}(z_0) \subseteq A$ \\
      $A^o := \{ z\in A | z $ innerer Punkt von A$ \}$ heißt das \begriff{Innere von A}. Klar ist: $A^o \subseteq A$\\
      $A$ heißt offen $:\gdw$ $A=A^o$
\item $A$ heißt \begriff{abgeschlossen} $:\gdw$ $\MdC \backslash A$ ist offen.
\item $A$ heißt \begriff{beschränkt} $:\gdw$ $\exists c \geq 0 : |a| \leq c$ $\forall a \in A$
\item $A$ heißt \begriff{kompakt} $:\gdw$ $A$ ist beschränkt und abgeschlossen.
\item $z_0$ heißt ein \begriff{Häufungspunkt} von $A$ $:\gdw$ $\forall \epsilon > 0:  \dot{U}_{\epsilon}(z_0) \cap A \neq \emptyset$.\\
      $\bar{A} := \{ z \in \MdC | z $ ist HP von $A$$ \} \cup A$ heißt die \begriff{Abschließung} von A
\item $z_0$ heißt ein \begriff{Randpunkt} von $A$ \\ $:\gdw$ $\forall \epsilon > 0: U_{\epsilon}(z_0) \cap A \neq \emptyset$ und $U_{\epsilon}(z_0) \cap (\MdC \backslash A) \neq \emptyset$ \\
	  $\partial A := \{ z \in \MdC | z $ ist Randpunkt von $A$ $\}$ wird als \begriff{Rand von A} bezeichnet
\end{liste}
\end{definition}

Wie in $\MdR$ zeigt man:
%Satz 2.3 
\begin{satz}
\begin{liste}
\item $A$ heißt abgeschlossen $\gdw$ $A = \bar{A}$ $\gdw$ der Grenzwert jeder konvergenten Folge aus $A$ gehört zu $A$.
\item $z_0$ ist HP von $A$ $\gdw$ $\exists$ Folge $(z_n)$ in $A \backslash \{z_0\}: z_n \to z_0$
\item $A$ ist kompakt $:\gdw$ jede Folge in $A$ enthält eine konvergente Teilfolge deren Limes zu $A$ gehört \\
	  $\gdw$ jede offene Überdeckung von $A$ enthält eine endliche Überdeckung von $A$.
\end{liste}
\end{satz}

\chapter{Stetigkeit, Zusammenhang, Gebiete}
In diesem Paragraphen seien $D, E \subseteq \MdC$, $D \neq \emptyset \neq E$ und $f: D \to \MdC$ eine Funktion.
Die Funktionen  Re $f$, Im $f$, $|f|: D \to \MdR$ sind definiert durch: \\
\centerline{$($Re $f)(z) := $Re $f(z)$, $($Im $f)(z) := $Im $f(z)$, $|f|(z) := |f(z)|$.}

\begin{definition}
Sei $z_0$ ein HP von D und $a \in \MdC.$ \\
$\lim_{z \to z_0} f(z) = a $ $:\gdw$ $\forall  \varepsilon > 0 \exists \delta > 0 : |f(z)-a| <  \varepsilon$ $\forall z\in \dot{U}_{ \delta}(z_0) \cap D $\\
In diesem Fall schreibt man $f(z) \to a$ $(z\to z_0)$ \\
$\lim_{z \to z_0} f(z)$ existiert $:\gdw$ $\exists a \in \MdC : \lim_{z \to z_0} f(z) = a $. Es gelten die üblichen Rechenregeln.
\end{definition}

\begin{definition}
\begin{liste}
\item Sei $z_0 \in D$. $f$ heißt \begriff{stetig} in $z_0$ $:\gdw$ $\forall  \varepsilon > 0 \exists \delta > 0 : |f(z)-f(z_0)| <  \varepsilon$ $\forall z\in \dot{U}_{ \delta}(z_0) \cap D$
\item $f$ heißt stetig auf D $:\gdw$ $f$ ist in jedem $z \in D$ stetig. In diesem Fall schreiben wir $f \in C(D)$.
\end{liste}
\end{definition}

\begin{beispiel}
\begin{liste}
\item $p(z) = a_0 + a_1z+\cdots+a_nz^n$ $(a_0,...,a_n \in \MdC)$. Klar: $p \in C(\MdC)$ (Linearkombination stetiger Funktionen).
\item  $f(z) =\begin{cases}
		\frac{\text{Re } z}{z} &, \text{falls } z\neq 0 \\
		0 &, \text{falls } z = 0.
		\end{cases}
		$\\
		Klar: $f\in C(\MdC \backslash \{0\})$. Für $z \in \MdR \backslash \{0\} $ ist $f(z) = 1 \not\to f(0) = 0$ $(z \to 0)$. $f$ ist in $z_0 = 0$ nicht stetig.
\item $f(z) =\begin{cases}
		\frac{(\text{Re } z)^2}{z} &, \text{falls } z\neq 0 \\
		0 &, \text{falls } z = 0.\\
		\end{cases}
		$\\
      Für $z \neq 0: |f(z)| = \frac{| \text{Re} z |^2}{|z|} \leq \frac{|z|^2}{|z|} \leq |z|$ $\Rightarrow$ $f$ ist in $z_0 = 0$ stetig. Insgesamt: $f \in C(\MdC)$.
\end{liste}
\end{beispiel}

\begin{beispiel}
$D = \MdC \backslash \{0\}$; für $z = |z|(\text{cos} \varphi + i \text{sin}\varphi) \in D$ mit $\varphi \in (-\pi,\pi]$ sei $f(z) := \varphi = Arg$ $ z$.
Behauptung: Ist $z_0 \in \MdR$ und $z_0 < 0$ $\Rightarrow$ $f$ ist in $z_0$ nicht stetig. Denn: \\
Sei $z_n := |z_0|(\text{cos}(\pi-\frac{1}{n})+i\text{sin}(\pi-\frac{1}{n}))$, $w_n := |z_0|(\text{cos}(-\pi+\frac{1}{n})+i\text{sin}(-\pi+\frac{1}{n}))$ 
$\Rightarrow z_n \to -|z_0| = z_0, w_n \to -|z_0| = z_0$ und $f(z_n) = Arg$ $ z_n = \pi - \frac{1}{n} \to \pi, f(w_n) = Arg$ $ w_n = -\pi + \frac{1}{n} \to -\pi$

\end{beispiel}

%ab hier Bernhard Konrad vom 3. Mai 2006

Wie im $\MdR^n$ beweist man die folgenden S"atze 3.1,3.2 und 3.3

%Satz 3.1
\begin{satz}
Sei $z_0 \in D$.
\begin{liste}
\item $f$ ist stetig in $z_0 \Leftrightarrow $Re$f$ und Im $\! f$ sind stetig in $z_0 \Leftrightarrow$ f"ur jede Folge $(z_n)$ in $D$ mit $z_n \rightarrow z_0: f(z_n) \rightarrow f(z_0).$
\item Ist $z_0$ ein HP von $D$, so gilt: f ist in $z_0$ stetig $\Leftrightarrow \lim_{z \rightarrow z_0} f(z) = f(z_0)$
\item Sei $g:D \rightarrow \MdC$ eine weitere Funktion und $f$ und $g$ seien stetig in $z_0$. Dann sind $f+g, fg, |f|$ stetig in $z_0$; ist $f(z) \not= 0 \, \forall z \in D \Rightarrow \frac1{f}$ ist stetig in $z_0$.
\end{liste}
\end{satz}

%Satz 3.2
\begin{satz}
Sei $\emptyset \not= E \subseteq \MdC, g:E \rightarrow \MdC$ eine Funktion und $f(D) \subseteq E$. Ist $f$ stetig in $z_0$ und $g$ stetig in $f(z_0)$, so ist $g \circ f: D \rightarrow \MdC$ stetig in $z_0$.
\end{satz}

%Satz 3.3
\begin{satz}
$D$ sei \begriff{kompakt} und $f \in C(D)$
\begin{liste}
\item $f(D)$ ist kompakt
\item $\exists \max |f|(D), \, \exists \min |f|(D)$
\end{liste}
\end{satz}


\begin{definition}
Sei $[a,b] \subseteq \MdR \, (a < b).$ Eine stetige Funktion $\gamma: [a,b] \rightarrow \MdC$ hei"st ein \begriff{Weg} (in $\MdC$). $\gamma(a)$ hei"st \begriff{Anfangspunkt} von $\gamma$, $\gamma(b)$ hei"st \begriff{Endpunkt} von $\gamma$. $\gamma([a,b])$ hei"st der \begriff{Tr"ager} von $\gamma$. 3.3 $\Rightarrow \gamma([a,b])$ ist kompakt. ("\begriff{Rektifizierbarkeit}" \, und "\begriff{L"ange}" von $\gamma$: siehe Analysis II)
\end{definition}

\begin{beispiele}
\item Seien $z_0, z_1 \in \MdC; \gamma(t) := z_0 + t(z_1-z_0), t \in [0,1]. \, S[z_0,z_1] := \gamma([0,1])$ hei"st die \begriff{Verbindungsstrecke} von $z_0$ und $z_1$.
\item Sei $z_0 \in \MdC, r>0; \gamma(t) := z_0 + r(\cos t + i \sin t), t \in [0,2\pi]. \gamma(0) = z_0 + r = \gamma(2\pi), \gamma([0,2\pi]) = \partial \overline{U_r(z_0)}$
\end{beispiele}

F"ur den Rest des §en sei $\emptyset \not= M \subseteq \MdC$

\begin{definition}
$M$ hei"st \begriff{konvex} $:\Leftrightarrow$ aus $z_0, z_1 \in M$ folgt stets: $S[z_0,z_1] \subseteq M.$
\end{definition}

\begin{definition}
\begin{liste}
\item Eine Funktion $\varphi: M \rightarrow \MdC$ hei"st auf $M$ \begriff{lokalkonstant} $:\Leftrightarrow \, \forall a \in M \, \exists \delta = \delta(a) > 0: \varphi$ ist auf $U_{\delta}(a) \cap M$ konstant. Beachte: i.d.Fall: $\varphi \in C(M)$.
\item $M$ hei"st \begriff{zusammenh"angend} (zsh) $:\Leftrightarrow$ jede auf $M$ lokalkonstante Funktion ist auf $M$ konstant.
\item $M$ hei"st \begriff{wegzusammenh"angend} (wegzsh) $:\Leftrightarrow$ zu je zwei Punkten $z,w \in M$ existiert ein Weg $\gamma[a,b] \rightarrow \MdC: \gamma([a,b]) \subseteq M, \gamma(a)=z$ und $\gamma(b)=w$.
\item $M$ hei"st ein \begriff{Gebiet} $:\Leftrightarrow M$ ist offen und wegzsh.
\end{liste}
\end{definition}

\begin{bemerkung}
\item (1) Mengen die offen und konvex sind, sind Gebiete.
\item (2) wegzsh $\Rightarrow$ zsh ("$\Leftarrow$" \, ist i.a. falsch)
\end{bemerkung}

%Satz 3.4
\begin{satz}
$M$ sei offen, dann sind "aquivalent:
\begin{liste}
\item $M$ ist ein Gebiet
\item $M$ ist wegzsh
\item $M$ ist zsh
\item Aus $M = A \cup B, A \cap B = \emptyset, A,B$ offen folgt stets: $A = \emptyset$ oder $B = \emptyset$.
\end{liste}
\end{satz}

\begin{beweis}
(1) $\Leftrightarrow$ (2): klar,(2) $\Leftrightarrow$ (3): ohne Beweis.\\
(3) $\Rightarrow$ (4): Sei $M = A \cup B, A \cap B = \emptyset, A,B$ offen. Annahme: $A \not= \emptyset$ und $B \not= \emptyset$. Definiere $\varphi: M \rightarrow \MdC$ durch $\varphi(z):= \begin{cases} 1, z\in A\\ 0, z\in B \end{cases}$.\\
Sei $z_0 \in M$. 1. Fall (2. Fall) : $z_0 \in A (B), A (B)$ offen $\Rightarrow \, \exists \delta >0: U_{\delta}(z_0) \subseteq A (B) \Rightarrow \varphi$ ist auf $U_{\delta}(z_0)$ konstant. $\varphi$ ist also auf $M$ lokalkonstant. Vor $\Rightarrow \varphi$ ist auf $M$ konstant $\Rightarrow 1=0$, Wid!\\
(4) $\Rightarrow$ (3): Sei $\varphi: M \rightarrow \MdC$ lokalkonstant. Annahme: $\varphi$ ist nicht konstant auf $M$. $\exists z_0, w_0 \in M: \varphi(z_0) \not= \varphi(w_0). A := \{ z \in M: \varphi(z) = \varphi(z_0) \}; z_0 \in A$, also $A \not= \emptyset. B:= M \backslash A, w_0 \in B$, also $B \not= \emptyset$. Klar: $M = A \cup B, A \cap B = \emptyset$.\\
Sei $z_1 \in A. \varphi$ ist lokalkonstant $\Rightarrow \exists \delta > 0: U_{\delta}(z_1) \subseteq M$ und $\varphi$ ist auf $U_{\delta}(z_1)$ konstant. Sei $z \in U_{\delta}(z_1). \varphi(z) = \varphi(z_1) \stackrel{z_1 \in A}{=} \varphi(z_0) \Rightarrow z \in A$. Also: $U_{\delta}(z_1) \subseteq A. A$ ist also offen. "Ahnlich: $B$ ist offen. Fazit: $M = A \cup B, A \cap B = \emptyset, A,B$ offen, $A \not= \emptyset, \, B \not= \emptyset$. Wid zur Vor.
\end{beweis}

%Folgerung 3.5
\begin{folgerung}
Sei $A \subseteq \MdC, A$ sei offen und abgeschlossen. Dann: $A = \emptyset$ oder $A = \MdC$.
\end{folgerung}
\begin{beweis}
$B:= \MdC \backslash A;$ dann $A,B$ offen, $A \cap B = \emptyset$ und $\MdC = A \cup B. \MdC$ ist ein Gebiet $\stackrel{3.4}{\Rightarrow} A = \emptyset$ oder $B = \emptyset \Rightarrow A = \emptyset$ oder $A = \MdC$.
\end{beweis}

%Satz 3.6
\begin{satz}
Sei $M$ zsh und $g \in C(M)$. Dann ist $g(M)$ zsh.
\end{satz}
\begin{beweis}
Sei $\varphi: g(M) \rightarrow \MdC$ auf $g(M)$ lokalkonstant. Zu zeigen: $\varphi$ ist auf $g(M)$ konstant. $\psi := \varphi \circ g: M \rightarrow \MdC$. Sei $z_0 \in M \Rightarrow g(z_0) \in g(M) \Rightarrow \, \exists \varepsilon > 0$ und $c \in \MdC: (\ast) \, \varphi(w) = c \,\, \forall w \in U_{\varepsilon}(g(z_0)) \cap g(M). g$ stetig in $z_0 \Rightarrow \, \exists \delta >0: |g(z) - g(z_0)| < \varepsilon \,\, \forall z \in U_{\delta}(z_0) \cap M$.
Sei $z \in U_{\delta}(z_0) \cap M$. Dann: $g(z) \in U_{\varepsilon}(g(z_0)) \cap g(M) \stackrel{(\ast)}{\Rightarrow} \varphi(g(z)) = c \Rightarrow \psi(z) = c$.
Also ist $\psi$ auf $M$ lokalkonstant. $M$ zsh $\Rightarrow \psi(z) = c \,\, \forall z \in M.$ Sei $w \in g(M) \Rightarrow \, \exists z \in M: w=g(z) \Rightarrow \varphi(w) = \varphi(g(z)) = \psi(z) = c. \, \varphi$ ist also auf $g(M)$ konstant.
\end{beweis}

\begin{beispiele}
\item $[a,b] \subseteq \MdR$ ist zsh.
\item Ist $\gamma: [a,b] \rightarrow \MdC$ ein Weg, so ist $\gamma([a,b])$ zsh.
\end{beispiele}

\begin{beweis}
(2) folgt aus (1) und 3.6\\
(1) Sei $\varphi:[a,b] \rightarrow \MdC$ lokalkonstant. Also: $\forall t \in [a,b] \, \exists \delta(t) > 0: \varphi$ ist auf $U_{\delta(t)}(t) \cap [a,b]$ konstant. $[a,b] \subseteq \cup_{t \in [a,b]} U_{\delta(t)}(t) \stackrel{2.3}{\Rightarrow} \, \exists t_1,\dots,t_n \in [a,b]: [a,b] \subseteq \cup_{j=1}^n U_{\delta(t_j)}(t_j). \, \exists c_1,\dots,c_n \in \MdC: \varphi(t) = c_j \, \forall t \in U_{\delta(t_j)}(t_j) \cap [a,b] \Rightarrow \varphi([a,b]) = \{c_1,\dots,c_n\}.$ O.B.d.A: $c_1,\dots,c_n \in \MdR$. Annahme: $c_1 \not= c_2$ etwa $c_1 < c_2. \, \varphi \in C[a,b].$ ZWS $\Rightarrow [c_1,c_2] \subseteq \varphi([a,b])$ Wid! Also: $c_1 = c_2$. Analog: $c_2=c_3=\dots=c_n. \,\, \varphi$ ist also konstant.
\end{beweis}


\chapter{Komplexe Differenzierbarkeit, Holomorphie}

In diesem §en sei $\emptyset \neq D \subseteq \MdC$, $D$ offen und $f: D \rightarrow \MdC$ eine Funktion.

\begin{definition}
\begin{liste}
\item $f$ heißt in $z_0 \in D$ \begriff{komplex differenzierbar} (komplex differenzierbar) $:\gdw$ es ex. $\lim_{z \rightarrow z_0} \frac{f(z) - f(z_0)}{z - z_0} (= \lim_{h \rightarrow 0} \frac{f(z_0 + h) - f(z_0)}{h})$. I.d. Fall heißt obiger Grenzwert die Ableitung von $f$ in $z_0$ und wird mit $f'(z_0)$ bezeichnet.
\item $f$ heißt auf $D$ \begriff{holomorph} (analytisch) $: \gdw f$ ist zu jedem $z \in D$ differenzierbar.
\item $H(D) := \{g : D \rightarrow \MdC: g$ ist auf $D$ holomorph$\}$.
\end{liste}
\end{definition}

\begin{beispiele}
\item $D = \MdC, n \in \MdN, f(z) := z^n$. \\
Wie in $\MdR$ zeigt man: $f \in H(\MdC)$ und $f'(z) = nz^{n-1} \forall z \in \MdC$.
\item $D =  \MdC, f(z) = \overline{z}$. Sei $z_0 \in \MdC$ und $h \in \MdC \backslash\{0\}$. \\
$Q(h) := \frac{f(z_0 + h) - f(z_0)}{h} = \frac{\overline{z_0} + \overline{h} - \overline{z_0}}{h} = \frac{\overline{h}}{h}$; z.B. ist $Q(h) = 1$, falls $h \in \MdR$ und $Q(h) = -1$, falls $h \in i\MdR := \{it : t \in \MdR\}$. Also ex. $\lim_{h \rightarrow 0} Q(h)$ \textbf{nicht}. $f$ ist also in \textbf{keinem} $z \in \MdC$ komplex differenzierbar.
\end{beispiele}

Sei $u := \Re f$ und $v := \Im f$. Fasst man D als Teilmenge des $\MdR ^2$ auf, und schreibt man $z = (x,y)$ statt $z = x + iy $ $(x,y \in \MdR)$, so ist $f = (u,v): D \subseteq \MdR ^2 \rightarrow \MdR ^2$ eine vektorwertige Funktion. \\
$f(z) = u(z) + iv(z) = \big(u(z),v(z)\big) = \big(u(x,y),v(x,y)\big) = f(x,y)$.

\textbf{Erinnerung (Ana II)}: $f$ heißt im $(x_0, y_0) \in D$ reell differenzierbar $:\gdw$ es ex. relle $2\times 2$-Matrix $A$: \\
$$\lim_{(h,k) \rightarrow (0,0)} \frac{f(x_0 + h,y_0 + k) - f(x_0,y_0) - A \begin{pmatrix}h\\k\end{pmatrix}}{\|(h,k)\|} = 0$$

\begin{beispiel}
$D = \MdC, f(z) = \overline{z}$, reelle Auffassung: $f(x,y) = (x, -y). f$ ist in \textbf{jedem} $(x,y) \in \MdR ^2$ reell differenzierbar, aber in \textbf{keinem} $z \in \MdC$ komplex differenzierbar.
\end{beispiel}

%satz 4.1
\begin{satz}
Sei $u:=\Re f, v:=\Im f$; Sei $z_0 = (x_0,y_0) = x_0 + iy_0 \in D $ $(x_0,y_0 \in \MdR).$ \\ $ f$ ist in $z_0$ komplex differenzierbar. $:\gdw f$ ist in $(x_0,y_0)$ reell differenzierbar und es gelten die \begriff{Cauchy-Riemannschen Differentialgleichungen} (CRD): \\
\centerline{$u_x(z_0) = v_y(z_0), u_y(z_0) = -v_x(z_0)$} \\ \\
Ist $f$ in $z_0$ komplex differenzierbar, so ist $f'(z_0) = u_x(z_0) + iv_x(z_0) = v_y(z_0) - iu_y(z_0)$
\end{satz}

\begin{beweis}
Sei $ a = \alpha + i\beta \in \MdC$ und $s = h + ik \in \MdC \backslash\{0\} (\alpha ,\beta ,h,k \in \MdR )$ \\
$f$ ist in $z_0$ komplex differenzierbar und $f'(z_0) = a \gdw \lim_{s \rightarrow 0} \frac{f(z_0 + s) - f(z_0) - as}{|s|} = 0$ \\
\begin{eqnarray}\notag \stackrel{\text{Zerlegen}}{\gdw} \lim_{(h,k) \rightarrow (0,0)} & \Big( &\underbrace{\frac{u(x_0+h,y_0+k) - u(x_0,y_0) - (\alpha h+\beta k)}{\|(h,k)\|}}_{=:\varphi _1(h,k)} \\ 
\notag + & i &\underbrace{\frac{v(x_0+h,y_0+k) - v(x_0,y_0) - \beta h - \alpha k}{\|(h,k)\|}}_{=:\varphi _2(h,k)}\Big) = 0
\end{eqnarray}

$\gdw \varphi _1 (h,k) \rightarrow 0, \varphi _2 (h,k) \rightarrow 0 ((h,k) \rightarrow (0,0))$ \\
$\gdw u$ und $v$ sind in $(x_0,y_0)$ reell differenzierbar, $u'(x_0,y_0) = (\alpha , -\beta)$ und $v'(x_0,y_0) = (\beta ,\alpha)$ \\
$\gdw f$ ist in $(x_0,y_0)$ reell differenzierbar und es gelten die CRD. Ist $f$ in $z_0$ komplex differenzierbar $\folgt f'(z_0) = a = \alpha + i\beta = u_x(z_0) + iv_x(z_0)$
\end{beweis}

\begin{folgerung}%folgerung 4.2
Es sei $f \in H(D)$
\begin{liste}
\item $f$ ist auf $D$ lokal konstant $\gdw f' = 0$ auf $D$.
\item Ist $f(D) \subseteq \MdR$, so ist $f$ auf $D$ lokal konstant.
\item Ist $f(D) \subseteq  i\MdR$, so ist $f$ auf $D$ lokal konstant.
\item Ist $D$ ein \begriff{Gebiet} so gilt:
\begin{enumerate} 
\item $f$ ist auf $D$ konstant $\gdw f' = 0$ auf $D$.
\item ist $f(D) \subseteq \MdR$ oder $\subseteq i\MdR$, so ist $f$ auf $D$ konstant.
\end{enumerate}
\end{liste}
\end{folgerung}


\begin{beweis}
$u:=\Re f, v:=\Im f$.
\begin{liste}
\item $"\folgt "$ klar! \\
$"\Leftarrow "$ 4.1 $\folgt u_x = u_y = v_x = v_y = 0$ auf $D$. Ana II $\folgt u,v$ sind auf $D$ lokal konstant.
\item $f(D) \subseteq \MdR \folgt v = 0$ auf $D \folgt v_x = v_y = 0$ auf $D \folgtnach{CRD} u_x = u_y = 0$ auf $D$. Weiter wie bei (1).
\item Sei $f(D) \subseteq i\MdR, g:=if \folgt g \in H(D), g(D) \subseteq \MdR \folgtnach{(2)} g$ ist auf $D$ lokal konstant. $\folgt f$ ist auf $D$ lokal konstant.
\item folgt aus (1),(2),(3) und 3.4
\end{liste}
\end{beweis}

\begin{satz}
Sei $z_0 \in D$ und $f$ in $z_0$ komplex differenzierbar.
\begin{liste}
\item $f$ ist in $z_0$ stetig.
\item Sei $g: D \rightarrow \MdC$ eine weitere Funktion und $g$ sei komplex differenzierbar in $z_0$
\begin{enumerate}
\item Für $\alpha ,\beta \in \MdC$ ist $\alpha f+\beta g$ komplex differenzierbar in $z_0$ und 
$$(\alpha f+\beta g)'(z_0) = \alpha f'(z_0) + \beta g'(z_0)$$
\item $fg$ ist in $z_0$ komplex differenzierbar und 
$$(fg)'(z_0) = f'(z_0)g(z_0) + f(z_0)g'(z_0)$$
\item Ist $g(z_0) \neq 0$, so ex. ein $\delta > 0: U_\delta (z_0) \subseteq D, g(z) \neq 0 \forall z \in U_\delta (z_0)$, \\
$\frac{f}{g}: U_\delta (z_0) \rightarrow \MdC$ ist in $z_0$ komplex differenzierbar und 
$$\frac{f}{g}'(z_0) = \frac{f'(z_0)g(z_0) - f(z_0)g'(z_0)}{g(z_0)^2}$$
\end{enumerate}
\item \textbf{Kettenregel}: Sei $\emptyset \neq E \subseteq \MdC, E$ offen, $f(D) \subseteq E$ und $h: E \rightarrow \MdC$ komplex differenzierbar in $f(z_0)$. Dann ist $h\circ f: D \rightarrow \MdC$ komplex differenzierbar in $z_0$ und 
$$(h\circ f)'(z_0) = h'(f(z_0))\cdot f'(z_0)$$
\end{liste}
\end{satz}

\begin{definition}
Sei $f \in H(D)$ und $z_0 \in D$. $f$ heißt in $z_0$ \begriff{zweimal komplex differenzierbar} $:\gdw f'$ ist in $z_0$ komplex differenzierbar. I.d. Fall: $f''(z_0):=(f')'(z_0)$ (2. Ableitung von $f$ in $z_0$). Entsprechend definiert man höhere Ableitungen von $f$ in $z_0$, bzw. auf $D$. Übliche Bezeichnungen: $f'', f''', f^{(4)},\ldots,f^{(0)}:=f$
\end{definition}

\chapter{Potenzreihen}
Im Folgenden sei $\emptyset \ne A \subseteq \MdC$, $(f_n)$ eine Folge von Funktionen $f_n:A\to\MdC$  und $s_n:=f_1+f_2+\dots+f_n$ $(n\in \MdN)$\\
\begin{definition}
\begin{liste}
\item $(f_n)$ heisst auf A \begriff{punktweise konvergent} $:\gdw \forall z \in A$ ist $(f_n(z))$ konvergent.\\
  In diesem Fall heisst $f:A\to\MdC$, definiert durch $f(z):=\lim_{n\to\infty} f_n(z)$, die \begriff{Grenzfunktion} von $(f_n)$.
\item $(f_n)$ heisst auf A \begriff{gleichmaessig (glm) konvergent} $:\gdw \exists f:A\to\MdC$ mit:
  $$ \forall \epsilon > 0 \exists n_0 \in \MdN: |f_n(z)-f(z)|<\epsilon \ \forall n \geq n_0 \forall z \in A $$
  In diesem Fall sagt man : $(f_n)$ konvergiert auf A gleichmaessig gegen $f$.
\item $(f_n)$ heisst auf A \begriff{lokal gleichmaessig konvergent} $:\gdw (f_n)$ konvergiert auf jeder kompakten Teilmenge von A gleichmaessig. ($\gdw\forall a\in A\exists\rho>0:(f_n)$ konvergiert auf $U_\rho(a)\cap A$ gleichmaessig)
\item $ \sum\limits_{n=1}^\infty f_n \text{ konvergiert auf A punktweise } :\gdw (s_n) \text{ konvergiert auf A punktweise.}$\\
  $ \sum\limits_{n=1}^\infty f_n \text{ konvergiert auf A gleichmaessig } :\gdw (s_n) \text{ konvergiert auf A gleichmaessig.}$\\
  $ \sum\limits_{n=1}^\infty f_n \text{ konvergiert auf A lokal gleichmaessig } :\gdw (s_n) \text{ konvergiert auf A lokal gleichmaessig.}$\\
 
\end{liste}

\end{definition}
Klar: gleichmaessig Konvergenz $\folgt$ lokal gleichmaessig Konvergenz $\folgt$ punktweise Konvergenz.\\
Wie in der Analysis zeigt man:\\

%Satz 5.1
\begin{satz}
 \begin{liste}
  \item $(f_n)$ konvergiert auf A gleichmaessig gegen $f$, alle $f_n$ seien in $z_0 \in A$ stetig. 
    $\folgt f \text{ ist in } z_0 \text{ stetig.} $
  \item \begriff{Cauchykriterium}:\\ $(f_n)$ konvergiert auf A gleichmaessig $\gdw \forall\epsilon>0 $ $\exists n_0\in\MdN:|f_n(z)-f_m(z)|<\epsilon $ $\forall n,m \ge n_0 $ $\forall z\in A$
  \item \begriff{Kriterium von Weierstrass}: \\ Sei $(a_n)$ eine Folge in $[0,\infty)$, $\sum\limits_{n=1}^\infty (a_n)$ konvergiert und $|f_n(z)|  \leq a_n $ $\forall n\in\MdN $ $\forall z\in A$. Dann konvergiert $\sum\limits_{n=1}^\infty f_n$ auf A gleichmaessig.
 \end{liste}
\end{satz}

\begin{definition}
 Sei $(a_n)_{n=0}^\infty$ eine Folge in $\MdC$ und $z_0 \in \MdC$. \\
 Eine Reihe der Form $\sum\limits_{n=0}^\infty a_n(z-z_0)^n $ heisst eine \begriff{Potenzreihe (PR)}. \\
 Wir setzen $\rho :=\lim\sup \sqrt[n]{|a_n|}$ ($\rho = \infty$ falls $(\sqrt[n]{|a_n|})$ unbeschraenkt) und \\
 $r:=\begin{cases} 0 \text{ falls } \rho = \infty \\ \infty \text{ falls } \rho = 0 \\ \frac{1}{\rho} \text{ falls } 0<\rho<\infty \end{cases}$. \\ $r$ heisst der \begriff{Konvergenzradius (KR)} der Potenzreihe.
\end{definition}
Wie in der Analysis zeigt man:

%Satz 5.2
\begin{satz}
Die Summe $  \sum\limits_{n=0}^\infty a_n(z-z_0)^n \text{ habe den Konvergenzradius } r$\\
\begin{liste}
 \item Ist $r=0$, so konvergiert die Potenzreihe nur in $z=z_0$
 \item Ist $r=\infty$, so konvergiert die Potenzreihe in jedem $z\in\MdC$ absolut. \\
  Die Potenzreihe konvergiert auf $\MdC$ lokal gleichmaessig.
 \item Ist $0<r<\infty$ so gilt: 
  \begin{liste}
   \item die Potenzreihe konvergiert in jedem $z\in U_r(z_0)$ absolut.
   \item die Potenzreihe divergiert zu jedem $z \not\in \overline{U_r(z_0)}$.
   \item f"ur $z\in \partial U_r(z_0)$ ist keine allgemeine Aussage m"oglich.
   \item die Potenzreihe konvergiert auf $U_r(z_0)$ lokal gleichmaessig.
  \end{liste}
\end{liste}
\end{satz}
\textbf{Beispiel:}\\
 \begin{liste}
  \item $ \sum\limits_{n=0}^\infty z^n $ hat den Konvergenzradius $r=1$. F"ur $|z|=1$ ist $z^n$ keine Nullfolge $ \folgt \sum\limits_{n=0}^\infty z^n $ ist divergent zu jedem $z\in\MdC$ mit $|z|=1$.
  \item $ \sum\limits_{n=0}^\infty n^n z^n $ hat den Konvergenzradius $r=0$.
  \item $ \sum\limits_{n=0}^\infty \frac{z^n}{n^2}$ hat den Konvergenzradius $r=1$. Sei $|z|=1$, $|\frac{z^n}{n^2}|=\frac{1}{n^2}$; Majorantenkriterium $\folgt \sum\limits_{n=0}^\infty \frac{z^n}{n^2}$ konvergiert.
  \item $ \sum\limits_{n=0}^\infty \frac{z^n}{n!}$. Wie in der Analysis: die Potenzreihe hat den Konvergenzradius $r=\infty$.
 \end{liste}

%Satz 5.3
\begin{satz}
$\sum\limits_{n=0}^{\infty} a_n(z-z_0)^n$ habe den Konvergenzradius $r$. 
Dann hat die Potenzreihe $\sum\limits_{n=1}^{\infty} na_n(z-z_0)^{n-1}$ ebenfalls den Konvergenzradius $r$.
\end{satz}
\begin{beweis}
\[\alpha_n = n a_n; \sqrt[n]{|\alpha_n|} = \sqrt[n]{n}\sqrt[n]{|a_n|}; \sqrt[n]{n} \to 1 
\Rightarrow \text{lim sup } \sqrt[n]{|\alpha_n|} = \text{lim sup } \sqrt[n]{|a_n|} \]
\end{beweis}
\begin{definition}
Für $z_0 \in \MdC: U_{\infty}(z_0) := \MdC$.
\end{definition}

\begin{satz}
$\sum\limits_{n=0}^{\infty} a_n(z-z_n)^n$ habe den Konvergenzradius $r > 0$ ($r = \infty$ zugelassen). Die Funktion
$f: U_r(z_0) \to \MdC$  sei definiert durch $f(z) = \sum\limits_{n=0}^{\infty} a_n(z-z_0)^n$. Dann
\begin{liste}
\item $f\in H(U_r(z_0))$ und $f'(z) = \sum\limits_{n=1}^{\infty} na_n(z-z_0)^{n-1}$ $\forall z \in U_r(z_0)$
\item $f$ ist auf $U_r(z_0)$ beliebig oft komplex db und \\ $f^{(k)}(z) = \sum\limits_{n=k}^{\infty} n(n-1) \cdots (n-k+1)a_n(z-z_0)^{n-k}$ $\forall z \in U_r(z_0)$ $\forall n \in \MdN$
\item $a_n = \frac{f^{(n)}(z_0)}{n!}$
\end{liste}
\end{satz}
\begin{beweis}
\begin{liste}
\item O.B.d.A $z_0 = 0$. \\
      Für $w \in U_r(0) : g(w) := \sum\limits_{n=1}^{\infty} na_nw^{n-1}$. Sei $w \in U_r(0).$ Wähle $\rho > 0$, so daß
      $|w| < \rho < r$. \\
      $b_n := n^2 |a_n| \rho^{n-2} $ $(n \geq 2)$; $\sqrt[n]{|b_n|} \to \frac{\rho}{r} < 1 \Rightarrow $ 
      $\sum\limits_{n=2}^{\infty} b_n$ konvergent; $c := \sum\limits_{n=2}^{\infty} b_n $. 
      Sei $z \in U_{\rho}(0)$ und $z \neq w$. Betrachte dann \\
      $ \frac{f(z)-f(w)}{z-w} - g(w) = \frac{1}{z-w} \sum\limits_{n=0}^{\infty} a_n (z^n-w^n) - \sum\limits_{n=1}^{\infty} na_nw^{n-1} 
      = \sum\limits_{n=2}^{\infty} a_n(\underbrace{\frac{z^n-w^n}{z-w}-nw^{n-1}}_{=: \alpha_n})$.\\
	  Nachrechnen: $\alpha_n = (z-w) \sum\limits_{n=1}^{n-1} k w^{k-1}z^{n-k-1}$.\\ Dann gilt: \\
	  $ |\alpha_n| = |z-w| |\sum\limits_{k=1}^{n-1} k w^{k-1}z^{n-k-1}| \leq 
	  |z-w| \sum\limits_{k=1}^{n-1} k {\underbrace{|w|}_{< \rho}} ^{k-1}{\underbrace{|z|}_{<\rho}} ^{n-k-1} $ \\ 
      $ \leq |z-w| \sum\limits_{k=1}^{n-1} k \rho^{n-2} = |z-w| \rho^{n-2} \frac{n(n-1)}{2} \leq |z-w| \rho^{n-2} n^2 $ \\
	  $ \Rightarrow |\frac{f(z)-f(w)}{z-w} - g(w)| = |\sum\limits_{n=2}^{\infty} a_n \alpha_n| \leq \sum\limits_{n=2}^{\infty} |a_n| |\alpha_n| \\
	  \leq (\sum\limits_{n=2}^{\infty}  |a_n| n^2 \rho^{n-2})|z-w| = c|z-w| $ \\
	  $\Rightarrow (z \to w)$ $f$ ist in $w$ komplex db und $f'(w) = g(w)$
	  \item folgt aus (1) induktiv.
	  \item folgt aus (2) mit $z = z_0$.
\end{liste}
\end{beweis}
\begin{definition}
Seien $r_1, r_2 \in [0, \infty) \cup \{\infty\}$. Dann \\
$ \text{min}\{r_1,r_2\} =\begin{cases}
		\text{min}\{r_1, r_2\} &, \text{falls } r_1,r_2 < \infty \\
		r_2 &, \text{falls } r_1 = \infty \\
		r_1 &, \text{falls } r_2 = \infty 
		\end{cases}$
\end{definition}
\begin{satz}
$\sum\limits_{n=0}^{\infty} a_n(z-z_0)^{n}$ und $\sum\limits_{n=0}^{\infty} b_n(z-z_0)^{n}$ seien Potenzreihen mit den Konvergenzradien $r_1$ und $r_2$. 
Dann hat für $\alpha, \beta \in \MdC$ die Potenzreihe $\sum\limits_{n=0}^{\infty} (\alpha a_n+\beta b_n)(z-z_0)^{n}$ einen Konvergenzradius $r \geq \text{min}\{r_1, r_2\}$
\end{satz}
\begin{beweis}
Klar.
\end{beweis}
\begin{beispiel}
$a_n = b_n, \alpha = 1, \beta = -1$
\end{beispiel}

\chapter{Exponentialfunktion und trigonometrische Funktionen}
Bekannt aus §5: $\sum_{n=0}^{\infty}\frac{z^n}{n!}$ konvergiert absolut in jedem $z \in \MdC$
$$ e^z := \exp(z):=\sum_{n=0}^{\infty}\frac{z^n}{n!} (z \in \MdC)$$

klar: $e^0 = 1, e^1 = e$

\begin{satz} %6.1
\begin{liste}
\item $\sum_{n=0}^{\infty}\frac{z^n}{n!}$ konvergiert auf $\MdC$ lokal gleichmäßig.
\item $\exp \in H(\MdC )$ und $\exp '(z) = \exp (z) \forall z \in \MdC$
\item \begriff{Additionstheorem}: $ e^{z+w} = e^ze^w \forall z,w \in \MdC$
\item $e^z\cdot e^{-z} = 1$, insbesondere $e^z \neq 0$
\item Für $z = x + iy (x,y \in \MdR): e^z = e^xe^{iy}, |e^{iy}| = 1, |e^z| = e^x$
\end{liste}
\end{satz} 

\begin{beweis}
\begin{liste}
\item folgt aus 5.2
\item $5.4 \folgt \exp \in H(\MdC )$ und $\exp '(z) = \sum_{n=1}^{\infty}\frac{z^{n-1}}{(n-1)!} = \exp (z) (z \in \MdC )$
\item Sei $c \in \MdC$ zunächst fest. \\
$f(z) := e^ze^{c-z} (z \in \MdC )$, \\
$f\in H(\MdC )$ und $f'(z) = e^ze^{c-z} + e^ze^{c-z}(-1) = 0 \quad \forall z \in \MdC$ \\
$\MdC$ ist ein Gebiet $\folgtnach{4.2} f$ ist auf $\MdC$ konstant. \\
$f(0) = e^c$. Also: $e^ze^{c-z} = e^c \quad \forall z\in \MdC \forall c\in \MdC$ \\
Setze $c := z + w$
\item folgt aus (3)
\item Nur zu zeigen: $|e^{iy}| = 1 (y \in \MdR)$ \\
$$ \overline{e^{iy}} = \overline{\sum_{n=0}^{\infty}\frac{(iy)^n}{n!}} = \sum_{n=0}^{\infty}\frac{(\overline{{iy}})^n}{n!} = \sum_{n=0}^{\infty}\frac{(-iy)^n}{n!} = e^{-iy}$$ \\
$\folgt |e^{iy}|^2 = e^{iy}\overline{e^{iy}} = e^{iy}e^{-iy} \gleichnach{(4)} 1$ 
\end{liste}
\end{beweis}

\begin{definition}
Für $z \in \MdC$ 
\begin{eqnarray}\notag \cos z &:=& \frac{1}{2}(e^{iz} + e^{-iz}) \quad \begriff{Cosinus}\\
\notag\sin z &:=& \frac{1}{2i}(e^{iz} - e^{-iz}) \quad \begriff{Sinus} \end{eqnarray}
\end{definition}

\begin{satz} %6.2
\begin{liste}
\item $$\cos z = \sum_{n=0}^{\infty}(-1)^n\frac{z^{2n}}{(2n)!}$$\\
$$\sin z = \sum_{n=0}^{\infty}(-1)^n\frac{z^{2n+1}}{(2n+1)!} \quad \forall z \in \MdC$$
\item $\cos , \sin \in H(\MdC )$\\
$\cos 'z = -\sin z,\; \sin 'z = \cos z \;\forall z \in \MdC$
\item $e^{iz} = \cos z + i\sin z \; \forall z \in \MdC.$ \\
Insbesondere: $e^{i\varphi} = \cos\varphi + i\sin\varphi \;\forall\varphi\in\MdR$. Damit lautet für $z \in \MdC\backslash\{0\}$ die Darstellung in Polarkoordinaten: $z = |z|e^{i\arg z}$.
\item Additionstheoreme:\begin{eqnarray}\notag\cos (z+w) &=& \cos z\cos w - \sin z\sin w \\
\notag \sin (z+w) &=& \sin z\cos w + \sin w\cos z \quad\forall z,w \in \MdC \end{eqnarray}
\item $\cos ^2z + \sin ^2z = 1 \;\forall z\in\MdC$
\end{liste}
\end{satz}



%\pagebreak




\begin{beweis}
\begin{liste}
\item nur für $\cos$: \\
$\forall z \in \MdC :$ 
$$\cos z = \frac{1}{2}\sum_{n=0}^{\infty}\underbrace{\frac{i^n + (-i)^n}{n!}}_{\begin{cases}0,\; n \text{ ungerade} \\ 2(-1)^n,\;n = 2k\end{cases}}z^n$$ \\
$\folgt \cos z = \sum_{k=0}^{\infty}(-1)^k\frac{z^{2k}}{(2k)!}$
\item Aus der Definition folgt: $\cos \in H(\MdC)$ und \\
$\cos 'z = \frac{1}{2}(ie^{iz} - ie^{-iz}) = \frac{i}{2}(e^{iz} - e^{-iz}) = \frac{-1}{2i}(e^{iz} - e^{-iz}) = -\sin z$ \\
Analog für den Sinus.
\item , (4) , (5) folgen aus der Definition.
\end{liste}
\end{beweis}

\begin{folgerung}
\begin{liste}
\item $e^{2k\pi i} = 1 \; \forall k \in \MdZ$; insbesondere: $e^{2\pi i} = 1$
\item $e^{i\pi} + 1 = 0$
\item Für $z \in \MdC : e^z = 1 \gdw \exists k\in \MdZ : z = 2k\pi i$
\item $e^{z + 2\pi i} = e^z \;\forall z\in \MdC$ (Die Exponentialfkt. hat die Periode $2\pi$)
\item Für $z \in \MdC :$\\ 
$\sin z = 0 \folgt\exists k \in \MdZ : z = k\pi$\\
$\cos z = 0 \folgt\exists k \in \MdZ : z = \frac{2k+1}{2}\pi$
\end{liste}
\end{folgerung}

\begin{beweis}
\begin{liste}
\item 6.2 (3) $\folgt e^{2k\pi i} = \cos (2k\pi) + i\sin (2k\pi) = 1 (k \in \MdZ)$
\item $e^{i\pi} \gleichnach{6.2(3)} \cos\pi + i\sin\pi = -1$
\item $"\folgt "$ Sei $z = x + iy \in \MdC (x,y \in \MdR )$ und $e^z = 1$ \\
$\folgt e^xe^{iy} = e^x(\cos y +i\sin y) = 1$ \\
$\folgt e^x\cos y = 1, e^x\sin y = 0 \folgt \sin y = 0 \folgt \exists k \in \MdZ : y = k\pi$\\
$ 1 = |e^z| = e^x \folgt x=0 \folgt \cos y = 1 \folgt k=2j (j\in \MdZ) \folgt z = i2j\pi$
\item $e^{z+2\pi i} = e^ze^{2\pi i} = e^z$
\item Nur für $\sin$. Sei $z \in \MdC :$ \\
$\sin z = 0 \gdw e^iz = e^{-iz} \gdw e^{2iz} = 1 \stackrel{(3)}{\gdw} \exists k \in \MdZ : 2iz = 2k\pi i$\\
$\gdw \exists k \in \MdZ : z = k\pi$.
\end{liste}
\end{beweis}

%Kein Plan ob das hier rein gehört, naja... (Bernhard)

\begin{definition}
F"ur $z \in \MdC$:
\begin{eqnarray} 
\notag \tan z &:=& \frac{\sin z}{\cos z}, \quad z \in \MdC \backslash \{ \frac{2k+1}{2}\pi: k\in \MdZ \} \quad \begriff{Tangens}\\
\notag \\
\notag \cot z &:=& \frac{\cos z}{\sin z}, \quad z \in \MdC \backslash \{ k\pi: k\in \MdZ \}\quad \begriff{Cotangens}
\end{eqnarray}
\end{definition}
tan und cot sind auf ihrem Definitionsbereichen holomorph.


\chapter{Der komplexe Logarithmus}

\begin{definition}
Sei $w \in \MdC \backslash\{0\}$ Jedes $z \in \MdC$ mit $e^z = w$ heißt \begriff{ein Logarithmus von $w$}. Man schreibt in diesem Fall (ungenau): $z = \log w$.
\end{definition}

\begin{satz} %7.1
Sei $w \in \MdC\backslash\{0\}, w = |w|e^{i\Arg w} (\Arg w \in (-\pi , \pi])$ \\
Für $z \in \MdC$ gilt: $e^z = w \gdw \exists k \in \MdZ : z = \log |w| + i\Arg w + 2k\pi i$
($\log |w|$ ist der reelle $\Log$)
\end{satz}

\begin{beweis}
$"\Longleftarrow ": e^z = \underbrace{e^{\log|w|}}_{|w|} e^{i\Arg w}\underbrace{e^{2k\pi i}}_{1} = |w|e^{i\Arg w} = w$ \\
$"\folgt "$ Sei $z = x + iy (x,y \in \MdR)$ und $e^z = w$. Dann: $|w| = |e^z| = e^x \folgt x = \log|w|$ \\
$|w|e^{i\Arg w} = w = e^z = e^xe^{iy} = |w| e^{iy}$ \\
$\folgt e^{iy} = e^{i\Arg w} \folgt e^{i(y-\Arg w)} = 1 \stackrel{6.3}{\folgt} \exists k \in \MdZ : iy - i\Arg w = 2k\pi i$ \\
$\folgt z = \log |w| + i\Arg w + 2k\pi i$
\end{beweis}

\begin{definition}
Die Funktion $\Log : \MdC \backslash\{0\}\rightarrow \MdC$ def. durch $\Log w := \log |w| + i\Arg w$ heißt der \begriff{Hauptzweig des Logarithmus}.
\end{definition}

\begin{beispiele}
\item Alle $\Log$ von $w = 1$: $2k\pi i$ ($k \in \MdZ$) \\
$\Log \ 1 = 0$
\item $\Log (-1) = i\pi$
\item $w = 1+i, \; |w| = \sqrt 2, \; \Arg w = \frac{\pi}{4}$ \\
$\Log (1+i) = \log \sqrt 2 + i\frac{\pi}{4}$
\end{beispiele}

\newpage

\begin{satz} %7.2
Sei $A = \{z \in \MdC : -\pi < \Im z \leq \pi\}$ \\
$f := \exp _{|A}$
\begin{liste}
\item $f$ ist auf $A$ injektiv.
\item $f(A) = \MdC\backslash\{0\}$
\item $f^{-1}(w) = \Log w$ ($w \in \MdC\backslash\{0\}$)
\item Die Funktion $\Log$ ist unstetig in jedem $w \in \MdR$ und $w < 0$
\end{liste}
\end{satz}

\begin{beweis}
\begin{liste}
\item 6.3, 7.1
\item 6.3, 7.1
\item 6.3, 7.1
\item §3 Beispiel: $w \mapsto \Arg w$ ist in $w < 0$ unstetig.
\end{liste}
\end{beweis}


%Ab hier Bernhard Konrad und Matej Belica, Mi 17.05.2006

\begin{definition}
$\MdC\_ := \MdC \backslash \{ t \in \MdR: t \leq 0 \} \; (\subseteq \MdC \backslash \{0\})$\\
\\
F"ur $w \in \MdC\_$ ist $\Arg w \in (-\pi,\pi)$.
\end{definition}

%Satz 7.3
\begin{satz}
$\Log \in C(\MdC\_)$
\end{satz}

\begin{beweis}
Sei $w_0 \in \MdC\_$,
$z_0 := \Log w_0$, $x_0 := \Re z_0$, $y_0:=\Im z_0$; also: $x_0 = \log |w_0|$,
$y_0 = \Arg w_0 \in (-\pi,\pi)$. $R:= \{ z= x+iy: x,y \in \MdR, |x-x_0| \leq \log 2, |y| \leq  \pi \}$.\\
Sei $\varepsilon > 0$ so klein, dass $K:= R \cap \left( \MdC \backslash U_{\varepsilon}(z_0)\right) \not= \emptyset$. \; Klar: $K$ ist kompakt, $z_0 \notin K$.\\
Definiere $\varphi: K \rightarrow \MdR$ durch $\varphi(z):= |e^z - w_0| = |e^z - e^{z_0}|$.\\
Dann: $\varphi \in C(K)$. 3.3 $\Rightarrow \exists \varrho := \min \varphi(K)$. \\
Annahme: $\varrho = 0$. Also existiert ein $z \in K: e^z = e^{z_0} \Rightarrow e^{z-z_0}=1$. \; 6.3 $\Rightarrow \exists j \in \MdZ: z-z_0 = 2j\pi i \Rightarrow 2j\pi = \Im(z-z_0) = \Im z - \Im z_0 \Rightarrow 2|j|\pi = |\Im z - \Im z_0| \leq \underbrace{|\Im z|}_{\leq \pi} + \underbrace{|\Im z_0|}_{< \pi} < 2\pi \Rightarrow j = 0 \Rightarrow z_0 = z \in K$. Wid!\\
Also: $\varrho > 0$\\
$\delta := \min\{ \varrho, \frac{1}{2}e^{x_0}\}$. Sei $w \in \MdC\_$ und $|w-w_0| < \delta$; \; $z:=\log w$. \; Z.z: $|z-z_0|<\varepsilon$. \\
Sei $z=x+iy$ ($x,y \in \MdR$); \, $y = \Arg w \in (-\pi,\pi)$, also: $|y| \leq \pi$. \\
Annahme: $x>x_0 + \log 2$. Dann:\\
\[
\frac{1}{2} e^{x_0} \geq \delta > |w-w_0| = |e^z - e^{z_0}| \geq | |e^z| - |e^{z_0}| | = |e^x - e^{x_0}| \geq e^x - e^{x_0} > e^{x_0 + \log 2} - e^{x_0} = e^{x_0} \quad \mbox{Wid!}
\]
Also: $x \leq x_0 + \log 2$. Analog: $x \geq x_0 - \log 2.$ \\
Fazit: $z\in R.$\\
Annahme: $|z-z_0| \geq \varepsilon \, \Rightarrow z\in K \Rightarrow \delta \leq \varrho \leq \varphi(z) = |e^z - e^{z_0}| = |w-w_0| < \delta. \quad \mbox{Wid!}$
\end{beweis}

%Satz 7.4
\begin{satz}
$\Log \in H(\MdC\_)$ und $\Log'w= \frac1w \; \forall w \in \MdC\_$
\end{satz}

\begin{beweis}
Sei $w_0 \in \MdC\_$; $(w_n)$ eine Folge in $\MdC\_$ mit: $w_n \not= w_0 \ \forall n \in \MdN$ und $w_n \rightarrow w_0$, \, $z_0 := \Log w_0$, \, $z_n:=\Log w_n$. \; 7.3 $\Rightarrow z_n \rightarrow z_0$. Dann:\\
\[
\frac{\Log w_n - \Log w_0}{w_n - w_0} = \frac{z_n-z_0}{e^{z_n}-e^{z_0}} = \left(\frac{e^{z_n}-e^{z_0}}{z_n -z_0}\right)^{-1} \rightarrow \frac1{e^{z_0}} = \frac1{w_0}
\]
D.h. $\Log$ ist in $w_0$ komplex differenzierbar und $\Log'w_0 = \frac{1}{w_0}$
\end{beweis}

\textbf{Bezeichnung:} $\; \MdD := \{ z \in \MdC : |z| < 1 \} = U_1(0)$ \\
Beachte: F"ur $z \in \MdD$ ist $1-z \in \MdC\_$

%Satz 7.5
\begin{satz}
F"ur alle $ z \in \MdD$ gilt:
\[
\Log (1+z) = \sum_{n=1}^{\infty} (-1)^{n+1} \frac{z^n}{n} 
\]
\end{satz}

\begin{beweis}
7.4, 5.4 $\Rightarrow f(z) := \Log(1+z) - \sum_{n=1}^{\infty} (-1)^{n+1} \frac{z^n}{n}$ ist auf $\MdD$ holomorph und \\
$f'(z) = \frac1{1+z} - \sum_{n=1}^{\infty} (-1)^{n+1} z^{n-1} = \frac1{1+z} - \sum_{n=1}^{\infty} (-z)^{n-1} = \frac1{1+z} - \frac1{1-(-z)} = 0 \ \forall z \in \MdD$\\
$\MdD$ ist ein Gebiet $\stackrel{4.2}{\Rightarrow} f$ ist auf $\MdD$ konstant. $f(0) = 0 \Rightarrow$ Beh.
\end{beweis}

\begin{definition}
Sei $w \in \MdC \backslash \{0\}$ und $a \in \MdC$.\\
$w^a := e^{a \Log w} \quad $ (\begriff{Hauptzweig der allgemeinen Potenz})
\end{definition}

\begin{beispiele}
\item F"ur $a= k \in \MdZ$ ist obige Definition die fr"uhere Potenz von $w$. Denn: $\forall k \in \MdN:$ \\
$e^{k \Log w} = e^{\Log w + \Log w + \dots + \Log w} = \left(e^{\Log w}\right)^k = w^k$\\
$e^{-k \Log w} = \frac1{e^{k \log w}} \stackrel{s.o.}{=} \frac1{w^k} = w^{-k}$
\item $w=a=i, \, \log|w| = 0, \, \Arg w = \frac{\pi}2, \, \Log w = i \frac{\pi}{2} \, \Rightarrow i^i = e^{i\cdot i \frac{\pi}{2}} = e^{-\frac{\pi}{2}} \in \MdR$
\end{beispiele}

%Satz 7.6
\begin{satz}
Sei $a \in \MdC$ und $f: \MdC\_ \rightarrow \MdC$ definiert durch $f(w) := w^a$. Dann: \\
$f \in H(\MdC\_)$ und $f'(w) = aw^{a-1} \ \forall w \in \MdC\_$
\end{satz}

\begin{beweis}
$7.4, 4.4 \Rightarrow f \in H(\MdC\_)$ und $f'(w) = e^{a \Log w}(a \Log w)' = ae^{a \Log w}\frac1w \stackrel{Bsp(1)}{=} ae^{a \Log w}e^{- \Log w} = ae^{(a-1) \Log w} = aw^{a-1}$
\end{beweis}

\chapter{Komplexe Wegintegrale}

Im Folgenden sei $I= [a,b] \subseteq \MdR \; (a<b)$ und $\varphi,\psi: I \rightarrow \MdC$ Funktionen.

\begin{definition}
\begin{liste}
\item Ist $\varphi$ auf $I$ stetig, so setze: $\int_a^b \varphi(t)dt := \int_a^b \Re \varphi(t)dt + i \int_a^b \Im \varphi(t)dt$; \quad $\int_b^a \varphi(t)dt := -\int_a^b \varphi(t)dt; \quad \int_a^a \varphi(t)dt := 0$
\item $\varphi$ hei"st auf $I$ \begriff{differenzierbar} (db) $\Leftrightarrow \Re \varphi, \, \Im \varphi$ sind auf $I$ differenzierbar.
In diesem Fall: $\varphi' := (\Re \varphi)' + i(\Im \varphi)'$
\item $\varphi$ hei"st auf $I$ \begriff{stetig differenzierbar} $\Leftrightarrow \Re \varphi, \, \Im \varphi$ sind auf $I$ stetig differenzierbar.\\
\end{liste}
\end{definition}

%Satz 8.1
\begin{samepage}\begin{satz}
Sei $D \subseteq \MdC$ offen, $f\in H(D), \, \varphi(I) \subseteq D$ und $\varphi$ auf $I$ differenzierbar.\\
 Dann ist $f \circ \varphi: I \rightarrow \MdC$ differenzierbar auf $I$ und $(f \circ \varphi)'(t) = f'(\varphi(t))\varphi'(t) \, \forall t \in I.$
\end{satz}\end{samepage}

\begin{beweis}
"Ubung!
\end{beweis}

%Satz 8.2)
\begin{samepage}\begin{satz}
\begin{liste}
\item Sei $\varphi$ stetig auf $I$ und $\Phi: I \rightarrow \MdC$ definiert durch $\Phi(s):= \int_a^s \varphi(t)dt$. Dann ist $\Phi$ stetig differenzierbar auf $I$ und $\Phi' = \varphi$ auf $I$
\item Sei $\varphi$ auf $I$ stetig differenzierbar $\Rightarrow \int_a^b \varphi'(t)dt = \varphi(b) - \varphi(a)$
\end{liste}
\end{satz}\end{samepage}

\begin{beweis}
"Ubung!
\end{beweis}

\begin{definition}
Sei $\gamma: [a,b] \rightarrow \MdC$ ein Weg (also $\gamma$ stetig).
\begin{liste}
\item $\Tr(\gamma):= \gamma([a,b])$ \quad \begriff{Tr"ager von $\gamma$}
\item $\gamma$ hei"st \begriff{geschlossen} $:\Leftrightarrow \gamma(a) = \gamma(b)$
\item $\gamma$ hei"st \begriff{glatt} $:\Leftrightarrow \gamma$ ist auf $[a,b]$ stetig differenzierbar.
\end{liste}
\end{definition}

\begin{definition}
Sei $n \in \MdN, a_1,\, \dots,\, a_{n+1} \in \MdR, \; a_1<a_2<\dots<a_{n+1}$ und $\gamma_j:=[a_j,a_{j+1}] \rightarrow \MdC$ seien Wege ($j=1,\dots,n$) mit $\gamma_j(a_{j+1}) = \gamma_{j+1}(a_{j+1}) \, (j=1,\dots,n-1)$. \\
Definiere $\gamma:[a_1,a_{n+1}] \rightarrow \MdC$ durch $\gamma(t) := \gamma_j(t)$, falls $t\in [a_j,a_{j+1}].$\\ 
Dann ist $\gamma$ ein Weg und man schreibt: $\gamma = \gamma_1 \oplus \gamma_2 \oplus \dots \oplus \gamma_n$\\
$\gamma$ hei"st \begriff{st"uckweise glatt} $: \Leftrightarrow \gamma_1,\ldots,\gamma_n$ sind glatt.
\end{definition}


\begin{bemerkungen}
\item Sei $\gamma:[a,b] \rightarrow \MdC$ ein Weg. $\gamma$ ist st"uckweise glatt $\Leftrightarrow \exists \, a_1,\dots,a_{n+1} \in [a,b]: \, a=a_1<a_2<\dots<a_{n+1}=b$ und $\gamma_{|[a_j,a_{j+1}]}$ ist glatt ($j=1,\dots,n$)
\item st"uckweise glatte Wege sind rektifizierbar.
\item glatt $\Rightarrow$ st"uckweise glatt
\end{bemerkungen}

\begin{beispiel}
$\gamma_1(t) := t \, (t \in [0,1]), \; \gamma_2(t):= 1+ i(t-1) \, (t \in [1,2]). \\
\gamma := \gamma_1 \oplus \gamma_2, \;\, \gamma_1, \gamma_2$ sind glatt, \, $\gamma_1'(t) = 1 \not= \gamma_2'(1) = i \Rightarrow \gamma$ in 1 nicht differenziebar.
\end{beispiel}

Wie in $\MdR$ zeigt man: Ist $\phi:\abnC$ stetig, so gilt: $|\iint_a^b\phi(t)dt|\le\iint_a^b|\phi(t)|dt$.\\
F"ur den Rest des Paragraphen sei $\gamma:\abnC$ stets ein Weg (also stetig).

\begin{definition}
$\gamma^-:\abnC,\gamma^-(t):=\gamma(b+a-t);\gamma^-$ hei"st der zu $\gamma$ \begriff{inverse Weg}.\\
Klar: $\Tr(\gamma)=\Tr(\gamma^-)$
\end{definition}
\begin{definition}
  Ist $\gamma$ glatt, so setze $L(\gamma):=\iint_a^b|\gamma'(t)|dt$. \\
  Ist $\gamma = \gamma_1 \oplus \cdots \oplus \gamma_n$ st"uckweise glatt (mit $\gamma_1,\ldots,\gamma_n$ glatt), so setze:\\
  $L(\gamma):=L(\gamma_1)+\dots+L(\gamma_n)$\\
  $L(\gamma)$ hei"st \begriff{Wegl"ange} von $\gamma$.
\end{definition}

\begin{beispiele}
    \item Seien $z_1,z_2\in\MdC,\gamma(t):=z_1+t(z_2-z_1)(t\in[0,1]),\gamma$ ist glatt.\\
      $\gamma'(t)=z_2-z_1$. 
      \folgt $L(\gamma)=\iint_0^1|z_2-z_1|dt=|z_2-z_1|$
    \item Sei $z_0\in\MdC,r>0$ und $\gamma(t):=z_0+re^{it}(t\in[0,2\pi])$\\
     
      $\gamma$ ist glatt, $\gamma'(t)=rie^{it}$, $|\gamma'(t)|=r$\folgt $L(\gamma) = \iint_0^{2\pi}rdt=2\pi r$.
\end{beispiele}

\begin{definition}
Sei $[\alpha,\beta]\subseteq\MdR$ und $h:[\alpha,\beta]\to[a,b]$ stetig differenzierbar, bijektiv und $h(\alpha)=a,h(\beta)=b$. Ist $\gamma$ st"uckweise glatt, so setze $\Gamma:=\gamma \circ h$, also $\Gamma(s)=\gamma(h(s))(s\in[\alpha,\beta])$.\\
Dann ist $\Gamma$ ein st"uckweise glatter Weg mit $\Tr(\Gamma)=\Tr(\gamma)$.\\
$h$ hei"st eine \begriff{Parametertransformation}. Man schreibt $\Gamma \sim \gamma$.
\end{definition}

\begin{definition}
Sei $f \in C(\Tr(\gamma));$\\
Ist $\gamma$ glatt, so setze $\wegint f(z)dz:=\iint_a^bf(\gamma(t))\gamma'(t)dt$\\
Ist $\gamma=\gamma_1 \oplus \cdots \oplus \gamma_n$ st"uckweise glatt mit $\gamma_1,\ldots,\gamma_n$ glatt, so setze $\wegint f(z)dz:=\sum\limits_{j=1}^n\iint_{\gamma_j} f(z)dz$.\\
$\wegint f(z)dz$ heisst ein (komplexes) \begriff{Wegintegral} (von f l"angs $\gamma$)
\end{definition}

\begin{beispiel}
$\gamma(t):=3e^{it}(t\in[0,2\pi])$ 
 \begin{liste}
 \item $f(z)=\overline{z},\wegint \overline{z}dz=\iint_0^{2\pi} 3e^{-it}i3e^{it}dt=18\pi i$.
 \item $f(z)=z^2,\wegint z^2dz=\iint_0^{2\pi} 9e^{2it}i3e^{it}dt=0$.
 \end{liste}
\end{beispiel}

Wie in der Analysis zeigt man:


%Satz 8.3
\begin{samepage}\begin{satz}
$\gamma$ sei st"uckweise glatt, $f,g:\Tr(\gamma)\to\MdC$ seien stetig, $\alpha,\beta\in\MdC$ und $\Gamma$ sei ein st"uckweise glatter Weg mit $\Gamma \sim \gamma$.
 \begin{liste}
  \item $\wegint(\alpha f(z)+\beta g(z))dz=\alpha\wegint f(z)dz + \beta\wegint g(z)dz$
  \item $\iint_\Gamma f(z)dz=\wegint f(z)dz$ (und $L(\Gamma) = L(\gamma)$)
  \item $\iint_{\gamma^-}f(z)dz=-\wegint f(z)dz$
 \end{liste}
\end{satz}\end{samepage}


%Satz 8.4
\begin{samepage}
\begin{satz}
$\gamma$ und $f$ seien wie in 8.3. $f_n$ sei eine Folge in $C(\Tr(\gamma))$ und es sei $M:=\max_{z\in\Tr(\gamma)}|f(z)|$. Dann: 
\begin{liste}
\item $|\wegint f(z)dz| \le M L(\gamma)$\\
\item Konvergiert die Folge $(f_n)$ auf $\Tr(\gamma)$ gleichmaessig gegen f, so gilt: \\ \\ 
\centerline{$\wegint f_n(z)dz \to\wegint f(z)dz (n\to\infty)$}\\
(also: $\lim_{n\to\infty}\wegint f_n(z)dz = \wegint(\lim_{n\to\infty}f_n(z))dz$ )
\end{liste}
\end{satz}
\end{samepage}

\begin{beweis}
\begin{liste}
   \item $|\wegint f(z)dz| = |\iint_a^bf(\gamma(t))\gamma'(t)dt|\le \iint_a^b|f(\gamma(t))||\gamma'(t)|dt\le\iint_a^b M |\gamma'(t)|dt=M L(\gamma)$.
   \item $M_n:=\max_{z\in\Tr(\gamma)}|f_n(z)-f(z)|$. Vorraussetzung \folgt $M_n\to 0(n\to\infty)$.\\
        $|\wegint f_n(z)dz-\wegint f(z)dz|=|\wegint(f_n(z)-f(z))dz| \stackrel{(1)}{\leq} M_n L(\gamma)$.
\end{liste}
\end{beweis}

\begin{definition}
Sei $\emptyset \neq D \subseteq \MdC$, $D$ offen und $f\in C(D)$. $f$ besitzt auf D eine \begriff{Stammfunktion} (SF) :\gdw $\exists F\in H(D):F'=f$ auf $D$.
\end{definition}

%Satz 8.5
\begin{samepage}\begin{satz}
Sei $\emptyset \neq D \subseteq \MdC$, $D$ offen und $f\in C(D)$ besitze auf $D$ die Stammfunktion $F$ und $\gamma$ sei ein st"uckweiser glatter Weg mit $\Tr(\gamma)\subseteq D$. Dann:\\
\\ \centerline{$\wegint f(z)dz = F(\gamma(b))-F(\gamma(a))$}
\end{satz}\end{samepage}

\begin{beweis}
O.B.d.A: $\gamma$ glatt. $\wegint f(z)dz = \iint_a^bf(\gamma(t))\gamma'(t)dt=\iint_a^b F'(\gamma(t))\gamma'(t)dt=\iint_a^b(F\circ \gamma)'(t)dt \gleichnach{8.2} (F \circ \gamma)(b) - (F\circ \gamma)(a)$.
\end{beweis}

%Folgerung 8.6
\begin{folgerung}
 $D$, $f$ und $\gamma$ seien wie in 8.5. Ist $\gamma$ geschlossen \folgt $\wegint f(z)dz = 0$.
\end{folgerung}

%Beispiel 8.7
\begin{wichtigesbeispiel}
Sei $z_0 \in \MdC$, $r>0$. $\gamma(t)=z_0+re^{it} (t\in[0,2\pi])$; $\gamma$ ist geschlossen. \\
F"ur $k\in\MdZ$ sei $f_k(z):=(z-z_0)^k $ $(z\in\MdC \backslash \{z_0\})$\\
\begin{liste}
   \item Sei $k\neq -1$. $f_k$ hat auf $\MdC \backslash \{z_0\}$ die Stammfunktion $z\mapsto \frac{1}{k+1}(z-z_0)^{k+1}$.\\
     8.6\folgt $\wegint(z-z_0)^kdz=0$
   \item Sei $k=-1$: $\wegint \frac{dz}{z-z_0} = \iint_0^{2\pi} \frac1{re^{it}}ire^{it}dt = 2\pi i$.
\end{liste}
Also: 
$$\frac1{2\pi i} \wegint \frac{dz}{z-z_0} = 1$$ \\ 
Wegen 8.6: Die Funktion $z\mapsto\frac1{z-z_0}$ hat auf $\MdC \backslash \{z_0\}$ keine Stammfunktion!\\
Aber:  im Falle $z_0=0$ hat die Funktion $z \mapsto \frac1z$ die Stammfunktion $\Log z$ auf $\MdC_-$.
\end{wichtigesbeispiel}

%satz 8.8
\begin{samepage}\begin{satz}
   Sei $[a_j,b_j]\subseteq \MdR$ und $\gamma_j:[a_j,b_j]\to\MdC$ glatte Wege mit $\gamma_j(b_j)=\gamma_{j+1}(a_{j+1}) $ $ (j=1,\ldots,n)$ \\
   Dann exisitert ein st"uckweise glatter Weg $\gamma$ mit: $\Tr(\gamma) = \bigcup\limits_{j=1}^n \Tr(\gamma_j),L(\gamma)=L(\gamma_1)+\cdots+L(\gamma_n)$ und $\wegint f(z)dz = \iint_{\gamma_1} f(z)dz+\cdots+\iint_{\gamma_n} f(z)dz $ $\forall f\in C(\Tr(\gamma))$
   
\end{satz}\end{samepage}

\begin{beweis}
mit 8.3(2).
\end{beweis}
\begin{beispiel}
$\gamma_1(t)=t, \,(t\in[0,1]);$ $\gamma_2(t)=1+it, t\in[0,1]$\\
$\tilde\gamma_2(t):=1+i(t-1), t\in[1,2]$. Dann: $\tilde\gamma_2 \sim \gamma_2;$ $\gamma:=\gamma_1\oplus\tilde\gamma_2:[0,2]\to\MdC$.
\end{beispiel}

%satz 8.9
\begin{samepage}\begin{satz}
  Sei $(K_n)$ eine Folge kompakter Mengen in $\MdC$ mit: $K_n\neq\emptyset $ $ (\forall n\in\MdN), K_1\supseteq K_2\supseteq K_3\supseteq \ldots$ und $d(K_n):=\max\limits_{z,w\in K_n}|z-w| \to 0 $ $(n\to\infty)$. Dann: \\
  \centerline{ $\bigcap\limits_{n\in\MdN} K_n \neq \emptyset.$}
\end{satz}\end{samepage}
\begin{beweis}
W"ahle in jedem $K_n$ ein $z_n$. F"ur n,k $\in \MdN:z_n,z_{n+k}\in K_n$\\
Dann: $|z_n-z_{n+k}|\leq d(K_n) \folgt (z_n)$ ist eine Cauchy-Folge \folgt $\exists z_0\in\MdC:z_n\to z_0$.\\ \\
Sei $N\in\MdN$. $z_n \in K_N \forall n\ge N$. $K_N$ abgeschlossen \folgt $z_0\in K_N$.
\end{beweis}

\chapter{Cauchyscher Integralsatz und Cauchysche Integralformeln} 
\begin{definition}
Seien $z_1,z_2,z_3 \in \MdC. \Delta := \Delta_{z_1,z_2,z_3} := \{t_1z_1+t_2z_2+t_3z_3: t_1,t_2,t_3 \geq 0, t_1+t_2+t_3 = 1 \}$
\end{definition}

$\Delta$ hei"st ein \begriff{Dreieck} ($\Delta$ ist kompakt)
\[
\gamma_1(t):=z_1+t(z_2-z_1) (t \in [0,1])
\]
\[
\gamma_2(t):=z_2+(t-1)(z_3-z_2) (t \in [1,2])
\]
\[
\gamma_3(t):=z_3+(t-2)(z_1-z_3) (t \in [2,3])
\]
$\gamma := \gamma_1\oplus\gamma_2\oplus\gamma_3: [0,3] \rightarrow \MdC$ ist ein st"uckweise glatter Weg mit $\Tr(\gamma) = \partial\Delta.$\\
Wir setzen (ausnahmsweise): $L(\partial\Delta) = L(\gamma)$ und $\int_{\partial\Delta} f(z)dz:=\int_{\gamma} f(z)dz.\ (f \in C(\partial\Delta)) $


%Eigentlich Lemma 9.1
\begin{satz}[Lemma von Goursat]
Sei $\emptyset \not= D \subseteq \MdC$, $D$ offen und $f\in H(D)$.\\
Ist $\Delta \subseteq D$ ein Dreieck, so gilt: $\int_{\partial\Delta}f(z)dz = 0$
\end{satz}

\begin{beweis}
Sei $\Delta = \Delta_{z_1,z_2,z_3}, \gamma_1,\gamma_2,\gamma_3,\gamma$ wie oben.\\
Fall 1: $z_1=z_2$\\
Fall 1.1: $z_3=z_1: \gamma(t)=z_3 \,\forall t\in[0,3]$. Dann: $\gamma'_1,\gamma'_2,\gamma'_3=0 \Rightarrow \int_{\partial\Delta}f(z)dz = \int_{\gamma_1}0 + \int_{\gamma_2}0 + \int_{\gamma_3}0 = 0$.\\
Fall 1.2: $z_3 \not= z_1. \gamma_1(t)=z_1, \gamma'_1=0$, also $\int_{\gamma_1}=0, \gamma_2^{-} \sim \gamma_3 \Rightarrow \int_{\gamma_3} = \int_{\gamma_2^-} \stackrel{8.3}{=} - \int_{\gamma_2} \Rightarrow \int_{\partial\Delta}f(z)dz = \int_{\gamma_1} + \int_{\gamma_2} - \int_{\gamma_2} = 0.$\\
Fall 2: $\Delta$ ist ein echtes Dreieck ($z_1\not=z_2\not=z_3, z_3\not=z_1$). Verbinde die Mittelpunkte der Kanten von $\Delta$ durch Geraden.

%
%Skizze?
%

Wir erhalten 4 Dreiecke\footnote{Die Skizze taucht hier leider nicht auf, ich versuchs mal zu erkl"aren: Verbindet man alle Seitenhalbierenden miteinander, so entstehen in einem Dreieck vier kleine Dreiecke. Diese nummeriert man nun gegen den Uhrzeigersinn mit 1,2,3, das mittlere aber nennt man 4.} $\Delta_1,\Delta_2,\Delta_3,\Delta_4$.\\
Es existieren st"uckweise glatte Wege $\alpha_1,\alpha_2,\alpha_3,\alpha_4$ mit $\Tr(\alpha_j) = \partial\Delta_j\ (j=1,\ldots,4)$.\footnote{gegen den Uhrzeigersinn}\\
Die Summe der Integrale in entgegengesetzten Richtungen l"angs der Kanten von $\Delta_4=0.$\\ %verstehe wer will
Also: 
\[
\sum_{j=1}^4 \int_{\partial\Delta_j}f(z)dz = \int_{\partial\Delta} f(z)dz
\]
Somit:  
\[
\left|\int_{\partial\Delta} f(z)dz \right| \leq \sum_{j=1}^4  \underbrace{\left| \int_{\partial\Delta_j} f(z)dz \right|}_{=: a_j}
\]
O.B.d.A: $a_1 = \max\{a_1,\ldots,a_4\}. \Delta^{(1)}:=\Delta_1$. Fazit: $|\int_{\partial\Delta} f(z)dz| \leq 4 | \int_{\partial\Delta^{(1)}} f(z)dz|$ und $L(\partial\Delta^{(1)}) = \frac12L(\partial\Delta)$ \footnote{Diese Gleichheit folgt aus geometrischen "Uberlegungen an Dreiecken.}
\\
Verfahre mit $\Delta^{(1)}$ genauso wie mit $\Delta$. Wir erhalten ein Dreieck $\Delta^{(2)} \subseteq \Delta^{(1)} \subseteq \Delta: |\int_{\partial\Delta^{(1)}} f(z)dz| \leq 4 |\int_{\partial\Delta^{(2)}} f(z)dz|, L(\partial\Delta^{(2)})= \frac12L(\partial\Delta^{(1)}).$\\

Also: $|\int_{\partial\Delta} f(z)dz| \leq 4^2 |\int_{\partial\Delta^{(2)}} f(z)dz|$ und $L(\partial\Delta^{(2)}) = \frac1{2^2} L(\partial\Delta)$.\\

Induktiv erhaelt man eine Folge $(\Delta^{(n)})$ von Dreiecken mit:\\
$\Delta \supseteq \Delta^{(1)} \supseteq \Delta^{(2)} \supseteq \dots, |\int_{\partial\Delta} f(z)dz| \leq 4^n |\int_{\partial\Delta^{(n)}} f(z)dz|$ und $L(\partial\Delta^{(n)}) = \frac1{2^n} L(\partial\Delta) (n \in \MdN)$\\
$8.9 \Rightarrow \,\exists z_0 \in D: z_0 \in \Delta^{(n)} \,\forall n \in \MdN.$\\

Definiere: $\varphi: D \rightarrow \MdC$ durch
\[
\varphi(z) = \begin{cases} \frac{f(z)-f(z_0)}{z-z_0} & , z \not= z_0 \\ f'(z_0) & , z= z_0 \end{cases}
\]
$\Rightarrow f \in H(D) \Rightarrow \varphi \in C(D)$. Es ist \\ 
 $f(z) = f(z_0) + \varphi(z)(z-z_0) =  \\ \underbrace{f(z_0) +
 f'(z_0)(z-z_0)}_{=:f_1(z)}
 + \underbrace{(\varphi(z) - f'(z_0))(z-z_0)}_{=:f_2(z)} \,\forall z\in D$.\\

Sei $\varepsilon > 0: \,\exists \delta>0: U_{\delta}(z_0) \subseteq D$ und $|\varphi(z)-f'(z_0)| \leq \varepsilon \,\forall z \in U_{\delta}(z_0) \footnote{Folgt aus der Stetigkeit.}. \,\exists m \in \MdN: \Delta^{(m)} \subseteq U_{\delta}(z_0).$
F"ur $z \in \partial\Delta^{(m)}: |z-z_0| \leq \footnote{$\max_{w,z \in \Delta} |w-z| \leq L(\partial \Delta)$} L(\partial\Delta^{(m)})$ und
$|\varphi(z)-f'(z_0)| \leq \varepsilon$. \\
Dann: $|f_2(z)| \leq \varepsilon L(\partial\Delta^{(m)}) \,\forall z \in \partial\Delta^{(m)}$. Also: $|\int_{\partial\Delta^{(m)}} f_2(z) dz| \stackrel{8.4}{\leq} \varepsilon L(\partial\Delta^{(m)})^2$.\\

$f_1$ hat auf $D$ die Stammfunktion $f(z_0)z+ \frac12 f'(z_0)(z-z_0)^2 \stackrel{8.6}{\Rightarrow} |\int_{\partial\Delta^{(m)}}f_1(z)dz| = 0$.

Dann: $|\int_{\partial\Delta} f(z)dz| \leq 4^m |\int_{\partial\Delta^{(m)}}f(z)dz| = 4^m |\int_{\partial\Delta^{(m)}}f_2(z)dz| \leq 4^m \varepsilon L(\partial\Delta^{(m)})^2 = 4^m \varepsilon(\frac1{2^m}L(\partial\Delta))^2 = 4^m \varepsilon \frac1{4^m} L(\partial\Delta)^2 = \varepsilon L(\partial\Delta)^2$.\\

Fazit: $\forall \varepsilon > 0$ gilt: $|\int_{\partial\Delta} f(z)dz| \leq \varepsilon L(\partial\Delta)^2$.
\end{beweis}
\emph{Hilfssatz 1:}\\
Sei $z_0 \in \MdC, r>0$ und $f \in C(U_r(z_0)).$ F"ur jedes Dreieck $\Delta \subseteq U_r(z_0)$ gelte: $\int_{\partial\Delta} f(z)dz = 0.$ Dann besitzt $f$ auf $U_r(z_0)$ eine Stammfunktion.\\

%Beweis vom Hilfssatz, Schmöger hat ihn später erst gebracht
\begin{beweis}
Definiere: $F: U_r(z_0) \rightarrow \MdC$ wie folgt:
F"ur $z \in U_r(z_0)$ sei $\gamma_z(t) := z_0 + t(z-z_0) $ $(t \in [0,1]). $ $ F(z) := \int_{\gamma_z} f(w)dw$.

Sei $z_1 \in U_r(z_0)$. Sei $h \in \MdC \backslash\{0\}$ so, dass $\Delta_{z_0,z_1,z_1+h} \subseteq U_r(z_0)$.

$\gamma_0(t) := z_1 + th (t \in [0,1]).$\\
$\gamma_1 := \gamma_{z_1+h}^-$\\
Vorraussetzungen   $\Rightarrow$ $ 0 = \underbrace{\int_{\gamma_{z_1}}f(w)dw}_{=F(z_1)} +
\int_{\gamma_0} f(w)dw + \underbrace{\int_{\gamma_1}f(w)dw}_{=-F(z_1+h)} \Rightarrow F(z_1+h) - F(z_1) \\ 
= \int_{\gamma_0} f(w)dw;$\\
$\int_{\gamma_0}f(z_1)dw = \int_0^1 f(z_1)hdt = f(z_1)h.$\\ \\
Also:
\begin{eqnarray}\nonumber
\left| \frac{F(z_1+h) - F(z_1)}h - f(z_1) \right|&=&\left| \frac1h \int_{\gamma_0}
(f(w)-f(z_1))dw \right| = \left| \frac1h \int_0^1(f(z_1+th)-f(z_1))h dt \right|\\\nonumber 
&=& \left| \int_0^1(f(z_1+th) - f(z_1))dt \right| \leq \int_0^1 \left|(f(z_1+th) - f(z_1))\right| dt
\end{eqnarray}
Sei $\varepsilon > 0: \,\exists \delta > 0: |f(z_1+th) - f(z_1)| \leq \varepsilon$ f"ur $0 < |h| < \delta$ und f"ur $t \in [0,1] \Rightarrow \\ | \frac{F(z_1+h) - F(z_1)}h - f(z_1)| \leq \varepsilon$ f"ur $0 < |h| < \delta$. D.h. $F$ ist in $z_1$ komplex differenziebar und $F'(z_1) = f(z_1)$
\end{beweis}



\emph{Folgerung:}\\
Sei $\emptyset \not= D \subseteq \MdC$ offen und $f \in H(D)$. Ist $z_0 \in D$, so existiert ein $\delta > 0: U_{\delta}(z_0) \subseteq D$ und $f$ besitzt auf $U_{\delta}(z_0)$ eine Stammfunktion.\\

\begin{beweis}
9.1 und Hilfssatz 1
\end{beweis}


%Ab hier also: Montag, 29. Mai 2006
%Es fehlen die Zeichnungen. Kann die jemand machen? Gruß, Ferdi

\begin{definition}
Sei $G \subseteq \MdC$
\begin{liste}
\item G heißt \begriff{sternförmig} : $\gdw \exists z^* \in G$ mit: $S[z,z^*] \subseteq G$
I.d. Fall heißt $z^*$ ein \begriff{Sternmittelpunkt} von $G$.\newline
\text{Beachte}: sternförmig $\folgt$ \begriff{Wegzusammenhang}.
\item Ist $G$ offen und sternförmig, so heißt $G$ ein \begriff{Sterngebiet}
\end{liste}
\end{definition}

\begin{beispiel}
\begin{liste}
\item Konvexe Mengen sind sternförmig
\item $\MdC, U_{\epsilon}(z_0)$ sind Sterngebiete. $\MdC\backslash\{0\}, \dot U_{\epsilon}(z_0)$ sind Gebiete, aber keine Sterngebiete.
\item $\MdC_{\_}$ ist ein Sterngebiet. Jedes $z^* \in (0,\infty)$ ist ein Sternmittelpunkt von $\MdC_{\_}$.
\end{liste}
\end{beispiel}
\begin{satz}[Cauchyscher Integralsatz für Sterngebiete]
Sei $G \subseteq \MdC$ ein Sterngebiet, es sei $f \in H(G)$ und es sei $\gamma:[a,b] \rightarrow \MdC$ ein stückweise glatter Weg mit $\Tr (\gamma) \subseteq G$. Dann:
\begin{liste}
\item $f$ besitzt auf $G$ eine Stammfunktion $F$.
\item $\wegint f(z) dz = F(\gamma(b))-F(\gamma(a))$
\item Ist $\gamma$ geschlossen, so ist $\wegint f(z) dz = 0$
\end{liste}
\end{satz}
\begin{bemerkung}
Für beliebige Gebiete ist 9.2 i.a. falsch. \\
Beispiel: $G = \MdC\backslash\{0\}$, $f(z) = \frac{1}{z}$ (s. 8.7)
\end{bemerkung}

\begin{beweis}
\begin{liste}
\item Sei $z^*$ ein Sternmittelpunkt von $G$. Definiere $F : G \to \MdC$ wie folgt:
für $z \in G$ sei $\gamma_z (t) := z^* + t (z-z^*) (t \in [0,1])$.
$\Tr(\gamma_z) = S[z,z^*] \subseteq G$. $F(z) := \int\limits_{\gamma _z} f(w) dw$.
$f \in H(G) \stackrel{9.1}{\folgt} \int_{\partial \Delta} f(w) dw = 0$ für jedes Dreieck $\Delta \subseteq G$
Fast wörtlich wie in HS 1 zeigt man: $F \in H(G)$ und $F'=f$ auf $G$.
\item folgt aus (1) und 8.5
\item folgt aus (1) und 8.6
\end{liste}
\end{beweis}

\textbf{Bezeichnung} \\
Seien $G$ und $f$ wie in 9.2. $z^* \in G$ sei ein Sternmittelpunkt von $G$. Für $z \in G$ setze: $F(z) := \int_{z^*}^{z} f(w) dw := \wegint f(w) dw$, wobei $\gamma$ \begriff{irgendein} stückweise glatter Weg mit $\Tr(\gamma) \subseteq G$, Anfangspunkt von $\gamma = z^*$ und Endpunkt von $\gamma = z$ ist.\\ Wegen 9.2(2) ist $F$ wohldefiniert. Der Beweis von 9.2(1) zeigt: $F \in H(G)$ und $F'=f$ auf $G$.

\begin{beispiel}\footnote{Dieses Beispiel tr"agt die Nummer 9.3}
$G = \MdC_{\_}$, $f(z) = \frac{1}{z}$, $z^* = 1$,
$F(z) := \int_{1}^{z} \frac{1}{w} dw$ \\
Dann: $F'(z) = \frac{1}{z} = \Log ' z \; \forall z \in G$ \\
Dann existiert $c \in \MdC : F(z) = \Log z + c \;\forall z \in G$. $F(1) = 0 = \Log 1 \folgt c = 0$\\
Also: $\Log z = \int_{1}^{z} \frac{1}{w} dw (z \in \MdC_{\_})$
\end{beispiel}

\textbf{Hilfssatz 2} \\
Sei $D \subseteq \MdC$ offen. $z_0 \in D$, $r > 0$, $ \overline{U_r (z_0)} \subseteq D$ und $\gamma(t) := z_0 + r \cdot e^{it} \; (t \in [0,2\pi])$
Weiter sei $z_1 \in U_r (z_0)$, $\rho > 0$ so, daß $\overline{U_{\rho} (z_1)} \subseteq U_r (z_0)$ und \\
$\gamma_0 (t) := z_1 + \rho \cdot e^{it} (t \in [0,2\pi])$
Ist $g \in H (D\backslash \{z_1\})$, so gilt: 
$$\wegint g(w) dw = \int\limits_{\gamma _0} g(w) dw$$
\begin{beweis}
O.B.d.A $\Re z_o = \Re z_1, \gamma _1$ und $\gamma _2$ seien stückweise glatte Wege. Wähle $R > r$ so, dass $U_R(z_0) \subseteq D$. \\
$G_1 := U_R(z_0) \backslash \{z_1 + t : t \leq 0\}$. $G_1$ ist ein Sterngebiet, $\Tr (\gamma _1) \subseteq G_1$. $\gamma _1$ ist geschlossen, $g \in H(G_1). 9.2 \folgt \int\limits _{\gamma _1} g(w) dw = 0$. Analog: $\int\limits _{\gamma _2} g(w) dw = 0$. Also:\\
$0 = \int\limits _{\gamma _1} g(w) dw + \int\limits _{\gamma _2} g(w) dw = \int\limits _{\gamma} g(w) dw + \int\limits _{\gamma _0^-} g(w) dw = \int\limits _{\gamma} g(w) dw - \int\limits _{\gamma _0} g(w) dw$
\end{beweis}

\setcounter{satz}{3}
%Satz 9.4
\begin{satz}[Cauchysche Integralformel für Kreisscheiben]
$D\subseteq \MdC$ sei offen, $z_0 \in D, r>0$ und $\overline{U_r(z_0)} \subseteq D$. Weiter sei $f \in H(D)$ und $\gamma (t) := z_0 + r\cdot e^{it} \; (t \in [0,2\pi ])$.\\
Dann gilt:
$$f(z) = \frac{1}{2\pi i}\wegint\frac{f(w)}{w-z}dw\quad\forall z\in U_r(z_0)$$
\end{satz}



\textbf{Bemerkungen}
\begin{liste}
\item Die Werte von $f$ in $U_r(z_0)$ sind festgelegt durch die Werte von $f$ auf $\partial U_r(z_0)$
\item Für $z = z_0: f(z_0) = \frac{1}{2\pi}\int\limits _0^{2\pi}f(z_0 + r\cdot e^{it}) dt$ (Mittelwertgleichung)
\end{liste}

\begin{beweis}
Sei $z_1 \in U_r(z_0)$. Sei $\epsilon > 0.\; \exists \delta > 0: U_\delta (z_1) \subseteq U_r(z_0)$ und \\
$|f(w) - f(z_1)| \leq \epsilon\;\forall w\in U_\delta (z_1)$. \\
Sei $ 0<\rho <\delta ; \gamma _0(t) := z_1 + \rho\cdot e^{it}\; (t \in [0,2\pi])$.\\
Für $w \in\Tr (\gamma _0): |w - z_1| = \rho < \delta$, also $|f(w) - f(z_1)|\leq \epsilon$.\\
Also: $|\frac{f(w) - f(z_1)}{w -z_1}| \leq \frac{\epsilon}{\rho}\;\forall w\in\Tr (\gamma _0)$.\\
$8.4 \folgt |\int\limits _{\gamma _0}\frac{f(w) - f(z_1)}{w -z_1} dw|\leq \frac{\epsilon}{\rho}L(\gamma _0) = \frac{\epsilon}{\rho} 2\pi\rho = 2\pi\epsilon$.\\
Definiere $g: D \backslash \{ z_1 \} \to \MdC$ durch $g(w) := \frac{f(w)}{w -z_1}$ .\\
Dann: $g \in H(D \backslash \{z_1 \})$.\\ Somit:
\begin{eqnarray}
\notag\wegint \frac{f(w)}{w - z_1} dw & = & \wegint g(w) dw = \int\limits_{\gamma _0} g(w) dw \\
\notag & = & \int\limits _{\gamma _0}\frac{f(z_1)+f(w)-f(z_1)}{w - z_1} dw\\
\notag & = & f(z_1) \cdot \underbrace{\int\limits _{\gamma _0}\frac{dw}{w - z_1}}_{\stackrel{8.7}{=}2\pi i} + \underbrace{\int\limits _{\gamma_0}\frac{f(w)-f(z_1)}{w - z_1} dw}_{=: A}\\
\notag & = & 2\pi if(z_1) + A
\end{eqnarray}
$\folgt |\wegint \frac{f(w)}{w - z_1} dw - 2\pi if(z_1)| = |A| \stackrel{\text{s.o.}}{\leq} 2\pi \epsilon$.\\
$\epsilon > 0$ beliebig $\folgt f(z_1) = \frac{1}{2\pi i}\wegint \frac{f(w)}{w - z_1} dw$
\end{beweis}

\begin{beispiel}
Berechne $I = \wegint \frac{e^{\sin z} + \cos(e^z)z^2}{z}dz\, , \gamma(t) = e^{it} (t \in [0,2\pi ])$\\
$f(z) := e^{\sin z } + \cos (e^z)z^2$\\
9.4 $\folgt I = 2\pi if(0) = 2\pi i$
\end{beispiel}


%Satz 9.5
\begin{satz}
$\gamma$ sei ein stückweise glatter Weg in $\MdC$, es sei $D:= \MdC \backslash
\Tr(\gamma)$ ($D$ offen). Für $n \in \MdN$ sei $F_n: D \to \MdC$ definiert durch
\[ F_n(z) := \int_{\gamma} \frac{\varphi(w)}{(w-z)^n} dw,\]\\ 
wobei $\varphi \in C(\Tr(\gamma))$. \\
Dann ist $F_n \in H(D)$ und $F_n ' = n F_{n+1}$ auf $D$ $(n\in \MdN)$
\end{satz}

\begin{beweis}
Sei $z_0 \in D.$ Wir zeigen : $F_n$ ist in $z_0$ komplex differenziebar und $F_n'(z_0) =
nF_{n+1}(z_0)$. \\
o.B.d.A: $z_0 = 0$. Dann ist $ 0 \in D$, also $0 \not\in \Tr(\gamma)$. 
Sei $w \in Tr(\gamma)$ und $z \in D \backslash \{0\} $: \\ 
Nachrechnen: 

\[ \frac{1}{(w-z)^n}-\frac{1}{w^n} = \frac{z}{(w-z)^n w^n} 
\sum_{k=0}^{n-1}w^{n-k-1}(w-z)^k \]

$h(z,w) := \frac{1}{(w-z)^n} \sum_{k=0}^{n-1}w^{n-k-1}(w-z)^k$  $-\frac{n}{w}$
\\ \\ Dann folgt (Nachrechnen!): \[ \frac{F_n(z)-F_n(0)}{z}- n F_{n+1}(0) =
\int_{\gamma} \frac{\varphi(w)}{w^n} h(z,w) dw\] 
Weiter gilt $\exists r > 0 : U_r(z_0) \subseteq D$. Sei $\epsilon > 0$.
$\overline{U_{\frac{r}{2}}(z_0)} \times \Tr(\gamma)$ ist kompakt und h ist auf 
$\overline{U_{\frac{r}{2}}(z_0)} \times \Tr(\gamma)$ gleichmäßig stetig.
Dann existiert ein $\delta >0: \delta < \frac{r}{2}$ und $|h(z_1,w) - h(z_2,w)|
\leq \epsilon$ $\forall z_1, z_2 \in U_{\delta}(0)$ $\forall w \in \Tr(\gamma)$.
\\ Es ist $h(0,w) = 0$ $\forall w \in \Tr(\gamma)$ $\Rightarrow |h(z, w)| \leq
\epsilon $ $\forall z \in U_{\delta}(0)$ $ \forall w \in \Tr(\gamma)$ \\
$M := \text{max}_{w \in \Tr(\gamma)}|\varphi(w)|$; $w \in \Tr(\gamma)$ $\Rightarrow |w| =
|w-0| \geq \frac{r}{2}$ \\ $\Rightarrow |w|^n \geq \frac{r^n}{2^n} \Rightarrow
\frac{1}{|w|^n} \leq \frac{2^n}{r^n}$ \\
$\Rightarrow \frac{| \varphi(w) |}{|w|^n }|h(z,w)|  \leq M \frac{2^n}{r^n} \epsilon$ $\forall z \in U_{\delta}(0)$ \\
$\Rightarrow  \underbrace{\int_{\gamma} |\frac{ \varphi(w) }{w^n }h(z,w)dw|}_{= 
|\frac{F_n(z)-F_n(0)}{z}- n F_{n+1}(0)|} \leq  M
\frac{2^n}{r^n} \epsilon L(\gamma) = \epsilon (\frac{M 2^n}{r^n} L(\gamma))$ 
$\forall z \in U_{\delta}(0)$
\end{beweis}

%satz 9.6
\begin{satz}
Sei $\emptyset \neq D \subseteq \MdC$, $D$ offen und $f \in H(D)$. Dann:
\begin{liste}
\item $f' \in H(D)$
\item $f$ ist auf $D$ beliebig oft komplex differenzierbar
\item \begriff{Cauchysche Integralformeln für Ableitungen} \\
      Ist $z_0 \in D, r>0, \overline{U_r(z_0)} \subseteq D$ und $\gamma(t) = z_0+ r e^{\ie t}$
       $(t \in [0, 2\pi])$, so gilt: \\
       \[f^{(n)}(z) = \frac{n!}{2\pi \ie} \int_{\gamma}
       \frac{f(w)}{(w-z)^{n+1}} dw \quad \forall z \in U_r(z_0) \quad \forall n
       \in \MdN_0 \] 
      
\end{liste}

\end{satz}
\begin{beweis}
Sei $z_0, r, \gamma$ wie in (3). \\
\[ F_n(z) :=\frac{1}{2\pi i} \int_{\gamma} \frac{f(w)}{(w-z)^n} dw \quad \text{für } z\in \MdC
\backslash \Tr(\gamma), n \in \MdN\] \\
$9.4 \Rightarrow f = F_1$ auf $U_r(z_0)$; \\
$9.5 \Rightarrow F_1 \in H(U_r(z_0))$ und $F_1'=F_2$ auf $U_r(z_o)$. Also: $f'=F_2$ auf $U_r(z_0)$. $9.5 \Rightarrow F_2\in H(U_r(z_0))$, also $f' \in H(U_r(z_0))$. $z_0\in D$
beliebig $\Rightarrow$ (1). \\
$f' = F_2$ auf $U_r(z_0)$ $\Rightarrow$ 
\[
	f'(z) = \frac{1}{2\pi\ie}\int_{\gamma} \frac{f(w)}{(w-z)^2} dw \quad \forall z
	\in U_r(z_0)
 \]

$f'' = F_2' = 2F_3$ auf $U_r(z_0)$ $\Rightarrow$  \\
\[
	f''(z) = \frac{2}{2\pi\ie}\int_{\gamma} \frac{f(w)}{(w-z)^3} dw \quad \forall z
	\in U_r(z_0)
\]
Weiter mit Induktion und 9.5
\end{beweis}
%satz 9.7
\begin{satz}[Satz von Morera]
Sei $\emptyset \neq D \subseteq \MdC$, $D$ offen und $f \in C(D)$\\
Dann: \[f\in H(D) \gdw \int_{\partial \Delta} f(z) dz = 0 \text{ für jedes
Dreieck } \Delta \subseteq D\]
\end{satz}
\begin{beweis}
$"\Rightarrow":$ 9.1 \\
$"\Leftarrow:$ Sei $z_0 \in D, r > 0$ und $U_r(z_0) \subseteq D$. Dann mit
HilStammfunktionsatz 1 und den Vorraussetzungen $\Rightarrow$ $\exists F \in H(U_r(z_0)):
F' = f$ auf $U_r(z_0)$ \\ 
9.6 $\Rightarrow$ $f \in H(U_r(z_0))$. Da $z_0 \in D$ beliebig $\Rightarrow$ $f
\in H(D)$
\end{beweis}

\textbf{Hilfssatz 3}
Seien $G_1$ und $G_2$ Gebiete in $\MdC$ und es sei $G_1 \cap G_2 \neq \emptyset$.  \\
Dann ist $G_1 \cup G_2$ ein Gebiet.

\begin{beweis}
$G_1 \cup G_2$ ist offen. Sei $\varphi: G_1 \cup G_2 \to \MdC$ lokal konstant.
$\varphi_j := \varphi_{|_{G_j}}$ $(j = 1,2)$ \\
$G_j$ Gebiet $\Rightarrow \quad \varphi_j $ ist auf $G_j$ konstant. $G_1 \cap G_2
\neq \emptyset \Rightarrow \quad \varphi$ ist auf $G_1 \cup G_2$ konstant.
\end{beweis}

\begin{definition}
Sei $G \subseteq \MdC$ ein Gebiet. \\$G$ heißt ein \begriff{Elementargebiet} (EG)
$:\gdw$ $\forall f \in H(G) \exists F \in H(G) : F' = f$ auf $G$.
\end{definition}

\begin{beispiel}
\begin{liste}
\item Aus 9.2: Sterngebiete sind Elementargebiete
\item $\MdC \backslash \{0\}$ ist kein Elementargebiet, denn die Funktion
$\frac{1}{z}$ hat auf $\MdC \backslash \{0\} $ keine Stammfunktion (siehe 8.7).
\end{liste}
\end{beispiel}

\begin{satz}
Seien $G_1$ und $G_2$ Elementargebiete, $G_1\cap G_2 \neq \emptyset$ und es sei
$G_1 \cap G_2$ zusammenhängend. \\
Dann ist $G_1 \cup G_2$ ein Elementargebiet.
\end{satz}

\begin{bemerkung}
 
\begin{liste}
\item Sind $G_1$ und $G_2$ Gebiete, so muß $G_1 \cap G_2$ nicht zusammenhängend
sein. 
\item Es gibt Elementargebiete, die keine Sterngebiete sind.
\end{liste}
\end{bemerkung}

\begin{beweis}
Hilfssatz 3 $\Rightarrow$ $G_1 \cup G_2$ ist ein Gebiet. \\
Vorraussetzungen $\Rightarrow$ $G_1 \cap G_2$ ist ein Gebiet. \\
Sei $f \in H(G_1 \cup G_2), f_j := f_{|_{G_j}}$ $( j = 1,2)$, \\ 
$\exists F_j \in H(G_j): F_j ' = f_j = f$ auf $G_j$ $(j=1,2)$ \\
Für $z \in G_1 \cap G_2: (F_1 - F_2)'(z) = f(z) - f(z) = 0$ \\
4.2 $\Rightarrow$ $\exists c \in \MdC: F_1(z)=F_2(z) + c \quad \forall z \in G_1
\cap G_2$

\[ F(z) := \begin{cases}
				F_1(z) &, z \in G_1 \\
				F_2(z)+c &, z \in G_2
		   \end{cases} \] 
Dann ist $F$ eine Stammfunktion von $f$ auf $G_1 \cup G_2$
\end{beweis}

\begin{definition}
Sei $\emptyset \neq D \subseteq \MdC$ und $g: D \to \MdC$ eine Funktion. 
$g$ ist auf $D$ \begriff{beschränkt} $:\gdw$ $\exists c \geq 0 : |g(z)| \leq c
\quad \forall z \in D$
\end{definition}
\begin{definition}
Eine Funktion $f \in H(\MdC)$ heißt eine \begriff{ganze
Funktion}.(\begriff{entire function})
\end{definition}


\chapter{Folgerungen aus den Integralformeln}

%Satz 10.1
\begin{satz}[Cauchysche Abschätzungen]
Sei $z_0 \in \MdC, r>0, f \in H(U_r(z_0))$ und $f$ sei auf $U_r(z_0)$ beschränkt mit $M := \sup\limits _{U_r(z_0)} |f(z)|$.\\
Dann: $|f^{(n)}(z_0)| \leq \frac{Mn!}{r^n} \,\forall n \in \MdN _0$.
\end{satz}

\begin{beweis}
Sei $0<\rho < r, \gamma (t) := z_0 + \rho e^{it} (t \in [0,2\pi ]).\; 9.6 \folgt f^{(n)}(z_0) = \frac{n!}{2\pi i} \wegint \frac{f(w)}{(w - z_0)^{n+1}}dw$.\\
Für $w \in \Tr (\gamma ) : |w - z_0| = \rho$, also $\frac{|f(w)|}{|w - z_0|^{n+1}} \leq \frac{M}{\rho ^ {n+1}}$ \\
$\stackrel{8.4}{\folgt} |f^{(n)}(z_0)| \leq \frac{n!}{2\pi} \cdot \frac{M}{\rho^{n+1}}2\pi\rho = \frac{Mn!}{\rho ^n}\stackrel{\rho \to r}{\folgt}$ Beh.
\end{beweis}

%Satz 10.2
\begin{satz}[Satz von Liouville]
Ist $f \in H(\MdC )$ auf $\MdC$ beschränkt, so ist $f$ konstant.
\end{satz}

\begin{beweis}
Sei $z_0 \in \MdC$ und $r > 0$. 10.1 $\folgt |f'(z_0)| \leq \frac{M}{r}; r>0$ beliebig. \\
$\stackrel{r \to \infty}{\folgt} f'(z_0) = 0, z_0 \in \MdC$ beliebig $\folgt f' = 0$ auf $\MdC. 4.2 \folgt$ Beh.
\end{beweis}

\begin{bemerkung}
10.2 ist in $\MdR$ falsch. Z.B. ist $x \to \cos x$ auf $\MdR$ beschränkt aber nicht konstant. Für $t \in \MdR: \cos (it) = \frac{1}{2}(e^{i(it)} + e^{i(-it)}) = \frac{1}{2}(e^t + e^{-t})$\\ 
$= \cosh t \to \infty (t \to \pm\infty )$
\end{bemerkung}

\textbf{Hilfssatz}\\
Sei $n \in \MdN. a_0,\ldots ,a_n \in \MdC, a_n \neq 0$ und $p(z):=a_0 + a_1z + \ldots + a_nz^n$. \\
Dann ex. ein $R>0: |p(z)| \geq 1 \;\forall z \in \MdC$ mit $|z| > R$.

\begin{beweis}
Für $z \neq 0: \varphi (z) := \frac{|a_0|}{|z^n|} + \frac{|a_1|}{|z^{n-1}|} + \ldots + \frac{|a_{n-1}|}{|z|} + |a_n|$.\\
$\folgt \varphi (z) \to \underbrace{|a_n|}_{\neq 0} (|z| \to \infty) \folgt |p(z)| = |z|^n |\varphi(z)| \to \infty (|z| \to \infty) \folgt$ Beh.
\end{beweis}

%Satz 10.3
\begin{satz}[Fundamentalsatz der Algebra]
Sei $p$ wie in obigem Hilfssatz. Dann ex. ein $z_0 \in \MdC: p(z_0) = 0$
\end{satz}

\begin{beweis}
Hilfssatz $\folgt \exists R>0 : |p(z)|\geq 1 \forall z \in \MdC$ mit $|z|>R$.\\
Annahme: $p(z) \neq 0 \forall z \in \MdC$. Dann $q:=\frac{1}{p} \in H(\MdC)$ und $|q(z)|\leq 1$ für $z\in \MdC$ mit $|z| > R$. \\
$q$ ist stetig $\folgt q$ ist beschränkt auf $\overline{U_R(0)} \folgt q$ ist auf $\MdC$ beschränkt. \\
$10.2 \folgt q$ ist konstant $\folgt p$ ist konstant, Wid!
\end{beweis}

%Satz 10.4
\begin{satz}[Potenzreihenentwicklung]
Sei $D \subseteq \MdC$ offen, $f \in H(D), z_0 \in D$ und $r > 0$ so, dass $U_r(z_0) \subseteq D$. Dann:
$$(*)f(z) = \sum _{n=0}^{\infty}a_n(z-z_0)^n \quad \forall z \in U_r(z_0)$$
wobei
$$(**) a_n = \frac{f^{(n)}(z_0)}{n!} = \frac{1}{2\pi i}\wegint\frac{f(w)}{(w-z_0)^{n+1}}dw$$
mit $\gamma (t) = z_0 + \rho e^{it}, t \in [0,2\pi], 0<\rho <r$
\end{satz}

\begin{beweis}
(**) folgt aus (*), 5.4 und 9.6. O.B.d.A.:$\;z_0 = 0$. \\
Sei $z \in U_r(0)$ und sei $R>0$ so, dass $|z|<R<r$;\\
$\gamma _0(t) := z_0 + R\cdot e^{it} \;(t \in [0,2\pi])$. \\
Sei $w \in \Tr (\gamma _0)$. Dann $\frac{|z|}{|w|} = \frac{|z|}{R} < 1$, also $\frac{f(w)}{w-z} = \frac{f(w)}{w}\cdot \frac{1}{1-\frac{z}{w}} = \sum\limits _{n=0}^{\infty}\frac{f(w)}{w^{n+1}}z^n$.\\
Also $\underbrace{\int\limits _{\gamma _0} \frac{f(w)}{w-z} dw}_{\stackrel{9.4}{=}2\pi if(z)} = \int\limits _{\gamma _0}(\sum\limits _{n=0}^{\infty}\frac{f(w)}{w^{n+1}}z^n)dw $\\
$\stackrel{8.4}{=}\sum\limits _{n=0}^{\infty}(\underbrace{\int\limits _{\gamma _0}\frac{f(w)}{w^{n+1}}dw}_{\stackrel{9.6}{=}2\pi i\cdot\frac{f^{(n)}(0)}{n!}})z^n \folgt f(z) = \sum\limits _{n=0}^{\infty}(\frac{f^{(n)}(0)}{n!})z^n$
\end{beweis} 

\begin{bemerkungen}
\item 10.4 ist in $\MdR$ falsch. \\Bekannt aus der Analysis: Die Funktion
$$f(x) := \begin{cases}e^{-1/x^2} &, x\in\MdR\backslash\{0\} \\ 0 &, x=0\end{cases}$$ ist auf $\MdR$ bel. oft db und $f^{(n)}(0) = 0 \,\forall n \in \MdN _0.$ \\
Also: $\sum\limits _{n=0}^{\infty}\frac{f^{(n)}(0)}{n!}x^n \equiv 0$ auf $\MdR$.
\item Die Entwicklung (*) gilt in der größten offenen Kreisscheibe um $z_0$, die noch ganz in $D$ liegt. Sei $r_0$ der Radius dieser Kreisscheibe (ist $D = \MdC$, so ist $r = \infty)$. Sei R der KR der PR in (*). Also : $R \geq r_0$.
\end{bemerkungen}

%\begin{beispiel}
%Fehlt noch, kommt aber
%\end{beispiel}

% Satz 10.5
\begin{satz}[Konvergenzsatz von Weierstraß]
$D \subseteq \MdC$ sei offen, $(f_n)$ sei eine Folge in $H(D)$ und $(f_n)$ konvergiere auf $D$ \begriff{lokal gleichmäßig} gegen eine Funktion $f: D \rightarrow \MdC$.
\begin{liste}
\item $f \in H(D)$
\item $(f_n')$ konvergiere auf $D$ lokal gleichmäßig gegen $f'$.
\end{liste}
\end{satz}

\begin{beweis}
\begin{liste}
\item 5.1 $\Rightarrow f \in C(D)$. Sei $\Delta \subseteq D$ ein Dreieck. $(f_n)$ konvergiere auf $\partial\Delta$ \\
gleichmäßig $\Rightarrow$ $\int_{\partial\Delta} f(z) dz \stackrel{8.4}{=} \lim_{n\to\infty}\int_{\partial\Delta}f_n(z) dz \stackrel{9.1}{=} 0$.\\
9.7 $\Rightarrow f \in H(D)$.
\item O.B.d.A. $f = 0$ auf $D$ (ansonsten betrachte $f_n - f$). Sei $z_0 \in D$ und $r > 0$ \\
so, daß $\overline{U_r(z_0)} \subseteq D$. Es genügt zu zeigen: \\
$(f_n')$ konvergiert auf $\overline{U_{\frac{r}{2}}(z_0)}$ gleichmäßig gegen $0$.\\
$\gamma (t) := z_0 + r \cdot e^{it} (t \in [0,2\pi])$. $M_n := \max\limits _{w \in \Tr(\gamma)} |f_n(w)|$\\
Vor $\Rightarrow M_n \to 0$.\\
Sei $z \in \overline{U_{\frac{r}{2}}(z_0)}. {f'}_n (z) \stackrel{9.6}{=} \frac{1}{2 \pi i} \int_{\gamma} \frac{f_n(w)}{(w-z)^2}dw$\\
$w \in \Tr(\gamma) : |w-z| \geq \frac{r}{2} \Rightarrow \frac{|f_n(w)|}{|w-z|^2} \leq \frac{4 M_n}{r^2}$\\
$\Rightarrow |f_n ' (z)| \leq \frac{1}{2\pi} \frac{4 M_n}{r^2} 2 \pi r = \frac{4 M_n}{r}$\\
Also: $|f_n'(z)| \leq \frac{4 M_n}{r} \forall z \in \overline{U_{\frac{r}{2}}(z_0)} \forall n \in \MdN$ und $M_n \to 0$.
\end{liste}
\end{beweis}

\chapter{Weitere Eigenschaften holomorpher Funktionen}

In diesem Paragraphen sei $G \subseteq \MdC$ stets ein \begriff{Gebiet}. Fast wörtlich wie in Analysis I zeigt man:

\begin{satz}[Identitätssatz für Potenzreihen]
$\sum\limits_{n=0}^{\infty}a_n(z-z_0)^n$ sei eine Potenzreihe mit Konvergenzradius $r>0$, \\
es sei $f(z)=\sum\limits_{n=0}^{\infty}a_n(z-z_0)^n$ für $z \in U_r(z_0)$, es sei $(z_k)$ eine \\
Folge in $\dot U_r(z_0)$ mit $z_k \to z_0$ und es gelte $f(z_k) = 0$  $\forall$ $k$ $\in$ $\MdN$. \\
Dann: $a_n = 0$ $\forall$ $n$ $\in$ $\MdN_0$.
\end{satz}

\begin{satz}[Identitätssatz für holomorphe Funktionen]
Es sei $f \in H(G)$, $z_0 \in G$, $(z_k)$ eine Folge in $G\backslash\{z_0\}$ mit $f(z_k) = 0$ $\forall$ $k$ $\in$ $\MdN$\\
und $ z_k \to z_0$.\\
Dann: $f = 0$ auf $G$.
\end{satz}

\begin{beweis}
$\exists r > 0$: $U_r(z_0) \subseteq G$. 10.4 $\Rightarrow f(z) = \sum\limits_{n=0}^{\infty} \frac{f^{(n)}(z_0)}{n!}(z-z_0)^n$ $\forall$ $z$ $\in$ $U_r(z_0)$\\
$\exists k_0 \in \MdN$: $z_k \in U_r(z_0)$ $\forall$ $k$ $\geq$ $k_0$. 11.1 $\Rightarrow f^{(n)}(z_0) = 0$ $\forall$ $n \in \MdN_0$\\
$\Rightarrow z_0 \in A := \{z \in G: f^{(n)}(z) = 0$ $\forall$ $n$ $\in$ $\MdN_0\}$. $B:= G\backslash A$, $A \cap B = \emptyset$\\
Sei $ a \in A$. $\exists \delta > 0: U_{\delta}(a) \subseteq G$. 10.4 $\Rightarrow f(z) = \sum\limits_{n=0}^{\infty} \frac{f^{(n)}(a)}{n!}(z-a)^n$ $\forall$ $z$ $\in$ $U_{\delta}(a)$\\
$a \in A \Rightarrow f^{(n)}(a) = 0$ $\forall$ $n$ $\in$ $\MdN_0$ $\Rightarrow f \equiv 0$ auf $U_{\delta}(a)$\\
$\Rightarrow f^{(n)} \equiv 0$ auf $U_{\delta}(a)$ $\forall$ $n$ $\in$ $\MdN_0$\\
$\Rightarrow U_{(\delta)}(a) \subseteq A$. $A$ ist also offen. Sei $b \in B$. $\exists k \in \MdN_0: f^{(k)}(b) \neq 0$; \\
$f^{(k)}$ stetig $\Rightarrow \exists \epsilon > 0: U_{\epsilon}(b) \subseteq G$ und $f^{(k)}(z) \neq 0$ $\forall$ $z \in U_{\epsilon}(b)$\\
$\Rightarrow U{\epsilon}(b) \subseteq B$; d.h. $B$ ist offen. $G$ ist ein Gebiet $\Rightarrow B = \emptyset \Rightarrow G = A \Rightarrow$ Beh.
\end{beweis}

\textbf{Bezeichnung} \\
für $f \in H(G)$: $Z(f) := \{z\in G: f(z) = 0\}$.

\begin{folgerung}
\begin{liste}
\item Ist $f \in H(G)$, $f \not\equiv$ %%nicht identisch!!!!!%% 
$0$ auf $G$ und $z_0 \in Z(f)$, so existiert ein $\epsilon > 0$: 
\\$U_{\epsilon}(z_0) \subseteq G$, $f(z) \neq 0$ $\forall$ $z$ $\in$ $\dot U_{\epsilon}(z_0)$
\item Ist $f \in H(G)$, $z_0 \in G$ und gilt: $f^{(n)}(z_0) = 0$ $\forall$ $n$ $\in$ $\MdN_0$, so ist $f = 0$ auf $G$. 	
\end{liste}
\end{folgerung}

\begin{beweis}
\begin{liste}
\item folgt aus 11.2
\item Verfahre wie im Beweis von 11.2
\end{liste}
\end{beweis}

\begin{satz}
Sei $G$ ein EG und $f \in H(G)$ mit $Z(f) = \emptyset$
\begin{liste}
\item $\exists h \in H(G)$: $e^h = f$ auf $G$
\item Ist $n \in \MdN$, so existiert ein $g \in H(G)$: $g^n = f$ auf $G$
\end{liste}
\end{satz}

\begin{beweis}
\begin{liste}
\item Es ist $\frac{f'}{f} \in H(G)$. $G$ ist ein EG $\Rightarrow \exists F \in H(G)$: $F'= \frac{f'}{f}$ auf $G$. $\phi := \frac{e^F}{f}$.\\
Dann: $\phi \in H(G)$ und $\phi ' = 0$ auf $G$. (nachrechnen!)\\
$\exists c \in \MdC$: $e^F = c \cdot f$ auf $G$.\\
Klar: $c \neq 0$. 7.1 $\Rightarrow \exists a \in \MdC$: $c = e^a \Rightarrow f = e^{F - a}$ auf $G$.
\item Sei $h$ wie in (1), $g := e^{\frac{1}{n} h}$. Dann: $g^n = e^h = f$ auf $G$.
\end{liste}
\end{beweis}

\begin{satz}
Sei $D \subseteq \MdC$ offen.
\begin{liste}
\item Ist $F \in H(D)$, $0 \in D$, $F(0) = 0$ und $F'(0) \neq 0$, so gilt: $0 \in {(F(D))}^o$
\item Ist $f \in H(G)$ \begriff{nicht} konstant, so ist $f(G)$ offen.
\item \begriff{Satz von der Gebietstreue:}\\
Ist $f \in H(G)$ \begriff{nicht} konstant, so ist $f(G)$ ein Gebiet.
\end{liste}
\end{satz}

\begin{beweis}
\begin{liste}
\item $u := \Re F$, $v := \Im F$. 4.1 $\Rightarrow u_x(0) = v_y(0), u_y(0) = - v_x(0)$ \\
und $F'(0) = u_x(0) + i v_x(0)$\\
$\Rightarrow det 
\left( \begin{array}{ccc}
u_x(0) & u_y(0) \\
v_x(0) & v_y(0) \\
\end{array} \right)
= u_x(0)^2 + v_x(0)^2 = |F'(0)|^2 \neq 0$\\
Umkehrsatz (Analysis II) $\Rightarrow \exists U \subseteq D$: $0 \in U$, $U$ ist offen und $F(U)$ ist offen.
$F(0) = 0 \Rightarrow 0 \in F(U) \Rightarrow \exists \delta > 0$: $U_{\delta}(0) \subseteq F(U) \subseteq F(D)$.
\item Sei $w_0 \in f(D)$. z.z. $\exists \delta > 0$: $U_{\delta}(w_0) \subseteq f(D)$.\\
O.B.d.A. $w_0 = 0$. $\exists z_0 \in D$: $f(z_0) = w_0 = 0$. O.B.d.A. $z_0 = 0$.\\
Also: $f(0) = 0$. $\exists \varepsilon > 0$: $U_{\varepsilon}(z_0) \subseteq D$.\\
10.4 $\Rightarrow f(z) = a_0 + a_1 z + a_2 z^2 + \dots$  $\forall z \in U_{\varepsilon}(0)$;\\
$f(0) = 0 \Rightarrow a_0 = 0$. 11.3 $\Rightarrow \exists n \in \MdN$: $a_n \neq 0$\\
$m:= min \{n \in \MdN : a_n \neq 0 \}$ ($\geq 1$)\\
Dann: $f(z) = z^m (a_m + a_{m+1} z + a_{m+2} z^2 + \dots) = z^m \cdot g(z)$  $\forall z \in U_{\varepsilon}(0)$,\\
wobei $g \in H(U_{\varepsilon}(0))$ und $g(0) = a_m \neq 0$.\\
$g$ stetig $\Rightarrow \exists r \in (0,\varepsilon)$: $g(z) \neq 0$ $\forall z \in U_r(0)$\\
$U_r(0)$ ist ein EG $\stackrel{11.4}{\Rightarrow} \exists h \in H(U_r(0))$: $h^m = g$ auf $U_r(0)$\\
Def. $F \in H(U_r(0))$ durch $F(z) := z h(z)$. \\
Dann: $F(0)=0$, $F'(z) = h(z) + zh'(z)$\\
also $F'(0)^m = h(0)^m = g(0) \neq 0$, also $F'(0) \neq 0$.\\
Weiter: $F^m = f$ auf $U_r(0)$. (1)$\Rightarrow \exists R > 0$: $U_R(0) \subseteq F(U_r(0))$.\\
$\delta := R^m$. Sei $w \in U_{\delta}(0)$. 1.5 $\Rightarrow \exists v \in \MdC$: $v^m = w$\\
Dann: $|v|^m = |w| < \delta = R^m \Rightarrow |v| < R \Rightarrow v \in U_R(0) \subseteq F(U_r(0))$\\
$\Rightarrow \exists z \in U_r(0) \subseteq D$ mit: $F(z) = v$.\\
$\Rightarrow w = v^m = F(z)^m = f(z) \in f(D)$\\
Also: $U_{\delta}(0) \subseteq f(D)$
\item 3.6 $\Rightarrow f(G)$ ist zusammenhängend $\stackrel{(2)}{\Rightarrow} f(G)$ ist ein Gebiet.

\end{liste}
\end{beweis}

% Satz 11.6
\begin{satz} [Maximimum-, Minimimumsprinzip (I)]
$f \in H(G)$ sei nicht konstant.
\begin{liste}
\item $|f|$ hat auf G kein lokales Maximum
\item Ist $Z(f) = \emptyset$, so hat $|f|$ auf G kein lokales Minimum.
\end{liste}
\end{satz}

\begin{beweis}
\begin{liste}
\item Sei $z_0 \in G$ und $\epsilon > 0$ so, dass $U_{\epsilon}(z_0) \subseteq
G.$ $ w_0 := f(z_0). $ $11.5 \Rightarrow f(U_{\epsilon}(z_0))$ ist offen. $w_0
\in f(U_{\epsilon}(z_0)) \Rightarrow \exists \delta > 0: U_{\delta}(w_0)
\subseteq f(U_{\epsilon}(z_0)).$ \\ $\exists w \in U_{\delta}(w_0) : |w| >
|w_0|.$ $
\exists z \in U_{\epsilon}(z_0): w = f(z)$.\\ Dann: $|f(z)| = |w| > |w_0| = |f(z_0)|$
\item Wende (1) auf $\frac{1}{f}$ an.
\end{liste} 
\end{beweis}

% Satz 11.7
\begin{satz}[Maximimum-, Minimimumsprinzip (II)]
G sei beschränkt, $f \in C(\overline{G})$ und es sei $f \in H(G)$.
\begin{liste}
\item $|f(z)|  \leq \max\limits_{w \in \partial G} |f(w)| $ $\forall z \in
\overline{G}$ 
\item Ist $f(z) \neq 0$ $\forall z \in G$ , so gilt $|f(z)| \geq \min\limits_{w
\in \partial G} |f(w)|$ $\forall z \in \overline{G}$
\end{liste}
\end{satz} 
\begin{beweis}
\begin{liste}
\item $\overline{G}$ ist kompakt, 3.3 $\Rightarrow $ $\exists w_0 \in
\overline{G}:$ $|f(z)| \leq |f(w_0)|$ $\forall z \in \overline{G}$ \\
Fall 1: $w_0 \in \partial G$: fertig \\
Fall 2: $w_0 \in G.$  Dann: $|f(z)| \leq |f(w_0)|$ $\forall z \in G.$ $11.6
\Rightarrow f$ ist konstant auf G. $f$ stetig $\Rightarrow$ $f$ konstant auf
$\overline{G}$ $\Rightarrow$ Beh.
\item Fall1: $f(z) \neq 0$ $\forall z \in \overline{G}$. Wende (1) auf
$\frac{1}{f}$ an. \\
Fall 2: $\exists z_0 \in \overline{G}: $ $f(z_0) = 0$ Vor. $\Rightarrow$ $z_0
\in \partial G$ $\Rightarrow$ $\min\limits_{w\in\partial G} |f(w)| = 0$
$\Rightarrow $ Behauptung. 
\end{liste}
\end{beweis}

\begin{definition}
Sei $A \subseteq G$. $A$ heißt \begriff{diskret in G} $:\equizu$ $A$ hat in G
keinen Häufungspunkt. ($\equizu \forall z_0 \in G \;\exists r = r(z_0) > 0 : A \cap
\dot{U}_r(z_0) = \emptyset$)
\end{definition}
\emph{Aufgabe}: Ist $A$ diskret in $G$, so ist $A$ höchstens abzählbar.

% Satz 11.8
\begin{satz}
Sei $f \in H(G)$ und $f$ nicht identisch $0$ auf G. \\
Dann ist $Z(f)$ diskret in $G$. \\
Ist $z_0 \in Z(f)$, so existiert ein $m \in \MdN$ und ein $g \in H(G)$: \\ \\ 
\centerline{$f(z)
= (z-z_0)^m g(z)$ $\forall z \in G$ \underline{und} $g(z_0) \neq 0$} \\ \\$m$ und
$g$ sind eindeutig bestimmt. $m$ heißt \begriff{Ordnung} (oder
\begriff{Vielfachheit}) der Nullstelle $z_0$ von $f$. ("$f$ hat eine $m$-fache Nullstelle")
\end{satz}

\begin{beweis}
11.3 $\Rightarrow $ $Z(f)$ ist diskret in G. O.B.d.A: $z_0 = 0$. $\exists r > 0:
U_r(0) \subseteq G.$ \\
10.4 $\Rightarrow$ $f(z) = a_0 + a_1z + a_2 z^2 + \dots$ $\forall z \in U_r(0)$.
 $f(0) = 0 $ $\Rightarrow$ $a_0 = 0$ \\
11.2 $\Rightarrow $ $\exists n \in \MdN: a_n \neq 0$, $m := \min \{n \in \MdN:
a_n \neq 0\}$ \\
Dann: $f(z) = z^m \underbrace{(a_m + a_{m+1}z + \dots)}_{:= \varphi(z)} = z^m 
\varphi(z)$ $\forall z \in U_r(0)$ \\
Es ist $\varphi \in H(U_r(0))$ und $\varphi(0) = a_m \neq 0$ \\
Definiere $g: G \to \MdC$ durch 
\[g(z) := \begin{cases}
         	\frac{f(z)}{z^m} &, z \neq 0 \\
         	a_m              &, z = 0 
         \end{cases} \]
Dann: $f(z) = z^m g(z)$ $\forall z \in G$, $g(0) = a_m \neq 0, g = \varphi$ auf
$U_r(0)$, also $g \in H(G)$
\end{beweis}
\emph{Aufgabe}: Sei $f$ wie in 11.8, $z_0 \in G$ und $m \in \MdN$. Dann: \\ $f$
hat in $z_0$ eine m-fache Nullstelle $\equizu$ $f(z_0) = f'(z_0) = \ldots =
f^{(m-1)}(z_0)=0$ und $f^{(m)}(z_0) \neq 0$

% Satz 11.9
\begin{satz}
Sei $f \in H(G)$.
\begin{liste}
\item Sei $g: G \times G \to \MdC$ definiert durch 
\[g(z,w):= \begin{cases} 
           \frac{f(z)-f(w)}{z-w}&, z \neq w \\
           f'(z)&, z = w
           \end{cases}\]. \\
Dann ist g stetig.
\item Ist $z_0 \in G$, so existiert ein $\epsilon > 0$: \\ $ U_\epsilon(z_0)
\subseteq G$ und (*) $|f(z)-f(w)| \geq \frac{1}{2}|f'(z_0)||z-w|$ $\forall z, w
\in U_\epsilon(z_0)$ \\ Ist $f'(z_0) \neq 0$, so ist $f$ auf $U_\epsilon(z_0)$
injektiv und $f^{-1}$ ist auf $f(U_{\epsilon}(z_0))$ stetig.

\end{liste}
\end{satz}

\begin{beweis}
\begin{liste}
\item Es genügt zu zeigen: ist $z_0 \in G$, so ist $g$ stetig in $(z_0, z_0) \in G
\times G$, $\epsilon > 0$: $|g(z,w)-f'(z_0)| < \epsilon$ \\
Sei $\epsilon > 0$. $\exists \delta > 0: U_\delta(z_0) \subseteq G$ und
$|f'(w)-f'(z_0)| \leq \epsilon$ $\forall w \in U_\delta(z_0)$ \\
Seien $z,w \in U_\delta(z_0). \gamma(t):= z + t(w - z)$ $(t\in [0,1])$, dann : 
\\ $Tr(\gamma) \subseteq U_\delta(z_0)$. \\$U_\delta(z_0)$ ist ein Sterngebiet und
$f'$ hat auf $U_\delta(z_0)$ die Stammfunktion $f$. \\
9.2 $\Rightarrow $ $\int\limits_{\gamma} f'(\xi)d \xi = f(w) - f(z)$
$\Rightarrow$ $ f(w) - f(z) = \int\limits_{0}^1 f'(\gamma(t))(w-z)dt $ \\
Ist $z \neq w$ $\Rightarrow$ $g(z,w) =  \int\limits_{0}^1 f'(\gamma(t))dt $ \\
Ist $z = w$  $\Rightarrow$ $\gamma(t) = z$ $\forall t \in [0,1]$\\ $\Rightarrow$  
$\int\limits_{0}^1 f'(\gamma(t)) dt =  \int\limits_{0}^1 f'(z)dt = f'(z) =
g(z,z)$ \\ Also: $g(z,w) =  \int\limits_{0}^1 f'(\gamma(t))dt $ \\
Dann: \\
\centerline{$|g(z,w)-f'(z_0)| = | \int\limits_{0}^1 f'(\gamma(t))-f'(z_0)dt |$ $\leq 
\int\limits_{0}^1 \underbrace{|f'(\gamma(t))-f'(z_0)|}_{\leq\epsilon}dt \leq \epsilon$}
\item Aus (1): $|g(z,w)| \to |f'(z_0)| $ $ ((z,w) \to (z_0, z_0))$ $\Rightarrow$
$\exists \,\epsilon>0:\;U_\epsilon(z_0) \subseteq G$ und $|g(z,w)| \geq \frac{1}{2}|f'(z_0)|$ $\forall
z, w \in U_\epsilon(z_0)$ $\Rightarrow (*)$\\
Sei $f'(z_0) \neq 0. (*) \Rightarrow$ $f$ ist injektiv auf $U_\epsilon(z_0)$ \\
Seien $\lambda, \mu \in f(U_\epsilon(z_0)); z := f^{-1}(\lambda), w :=
f^{-1}(\mu)$ \\
$|f^{-1}(\lambda) - f^{-1}(\mu) | = |z -w | \leq \frac{2}{|f'(z_0)|}|\lambda - \mu|$
\end{liste}
\end{beweis}
% Satz 11.10
\begin{satz}
Sei $f \in H(G)$, $z_0 \in G$ und $f'(z_0) \neq 0$ \\
Dann existiert ein $r > 0$: $U_r(z_0) \subseteq G$,
\begin{liste}
\item $f$ ist auf $U_r(z_0)$ injektiv und  $f'(z) \neq 0$ $\forall z \in U_r(z_0)$
\item $f(U_r(z_0))$ ist ein Gebiet
\item $f^{-1} \in H(f(U_r(z_0)))$ und $(f^{-1})'(w) = \frac{1}{f'(f^{-1}(w))}$
$\forall w \in f(U_r(z_0))$
\end{liste}
\end{satz}

\begin{beweis}
\begin{liste}
 \item Sei $\epsilon > 0$ wie in 11.9(2), $f'$ ist stetig $\Rightarrow$ $\exists
  r \in (0,\epsilon): f'(z) \neq 0$ $\forall z \in U_r(z_0)$
 \item folgt aus 11.5
 \item Sei $w_0 \in f(U_r(z_0))$ und $(w_n)$ eine Folge in $f(U_r(z_0)) \backslash
  \{w_0\}$ mit: $w_n \to w_0$. \\ $z_n := f^{-1}(w_n)$, $\tilde{z} := f^{-1}(w_0)$. 11.4 $\Rightarrow$
  $f^{-1}$ stetig in $w_0$ $\Rightarrow$ $z_n \to \tilde{z}$ \\
  $\Rightarrow$ $\frac{f^{-1}(w_n)-f^{-1}(w_0)}{w_n-w_0} =
  \frac{z_n-\tilde{z}}{f(z_n)-f(\tilde{z})} \to \frac{1}{f'(\tilde{z})}
  = \frac{1}{f'(f^{-1}(w_0))}$ \\
  Also ist $f^{-1}$ in $w_0$ komplex differenzierbar und $(f^{-1})'(w_0) =
  \frac{1}{f'(f^{-1}(w_0))}$ 
\end{liste}
\end{beweis}

% Satz 11.11
\begin{satz}Sei $f \in H(G)$ auf $G$ injektiv. Dann: 
	\begin{liste}
		\item $Z(f') = \emptyset$
		\item $f^{-1} \in H(f(G))$ und $(f^{-1})'(w) = \frac{1}{f'(f^{-1}(w))}$ für $\forall w \in f(G)$
	\end{liste}
\end{satz}

\begin{beweis}
	\begin{liste}
		\item[ (1) ] Annahme: Sei $z_0 \in G$ mit $f'(z_0) = 0$, $w_0 := f(z_0)$. O.B.d.A. $w_0 = 0 = z_0$. Also $f(0) = f'(0) = 0$\\
			11.8 $\Rightarrow \exists m \geq 2; \exists g \in H(G)$ mit $f(z) = z^mg(z)\ \forall z \in G$ und $g(0) \neq 0$.\\
			11.3 $\Rightarrow \exists \varepsilon > 0: f(z) \neq 0\ \forall z \in \dot{U}_{\varepsilon}(0)$ und 
			$U_{\varepsilon}(0) \subseteq G$. Also $g(z) \neq 0\ \forall z \in U_{\varepsilon}(0)$. 11.4 $\Rightarrow 
			\exists \psi \in H(U_{\varepsilon}(0))$ mit $\psi^m = g$ auf $U_{\varepsilon}(0)$. Def. $\varphi \in H(U_{\varepsilon}(0))$
			durch $\varphi(z) := z\psi (z)\ (z \in U_{\varepsilon}(0))$. Dann: $\varphi^m = f$ auf $U_{\varepsilon}(0)$; $\varphi(0) = 0$, 
			$\varphi'(z) = \psi(z) + z\psi'(z)$, $\varphi'(0)^m = \psi(0)^m = g(0) \neq 0$ also $\varphi'(0) \neq 0$. 
			O.B.d.A. $\varphi'(z) \neq 0\ \forall z \in U_{\varepsilon}(0)$. Klar: $\varphi$ ist auf $U_{\varepsilon}(0)$ injektiv. \\
			$0 = \varphi(0) \in \varphi(U_{\varepsilon}(0))$. 11.5 $\Rightarrow \exists \delta > 0$: $U_{\delta}(0) \subseteq  \varphi(U_{\varepsilon}(0))$
			11.10 $\Rightarrow \varphi^{-1} \in H( \varphi(U_{\varepsilon}(0)))$, 11.5 $\Rightarrow U:=  \varphi^{-1}(U_{\delta}(0))$ ist offen.
			Klar: $0 \in U$, $U \subseteq U_{\varepsilon}(0)$ und $(*) \varphi(U) = U_{\delta}(0).$\\
			Sei $z_1 \in U\backslash\{0\}$; $a_1 := \varphi(z_1)$; $w_1 := f(z_1) \neq 0$.
			$a_1^m = \varphi(z_1)^m = f(z_1) = w_1 \Rightarrow a_1 \neq 0$.
			1.5 $\Rightarrow a_1$ ist eine m-te Wurzel von $w_1$; $m \geq 2 \Rightarrow \exists a_2: a_2^m = a_1^m = w_1$ mit $a_1 \neq a_2$. \\
			$a_2^m = w_1 = \varphi(z_1)^m$; $|a_2| = |\varphi(z_1)| \stackrel{(*)}{<} \delta \Rightarrow a_2 \in \varphi(U)
			\Rightarrow \exists z_2 \in U: a_2 =\varphi(z_2) \Rightarrow f(z_2) = \varphi(z_2)^m = a_2^m = w_1 = a_1^m = f(z_1) \Rightarrow f(z_1) = f(z_2)$
			Widerspruch zu f injektiv!
		\item[ (2) ] folgt aus (1) und 11.10
 	\end{liste}
\end{beweis}

\begin{definition}
	Sei $z_0 \in G$; $a > 0$; $\gamma_1, \gamma_2: [0,a]\rightarrow \MdC$ seien glatte Wege und \\
	$\gamma'_j(t) \neq 0\ \forall t \in [0,a], j = 1,2$ und $\gamma_1(0) = z_0 = \gamma_2(0)$.
	$\angle(\gamma_1, \gamma_2, z_0) := arg \gamma_2'(0) - arg \gamma'_1(0) = arg \frac{\gamma'_2(0)}{\gamma'_1(0)}$ 
	Orientierter Winkel von $\gamma_1$ nach $\gamma_2$ in $z_0$. 
\end{definition}

% Satz 11.12
\begin{satz}[Winkeltreue] % vielleicht mag irgendjemand das korrekte Winkelzeichen suchen? Dankeschön ;)
	Sei $f \in H(G)$, $z_0 \in G$ und $f'(z_0) \neq 0$. Dann: \\
	$\angle( f \circ \gamma_1, f \circ \gamma_2, f(z_0) ) = \angle( \gamma_1, \gamma_2, z_0 )$
\end{satz}

\begin{beweis}
	$\Gamma_j := f \circ \gamma_j\ ( j = 1, 2 )$. $\Gamma_j'(t) = f'(\gamma_j(t)) \gamma_j'(t)\ 
	\Gamma_j'(0) = f'(z_0)\gamma_j'(0) \neq 0$. \\
	$\exists b \in (0,a)$ mit $\Gamma_j'(t) \neq 0\ \forall t \in [0,b]$.\\
	$\angle(\Gamma_1, \Gamma_2, f(z_0)) = arg  \frac{\Gamma'_2(0)}{\Gamma'_1(0)} = arg \frac{\gamma'_2(0)}{\gamma'_1(0)} = \angle( \gamma_1, \gamma_2, z_0 )$
\end{beweis}

\begin{definition}
	\begin{liste}
		\item[ (1) ] $G_1$ und $G_2$ seien Gebiete in $\MdC$. Ist $f \in H(G_1)$ injektiv auf $G_1$ und gilt
			$f(G_1) = G_2$, so heißt f eine \begriff{konforme Abbildung} von $G_1$ auf $G_2$.
		\item[ (2) ] Ist $f: G \rightarrow G$ eine konforme Abbildung von $G$ auf $G$, so heißt $f$ ein \begriff{Automorphismus} von $G$: \\
			$f \in \mbox{Aut}(G)$.
	\end{liste}
\end{definition}

% Satz 11.13
\begin{satz}
	$G_1, G_2$ seien  Gebiete, $f: G_1 \rightarrow G_2$ sei eine konforme Abbildung von $G_1$ auf $G_2$ und $G_1$ sei ein Elementargebiet. 
	Dann ist $G_2$ ebenfalls ein Elementargebiet.
\end{satz}

\begin{beweis}
Sei $g \in H(G_2),\ h := (g \circ f) f'$.
Dann $h \in H(G_1), G_1$ EG $\Rightarrow \exists$ eine Stammfunktion $\Phi$ von h, $F := \Phi \circ f^{-1}$ ist dann SF von g, g war beliebig $\Rightarrow G_2$ ebenfalls EG. 
\end{beweis}

\chapter{Das Schwarzsche Lemma} % Sorry, latexki blickt das nicht mit: ; $\text{Aut}(\MdD)$}

$\MdD := \{ z\in\MdC: |z|<1\}$.

% Satz 12.1
\begin{satz}[Schwarzsches Lemma]
Es sei $f\in H(\MdD), f(\MdD) \subseteq \MdD$ und $f(0) = 0$. \\
Dann: \\  \centerline{$|f(z)| \leq |z| \forall z \in \MdD$ und $|f'(0)|\leq 1$ $(*)$. } \\  \\ Ist $|f'(0)|=1$ oder $|f(z_0)| = |z_0|$ für ein $z_0 \in \MdD \backslash \{0\}$, so ex. ein $\lambda \in \Rand \MdD$ mit: $f(z)=\lambda z$.
\end{satz}

\begin{beweis}
O.b.d.A.: $f\not\equiv 0$. 11.8 $\Rightarrow \exists g \in H(\MdD): f(z) = zg(z)$. Sei $z \in \MdD$. Wähle $r>0$ so, dass $r<1$ und $|z| < r$. Dann: $|g(z)| \stackrel{\text{11.7}}{\leq} \max\limits_{|w|=r} |g(w)| = \max\limits_{|w|=r} \frac{|f(w)|}{|w|} \leq \frac{1}{r} $ $\stackrel{r \rightarrow 1}{\Rightarrow} |g(z)| \leq1$. Also $|g(z)| \leq 1 \forall z \in \MdD$. 
$f'(z) = g(z) + zg'(z) \Rightarrow f'(0) = g(0)$ Also gilt $(*)$.
Es sei $|f'(0)| = 1$ oder $|f(z_0)| = |z_0|$ für ein $z_0 \in \MdD \backslash \{0\} \Rightarrow  |g(0)| = 1$ oder $|g(z_0)| = 1 \Rightarrow |g|$ hat ein Maximum in $\MdD$. 11.6 $\Rightarrow$ g konstant $\Rightarrow \exists \lambda \in \MdC: g(z) = \lambda \;\forall z \in \MdD$. Dann: $f(z) = \lambda z$. Es ist $|\lambda| = |g(0)| = 1$ oder $|\lambda| = |g(z_0)| = 1 \Rightarrow \lambda \in \partial \MdD$.
\end{beweis}

\begin{definition}
Sei $a\in\MdD$ und $S_a \in H(\MdC \backslash \{ \frac{1}{\overline{a}}\})$ def. durch $S_a(z) := \frac{z-a}{1-\overline{a}z}$
\end{definition}

\begin{beachte}
$|\frac{1}{\overline{a}}| = \frac{1}{|a|} > 1$, also $\frac{1}{\overline{a}} \notin \overline{\MdD}$. $S_a(a) = 0$, $S_a(0) = -a$.
\end{beachte}

% Satz 12.2
\begin{satz}
Sei $a \in \MdD$. Dann:
\begin{liste}
\item $S_a$ ist auf $\MdC \backslash \{\frac{1}{\overline{a}}\}$ injektiv.
\item $S_a^{-1} = S_{-a}$ auf $\overline{\MdD}$
\item $S_a(\partial \MdD) = \partial \MdD$
\item $S_a(\MdD) = \MdD$
\item Ist $\lambda \in \partial \MdD$, so ist $\lambda S_a \in \Aut(\MdD)$.
\end{liste}
\end{satz}

\begin{beweis}
\begin{liste}
\item[(1)] Nachrechnen.
\item[(2)] $w = S_a(z) = \frac{z-a}{1-\overline{a}z} \gdw z-a = w - \overline{a}zw \gdw z(1+\overline{a}w) = w+a \gdw z = \frac{w+a}{1+\overline{a}w} = S_{-a}(w)$
\item[(3)] Sei $|z| = 1$, also $z=e^{it} (t\in\MdR)$.$|S_a(z)| = |\frac{e^{it} - a}{1-\overline{a}e^{it}}| = |\frac{e^{it} -a}{e^{it}(e^{-it}-\overline{a})}| = \frac{|e^{it} -a|}{|e^{it}||\overline{e^{it}-a}|} = 1$. Also: $S_a(\partial \MdD) \subseteq \partial \MdD$, $\partial \MdD \stackrel{(2)}{=} S_a( \underbrace{S_{-a}(\partial \MdD)}_{\stackrel{\text{wie oben}}{\subseteq \partial \MdD}} ) \subseteq S_a(\partial \MdD)$.
\item[(4)] Sei $z \in \MdD$. $|S_a(z)| \stackrel{\text{11.7}}{\leq} \max\limits_{|w|=1}|S_a(w)| \stackrel{(3)}{=} 1 \Rightarrow S_a(\MdD) \subseteq \overline{\MdD}$
Sei $z \in \MdD$, $w:=S_a(z)$. Annahme: $|w|=1 \stackrel{(3)}{\Rightarrow} |z| = |S_{-a}(w)| = 1$ Wid.
Also $S_a(\MdD) \subseteq \MdD$. Genauso $S_{-a}(\MdD) \subseteq \MdD$. Dann $\MdD \stackrel{(2)}{=} S_a(S_{-a}(\MdD)) \subseteq S_a(\MdD)$
\item[(5)] folgt aus (1) und (4).
\end{liste}
\end{beweis}

% Satz 12.3
\begin {satz}
 Sei $f \in H(\MdD)$\\ $f \in \mbox{Aut}(\MdD)$ und $f(0)=0 \equizu \exists \lambda \in \partial \MdD:\ f(z) = \lambda z$.
\end{satz}
\begin{beweis}
  "$\Leftarrow$": Klar \\
  "$\Rightarrow$": Dann $f^{-1} \in \mbox{Aut}(\MdD)$, $f^{-1}(0) = 0$. Sei $z \in \MdD$, $w := f(z)$; dann: \\
   $z = f^{-1}(w)$, $|z| = |f^{-1}(w)| \stackrel{12.1}{\leq} |w| = |f(z)|\stackrel{12.1}{\leq} |z|$ Also 
   $|f(z)| = |z|\  \forall z \in \MdD$. $12.1 \Rightarrow \exists \lambda \in \partial \MdD:\ f(z) = \lambda z\ \forall z \in \partial \MdD$
\end{beweis}

% Satz 12.4
\begin{satz}
 $\mbox{Aut}(\MdD) = \{ \lambda S_a: \lambda \in \partial \MdD, a \in \MdD\}$
\end{satz}

\begin{beweis}
 "$\supseteq$": 12.2 (5)\\
 "$\subseteq$": Sei $f \in \mbox{Aut}(\MdD)$, $a := f^{-1}(0) \in \MdD$. 
 $g := f \circ S_a$; $g \in \mbox{Aut}(\MdD)$ und $g(0) = f(S_a(0)) = f(a) = 0$. 
 12.3 $\Rightarrow \exists \lambda \in \partial \MdD: g(z) = \lambda z$. Es ist $f = g \circ S_a = \lambda S_a$
\end{beweis}


%In diesem Paragraphen sei $G \subseteq \MdC$ stets ein \begriff{Gebiet}. Fast wörtlich wie in Analysis I zeigt man:
%
%\begin{satz}[Identitätssatz für Potenzreihen]
%$\sum\limits_{n=0}^{\infty}a_n(z-z_0)^n$ sei eine Potenzreihe mit Konvergenzradius $r>0$, \\
%es sei $f(z)=\sum\limits_{n=0}^{\infty}a_n(z-z_0)^n$ für $z \in U_r(z_0)$, es sei $(z_k)$ eine \\
%Folge in $\dot U_r(z_0)$ mit $z_k \to z_0$ und es gelte $f(z_k) = 0$  $\forall$ $k$ $\in$ $\MdN$. \\
%Dann: $a_n = 0$ $\forall$ $n$ $\in$ $\MdN_0$.
%\end{satz}
%
%\begin{satz}[Identitätssatz für holomorphe Funktionen]
%Es sei $f \in H(G)$, $z_0 \in G$, $(z_k)$ eine Folge in $G\backslash\{z_0\}$ mit $f(z_k) = 0$ $\forall$ $k$ $\in$ $\MdN$\\
%und $ z_k \to z_0$.\\
%Dann: $f = 0$ auf $G$.
%\end{satz}
%
%\begin{beweis}
%$\exists r > 0$: $U_r(z_0) \subseteq G$. 10.4 $\Rightarrow f(z) = \sum\limits_{n=0}^{\infty} \frac{f^{(n)}(z_0)}{n!}(z-z_0)^n$ $\forall$ $z$ $\in$ $U_r(z_0)$\\
%$\exists k_0 \in \MdN$: $z_k \in U_r(z_0)$ $\forall$ $k$ $\geq$ $k_0$. 11.1 $\Rightarrow f^{(n)}(z_0) = 0$ $\forall$ $n \in \MdN_0$\\
%$\Rightarrow z_0 \in A := \{z \in G: f^{(n)}(z) = 0$ $\forall$ $n$ $\in$ $\MdN_0\}$. $B:= G\backslash A$, $A \cap B = \emptyset$\\
%Sei $ a \in A$. $\exists \delta > 0: U_{\delta}(a) \subseteq G$. 10.4 $\Rightarrow f(z) = \sum\limits_{n=0}^{\infty} \frac{f^{(n)}(a)}{n!}(z-a)^n$ $\forall$ $z$ $\in$ $U_{\delta}(a)$\\
%$a \in A \Rightarrow f^{(n)}(a) = 0$ $\forall$ $n$ $\in$ $\MdN_0$ $\Rightarrow f \equiv 0$ auf $U_{\delta}(a)$\\
%$\Rightarrow f^{(n)} \equiv 0$ auf $U_{\delta}(a)$ $\forall$ $n$ $\in$ $\MdN_0$\\
%$\Rightarrow U_{(\delta)}(a) \subseteq A$. $A$ ist also offen. Sei $b \in B$. $\exists k \in \MdN_0: f^{(k)}(b) \neq 0$; \\
%$f^{(k)}$ stetig $\Rightarrow \exists \epsilon > 0: U_{\epsilon}(b) \subseteq G$ und $f^{(k)}(z) \neq 0$ $\forall$ $z \in U_{\epsilon}(b)$\\
%$\Rightarrow U{\epsilon}(b) \subseteq B$; d.h. $B$ ist offen. $G$ ist ein Gebiet $\Rightarrow B = \emptyset \Rightarrow G = A \Rightarrow$ Beh.
%\end{beweis}
%
%\textbf{Bezeichnung} \\
%für $f \in H(G)$: $Z(f) := \{z\in G: f(z) = 0\}$.
%
%\begin{folgerung}
%\begin{liste}
%\item Ist $f \in H(G)$, $f \not\equiv$ %%nicht identisch!!!!!%% 
%$0$ auf $G$ und $z_0 \in Z(f)$, so existiert ein $\epsilon > 0$: 
%\\$U_{\epsilon}(z_0) \subseteq G$, $f(z) \neq 0$ $\forall$ $z$ $\in$ $\dot U_{\epsilon}(z_0)$
%\item Ist $f \in H(G)$, $z_0 \in G$ und gilt: $f^{(n)}(z_0) = 0$ $\forall$ $n$ $\in$ $\MdN_0$, so ist $f = 0$ auf $G$. 	
%\end{liste}
%\end{folgerung}
%
%\begin{beweis}
%\begin{liste}
%\item folgt aus 11.2
%\item Verfahre wie im Beweis von 11.2
%\end{liste}
%\end{beweis}
%
%\begin{satz}
%Sei $G$ ein EG und $f \in H(G)$ mit $Z(f) = \emptyset$
%\begin{liste}
%\item $\exists h \in H(G)$: $e^h = f$ auf $G$
%\item Ist $n \in \MdN$, so existiert ein $g \in H(G)$: $g^n = f$ auf $G$
%\end{liste}
%\end{satz}
%
%\begin{beweis}
%\begin{liste}
%\item Es ist $\frac{f'}{f} \in H(G)$. $G$ ist ein EG $\Rightarrow \exists F \in H(G)$: $F'= \frac{f'}{f}$ auf $G$. $\phi := \frac{e^F}{f}$.\\
%Dann: $\phi \in H(G)$ und $f' = 0$ auf $G$. (nachrechnen!)\\
%$\exists c \in \MdC$: $e^F = c \cdot f$ auf $G$.\\
%Klar: $c \neq 0$. 7.1 $\Rightarrow \exists a \in \MdC$: $c = e^a \Rightarrow f = e^{F - a}$ auf $G$.
%\item Sei $h$ wie in (1), $g := e^{\frac{1}{n} h}$. Dann: $g^n = e^h = f$ auf $G$.
%\end{liste}
%\end{beweis}
%
%\begin{satz}
%Sei $D \subseteq \MdC$ offen.
%\begin{liste}
%\item Ist $F \in H(D)$, $0 \in D$, $F(0) = 0$ und $F'(0) \neq 0$, so gilt: $0 \in F(D)°$
%\item Ist $f \in H(D)$ \begriff{nicht} konstant, so ist $f(D)$ offen.
%\item \begriff{Satz von der Gebietstreue:}\\
%Ist $f \in H(G)$ \begriff{nicht} konstant, so ist $f(G)$ ein Gebiet.
%\end{liste}
%\end{satz}
%
%\begin{beweis}
%\begin{liste}
%\item $u := \Re F$, $v := \Im F$. 4.1 $\Rightarrow u_x(0) = v_y(0), u_y(0) = - v_x(0)$ \\
%und $F'(0) = u_x(0) + i v_x(0)$\\
%$\Rightarrow det 
%\left( \begin{array}{ccc}
%u_x(0) & u_y(0) \\
%v_x(0) & v_y(0) \\
%\end{array} \right)
%= u_x(0)^2 + v_x(0)^2 = |F'(0)|^2 \neq 0$\\
%Umkehrsatz (Analysis II) $\Rightarrow \exists U \subseteq D$: $0 \in U$, $U$ ist offen und $F(U)$ ist offen.
%$F(0) = 0 \Rightarrow 0 \in F(U) \Rightarrow \exists \delta > 0$: $U_{\delta}(0) \subseteq F(U) \subseteq F(D)$.
%\item Sei $w_0 \in f(D)$. z.z. $\exists \delta > 0$: $U_{\delta}(w_0) \subseteq f(D)$.\\
%O.B.d.A. $w_0 = 0$. $\exists z_0 \in D$: $f(z_0) = w_0 = 0$. O.B.d.A. $z_0 = 0$.\\
%Also: $f(0) = 0$. $\exists \varepsilon > 0$: $U_{\varepsilon}(z_0) \subseteq D$.\\
%10.4 $\Rightarrow f(z) = a_0 + a_1 z + a_2 z^2 + \dots$  $\forall z \in U_{\varepsilon}(0)$;\\
%$f(0) = 0 \Rightarrow a_0 = 0$. 11.3 $\Rightarrow \exists n \in \MdN$: $a_n \neq 0$\\
%$m:= min \{n \in \MdN : a_n \neq 0 \}$ ($\geq 1$)\\
%Dann: $f(z) = z^m (a_m + a_{m+1} z + a_{m+2} z^2 + \dots) = z^m \cdot g(z)$  $\forall z \in U_{\varepsilon}(0)$,\\
%wobei $g \in H(U_{\varepsilon}(0))$ und $g(0) = a_m \neq 0$.\\
%$g$ stetig $\Rightarrow \exists r \in (0,\varepsilon)$: $g(z) \neq 0$ $\forall z \in U_r(0)$\\
%$U_r(0)$ ist ein EG $\stackrel{11.4}{\Rightarrow} \exists h \in H(U_r(0))$: $h^m = g$ auf $U_r(0)$\\
%Def. $F \in H(U_r(0))$ durch $F(z) := z h(z)$. \\
%Dann: $F(0)=0$, $F'(z) = h(z) + zh'(z)$\\
%also $F'(0)^m = h(0)^m = g(0) \neq 0$, also $F'(0) \neq 0$.\\
%Weiter: $F^m = f$ auf $U_r(0)$. (1)$\Rightarrow \exists R > 0$: $U_R(0) \subseteq F(U_r(0))$.\\
%$\delta := \MdR^m$. Sei $w \in U_{\delta}(0)$. 1.5 $\Rightarrow \exists v \in \MdC$: $v^m = w$\\
%Dann: $|v|^m = |w| < \delta = \MdR^m \Rightarrow |v| <R \Rightarrow v \in U_R(0) \subseteq F(U_r(0))$\\
%$\Rightarrow \exists z \in U_r(0) \subseteq D$ mit: $F(z) = v$.\\
%$\Rightarrow w = v^m = F(z)^m = f(z) \in f(D)$\\
%Also: $U_{\delta}(0) \subseteq f(D)$
%\item 3.6 $\Rightarrow f(h)$ ist zusammenhängend $\stackrel{(2)}{=} f(G)$ ist ein Gebiet.
%
%\end{liste}
%\end{beweis}
%



\chapter{Isolierte Singularitäten}

Vereinbarung: In diesem Paragraphen sei stets $D \subseteq \MdC$ offen, $z_0 \in D$, $\dot{D} := D \backslash \{z_0\}$ und $f \in H(\dot{D})$.\\
$z_0$ hei"st dann eine \begriff{isolierte Singularit"at} von $f$.

\begin{definition}
$z_0$ hei"st eine \begriff{hebbare Singularit"at} von $f :\Leftrightarrow \, \exists h\in H(D): h=f$ auf $\dot{D}$. I.d. Fall ist $h$ eindeutig bestimmt und wir sagen kurz: $f \in H(D)$.
\end{definition}

\begin{beispiel}
$D=\MdC, z_0=0$
\[
f(z) = \frac{\sin z}{z} = \frac1z\left(z - \frac{z^3}{3!} + \frac{z^5}{5!} - \cdots + \cdots \right) = \underbrace{1 - \frac{z^2}{3!} + \frac{z^4}{5!} - \cdots + \cdots}_{=:h(z)}
\]
Dann: $h \in H(\MdC)$. $h=f$ auf $\MdC \backslash\{0\}$. $f$ hat also in $0$ eine hebbare Singularit"at.
\end{beispiel}

%satz 13.1
\begin{satz}[Riemannscher Hebbarkeitssatz]
$f$ hat in $z_0$ eine hebbare Singularit"at $\Leftrightarrow \, \exists \delta > 0: U_{\delta}(z_0) \subseteq D$ und $f$ ist auf $\dot U_{\delta}(z_0)$ beschr"ankt.
\end{satz}

\begin{beweis}
\item[$\Rightarrow:$] klar
\item[$\Leftarrow:$] $M:=sup_{z \in U_{\delta}(z_0)} |f(z)|$. Def: $g: D \rightarrow \MdC$ durch:
\[
g(z) := \begin{cases} (z-z_0)^2f(z) &, z \in \dot{D} \\
0 &, z=z_0 \end{cases}
\]
F"ur $z \in \dot{U}_{\delta}(z_0): \left| \frac{g(z)-g(z_0)}{z-z_0} \right| = \left| \frac{g(z)}{z-z_0} \right| = |f(z)(z-z_0)| \leq M|z-z_0|$\\
$\Rightarrow g$ ist komplex db in $z_0$, also $g \in H(D)$ und $g'(z_0) = 0$.\\
Fall 1: $g = 0$ auf $D$. Dann: $f=0$ auf $\dot{D}$\\
Fall 2: $g \not= 0$ auf $D$. Es ist $g(z_0) = g'(z_0) = 0.$ 11.8 $\Rightarrow \, \exists h \in H(D): g(z) = (z-z_0)^2 h(z) \, \forall z \in D.$ Dann: $h=f$ auf $\dot{D}$.
\end{beweis}

%satz 13.2
\begin{satz}
$z_0$ ist ein \begriff{Pol} von $f : \Leftrightarrow \,\exists m \in \MdN, \, \exists g \in H(D)$ mit:
\[
f(z) = \frac{g(z)}{(z-z_0)^m} \, \forall z \in \dot{D} \mbox{ und } g(z_0) \not= 0.
\]
I. d. Fall ist $m$ eindeutig bestimmt und hei"st die \begriff{Ordnung des Pols} $z_0$ von $f$
\end{satz}

\begin{beweis}
Seien $m,l \in \MdN$, $g,h \in H(D)$, $g(z_0) \not= 0 \not= h(z_0)$ und $\frac{g(z)}{(z-z_0)^m} = f(z) = \frac{h(z)}{(z-z_0)^l} \, \forall z \in \dot{D}$.\\
Annahme: $m > l$, also $m-l \geq 1$. $h(z_0) \not= 0$. $\, \exists \delta > 0: U_{\delta}(z_0) \subseteq D$ und $h(z) \not= 0 \, \forall z \in U_{\delta}(z_0)$. F"ur $z \in \dot{U}_{\delta}(z_0): \frac{g(z)}{h(z)} = (z-z_0)^{m-l} \stackrel{z \rightarrow z_0}{\Rightarrow} g(z_0) = 0$. Wid! Also: $m \leq l$. Analog: $l \leq m$.
\end{beweis}

%satz 13.3
\begin{satz}
Hat $f$ in $z_0$ einen Pol, so gilt: $|f(z)| \rightarrow \infty \, (z \rightarrow z_0)$
\end{satz}

\begin{beweis}
Folgt aus 13.2
\end{beweis}

\begin{beispiele}
\begin{liste}
\item[(1)] $f(z) = \frac{1}{z}$. $f$ hat im Nullpunkt einen einfachen Pol.
\item[(2)] $f(z) = \frac{e^z}{z^{17}}$. $f$ hat in $0$ einen Pol der Ordnung $17$.
\end{liste}
\end{beispiele}

\begin{definition}
$z_0$ hei"st eine \begriff{wesentliche Singularit"at} von $f :\Leftrightarrow z_0$ ist nicht hebbar und kein Pol von $f$.
\end{definition}

\begin{beispiel}
$f(z) = e^{\frac1z}$ \, $(D = \MdC, z_0=0)$\\
$z_n := \frac{1}{n}$, $f(z_n) = e^n \rightarrow \infty (n \rightarrow \infty)$, $z_n \rightarrow 0$. 13.1 $\Rightarrow 0$ ist nicht hebbar.\\
$w_n := \frac{i}n = - \frac1{in}$. $|f(w_n)| = | e^{-in} | = 1 \, \forall n \in \MdN$, $w_n \rightarrow 0$. 13.3 $\Rightarrow z_0 = 0$ ist kein Pol von $f$. $f$ hat also in $z_0 = 0$ eine wesentliche Singularit"at.
\end{beispiel}

%satz 13.4
\begin{satz}[Satz von Casorati-Weierstra"s]
$f$ habe in $z_0$ eine wesentliche Singularit"at und es sei $\delta > 0$ so, dass $U_{\delta}(z_0) \subseteq D$. Dann:
\[
\overline{f(\dot{U}_{\delta}(z_0))} = \MdC
\]
d.h. ist $b \in \MdC$ und $\varepsilon > 0$, so existiert ein $z \in \dot{U}_{\delta}(z_0): |f(z) - b| < \varepsilon$.
\end{satz}

\begin{beweis}
Sei $b \in \MdC$ und $\varepsilon > 0$. Ann: $|f(z) - b | \geq \varepsilon \, \forall z \in \dot{U}_{\delta}(z_0)$. $g:= \frac1{f-b}$. Dann: $g \in H(\dot{U}_{\delta}(z_0))$ und $|g| \leq \frac1{\varepsilon}$ auf $\dot{U}_{\delta}(z_0)$. 13.1 $\Rightarrow g$ hat in $z_0$ eine hebbare Singularit"at. Kurz: $g \in H (U_{\delta}(z_0))$\\
Fall 1: $g(z_0) \not= 0$. O.B.d.A: $g(z) \not= 0 \, \forall z \in U_{\delta}(z_0)$. $f = \frac{1}{g}+b$ auf $\dot{U}_{\delta}(z_0) \Rightarrow f$ hat in $z_0$ eine hebbare Singularit"at.\\
Fall 2: $g(z_0) = 0$. 11.8 $\Rightarrow \, \exists m \in \MdN, \varphi \in H(U_{\delta}(z_0)): g(z) = (z-z_0)^m \varphi(z) \, \forall z \in U_{\delta}(z_0)$ und $\varphi(z_0) \not= 0$. O.B.d.A: $\varphi(z) \not= 0 \, \forall z \in U_{\delta}(z_0)$. Def: $\Psi: D \rightarrow \MdC$ durch:
\[
\Psi(z) = \begin{cases}
\frac1{\varphi(z)} &, z \in U_{\delta}(z_0)\\
(z-z_0)^m(f(z)-b) &,z \in \dot{D}
\end{cases}
\]
$\Psi$ ist wohldefiniert: F"ur $z \in \dot{U}_{\delta}(z_0): \frac1{\varphi(z)} = \frac{(z-z_0)^m}{g(z)} = (z-z_0)^m (f(z)-b)$.
Dann: $\Psi \in H(D)$ und $\Psi(z_0) = \frac1{\varphi(z_0)} \not= 0$.\\
$h(z) := \Psi(z) + b(z-z_0)^m \, (z \in D)$. Klar: $h \in H(D)$\\
$h(z_0) = \Psi(z_0) \not= 0$.
Weiter: $\frac{h(z)}{(z-z_0)^m} = \frac{\Psi(z)}{(z-z_0)^m}+b = f(z) - b + b = f(z) \, \forall z \in \dot{D} \stackrel{13.2}{\Rightarrow} f$ hat in $z_0$ einen Pol. Wid!
\end{beweis}

%satz 13.5
\begin{satz}[Klassifikation]
Die isolierte Singularit"at $z_0$ von $f$ ist
\begin{liste}
\item hebbar $\Leftrightarrow \, \exists \delta > 0 : U_{\delta}(z_0) \subseteq D$ und $f$ ist auf $U_{\delta}(z_0)$ beschr"ankt.
\item ein Pol von $f$ $\Leftrightarrow |f(z)| \rightarrow \infty$ $(z \rightarrow z_0)$
\item wesentlich $\Leftrightarrow \, \forall \delta > 0$ mit $U_{\delta}(z_0) \subseteq D$ gilt: $\overline{f(\dot{U}_{\delta}(z_0))} = \MdC$
\end{liste}
\end{satz}

\begin{beweis}
\begin{liste}
\item 13.1
\item $\Rightarrow:$ 13.3\\
$\Leftarrow:$ Vorr. und 13.1 $\Rightarrow z_0$ nicht hebbar. Vorr. und 13.4 $\Rightarrow z_0$ nicht wesentlich
\item $\Rightarrow:$ 13.4\\
$\Leftarrow:$ Vorr. und 13.1 $\Rightarrow z_0$ ist nicht hebbar. Vorr. und 13.3 $\Rightarrow z_0$ ist kein Pol!
\end{liste}
\end{beweis}

\begin{beispiele}
\begin{liste}
\item $f(z) = e^{\frac{1}{z}}$. "Ubung: $f(\dot{U}_{\delta}(0)) = \MdC \backslash \{0\} \ \forall \delta > 0$.
\item $f(z) = \sin \frac{1}{z}$. "Ubung: $f(\dot{U}_{\delta}(0)) = \MdC \ \forall \delta > 0$.
\end{liste}
\end{beispiele}

\chapter{Laurententwicklung}

Für $z_0 \in \MdC$: $U_\infty(z_0) := \MdC$, $\dot{U}_\infty(z_0) = \MdC \backslash \{z_0\}$, 
$\frac{1}{0} := \infty$. Erinnerung: Satz 9.5: Sei $\gamma$ ein stückweise glatter Weg in $\MdC$, 
$\varphi \in C(\text{Tr}(\gamma))$ und $g(z) = \frac{1}{2 \pi i}\int_{\gamma}\frac{\varphi(w)}{w-z}dw$ 
$(z \in  \MdC \backslash \text{Tr}(\gamma) )$. Dann: \\
$g \in H( \MdC \backslash \text{Tr}(\gamma))$.

% Satz 14.1
\begin{satz}
  Seien $ 0 \le r < R \le \infty$; $A := \{ z \in \MdC : r < |z| < R \}$ und $f \in H(A)$. Für $s \in (r,R)$ sei 
  $\gamma_s(t) := se^{it}$, $t \in [0,2 \pi ]$ und $J(s) := \int_{\gamma_s} f(z) dz$. \\
  Dann ist $J$ konstant auf $(r, R)$.
\end{satz}

\begin{beweis} 
  $g(z) := z f(z)$  $(z \in A)$. Dann: 
  $f(z) = \frac{g(z)}{z}$ und $g \in H(A)$. \\
  $J(s) = \int_{\gamma_s} \frac{g(z)}{z}(z) dz = \int_0^{2 \pi} \frac{g(se^{it})}{se^{it}}sie^{it} dt =  
  \int_0^{2 \pi} ig(se^{it}) dt$\\
  $J$ ist auf $(r,R)$ db und $J'(s) = \int_0^{2 \pi} i \frac{d}{ds}g(se^{it}) dt = \int_0^{2 \pi} i g'(se^{it})e^{it}  dt = 
  \frac{1}{s} \int_0^{2 \pi} g'(se^{it})sie^{it}  dt =  \frac{1}{s} \int_0^{2 \pi} g'(\gamma_s(t))\gamma'_s(t) dt =
  \frac{1}{s} \int_{\gamma_s} g'(z) dz \stackrel{8.5}{=} 0 \Rightarrow J(s)$ konstant.
\end{beweis}

% Satz 14.2
\begin{satz} [Laurententwicklung]
  Sei $A$ wie in 14.1 und $f \in H(A)$. Dann existieren eindeutig bestimmte Funktionen $g \in H(U_R(0))$ und $h \in H(U_{\frac{1}{r}}(0))$ mit: \\
  (*) $f(z) = g(z) + h(\frac{1}{z})\ \forall z \in A$ und $h(0) = 0$ \\
  (*) heißt die Laurentzerlegung von $f$, $g$ heißt Nebenteil von $f$ und die Funktion $ z \rightarrow h(\frac{1}{z})$ ist der Hauptteil von $f$. 
\end{satz}

\begin{beispiel}
  $f(z) = e^{\frac{1}{z}}$ $A = \MdC \backslash \{0\}$ $(r = 0, R = \infty)$. Es gilt: \\
  $f(z) = 1 + (  e^{\frac{1}{z}} - 1 )$, also $g(z) = 1$, $h(z) = e^z -1 $
\end{beispiel}

\begin{beweis}
  \begin{itemize}
    \item[ 1. ] Eindeutigkeit: es sei $g, g_1 \in H(U_R(0))$, $h, h_1 \in H(U_{\frac{1}{r}}(0))$;
	$h_1(0) = 0 = h(0)$ und $g(z) + h(\frac{1}{z}) = f(z) = g_1(z) + h_1(\frac{1}{z})\ \forall z \in A$.\\
	$G := g - g_1 \in H(U_R(0))$, $H:= h_1 - h \in  H(U_{\frac{1}{r}}(0))$ \\
	$\Rightarrow G(z) = H(\frac{1}{z}) \ \forall z \in A$
	Dann ist $F:\MdC \rightarrow \MdC$, definiert durch \\
	$F(z) = \begin{cases} G(z) & (|z| < R) \\ H(\frac1z) & (|z| > r)\end{cases}$ \; auf $\MdC$ wohldefiniert. $F \in H(\MdC)$. \\
	Sei $(z_n)$ eine Folge in $\MdC$ und $|z_n| \rightarrow \infty$. Dann $(\frac{1}{z}) \rightarrow 0$ und
	$z_n > r \ \forall n \ge n_0$. $F(z_n) = H(\frac{1}{z_n}) = h_1(\frac{1}{z_n}) - h(\frac{1}{z_n}) \rightarrow h_1(0) - h(0) = 0$\\
	Also $F(z) \rightarrow 0 \ (|z| \rightarrow \infty)$ Somit: \\
	$\exists \varrho > 0: |F(z)| \le 1 \ \forall z \in \MdC \backslash U_{\varrho}(0)$. $F$ stetig auf $ \overline{U_{\varrho}(0)} \Rightarrow$
	$F$ ist auf $\MdC$ beschränkt. 10.2 $\Rightarrow$ $F$ ist auf $\MdC$ konstant; wegen 
	$F(z) \rightarrow 0 \ (z \rightarrow \infty)$ folgt: $F \equiv 0$
%%% hier gehts noch weiter!
%%% So müßt's jetzt stimmen (zumindest ist ja jetzt die Eindeutigkeit bewiesen, oder?)
    \item[ 2. ] Existenz: fehlt hier nicht was ? Doch! 
  \end{itemize}
\end{beweis}

\begin{definition}
Sei $(a_n)_{n \in \MdZ}$ eine Folge in $\MdC, z_0 \in \MdC$ und $A \subseteq
\MdC$. Eine Reihe der Form $\sum\limits_{n=-\infty}^{\infty} a_n(z-z_0)^n$ heißt eine
\begriff{Laurentreihe}. \\
Diese Reihe heißt in $z \in \MdC$ (absolut) konvergent $: \gdw
\sum\limits_{n=0}^{\infty} a_n(z-z_0)^n$  und $\sum\limits_{n=1}^{\infty}
a_{-n}(z-z_0)^{-n}$ konvergieren (absolut). \\ In diesem Fall: $\sum\limits_{n=-\infty}^{\infty}
a_n(z-z_0)^n := \sum\limits_{n=0}^{\infty} a_n(z-z_0)^n + \sum\limits_{n=1}^{\infty}
a_{-n}(z-z_0)^{-n}$. Die Laurentreihe heißt auf $A$ (lokal) gleichmäßig
konvergent $:\gdw$ $\sum\limits_{n=0}^{\infty} a_n(z-z_0)^n$ und $\sum\limits_{n=1}^{\infty}
a_{-n}(z-z_0)^{-n}$ konvergieren auf A (lokal) gleichmäßig.
\end{definition}

% Satz 14.3
\begin{satz}
Sei $0 \leq r < R \leq \infty$, $A := \{z \in \MdC: r < |z-z_0| < R\}$ und $f
\in H(A).$ \\Dann hat $f$ auf $A$ die \begriff{Laurententwicklung} 
$f(z) = \sum\limits_{n=-\infty}^{\infty} a_n(z-z_0)^n$. Die Laurentreihe
konvergiert auf $A$ absolut und lokal gleichmäßig. Die Koeffizienten $a_n$ $(n
\in \MdZ )$ sind eindeutig bestimmt. Ist $r < \rho < R$ und $\gamma(t) := z_0 +
\rho e^{it}$ $t \in [0,2\pi]$, so gilt: \\ \\
\centerline{$a_n = \frac{1}{2\pi i} \int\limits_{\gamma} 
\frac{f(w)}{(w-z_0)^{n+1}} dw$ $\forall n \in \MdZ$}. \\
$\sum\limits_{n=0}^{\infty} a_n(z-z_0)^n$ heißt \begriff{Nebenteil} von $f$, \\
$\sum\limits_{n=1}^{\infty} a_{-n}(z-z_0)^{-n} = \frac{a_{-1}}{z-z_0}+
\frac{a_{-2}}{(z-z_0)^2} + \ldots$ heißt der \begriff{Hauptteil} von f.
\end{satz}

\begin{beweis}
O.B.d.A: $z_0 = 0$. 14.2 $\Rightarrow$ $\exists g \in H(U_R(0)), \exists h \in
H(U_{\frac{1}{r}}(0)): f(z) = g(z) + h(\frac{1}{z})$ $\forall z \in A$ und $h(0)
= 0.$ 10.4 $\Rightarrow$ $g(z) = \sum\limits_{n=0}^{\infty} a_n z^n$
$\forall z \in U_R(0)$ und $h(z) = \sum\limits_{n=0}^{\infty} b_n z^n$
$\forall z \in U_{\frac{1}{r}}$. Setze $a_{-n} := b_n$ für $n \geq 1$. Dann: 
$f(z) = \sum\limits_{n=-\infty}^{\infty} a_n z^n$. 5.4 $\Rightarrow$ die
Laurentreihe konvergiert auf $A$ absolut und lokal gleichmäßig. \\
14.2 $\Rightarrow$ $g$ und $h$ sind eindeutig bestimmt \\ 5.4 $\Rightarrow$
$a_n$ eindeutig bestimmt für $n \in \MdZ$. Sei $n \in \MdZ$; $\gamma(t) := \rho
e^{it}$ $(t \in [0, 2 \pi])$ $r<\rho <R$. Sei $w \in Tr(\gamma)$: \\
\centerline{$\frac{f(w)}{w^{n+1}} = \sum\limits_{\nu=-\infty}^{\infty} a_\nu w^{\nu - n -1}$}.
Die letzte Reihe konvergiert auf Tr$(\gamma)$ gleichmäßig. \\
8.4 $\Rightarrow$ $\int\limits_{\gamma} \frac{f(w)}{w^{n+1}}= \sum\limits_{\nu=-\infty}^{\infty} a_\nu
\underbrace{\int\limits_{\gamma} w^{\nu -n -1}}_{
= \begin{cases} 0 &, \nu \neq n \\
  				2 \pi i & , \nu = n
  \end{cases}}$
\end{beweis}

% Satz 14.4
\begin{satz}
$D \subseteq \MdC$ sei offen, $z_0 \in D$, $\dot{D} := D \backslash \{ z_0 \}$
und $f \in H(\dot{D})$ ($z_0$ ist also eine isolierte Singularität). Sei $R > 0$
so, daß $U_R(z_0) \subseteq D$. $f$ hat also, nach 14.3, auf $\dot{U}_R(z_0)$ die
Laurententwicklung \\
\centerline{$f(z) = \sum\limits_{n=-\infty}^{\infty} a_n(z-z_0)^n$ $(z \in
\dot{U}_R(z_0))$} 
\begin{liste}
\item $f$ hat in $z_0$ eine hebbare Singularität $\gdw$ $a_{-n} = 0$ $\forall n
\in \MdN$
\item $f$ hat in $z_0$ einen Pol der Ordnung $m \in \MdN$ $\gdw$ $a_{-m} \neq 0$, 
$a_{-n} = 0$ $\forall n > m$
\item $f$ hat in $z_0$ eine wesentliche Singularität $\gdw$ $a_{-n} \neq 0$ für
unendlich viele $n \in \MdN$
\end{liste}
\end{satz}

\begin{definition}
Vorraussetzung wie in 14.4. Res$(f, z_0) := a_{-1}$ heißt das \begriff{Residuum}
von $f$ in $z_0$. \\
Ist $0 < \rho < R$ und $\gamma(t) = z_0 + \rho e ^{it}$ $( t \in [0, 2 \pi])$,
so folgt aus 14.3: \\ \centerline{Res$(f, z_0) = \frac{1}{2\pi i} \int\limits_{\gamma} f(z)
dz$}
\end{definition}

\begin{beweis}
\begin{liste}
\item Klar.
\item O.B.d.A: $z_0 = 0$. \\
`` $\Rightarrow$ ``: 13.2 $\Rightarrow$ $\exists g \in H(D): f(z) =
\frac{g(z)}{z^m}$ $\forall z \in \dot{D}$ und $g(z_0) \neq 0$  \\
10.4 $\Rightarrow$ $g(z) = c_0 + c_1z + c_2 z^2 + \ldots $ $\forall z \in
U_R(0)$ $\Rightarrow$ $f(z) = \frac{c_0}{z^m}  + \frac{c_1}{z^{m-1}} + \ldots +
\frac{c_{m-1}}{z} + \sum\limits_{n=m}^{\infty} c_n z^{n-m}$ $\forall z \in
\dot{U}_R(0)$. Eindeutigkeit der Laurententwicklung $\Rightarrow$ $c_0 =
a_{-m}$, also $a_{-m} = g(0) \neq 0$; weiter: $a_{-n} = 0$ $\forall n > m$ \\
``$\Leftarrow$'': $f(z) = \sum\limits_{n=0}^{\infty} a_n z^n + \frac{a_{-1}}{z}
+ \ldots + \frac{a_{-m}}{z^m}$ $\forall z \in \dot{U}_R(0)$ \\
$\Rightarrow$ $z^m f(z) = \underbrace{a_{-m} + \ldots + a_{-1} z ^{m-1} +
\sum\limits_{n=0}^{\infty} a_n z^{n+m}}_{ =: g(z)}$ $\forall z \in \dot{U}_R(0)$
\\
Es ist $g \in H(U_R(0))$, $g(0) = a_{-m} \neq 0$ und $f(z) = \frac{g(z)}{z^m}$
$\forall z \in \dot{U}_R(0)$. 13.2 $\Rightarrow$ $f$ hat einen Pol der Ordnung $m$.
\item folgt aus (1) und (2).
\end{liste}
\end{beweis}

\begin{beispiele}
\begin{liste}
\item $f(z) = \frac{1}{z-1}$ Laurententwicklung in $\MdC \backslash \{1\}: f(z)
= \frac{1}{z-1}$, Res$(f,1) = 1$
\item  $f(z) = \frac{1}{z-1}$ Laurententwicklung in $\{ z \in \MdC : 1 < |z| <
\infty \}$.\\ Für $|z| >
1: $ $\frac{1}{z-1} = \frac{1}{z} \frac{1}{1- \frac{1}{z}} = 
\frac{1}{z} \sum\limits_{n=0}^{\infty} \frac{1}{z^n}$
\item $f(z) = \frac{cos(z)}{z^3}.$ Laurententwicklung in $\MdC \backslash
\{0\}$. $f(z) = \frac{1}{z^3}(1 - \frac{z^2}{2!} + \frac{z^4}{4!} -+ \ldots)
= \underbrace{\frac{1}{z^3} - \frac{1}{2z}}_{\text{Hauptteil}} + \underbrace{\frac{z}{4!}-
\frac{z^3}{6!} +- \ldots}_{\text{Nebenteil}}$, Res$(f,0) = -\frac{1}{2}$
\end{liste}
\end{beispiele}

\chapter{meromorphe Funktionen, Moebiustransformationen}
\begin{definition}
Es sei $\infty$ irgendein Element $\not \in \MdC$. $\hat{\MdC} := \MdC \cup \{ \infty
\}$ heißt die \begriff{Vollebene}. $\infty$ heißt ``\begriff{der Punkt
$\infty$}``. Wir definieren: \\ \\
\centerline{$z + \infty := \infty + z := \infty -z := z - \infty := \infty$
$\forall z \in \MdC$;} \centerline{$\infty z := z \infty := \infty$ $\forall z \in \MdC
\{0\}$; $\frac{z}{\infty} := 0 (z \in \MdC)$, $\frac{z}{0}:= \infty$ ($ z \in
\hat{\MdC} \backslash \{0\} $)}\\ \\
$S := \{ (x_1, x_2, x_3) \in \MdR^3: x_1^2+ x_2^2 + (x_3 - \frac{1}{2})^2 =
\frac{1}{4}\}$ heißt \begriff{Riemannsche Zahlenkugel}. $N := (0,0,1)$ wird als
\begriff{Nordpol} bezeichnet .
\end{definition}

\begin{definition}
$\sigma: S \to \hat{\MdC}$ durch \\ \\
\centerline{$\sigma(N) := \infty$} \\
\centerline{$\sigma(x_1, x_2, x_3) := \frac{x_1}{1-x_3} + i \frac{x_2}{1-x_3}$ für
$(x_1,x_2,x_3) \in S \backslash \{N\}$}. \\ \\
$\sigma$ heißt \begriff{stereographische Projektion}. Anschaulich
(nachrechnen!): Ist $P \in S \backslash \{N\}$, so trifft die Gerade durch $N$ und $P$
die komplexe Ebene im Punkt $\sigma(P)$.
\end{definition}

%Satz 15.1
\begin{satz}
$\sigma$ ist injektiv auf $S$ und $\sigma(S) = \hat{\MdC}.\ \sigma^{-1}:
\hat{\MdC} \to S$ ist gegeben durch \\ $\sigma^{-1}(\infty) = N$, $\sigma^{-1}(z) =
\frac{1}{1+ |z|^2}(\text{Re} z, \text{Im} z, |z|^2)$, falls $z \in \MdC$.
\end{satz}
%
%
%Was ist mit dem Beweis von 15.1?
%den hat er nicht gebracht
%
%Ab hier: Bernhard

%Satz 15.2
\begin{satz}[Der chordale Abstand]
Seien $z,w \in \hat{\MdC}.\ d(z,w):= || \sigma^{-1}(z) - \sigma^{-1}(w) ||$ hei"st der \begriff{chordale Abstand} von $z$ und $w$ (wobei $||\cdot| |=$ eukl. Norm im $\MdR^3)$.\\
F"ur $z,w,u \in \hat{\MdC}: d(z,w) \geq 0;\ d(z,w)=0 \Leftrightarrow z=w;\ d(z,w)=d(w,z);\ d(z,w) \leq d(z,u) + d(u,w)\ (\triangle-$Ungl.)\\
$(\hat{\MdC},d)$ ist also ein metrischer Raum.\\
F"ur $z,w \in \MdC: d(z,\infty) = (1+|z|^2)^{-\frac12};\ d(z,w) = |z-w|(1+|z|^2)^{-\frac12}(1+|w|^2)^{-\frac12}$
\end{satz}

\begin{beweis}
"Ubung!
\end{beweis}

\begin{definition}
Sei $(z_n)$ eine Folge in $\hat{\MdC}$ und $z_0 \in \hat{\MdC}. (z_n)$ \begriff{konvergiert in} $\hat{\MdC}$ gegen $z_0 :\Leftrightarrow d(z_n,z_0) \rightarrow 0 (n \rightarrow \infty)$
\end{definition}

Aus 15.2 folgt:
%Satz 15.3
\begin{satz}
Sei $(z_n)$ eine Folge in $\hat{\MdC}, z_0 \in \hat{\MdC}$
\begin{liste}
\item $d(z_n,z_0) \rightarrow 0 \Leftrightarrow |z_n-z_0| \rightarrow 0$
\item $d(z_n,\infty) \rightarrow 0 \Leftrightarrow |z_n| \rightarrow \infty$
\end{liste}
\end{satz}

Ersetzt man $|z-w| \ (z,w \in \MdC)$ durch $d(z,w) \ (z,w \in \hat{\MdC})$, so lassen sich die topologischen Begriffe der §en 2,3 auch in $\hat{\MdC}$ definieren.

\begin{beispiele}
\item Sei $A \subseteq \hat{\MdC}$. Eine Funktion $f: A \rightarrow \hat{\MdC}$ hei"st stetig in $z_0 \in A :\Leftrightarrow$ f"ur jede Folge $(z_n)$ in $A$ mit $d(z_n,z_0) \rightarrow 0$ gilt: $d(f(z_n),f(z_0)) \rightarrow 0$.
\item $A \subseteq \hat{\MdC}$ hei"st offen $: \Leftrightarrow \ \forall a \in A \ \exists \delta = \delta(a) > 0: \{ z \in \hat{\MdC}: d(z,a) < \delta \} \subseteq A$.
\end{beispiele}

{\bf Konvention}\\
Sei $D \subseteq \MdC$ offen. $z_0 \in D. f \in H(D \backslash\{z_0\})$ und $z_0$ sei ein  Pol von $f$. Wegen 13.5 und 15.3 setzt man $f(z_0):=\infty$. Dann ist $f$ auf ganz $D$ definiert, also $f: D \rightarrow \hat{\MdC}$ und in jedem $z \in D$ stetig.

\begin{definition}
Sei $D \subseteq \MdC$ und $f: D \rightarrow \hat{\MdC}$ und $P(f) := \{ z \in D: f(z) = \infty\}. f$ hei"st auf $D$ \begriff{meromorph} $:\Leftrightarrow$
\begin{itemize}
\item[(i)] $P(f)$ ist in $D$ diskret
\item[(ii)] $f_{|D \backslash P(f)} \in H( D \backslash P(f))$
\item[(iii)] jedes $z_0 \in P(f)$ ist ein Pol von $f$.
\end{itemize}
\end{definition}
\[
M(D) := \{ f: D \rightarrow \hat{\MdC}: f \mbox{ ist auf } D \mbox{ meromorph } \}
\]

\begin{beispiele}
\item $P(f)= \emptyset$ zugelassen. Dann: $H(D) \subseteq M(D)$.
\item Seien $f,g \in H(D), g \not= 0$ auf $D$. Dann $\frac{f}{g} \in M(D). P(\frac{f}{g}) \subseteq Z(g)$.
\item $f(z) = \frac{1}{\sin(\frac{1}{z})}, P(f)= \{\frac{1}{k\pi}, k \in \MdZ \backslash \{0\}\}. 0$ ist kein Pol von $f$, $0$ ist HP der Pole $\frac1{k\pi}$. Also: $f \notin M(\MdC)$, aber $f \in M(\MdC \backslash \{0\}).$
\end{beispiele}

{\bf Moebiustransformationen:}\\
Seien $a,b,c,d \in \MdC$ und es gelte $ad-bc \not= 0$. Eine Abbildung der Form $T(z) := \frac{az+b}{cz+d}$ hei"st eine \begriff{Moebiustransformation} (MB) $(z \in \hat{\MdC})$. Also: $T: \hat{\MdC} \rightarrow \hat{\MdC}$. Die Matrix
$\left( \begin{array}{cc}
a & b \\
c & d
\end{array}
\right) := \Phi_T$ hei"st die zu $T$ geh"orende \begriff{Koeffizientenmatrix}.

\begin{bemerkungen}
\item Die Bedingung $ad-bc \not= 0$ sichert, dass T nicht konstant ist.
\item Sei $c = 0 \Rightarrow d \not= 0 \Rightarrow T(z) = \frac{a}{d}z + \frac{b}{d}.\ T(\infty) = \infty, T_{|\MdC} \in H(\MdC).$
\item $c \not= 0.\ T(\infty) = \frac{a}{c}; T(-\frac{d}{c})=\infty. -\frac{d}{c}$ ist ein Pol der Ordnung $1$ von $T;\ T \in M(\MdC)$.
\end{bemerkungen}
$\mathcal{M} := $Menge aller Moebiustransformationen.

%Satz 15.4
\begin{satz}
Seien $T,S \in \mathcal{M}.$
\begin{liste}
\item $T(\hat{\MdC}) = \hat{\MdC}; T$ ist stetig und injektiv auf $\hat{\MdC};\ T^{-1} \in \mathcal{M};\ T^{-1}(w) = \frac{-dw+b}{cw-a}$
\item $T \circ S \in \mathcal{M}.\ \Phi_{T \circ S} = \Phi_T \cdot \Phi_S$
\end{liste}
$\mathcal{M}$ ist also eine Gruppe.
\end{satz}

\begin{beweis}
"Ubung!
\end{beweis}

{\bf spezielle Moebiustransformationen:}\\
$T(z) := az$ \ (Drehstreckung)\\
$T(z) := z+a$ \ (Translation)\\
$T(z) := \frac1z$ \ (Inversion)

%Satz 15.5
\begin{satz}
$T \in \mathcal{M}$ l"asst sich darstellen als Hintereinandersausf"uhrung von Drehstreckung, Translation und Inversion.
\end{satz}

\begin{beweis}
Sei $T(z) = \frac{az+b}{cz+d}.$
\begin{liste}
\item[Fall 1:] $c=0$. Dann $d \not= 0$ und $T(z) = \frac{a}{d}z + \frac{b}{d}$. Setze $T_1=\frac{a}{d}z$ und $T_2=z+ \frac{b}{d} \Rightarrow T = T_2 \circ T_1.$
\item[Fall 2:] $c \not= 0.\ T(z) = \frac{a}{c}+\frac{\frac{b}{c}-\frac{ad}{c^2}}{z+\frac{d}{c}} = \alpha + \frac{\beta}{z+ \gamma}.\ T_1(z) := z+ \gamma;\ T_2(z) := \frac1z;\ T_3(z) := \beta z;\ T_4(z):=z + \alpha \Rightarrow T=T_4 \circ T_3 \circ T_2 \circ T_1.$
\end{liste}
\end{beweis}

%Satz 15.6
\begin{satz}
Sei $T \in \mathcal{M}$. Dann hat $T$ einen oder zwei Fixpunkte oder es ist $T(z) = z$.
\end{satz}

\begin{beweis}
Sei $T(z) = \frac{az+b}{cz+d}$
\begin{liste}
\item[Fall 1:] $T(\infty) = \infty.$ Dann ist $c = 0, d \not= 0. \Rightarrow T(z) = \frac{a}{d}z + \frac{b}{d} = \alpha z + \beta.$ Sei $z_1$ ein Fixpunkt von $T, z_1 \not= \infty$. Also $z_1 = \alpha z_1 + \beta \Leftrightarrow (1-\alpha)z_1=\beta.$
\begin{liste}
\item[Fall 1.1:] $\alpha = 1 \Rightarrow \beta = 0 \Rightarrow T(z) = z$.
\item[Fall 1.2:] $\alpha \not=1 \Rightarrow z_1 = \frac{\beta}{1-\alpha}.$
\end{liste}
\item[Fall 2:] $T(\infty) \not= \infty$. Sei $z_0 \in \MdC.\ T(z_0) = z_0 \Leftrightarrow az_0+b = z_0(cz_0+d)$ quadratische Gleichung $\Rightarrow$ ein oder zwei L"osungen.
\end{liste}
\end{beweis}

\begin{definition}
Seien $z_1,z_2,z_3 \in \hat{\MdC}$ paarweise verschieden. F"ur $z \in \hat{\MdC}$ hei"st
\[
DV(z,z_1,z_2,z_3) := \left\{
\begin{array}{cl}
\frac{z-z_1}{z-z_3}:\frac{z_2-z_1}{z_2-z_3} & \mbox{, falls } z_1,z_2,z_3 \in \MdC \\
\frac{z_2-z_3}{z-z_3} & \mbox{, falls } z_1 = \infty \\
\frac{z-z_1}{z-z_3} & \mbox{, falls } z_2 = \infty \\
\frac{z-z_1}{z_2-z_1} & \mbox{, falls } z_3 = \infty
\end{array} \right.
\]
das \begriff{Doppelverh"altnis} von $z,z_1,z_2,z_3$.
\end{definition}

%Satz 15.7
\begin{satz}
Seien $z_1,z_2,z_3 \in \hat{\MdC}$ wie oben.
\begin{liste}
\item Sind $T_1,T_2 \in \mathcal{M}$ und gilt $T_1(z_j) = T_2(z_j) \ (j=1,2,3) \Rightarrow T_1 = T_2.$
\item Es ist $T(z) := DV(z,z_1,z_2,z_3) \ (z \in \hat{\MdC})$ eine Moebiustransformation. $T$ ist die einzige Moebiustransformation mit $T(z_1)=0; \ T(z_2) = 1 \ ; T(z_3) = \infty$.
\item Sind $w_1,w_2,w_3 \in \hat{\MdC}$ paarweise verschieden, so existiert genau ein $S \in \mathcal{M}: S(z_j) = w_j \ (j = 1,2,3)$
\item $DV(z,z_1,z_2,z_3) = DV(S(z),S(z_1),S(z_2),S(z_3)) \ \forall z \in \hat{\MdC} \ \forall S \in \mathcal{M}$ \ (Invarianz des Doppelverh"altnisses)
\end{liste}
\end{satz}

\begin{beweis}
\begin{liste}
\item $T := T_2^{-1} \circ T_1.$ 15.4 $\Rightarrow T \in \mathcal{M}.\ T(z_j) = T_2^{-1}(T_1(z_j)) = T_2^{-1}(T_2(z_j)) = z_j \ (j=1,2,3).$ 15.6 $\Rightarrow T(z) = z \ \forall z \in \hat{\MdC} \Rightarrow T_1 = T_2$.
\item Klar: $T \in \mathcal{M}$. Nachrechnen: $T(z_1) = 0; \ T(z_2) = 1; \ T(z_3) = \infty$. Eindeutigkeit folgt aus (1).
\item Eindeutigkeit: (1). Existenz: $T_1(z):= DV(z,z_1,z_2,z_3); \ T_2(z) := DV(z,w_1,w_2,w_3).\ S:= T_2^{-1}\circ T_1.\ S(z_1) = T_2^{-1}(T_1(z_1)) \stackrel{(2)}{=} T_2^{-1}(0) \stackrel{(1)}{=} w_1.$ Analog: $S(z_2) = w_2; \ S(z_3) = w_3.$
\item "Ubung.
\end{liste}
\end{beweis}

{\bf Kreisgleichung:}\\
Sei $z_0 \in \MdC, r>0. |z-z_0|=r \Leftrightarrow (z-z_0)(\bar{z}-\bar{z_0}) = r^2 \Leftrightarrow |z|^2 - \bar{z_0}z-z_0\bar{z}+|z_0|^2-r^2=0 \Leftrightarrow |z|^2 + \bar{\alpha}z + \alpha\bar{z}+\beta=0,$ wobei $\alpha = -z_0 \in \MdC. \beta = |z_0|^2-r^2 \in \MdR$ und $|\alpha|^2-\beta = |z_0|^2-|z_0|^2+r^2 > 0$, also $ \beta < |\alpha|$.\\
\\
{\bf Geradengleichung:}\\
$mx+ny+d=0 \ (m,n,d,x,y \in \MdR). \ x = $Re$ z, \ y= $Im$ z; \alpha = \frac{m}{2}+i\frac{n}{2} \in \MdC, \ \beta := d \in \MdR. mx+ny+d = 0 \Leftrightarrow \bar{\alpha}z+\alpha\bar{z}+\beta=0$.\\
\\
{\bf Fazit:}\\
Sind $\alpha \in \MdC, \beta \in \MdR$, so ist $\varepsilon |z|^2 + \bar{\alpha}z + \alpha\bar{z}+\beta = 0$
\begin{liste}
\item[-] Die Gleichung eines Kreises, falls $\varepsilon = 1$ und $\beta < |\alpha|^2$
\item[-] Die Gleichung einer Geraden, falls $\varepsilon = 0.$
\end{liste}

%Satz 15.8
\begin{satz}
Sei $T \in \mathcal{M}.\ T$ bildet eine Gerade (einen Kreis) auf eine Gerade oder einen Kreis ab.
\end{satz}

\begin{beweis}
Die Behauptung ist klar f"ur Drehstreckungen und Translationen. Wegen 15.5 gen"ugt es die Behauptung f"ur Inversionen $(T(z) = \frac1z)$ zu zeigen. Sei $\varepsilon|z|^2 + \bar{\alpha}z + \alpha\bar{z}+\beta =0$. die Gleichung einer Geraden oder eines Kreises und $w = \frac1z$. Dann: $\varepsilon \frac1{|w|^2}+ \bar{\alpha}\frac1w + \alpha \frac1{\bar{w}}+ \beta = 0 \Rightarrow \varepsilon + \bar{\alpha}\bar{w}+\alpha w + \beta |w|^2 = 0.$
\begin{liste}
\item[Fall 1:] $\beta = 0 \rightarrow$ Gerade.
\item[Fall 2:] $\beta \not=0$. Dann: $\frac{\varepsilon}{\beta} + \overline{\left(\frac{\alpha}{\beta}\right)} \bar{w} + \frac{\alpha}{\beta}w + |w|^2 = 0 \rightarrow$ Kreis.
\end{liste}
\end{beweis}

\begin{beispiel}
Bestimme ein $T \in \mathcal{M}$ mit: $T(\partial \mathbb{D}) = \MdR \cup \{ \infty \}. z_1 = 1; \ z_2=i; \ z_3 = -1. T(z) := DV(z,1,i,-1) = -i\frac{z-1}{z+1}$. 15.7 $\Rightarrow T(1) = 0; \ T(i) = 1; \ T(-1) = \infty.$ 15.8 $\Rightarrow T(\partial \mathbb{D}) = \MdR \cup \{ \infty \}.$
\end{beispiel}


\chapter{Die Umlaufzahl}
{\bf Hilfssatz:}\\
Sei $\sigma$ eine Menge von zsh. Teilmengen von $\MdC$ mit $\bigcap_{A \in \sigma} A \not= \emptyset$. Dann ist $\bigcup_{A \in \sigma} A$ zsh.

\begin{beweis}
Fast w"ortlich wie Hilfssatz 3 in §9.
\end{beweis}

\begin{definition}
Sei $D \subseteq \MdC$ offen. $C \subseteq D$ hei"st eine (Zusammenhang-)\begriff{Komponente} von $D :\Leftrightarrow C$ ist zsh. und aus $C \subseteq C_1 \subseteq D$. $C_1$ zsh. folgt stets $C=C_1$.
\end{definition}

\begin{beispiel}
$D = U_1(0) \cup U_1(3)$ Dann nennt man $U_1(0)$ und $U_1(3)$ die Komponenten von $D$.
\end{beispiel}

%Satz 16.1
\begin{satz}
Sei $D \subseteq \MdC$ offen und $K \subseteq \MdC$ kompakt.
\begin{liste}
\item Ist $C \subseteq D$ eine Komponente von $D$, so ist $C$ ein Gebiet.
\item Sind $C_1,C_2$ Komponenten von $D$, so gilt: $C_1 \cap C_2 = \emptyset$ oder $C_1 = C_2$.
\item Ist $z_0 \in D$, so  existiert genau eine Komponente $C$ von $D: z_0 \in C$.
\item $\MdC \backslash K$ hat genau eine unbeschr"ankte Komponente.
\end{liste}
\end{satz}

\begin{beweis}
\begin{liste}
\item Sei $z_0 \in C$. $\exists \delta > 0: U_{\delta}(z_0) \subseteq D$. $C_1:=C \cup U_{\delta}(z_0) \subseteq D$. Klar: $C \subseteq C_1.$ HS $\Rightarrow C_1$ zsh. $C$ Komponente von $D \Rightarrow C=C_1 \Rightarrow U_{\delta}(z_0) \subseteq C \Rightarrow C$ offen $\Rightarrow C$ Gebiet.
\item Sei $C_1 \cap C_2 \not= \emptyset$. $C:=C_1 \cup C_2$. HS $\Rightarrow C$ zsh. Klar: $C_1 \subseteq C \subseteq D$.\\
$C_1$ Komponente von $D \Rightarrow C=C_1 \Rightarrow C_2 \subseteq C_1 \subseteq D$. \\
$C_2$ Komponente von $D \Rightarrow C_1 = C_2$.
\item $\sigma := \{ A \subseteq D: A\text{ ist zsh., } z_0 \in A \}$. $z_0 \in \bigcap_{A \in \sigma}A \stackrel{\mbox{HS}}{\Rightarrow} C:= \bigcup_{A \in \sigma} A$ zsh. Sei $C \subseteq C_1 \subseteq D$ und $C_1$ zsh. Dann: $C_1 \in \sigma \Rightarrow C_1 \subseteq C \Rightarrow C_1 = C$.
\item "Ubung.
\end{liste}
\end{beweis}

\begin{definition}
Sei $\gamma$ ein st"uckweise glatter und geschlossener Weg in $\MdC$ und es sei $z \notin \Tr(\gamma)$. $n(\gamma,z):= \frac1{2\pi i} \wegint \frac{dw}{w-z}$ hei"st die \begriff{Umlaufzahl} von $\gamma$ bez"uglich $z$. $n(\gamma^-,z) = -n(\gamma,z)$.
\end{definition}

%Satz 16.2
\begin{satz}
Sei $\gamma$ wie oben und $D:= \MdC \backslash \Tr(\gamma)$.
\begin{liste}
\item $n(\gamma,z) \in \MdZ \ \forall z \in D.$
\item Ist $C$ eine Komponente von $D$, so ist $z \mapsto n(\gamma,z)$ auf $C$ konstant.
\item Ist $C$ die unbeschr"ankte Komponente von $D$, so gilt: $n(\gamma,z) = 0 \ \forall z \in C.$
\end{liste}
\end{satz}

\begin{beispiele}
\item Sei $k \in \MdZ \backslash \{ 0 \}, r>0, z_0 \in \MdC$ und $\gamma(t) := z_0 + re^{ikt} \ (t \in [0,2\pi]). C_1 = U_r(z_0); C_2 = \MdC \backslash \overline{U_r(z_0)}$ und die Komponente von $\MdC \backslash \Tr(\gamma)$. Sei $z \in C_2.$ 16.2(3) $\Rightarrow n(\gamma,z)=0.$ Sei $z \in C_1. n(\gamma,z) \stackrel{16.2(2)}{=} n(\gamma,z_0) = \frac1{2\pi i} \int_0^{2\pi} \frac1{re^{ikt}}ikre^{ikt}dt = k$.
\item Sei
\[
\gamma(t) := \left\{
\begin{array}{cl}
t &, -2 \leq t \leq 2\\
2e^{i(t-2)} &, 2 < t \leq 2 + \pi
\end{array} \right.
\]
Berechne $n(\gamma,i)$. Sei $\gamma_1$ wie im Bild\footnote{$\gamma_1$ l"auft von $(2,0)$ aus nach $(-2,0)$ und dann den Halbkreis mit Radius $2$ und Mittelpunkt $(0,0)$ wieder zur"uck nach $(2,0)$}
und $\gamma_0(t) := 2e^{it} \ (t \in [0,2\pi])$.
\[
\underbrace{\frac1{2\pi i} \wegint \frac{dw}{w-i}}_{=n(\gamma,i)} + \underbrace{\frac1{2\pi i} \int\limits_{\gamma_1} \frac{dw}{w-i}}_{\stackrel{16.2(3)}{=}0} = \frac1{2\pi i} \int\limits_{\gamma_0} \frac{dw}{w-i} \stackrel{Bsp.1}{=} 1 \Rightarrow n(\gamma,i) = 1.
\]
\end{beispiele}
\begin{beweis}
\begin{liste}
\item O.B.d.A $\gamma$ glatt. $ \gamma : [a,b] \to \MdC$, $z \in D$ und $h:[a,b]
\to \MdC$ definiert durch $h(t) := \int\limits_{a}^{t}
\frac{\gamma'(s)}{\gamma(s)-z} ds$  
$\stackrel{\text{8.2}}{\Rightarrow}$ h ist auf $[a,b]$ differenzierbar und $h'(t) =
\frac{\gamma'(t)}{\gamma(t)-z}$.  \\ Sei $H(t) := e^{-h(t)}(\gamma(t)-z)).$
Nachrechnen: $H' = 0$ auf $[a,b]$. Also existiert ein $c \in \MdC$ mit $H(t) = c$
$\forall t \in [a,b]$. \\ $\stackrel{t = a}{\Rightarrow}c = 
e^{-h(a)}(\gamma(a)-z) = \gamma(a) -z$ \\
$\Rightarrow $ $e^{h(t)} = \frac{\gamma(t)-z}{\gamma(a)-z}$ $\forall t \in
[a,b]$ \\
$\stackrel{t = b}{\Rightarrow}$ $e^{h(b)} = \frac{\gamma(b)-z}{\gamma(a)-z}
\stackrel{\gamma \text{ geschlossen}}{=} 1$
\\ $\stackrel{\text{6.3}}{\Rightarrow} \exists k \in \MdZ: h(b) = 2k \pi \ie$ \\
$\Rightarrow$ $2k \pi \ie = h(b) = \int\limits_{a}^{b}
\frac{\gamma'(s)}{\gamma(s)-z} ds$ $= \int\limits_{\gamma}
\frac{1}{w-z} dw = 2 \pi \ie $ $n(\gamma,z)$ $\Rightarrow$ $ k = n(\gamma, z)$
\item Definiere $f: C \to \MdC$ durch $f(z) = n(\gamma, z) = \frac{1}{2 \pi \ie
} \int\limits_{\gamma}\frac{dw}{w-z}$ $\stackrel{9.5}{\Rightarrow}$ $f \in H(C).
$ $C$ ist ein Gebiet. $\stackrel{\text{11.5}}{\Rightarrow} f(C)$ ist ein Gebiet oder $f$
ist auf $C$ konstant. $\stackrel{\text{(1)}}{\Rightarrow} f(C) \subseteq \MdZ.$
Also ist $f$ auf $C$ konstant. 
\item Sei $f$ wie im Beweis von (2). Wähle $R > 0$, so daß Tr$(\gamma) \subseteq
U_R(0).$ $\stackrel{\text{(2)}}{\Rightarrow}$ $\exists c \in \MdC: f(z) = c$
$\forall z \in C$. Sei $z \in C$, so daß $|z| > 2R$ (geht, da C
unbeschränkt). \\
Für $w \in \text{Tr}(\gamma)$ gilt: $|w-z| \geq |z|-|w| > |z| -R > R > 0$. \\
Damit: $|c| = |f(z)| = \frac{1}{2 \pi} |\int\limits_{\gamma} \frac{dw}{w-z}|
\leq \frac{L(\gamma)}{2 \pi (|z| -R)}$. \\
Also: $|c| \leq \frac{L(\gamma)}{2 \pi (|z| -R)}$ $\forall z \in C$ mit $|z| >
2R$. $C$ unbeschränkt $\stackrel{R \to \infty}{\Rightarrow}$  Behauptung.
\end{liste}
\end{beweis}


\chapter{Der Residuensatz und Folgerungen} 

%17.1
\begin{satz}[Residuensatz]
  $G$ sei ein Elementargebiet, es seien $z_1, \ldots, z_k \in G$ ($z_j \neq z_l$ f\"ur
   $j \neq l$) und es sei $f \in H(G \backslash \{z_1, \ldots, z_k\})$. Jedes $z_j$
  ist also eine isolierte Singularit\"at von $f$. Weiter sei $\gamma$ ein
  geschlossener, st\"uckweise glatter Weg mit Tr$(\gamma) \subseteq G \backslash
  \{z_1, \ldots, z_k\}$. \\
  Dann: \\ \\
  \centerline{$\frac{1}{2 \pi \ie}$ 
  $ \int\limits_{\gamma} f(z) dz $
  $= \sum\limits_{j=1}^k n(\gamma, z_j) Res(f, z_j)$ }
\end{satz}
\begin{beweis}
  $\forall j \in \{1 \ldots k\}$ existiert ein $R_j > 0$: 
  $\overline{U_{R_j}(z_j)} \subseteq G$  
  und $\overline{U_{R_j}(z_j)} \cap \overline{U_{R_l}(z_l)} = \emptyset$ ($j \neq l$).
  Sei $j \in \{1\ldots k\}$. \\
  $\stackrel{\text{14.4}}{\Rightarrow} $ $f$ hat auf $U_{R_j}(z_j)$ die 
  Laurententwicklung \\ \\ \centerline{$f(z) = \sum\limits_{n=0}^{\infty}
  a_n^{(j)}(z-z_j)^n + 
  \underbrace{\sum\limits_{n=1}^{\infty}
  a_{-n}^{(j)}(z-z_j)^{-n}}_{\varphi_j(z)}$}, wobei $\varphi_j \in H(\MdC
  \backslash \{z_j \})$ \\ 
  Definiere $g \in H(G \backslash \{z_1,\ldots, z_k\})$ durch $g(z) = f(z) -
  \sum\limits_{j=1}^{k} \varphi_j(z)$. \\
  Dann hat $g$ in $z_j$ eine hebbare Singularit\"at$(j = 1\ldots k)$. Also $g \in
  H(G)$. $G$ ist ein Elementargebiet $\Rightarrow$ $g$ hat eine Stammfunktion auf
  $G$ $\stackrel{\text{8.6}}{\Rightarrow}$ $\int\limits_{\gamma} g(z)  dz = 0$
  $\Rightarrow$ $\wegint f(z) dz = \sum\limits_{j=1}^k \wegint \varphi_j(z) dz$.\\
  Noch zu zeigen: $\wegint  \varphi_j(z) dz = 2 \pi \ie $ $ n(\gamma, z_j)
  a_{-1}^{(j)}$ $(j = 1\ldots k)$. \\
  Die Reihe f\"ur $\varphi_j$ konvergiert lokal gleichm\"a{\ss}ig (14.3). \\
  $\stackrel{\text{8.4}}{\Rightarrow} \wegint \varphi_j(z) dz = \sum\limits_{n=1}^{\infty}
  a_{-n}^{(j)} \wegint (z-z_j)^{-n} dz$. Sei $n \in \{2,3,4, \ldots\}$. Die
  Funktion $\frac{1}{(z-z_j)^n}$ hat auf $G \backslash \{z_j\}$ die Stammfunktion
  $\frac{(z-z_j)^{-n+1}}{-n+1}$ \\ $\stackrel{8.6}{\Rightarrow}$ $\wegint
  (z-z_j)^{-n} dz = 0$ $\forall n \in \{2,3,4,\ldots\}$ \\ $\Rightarrow$ $\wegint
  \varphi_j dz = a_{-1}^{(j)} \wegint \frac{1}{(z-z_j)} dz = a_{-1}^{(j)}$
  $n(\gamma, z_j) 2 \pi \ie$
\end{beweis}

%17.2 Folgerung
\begin{folgerung} 
  $G \subseteq \MdC$ sei ein Elementargebiet, es sei $f \in H(G)$ und $\gamma$ sei
  ein geschlossener, st\"uckweise glatter Weg mit $\Tr(\gamma) \subseteq G$.\\
  Dann: 
  \begin{liste}
    \item \begriff{Cauchyscher Integralsatz f\"ur Elementargebiete} \\ \\
    \centerline{$\wegint f(z) dz = 0$}
    \item \begriff{Cauchysche Integralformel} \\ \\
    \centerline{$n(\gamma, z) f(z) = \frac{1}{2 \pi \ie} \wegint \frac{f(w)}{w-z}
    dw$ $\forall z \in G \backslash \text{Tr}(\gamma)$}
  \end{liste}
\end{folgerung}

\begin{beweis}
  \begin{liste}
    \item Alle $z_j$ in 17.1 sind hebbare Singularit\"aten.
    $\stackrel{\text{14.4}}{\Rightarrow}$ $Res(f(z_j)) = 0$ $\Rightarrow$
    Behauptung. 
    \item Sei $z_0 \in G \backslash \text{Tr}(\gamma)$. $g \in H(G \backslash
    \{z_0\})$ sei definiert durch $g(w) := \frac{f(w)}{w-z_0}$. Sei $r > 0$, so dass
    $U_r(z_0) \subseteq G$\\ $\stackrel{10.4}{\Rightarrow}$ $f(w) = a_0+a_1(w-z_0) +
    \ldots$ $\forall w \in U_r(z_0)$ \\ $\Rightarrow $ $g(w) = \frac{a_0}{w-z_0} +
    a_1 + a_2(w-z_0) + \ldots $ $\forall w \in \dot{U}_r(z_0)$ \\
    $\Rightarrow $ $Res(g, z_0) = a_0 = f(z_0)$ \\ 
    $\Rightarrow$ $\frac{1}{2 \pi \ie} \wegint \frac{f(w)}{w-z} dw = \frac{1}{2 \pi \ie
    } \wegint g(w) dw \stackrel{\text{17.1}}{=} n(\gamma, z_0) f(z_0)$
  \end{liste}
\end{beweis}

\underline{F\"ur die Berechnung von Residuen an Polstellen}

%17.3 Lemma
\begin{satz}
  Sei $D \subseteq \MdC$ offen, $z_0 \in D$, $f \in H(D \backslash \{z_0\})$ und
  $f$ habe in $z_0$ einen Pol der Ordnung $m \geq 1$. \\
  Es existiert also (siehe 13.2) ein $g \in H(D)$ mit: \\
  \centerline{$f(z) = \frac{g(z)}{(z-z_0)^m }$ $\forall z \in D \backslash \{z_0\}$} und
  $g(z_0) \neq 0.$ Dann: \\
  \begin{liste}
    \item Res$(f,z_0) = \frac{g^{(m-1)}(z_0)}{(m-1)!}$
    \item Ist $m=1$, so ist Res$(f, z_0) = \lim\limits_{z \to z_0} (z-z_0) f(z)$
  \end{liste}
\end{satz}
\begin{beweis}
  \begin{liste}
    \item Sei $r > 0$ so, dass $U_r(z_0) \subseteq D$. \\ 
    $\stackrel{10.4}{\Rightarrow}$ $g(z) = b_0 + b_1(z-z_0) + \ldots + b_m(z-z_0)^m
    + \ldots$ $\forall z \in U_r(z_0)$ \\
    $\Rightarrow$ $f(z) = \frac{b_0}{(z-z_0)^m} + \ldots + \frac{b_{m-1}}{(z-z_0)} + b_m +
    b_{m+1}(z-z_0) + \ldots$ $\forall z \in \dot{U}_r(z_0)$ $\Rightarrow$ Res$(f, z_0) =
    b_{m-1} \stackrel{\text{10.4}}{=} \frac{g^{(m-1)}(z_0)}{(m-1)!}$
    \item Aus (1) folgt: Res$(f, z_0) = g(z_0) = \lim\limits_{z\to z_0} g(z) =
    \lim\limits_{z\to z_0} (z-z_0) f(z)$
  \end{liste}
\end{beweis}
\begin{beispiel}
  \begin{liste}
    \item 
    $f(z) = \frac{1}{(z-i)(z+1)}$ hat in $z = i$ und in $z = -1$ jeweils einen
    Pol der Ordnung 1. Also: Res$(f, i) = \lim\limits_{z\to i} (z-i) f(z) =
    \frac{1}{i+1} = \frac{1}{2}- \ie \frac{1}{2}$; Res$(f, -1) = -\frac{1}{2}+ \ie
    \frac{1}{2}$ 
    \item $f(z) = \frac{1}{(z-i)^3 z}$ hat in $z = i$ einen Pol der Ordnung 3 und in
    $z = 0$ eine Pol der Ordnung 1. Hier ist $g(z) = \frac{1}{z}.$ $g'(z) = -
     \frac{1}{z^2}, g''(z) = \frac{2}{z^3}$ $\Rightarrow$ Res$(f,i) = \frac{2}{i^3
    2!} = i$
  \end{liste}
\end{beispiel}

%17.4 Satz
\begin{satz} [Das Argumentenprinzip]
  $G \subseteq \MdC$ sei ein Elementargebiet, es sei $f \in M(G)$, $f$ habe in $G$ genau die
  Pole $b_1, \ldots, b_m$ (jeder Pol sei so oft aufgef\"uhrt, wie seine Ordnung
  angibt), $f$ habe in $G$ genau die Nullstellen $a_1, \ldots , a_n$ (jede
  Nullstelle sei so oft aufgef\"uhrt, wie ihre Ordnung angibt) und $\gamma$ sei ein
  st\"uckweise glatter und geschlossener Weg mit Tr$(\gamma) \subseteq G \backslash
  \{b_1, \ldots, b_m, a_1, \ldots, a_n\}$. Dann: \\ \\
  \centerline{$\frac{1}{2 \pi \ie} \wegint \frac{f'(z)}{f(z)} dz =
  \sum\limits_{j=1}^{n} n(\gamma, a_j) - \sum\limits_{j=1}^{m} n(\gamma, b_j)$}
\end{satz}

\begin{bemerkung}
  \begin{liste}
    \item in 17.4 ist $\{b_1,\ldots, b_m\} = \emptyset$ oder $\{a_1,\ldots, a_n\} =
    \emptyset$ zugelassen. I.d.Fall: $ \sum\limits_{j=1}^{m} n(\gamma, b_j) = 0$ oder
    $\sum\limits_{j=1}^{n} n(\gamma, a_j) = 0$
    \item $n(\gamma, a_j) =  n(\gamma, b_k)$ $(j=1,\dots,n$, $k = 1,\dots,m)$. Dann: \\
    $\frac{1}{2 \pi \ie} \wegint \frac{f'(z)}{f(z)} dz = $ Anzahl der Nullstellen
    von $f$ $-$ Anzahl der Polstellen von $f$ (jeweils gez\"ahlt mit Vielfachheiten!)
  \end{liste}
\end{bemerkung}

\begin{beispiel}
  $f(z) = \frac{z}{(z-i)^2}$ $n = 1, a_n=0, m = 2, b_1 = b_2 = i; \gamma(t) = 2
  e^{it}$ $t \in [0,2 \pi]$. $\frac{1}{2 \pi \ie} \wegint \frac{f'(z)}{f(z)} dz = 1
  -2 = -1 $
\end{beispiel}
\begin{beweis} (von 17.4)
  Sei $\beta_1, \ldots, \beta_p$ die paarweise verschiedenen Pole von $f$ $( p
  \leq m)$ und $\alpha_1, \ldots , \alpha_q$ die paarweise verschiedenen
  Nullstellen $(q \leq n)$. \\
  $h := \frac{f'}{f}$. \\ Dann: $h \in H( G \backslash \{\alpha_1, \ldots ,
  \alpha_q,\beta_1, \ldots, \beta_p\})$. \\ Dann: \\
  $\frac{1}{2 \pi \ie} \wegint \frac{f'(z)}{f(z)} dz = \frac{1}{2 \pi \ie}
  \wegint h(z) dz$ $\stackrel{\text{17.1}}{=} \sum\limits_{j=1}^{q} n(\gamma,
  \alpha_j) \text{Res}(n, \alpha_j) + \sum\limits_{j=1}^{p} n(\gamma,
  \beta_j) \text{Res}(n,\beta_j)$. \\
  Sei $\alpha \in \{\alpha_1, \ldots , \alpha_q\}$, $\beta \in \{\beta_1, \ldots,
  \beta_p\}$, $\nu = $ Ordnung der Nullstelle von $\alpha$ von $f$ und $\mu = $
  Ordnung der Polstelle $\beta$ von $f$. \\ \\
  Zu zeigen: Res$(h, \alpha) = \nu$ und Res$(h, \beta) = -\mu$. \\
  $\stackrel{11.8}{=}$ $\exists \delta > 0: U_{\delta}(\alpha) \subseteq G$,
  $\exists \varphi \in H(U_{\delta}(\alpha))$ und $f(z) =
  (z-\alpha)^{\nu}\varphi(z)$ $\forall z \in U_{\delta}(\alpha)$ und $\varphi(z)
  \neq 0$ $\forall z \in U_{\delta}(\alpha)$. \\ \\
  Dann: $f'(z)=\nu (z-\alpha)^{\nu -1}\varphi(z) + (z-\alpha)^{\nu}\varphi'(z)$
  $\forall z \in U_{\delta}(\alpha)$\\ $\Rightarrow$ $h(z) = \frac{f'(z)}{f(z)} =
  \frac{\nu}{z - \alpha}+ \underbrace{\frac{\varphi'(z)}{\varphi(z)}}_{\text{holomorph
  auf} \ U_{\delta}(\alpha)} $ $\forall z \in U_{\delta}(\alpha)$ $\Rightarrow$
  Res$(h, \alpha) = -\nu$. \\
  Analog: Res$(h, \beta) = \mu$ (statt 11.8 nimmt man 13.2)
\end{beweis}

%17.5 Folgerungen
\begin{folgerungen}
  Sei $G \subseteq \MdC$ ein Gebiet, $z_0 \in G$, $r > 0$, $\overline{U_r(z_0)}
  \subseteq G$, $\gamma(t) = z_0 + r e^{\ie t}$ $(t \in [0, 2 \pi])$ und $f,g \in
  H(G)$. Sei $N_f := $ Anzahl der Nullstellen von $f$ in $U_r(z_0)$ (gez\"ahlt mit
  Vielfachheiten!).  
  \begin{liste}
    \item Ist $f(z) \neq 0$ $\forall z \in$ Tr$(\gamma)$ $\Rightarrow$
     $N_f = \frac{1}{2 \pi \ie} \wegint \frac{f'(z)}{f(z)} dz $ 
    \item \textbf{Satz von Rouch\'{e}} \\
     Gilt (*) $|g(z)- f(z)| < |f(z)|$ $\forall z \in \text{Tr}(\gamma)$, so gilt 
     $N_f = N_g$
  \end{liste} 
\end{folgerungen}
\begin{beweis}
  \begin{liste}
    \item $\exists R > r: \overline{U_r(z_0)} \subseteq \overline{U_R(z_0)}
    \subseteq G$. Also: $\overline{U_r(z_0)} \subseteq U_R(z_0)$. $U_R(z_0)$ ist ein
    Elementargebiet. Seien $a_1, \ldots, a_n$ die Nullstellen von $f$ in $U_R(z_0)$.
    (gez\"ahlt mit Vielfachheiten).\\
    $\stackrel{\text{17.4}}{\Rightarrow}$ $\frac{1}{2 \pi \ie} \wegint
    \frac{f'(z)}{f(z)} dz = \sum\limits_{j=1}^n \underbrace{n(\gamma, a_j)}_{
    \stackrel{\text{16.2}}{=}
    \begin{cases} 
    	1 & , a_j \in U_r(z_0) \\ 
	0 & ,a_j \not\in U_r(z_0)
    \end{cases}
    }$
    \item F\"ur $s \in [0,1]: h_s := f + s(g-f) \in H(G)$; $N(s) := N_{h_s}$. Aus (*)
    folgt $h_s(z) \neq 0$ $ \forall s \in [0,1]$ $\forall z \in Tr(\gamma)$. \\
    Aus (1): $N(s)= \frac{1}{2 \pi \ie} \wegint \frac{h_s'(z)}{h_s(z)} dz =  
    \frac{1}{2 \pi \ie} \wegint \frac{f'(z)+s(g'(z)-f'(z))}{f(z) + s(g(z)-f(z))} dz$
    \\ $\Rightarrow$ die Funtion $s \mapsto N(s)$ ist stetig. Wegen $N(s) \subseteq
    \MdN_0$ $\forall s \in [0,1]$: $N(s)$ ist konstant. Also $N_f = N(0) = N(1) = N_g$
  \end{liste}
\end{beweis}

%17.6
\begin{satz} [Satz  von Hurwitz]
  $G \subseteq \MdC$ sei ein Gebiet. $(f_n)$ sei eine Folge in $H(G)$ und $(f_n)$
  konvergiert auf G lokal gleichm\"a{\ss}ig gegen eine Funktion $f: G \to \MdC$.
  ($\stackrel{\text{10.5}}{\Rightarrow}$ $f \in H(G)$). \\ Dann:
  \begin{liste}
    \item Ist $Z(f_n) = \emptyset$ $\forall n \in \MdN$ $\Rightarrow$ $Z(f) =
    \emptyset$ oder $f \equiv 0$
    \item Sind alle $f_n$ auf $G$ injektiv $\Rightarrow$ $f$ ist auf $G$ injektiv
    oder $f$ ist auf $G$ konstant.
  \end{liste}
\end{satz}

\begin{beweis}
  \begin{liste}
    \item Sei $f \not \equiv 0$ auf $G$; $z_0 \in G$, $r > 0$ so, dass
    $\overline{U_r(z_0)} \subseteq G$ und $f(z) \neq 0$ $\forall z \in
    \overline{U_r(z_0)} \backslash \{z_0\} $. \\
    $\gamma(t) = z_0 + re^{\ie t}$ $(t \in [0, 2 \pi])$. $(f_n)$ konvergiert auf
    Tr$(\gamma)$ gleichm\"a{\ss}ig gegen $f$. $\stackrel{\text{10.5}}{\Rightarrow}$
    $(f_n')$ konvergiert auf Tr$(\gamma)$ gleichm\"a{\ss}ig gegen $f'$. \\
     \"Ubung: $(\frac{1}{f_n})$ konvergiert auf Tr$(\gamma)$ gleichm\"a{\ss}ig gegen
    $\frac{1}{f}$. \\
    Fazit: $(\frac{f_n'}{f_n})$ konvergiert auf Tr$(\gamma)$ gleichm\"a{\ss}ig gegen
    $(\frac{f'}{f})$. \\ \\
    \centerline{$\underbrace{\frac{1}{2 \pi \ie} \wegint \frac{f_n'}{f_n} dz}_{\stackrel{\text{17.5}}{N_{f_n}=0}} \to 
    \underbrace{\frac{1}{2 \pi \ie} \wegint \frac{f'}{f}
    dz}_{\stackrel{\text{17.5}}{N_{f}}} $} \\ Also: $N_f = 0$. Somit: $f(z_0) \neq
    0$ 
    \item Sei $z_0 \in G$. $g_n = f_n-f_n(z_0)$, $g := f - f(z_0)$. $\widetilde{G}:=G\backslash \{z_0\}$. Dann:\\
    $(g_n)$ konvergiert auf $\widetilde{G}$ lokal gleichm\"a{\ss}ig gegen g. $g_n(z)\neq 0 \forall z \in \widetilde{G}$\\
    (1) $\Rightarrow$ $g \equiv 0$ oder $g(z) \neq 0$ $\forall z \in
    \widetilde{G}$ $\Rightarrow$ $f$ ist auf $G$ konstant oder $f(z) \neq f(z_0)$
    $\forall z  \in G \backslash \{z_0\}$
   \end{liste}
\end{beweis}

\underline{Berechnung von Integralen}
%17.7 Satz
\begin{satz}
  Sei $R(x,y)= R(x+iy) = R(z)$ eine rationale Funktion ohne Pole auf $\partial\mathbb{D}$. Weiter sei 
  $R_1(z)=\frac{1}{iz}R(\frac{z+\frac{1}{z}}{2},\frac{z-\frac{1}{z}}{2i})$ und $M:=\{z\in\mathbb{D}:z$ ist ein Pol von $R_1\}$ (endlich)\\
  Dann:
  $$ \int\limits_0^{2\pi}R(\cos{t},\sin{t})dt=2\pi i \sum\limits_{z\in M} Res(R_1,z)$$
\end{satz}
\begin{beweis}

  $\iint_0^{2\pi}R(\cos{t},\sin{t})dt = \iint_0^{2\pi}\frac{1}{ie^{it}} R(\frac{e^{it}+e^{-it}}{2},\frac{e^{it}-e^{-it}}{2i}) ie^{it}dt$\\
  $= \wegint R_1(z)dz$, wobei $\gamma(t)=ie^{it}$ $(t\in[0,2\pi])$.\\
  Also: $\iint_0^{2\pi}R(\cos{t},\sin{t})dt = \wegint R_1(z)dz \stackrel{17.1}{=}2\pi i \sum\limits_{\text{z Pol von $R_1$}} \underbrace{n(\gamma, z)}_{
    =
     \begin{cases} 
       1 & , z \in M \\ 
       0 & ,z \not\in M
     \end{cases}
  }  Res(R_1,z)$.
\end{beweis}
%17.8 Satz
\begin{satz}
  $Z$ und $N$ seien Polynome. $R:=\frac{Z}{N}$ habe auf $\MdR$ keine Pole und es gelte (*) grad $N$ $\ge$ grad $Z+2$ ($\folgt \iint_\MdR R(x)dx$ konvergiert absolut). 
  Weiter sei $M:=\{z\in\MdC: \Im z >0\text{, } z\text{ ist Pol von }R\}$. Dann:\\
  $\iint_{-\infty}^\infty R(x) dx = 2\pi i \sum\limits_{z\in M} Res(R,z)$
\end{satz}
\begin{beweis}
  (*) $\folgt \exists m \ge 0 \exists c>0: |R(z)| \le \frac{m}{|z|^2} \forall z\in\MdC$ mit $|z|>c$. (**)\\
  Sei $\delta > c$ so gross, dass alle Pole von R in $U_\delta(0)$ liegen.\\
  $\gamma_1(t) := t$ $(t\in[-\delta,\delta])$; $\gamma_2(t):=\delta e^{it}$ $(t\in[0,\pi])$ $\gamma := \gamma_1 \oplus \gamma_2$.\\
  $\wegint R(z)dz = \iint_{\gamma_1}R(z)dz+\iint_{\gamma_2}R(z)dz.$ \\
  $\iint_{\gamma_1}R(z)dz = \iint_{-\delta}^\delta \to \iint_{-\infty}^\infty R(t)dt$ $(\delta\to\infty)$.\\
  Sei $z\in\Tr(\gamma_2)$. Dann: $|z| = \delta>0$, also nach (**): $|R(z)| \le \frac{m}{|z|^2} = \frac{m}{\delta^2} \folgt |\iint_{\gamma_2}R(z)dz| \le \frac{m}{|z|^2}L(\gamma_2)\le\frac{m\pi\delta}{\delta^2}=\frac{m\pi}{\delta}$\\
  $\folgt \iint_{\gamma_2}R(z)dz \to 0$ $(\delta\to\infty)$. Dann: $\wegint R(z)dz \to \iint_{-\infty}^\infty R(x) dx$ $(\delta \to \infty)$. 17.1 $\folgt \wegint R(z)dz = 2\pi i 
  \sum\limits_{\text{z Pol von R}} \underbrace{n(\gamma, z)}_{
      =
       \begin{cases} 
       1 & , z \in M \\ 
       0 & ,z \not\in M
       \end{cases}
  }  Res(R,z)$.
.				
\end{beweis}


\chapter{Der Satz von Montel} 

\begin{satz}[Satz von Montel]
  Sei $D\subseteq \MdC$ offen, $(f_n)$ eine Folge in $H(D)$ und es gelte mit einem $c \ge 0$: $|f_n(z)|\le c$ $\forall z\in D$ $\forall \natn$. (*)\\ 
  Dann enth\"alt $(f_n)$ eine auf D lokal gleichm\"a{\ss}ig konvergierende Teilfolge.
\end{satz}
\begin{beweis}
  Wegen (*) und des Satzes von Arzel\`{a}-Ascoli (Ana3) gen\"ugt es zu zeigen:\\
  $$ \text{Zu } \epsilon>0\text{ und } z_0 \in D\text{ existiert ein } \delta > 0: |f_n(z)-f_n(w)|<\epsilon\text{ }\forall\natn\text{ }\forall z,w\in U_\delta(z_0) $$ \\
  Sei $\epsilon > 0$ und $z_0 \in D$. $\exists r > 0: \overline{U_{2r}(z_0)} \subseteq D$\\
  $\gamma(t) := z_0+2re^{it}$ $(t\in [0,2\pi])$\\
  $\delta := \frac12 \min\{\frac{\epsilon r}{2c},2r\}$.\\
  Sei $\natn, z,w\in U_\delta(z_0).$ F\"ur $\lambda \in \Tr(\gamma)$: $|\lambda-z|,|\lambda-w| \ge r$ \\
  $\folgt \frac{|f_n(\lambda)|}{|\lambda-z||\lambda-w|}\le \frac{c}{r^2}$ \\
  Dann: $|f_n(z)-f_n(w)| \stackrel{\text{9.4}}{=} \frac1{2\pi} | \wegint \frac{f_n(\lambda)}{\lambda-z}-\frac{f_n(\lambda)}{\lambda-w} d\lambda |\\
  = \frac{|z-w|}{2\pi}|\wegint \frac{f_n(\lambda)}{(\lambda-z)(\lambda-w)} d\lambda | \le \frac{|z-w|}{2\pi} \frac{c}{r^2} 2\pi 2 r = \frac{2c}{r}|z-w| \\
  = \frac{2c}{r}|z-z_0+z_0-w| \stackrel{\Delta\text{-Ungl.}}{\le} \frac{2c}r (|z-z_0|+|w-z_0|) < \frac{2c}r 2\delta $\\
  
 $ \le \frac{2c}r \frac{\epsilon r}{2c} = \epsilon$.
  
\end{beweis}

\chapter{Der Riemannsche Abbildungssatz} 
\begin{definition}
  Zwei Gebiete $G_1$, $G_2\subseteq\MdC$ heissen \begriff{konform \"aquivalent} ($G_1 \sim G_2$) $:\equizu$ $\exists f\in H(G_1)$: $f(G_1)=G_2$, $f$ ist auf $G_1$ injektiv.\\

  ,,$\sim$'' ist eine \"Aquivalenzrelation auf der Menge der Gebiete in $\MdC$.
\end{definition}
% 19.1 Satz
\begin{satz}[Riemannscher Abbildungssatz]

  Sei  $G\subseteq\MdC$ ein Gebiet. \\
  Dann: $G \sim \mathbb{D}$ $\equizu$ $G\ne\MdC$ und $G$ ist ein Elementargebiet.
\end{satz}

\begin{beweis}
  ,,$\folgt$'':  \\
  10.2 (Satz von Liouville) $\folgt G\ne\MdC$\\
  11.13 $\folgt$ $G$ ist ein Elemenetargebiet.\\
  ''$\Longleftarrow$'': nach 19.5. \\
\end{beweis}
\begin{definition}
  Sei $G\subseteq\MdC$ ein Gebiet. $G$ hat die Eigenschaft (W) $:\equizu \forall f\in H(G)$ mit $Z(f) = \emptyset$ $\exists g\in H(G): g^2 = f$ auf G.
\end{definition}
\textbf{Beachte:} Elementargebiete haben die Eigenschaft (W) (siehe 11.4)

%19.2 Lemma
\begin{lemma}
  $G_1$, $G_2$ $\subseteq \MdC$ seien Gebiete, es gelte $G_1\sim G_2$ und $G_1$ habe die Eigenschaft (W). 
  Dann: $G_2$ hat die Eigenschaft (W).
\end{lemma}
\begin{beweis}
  \"Ubung.
\end{beweis}
% 19.3 Lemma
\begin{lemma}
  $G\subseteq\MdC$ sei ein Gebiet mit der Eigenschaft (W) und es sei $G\ne\MdC$. Dann existiert ein Gebiet $G^*$:\\
  $$ 0\in G^*\subseteq \mathbb{D} \text{ und } G \sim G^* \text{ ($G^*$ hat also die Eigenschaft (W))}$$
\end{lemma}
\begin{beweis}
  $G \ne \MdC$ $\folgt$ $\exists c\in \MdC: c\not\in G$. Dann: $f(z)=z-c$ hat keine Nullstelle in $G$. ($f\in H(G)$)\\
  (W) $\folgt$ $\exists g\in H(G)$: $g^2 = f$ auf G. F\"ur $z_1$, $z_2 \in G$:\\
  (+) aus $g(z_1)=\pm g(z_2)$ folgt $f(z_1)=f(z_2)$, also $z_1 = z_2$.\\
  Insbesondere: $g$ ist injektiv auf $G$. $G_1:=g(G)$. Also $G_1 \sim G$.\\
  Sei $a \in G_1$. $\exists r>0: U_r(a) \in G_1$. \\
  
  Sei $\omega \in G_1$.\\
  Annahme: $-\omega \in G_1$. \\
  $\exists z_1$, $z_2$ $\in G: g(z_1)=\omega=-g(z_2)$. (+) $\folgt z_1=z_2\folgt \omega=0\folgt g(z_1)^2=0\folgt f(z_1)=0$. Widerspruch.\\
  Also: $-\omega \not\in G_1$\\

  Insbesondere: $0\not\in G_1, -a\not\in G_1$.\\
  Definiere $\varphi \in H(G_1)$ durch $\varphi(w)=\frac1{w+a}$. (Wohl definiert und holomorph)\\
  \"Ubung: $\varphi$ injektiv.\\
  $G_2:=\varphi(G_1) \folgt G_2 \sim G_1$, also: $G\sim G_2$.\\
  Sei $\nu \in G_2$ $\folgt$ $\exists \omega \in G_1$: $\nu=\varphi(\omega)=\frac1{\omega+a}$.\\

  Annahme: $|\omega+a|<r$. Dann: $|-\omega-a|<r$ $\folgt$ $-\omega \in U_r(a) \subseteq G_1$. Widerspruch.\\
  Also: $|\omega+a| \ge r$.\\
  
  \folgt $|r| \le \frac1r$. $G_2$ also beschr\"ankt.\\
  Mit einer Abbildung $z\mapsto z+\alpha$: (Translation)\\ 
  $\exists$ Gebiet $ G_3$: $G_2 \sim G_3$, $0\in G_3$, $G_3$ beschr\"ankt. Somit: $G \sim G_3$.\\
  Mit einer geeigneten Abbildung $z\mapsto \delta z$ ($\delta>0$): $\exists$ Gebiet $G^*$: $G^* \sim G_3, 0\in G^*, G^* \subseteq \mathbb{D}$. \\
  Somit $G\sim G^*$.
\end{beweis}
\begin{lemma}
  Sei $G\subseteq \MdC$ ein Gebiet mit der Eigenschaft (W). Es gelte $0\in G\subseteq\MdD$ und es sei $G\ne\MdD$. \\
  Dann existiert $\varphi \in H(G)$: $\varphi(0)=0$, $\varphi$ ist auf G injektiv. $\varphi(G)\subseteq\MdD$\footnote{schwer zu entziffern.. wirklich teilmenge D?} und $|\varphi'(0)|>1$.
\end{lemma}
\begin{beweis}
   $G\ne\MdD$ $\folgt$ $\exists a\in \MdD: a\not\in G$. $f(z):=\frac{z-a}{\overline{a}z-1}$.\\
   $f\in H(G)$, 12.4 $\folgt f\in Aut(\MdD)$. $a\not\in G$. $f(a)=0\text{ (einzige Nullstelle)}$. f hat in $G$ keine Nullstelle. \\ 
   (W) $\folgt$ $\exists g \in H(G)$: $g^2 = f$ auf $G$.$|g|^2 = |f| \stackrel{\text{12.4}}{<} 1$, also $|g| < 1$ auf $G$. D.h.: $g(G)\subseteq\MdD$. 
   Dann: $c=g(0) \in \MdD$. $h(z):=\frac{z-c}{\overline{c}z-1}$, $\varphi:= h\circ g$. Klar: $\varphi \in H(G)$, $\varphi(0)=h(g(0))=h(c)=0$, $\varphi$ ist injektiv auf $G$, $\varphi(G) = h(\underbrace{g(G)}_{\subseteq \MdD}) \subseteq h(\MdD) \stackrel{=}{12.4} \MdD.$ Nachrechnen: $| \varphi'(0) |= \frac{|a|+1}{2\sqrt{|a|}} > 1.$
\end{beweis}


%Vorlesung 19.07.06
%19.5 Lemma
\begin{lemma}
  Sei $G\subseteq\MdC$ ein Gebiet mit der Eigenschft (W). Es gelte $0 \in G \subseteq \MdD$ und $\mathcal{F}:=\{\varphi\in H(G)$: $\varphi(0)=0$, $\varphi$ ist injektiv auf $G$ und $\varphi(G)\subseteq\MdD\}$.\\
  Weiter sei $\Psi \in \mathcal{F}$ und es gelte (*) $|\varphi'(0)|\le|\Psi'(0)|$ $\forall\varphi\in\mathcal{F}$. Dann: $\varphi(G)=\MdD$. Insbesondere $G \sim \MdD$.
\end{lemma}
\begin{beweis}
  $\Gs:=\Psi(G)$. 19.2 $\folgt$ $\Gs$ hat die Eigenschaft (W). Weiter: $0=\Psi(0)\in\Gs\subseteq\MdD$. \\

  Annahme: $Gs\ne\MdD$. Wende 19.4 auf $Gs$ an: $\exists \phis\in H(\Gs)$: $\phis(0)=0$, $\phis$ ist injektiv, \\
  $\phis(\Gs)\subseteq\MdD$ und $|\phis'(0)|>1$. $\varphi:=\phis\circ\Psi$. Dann: $\varphi\in H(G)$, $\varphi(0)=\phis(\Psi(0))=\phis(0)=0$. 
  $\varphi $ ist auf $G$ injektiv, $\varphi(G)=\phis(\Psi(G))=\phis(\Gs)\subseteq\MdD$. Also $\varphi\in\mathcal{F}$. Aber: \\
  $|\varphi'(0)|=|\phis'(\Psi(0)\Psi'(0)|=\underbrace{|\phis'(0)|}_{>1} \underbrace{|\Psi'(0)|}_{\stackrel{11.11}{\ne} 0}>|\Psi'(0)|$, Widerspruch zu (*).
\end{beweis}

\begin{beweis}
  Beweis ,,$\Longleftarrow$'' von 19.1:\\
  Sei $G$ ein Elementargebiet und $G\ne\MdC$. 11.4 $\folgt$ $G$ hat die Eigenschaft (W).\\
  ObdA: $0\in G\subseteq \MdD$ (wg 19.3). Sei $\mathcal{F}$ wie in 19.5. $\phi_0(z):=z$. Dann: $\phi_0\in\mathcal{F}$. Wegen 19.5 gen\"ugt es zu zeigen: \\
  $$\exists\Psi\in\mathcal{F}:|\varphi(0)|\le|\Psi(0)|\text{ }\forall \varphi\in\mathcal{F}$$
  $s:=$. $\exists$ Folge $(\varphi_n)$  in $\mathcal{F}$: $|\varphi_n'(0)|\to s$. $\varphi_n(G)\subseteq\MdD$ $\forall\natn$ \\
  $\folgt |\varphi_n(z)|\le 1$ $\forall\natn$ $\forall z\in G$. Satz von Montel $\folgt$ $(\varphi_n)$ enth\"alt eine auf G lokal gleichm\"a{\ss}ig konvergierende Teilfolge. \\
  ObdA: $(\varphi_n)$ konvergiert auf G lokal gleichm\"a{\ss}ig. $\Psi(z):=\lim\limits_{n\to\infty}\varphi_n(z)$ $(z\in G)$. 10.5 $\folgt$ $\Psi \in H(G)$ 
  und $\varphi_n'(0) \to \Psi'(0)$. Also: $|\Psi'(0)|=s$. $\Psi(0)=\lim \varphi_n(0)=0$. Es ist $|\varphi'(0)|=1\le|\Psi'(0)|$.
  Insbesondere ist $\Psi$ auf $G$ nicht konstant. $\varphi_n$ injektiv $\forall\natn$ $\stackrel{\text{17.6}}{\folgt}$ $\Psi$ it injektiv. $\varphi_n(G) \subseteq \MdD \forall\natn$
  $\folgt$ $|\Psi(z)|\le1\forall z\in G$ Annahme: $\exists z_0 \in G$: $|\Psi(z0)| = 1$. 11.6 $\folgt$ $\Psi$ konstant. Widerspruch! Also $\Psi(G)\subseteq\MdD$\\
  Fazit: $\Psi \in \mathcal{F}$ und $|\varphi'(0)|\le |\Psi'(0)| \forall \varphi\in\mathcal{F}$.
\end{beweis}
% 19.6 Satz
\begin{satz}[Charakterisierung von Elementargebieten, I]
  Sei $G\subseteq\MdC$ ein Gebiet.\\
  $$ G\text{ ist Elementargebiet } \equizu G \text{ hat die Eigenschaft (W)}$$
\end{satz}
\begin{beweis}
  ,,$\folgt$'': 11.4.\\
  ,,$\Longleftarrow$'':\\
  Fall 1: $G=\MdC$. $\surd$ \\
  Fall 2: $G\ne\MdC$. Im Beweisteil ,,$\Longleftarrow$'' von 19.1 wurde nur die Eigenschaft (W) benutzt. Also $G\sim\MdD$. $\MdD$ ist ein Elementargebiet $\stackrel{\text{11.13}}{\folgt} G$ ist ein Elementargebiet.
\end{beweis}

\chapter{Homotopie und einfacher Zusammenhang} 
%20.1 Lemma
\begin{lemma}
  Sei $\emptyset \ne K \subseteq D\subseteq\MdC$, $D$ offen und $K$ kompakt. Dann existiert ein $r>0$: $U_r(a)\subseteq D$ $\forall a\in K$.
\end{lemma}
\begin{beweis}
  $\forall a\in K$ $\exists r_a>0$: $U_{2r_a}(a) \subseteq D$. Dann: $K \subseteq \bigcup\limits_{a\in K} U_{r_a}(a)$. \\
  2.3 $\folgt$ $\exists a_1, \dots, a_n \in K$: $K\subseteq \bigcup\limits_{j=1}^n U_{r_{a_j}}(a_j)$.\\
  $r:= \min\{r_{a_1},\dots,r_{a_n}\}$. Sei $a\in K$ und $z\in U_r(a)$. \\
  Zu zeigen: $z\in D$.\\$\exists j\in\{1,\dots,n\}$: $a\in U_{r_{a_j}}(a_j)$. \\
  Dann: $|z-a_j| = |z-a+a-a_j| \stackrel{\Delta\text{-Ungl.}}{\le} |z-a|+|a-a_j| < r+r_{a_j} \le 2r_{a_j} \folgt$ $z\in U_{2r_{a_j}}(a_j) \subseteq D$
\end{beweis}
%20.2 Lemma
\begin{lemma}
  Sei $D\subseteq \MdC$ offen und $\gamma: \abnC$ ein Weg mit $\Tr(\gamma) \subseteq D$. ($\gamma$ also ,,nur'' stetig)\\
  Dann existiert ein $r>0$ und eine Zerlegung $Z=\{a_0,\dots,a_n\}$ von $[a,b]$ mit:
  \begin{liste}
    \item f\"ur $z_j:=\gamma(a_j)$ gilt: $U_r(z_j)\subseteq D$ ($j=0,\dots,n$)
    \item $\gamma([a_j,a_{j+1}])\subseteq U_r(z_j)\cap U_r(z_{j+1})$ ($j=0,\dots,n$)
  \end{liste}
\end{lemma}
\begin{beweis}
  20.1 $\folgt$ $\exists r>0$: $U_r(z) \subseteq D$ $\forall z\in K:=\Tr(\gamma)\folgt$ (1).\\
  OBdA: $[a,b] = [0,1]$. $\gamma$ ist auf $[0,1]$ gleichm\"a{\ss}ig stetig $\folgt$ $\exists \delta>0$: $|\gamma(s)-\gamma(t)|<r$ $\forall s,t\in [0,1]$ mit $|s-t|<\delta$.\\
  Sei $\natn$ so, da{\ss} $\frac1{n}<\delta$. $a_j:=\frac{j}{n}$ ($j=0,\dots,n$) und $Z:=\{a_0,\dots,a_n\}$. 
  Sei $t\in [a_j,a_{j+1}]$. $\folgt |t-a_j|<\delta$, $|t-a_{j+1}|<\delta$. $\folgt |\gamma(t)-\underbrace{\gamma(a_j)}_{=z_j}|<r$, $|\gamma(t)-\underbrace{\gamma(a_{j+1})}_{=z_{j+1}}|<r$ 
  $\folgt \gamma(t)\in U_r(z_j) \cap U_r(z_{j+1})$.
\end{beweis}

In  \S8 haben wir $\wegint f(z)dz$ definiert f\"ur $\gamma$ st\"uckweise glatt und $f \in C(\Tr(\gamma))$. Jetzt definieren wir $\wegint f(z)dz$ f\"ur $\gamma$ ,,nur'' stetig und $f$ holomorph.
\begin{definition}
  Sei $D\subseteq\MdC$ offen, $f\in H(D)$ und $\gamma$:$\abnC$ ein Weg mit $\Tr(\gamma) \subseteq D$. Seien $r$, $z_j$, $Z$ wie in 20.2. \\
  $z_0 = \gamma(a_0)=\gamma(a)$, $z_n = \gamma(a_n)=\gamma(b)$\\
  $\gamma_j(t):=z_j+t(z_{j+1}-z_j)$ ($t\in [0,1]$) ($j=0,\dots,n-1$).\\ $\Gamma:=\gamma_0\oplus\dots\oplus\gamma_{n-1}$ ist st\"uckweise glatt. 20.2 $\folgt$ Tr$(\Gamma) \subseteq D$. Setze\\
  $$\text{(+) } \wegint f(z)dz:=\iint_\Gamma f(z)dz$$\\
\end{definition}
%20.3 Lemma
\begin{lemma}
  Bezeichnungen wie in obiger Definition.
  \begin{liste}
    \item Ist $\gamma$ st\"uckweise glatt, so stimmt obige Definition (+) mit der Definition von $\wegint f(z)dz$ aus \S8 \"uberein.
    \item Die Definition (+) ist unabh\"angig von der Zerlegung $Z$, solange $Z$ die Eigenschaft aus 20.2 hat.
    \item $|\wegint f(z)dz|\le (\max\limits_{z\in\Tr(\Gamma)} |f(z)|) L(\Gamma)$.
  \end{liste}
\end{lemma}
\begin{beweis}
  \begin{liste}
  \item $\tilde{\gamma_j}:=\gamma_{|[a_j,a_{j+1}]}$. Dann: $\gamma=\tilde{\gamma_0}\oplus\tilde{\gamma_1}\oplus\dots\oplus\tilde{\gamma}_{n-1}$\\
    Sei $j\in \{0,\dots,n-1\}$: $\tilde{\gamma_j}\oplus\gamma_j^-$ ist ein geschlossener, st\"uckweise glatter Weg im Sterngebiet $U_r(z_j)$ (siehe 20.2).\\
    $\stackrel{\text{9.2}}{\folgt}$ $\iint_{\tilde{\gamma_j}\oplus\gamma_j^-} f(z)dz = 0 \folgt \iint_{\tilde{\gamma_j}} f(z)dz = \iint_{\gamma_j}f(z)dz$.\\
    $\stackrel{\text{Summation}}{\folgt} \wegint f(z)dz = \iint_\Gamma f(z)dz$. 
  \item \"Ubung. (Ist $\tilde{Z}$ eine weitere Zerlegung von $[a,b]$ mit den Eigenschaften aus 20.2, so betrachte die gemeinsame Verfeinerung $Z\cup\tilde{Z}$. Verfahre \"ahnlich wie in (1).)
  \item folgt aus 8.4
  \end{liste}  
\end{beweis}
\begin{definition}
$D\subseteq\MdC$ sei offen.
\begin{liste}
\item Seien $\gamma_0,\gamma_1:\ [0,1]\to\MdC$ Wege mit $\Tr(\gamma_0),\Tr(\gamma_1)\subseteq D$, $\gamma_0(0)=\gamma_1(0)$ und $\gamma_0(1)=\gamma_1(1)$.\\
$\gamma_0$ und $\gamma_1$ heißen \textbf{in D homotop} $:\Leftrightarrow \exists H:\ [0,1]^2\to\MdC$: H ist stetig, $H([0,1]^2)\subseteq D$ und
\[H(t,0)=\gamma_0(t),\ H(t,1)=\gamma_1(t)\quad\forall t\in[0,1]\]
\[H(0,s)=\gamma_0(0)=\gamma_1(0),\ H(1,s)=\gamma_0(1)=\gamma_1(1)\quad\forall s\in[0,1]\]
In diesem Fall heißt H eine \textbf{Homotopie von $\gamma_0$ nach $\gamma_1$ in D}.\\
~\\
\textbf{Anschaulich:} Sei $s\in[0,1]$.\\
$\Gamma_s(t):=H(t,s)\ (t\in[0,1])$, $\Gamma_s$ ist ein Weg mit $\Tr(\Gamma_s)\subseteq D$.
$\Gamma_s(0)=H(0,s)=\gamma_0(0)=\gamma_1(0), \Gamma_s(1)=H(1,s)=\gamma_0(1)=\gamma_1(1)$, $\Gamma_0=\gamma_0, \Gamma_1=\gamma_1$\\
"`$\gamma_0$ kann in D stetig nach $\gamma_1$ deformiert werden."'
\item $\gamma:\ [0,1]\to\MdC$ sei ein geschlossener Weg mit $\Tr(\gamma)\subseteq D$. $z_0:0\gamma(0)=\gamma(1)$.\\
$\gamma_{z_0}(t):=z_0\ (t\in[0,1])$ heißt ein \textbf{Punktweg}.\\
~\\
$\gamma$ heißt \textbf{nullhomotop in D} $:\Leftrightarrow \gamma$ und $\gamma_{z_0}$ sind in D homotop.\\
"`$\gamma$ lässt sich in D stetig auf einen Punkt zusammenziehen."'
\item $G\subseteq\MdC$ sei ein Gebiet. G heißt \textbf{einfach zusammenhängend} $:\Leftrightarrow$ jeder geschlossene Weg $\gamma:\ [0,1]\to\MdC$ mit $\Tr(\gamma)\subseteq G$ ist in G nullhomotop.\\
"`G hat keine Löcher."'
\end{liste}
\end{definition}
\begin{satz} 
Sei $G\subseteq\MdC$ ein konvexes Gebiet. 
\begin{liste}
\item Sind $\gamma_0,\gamma_1:\ [0,1]\to\MdC$ Wege mit $\gamma_0(0)=\gamma_1(0)$ und $\gamma_0(1)=\gamma_1(1)$ und $\Tr(\gamma_0),\Tr(\gamma_1)\subseteq G$, so sind $\gamma_0$ und $\gamma_1$ homotop in G.
\item G ist einfach zusammenhängend
\end{liste}
\end{satz}
\begin{beweis}
\begin{liste}
\item $H(s,t):=\gamma_0(t)+s(\gamma_1(t)-\gamma_0(t)), (s,t\in[0,1])$. H ist eine Homotopie von $\gamma_0$ nach $\gamma_1$ in G
\item folgt aus (1)
\end{liste}
\end{beweis}


\chapter{Cauchyscher Integralsatz (Homotopieversionen)} 

\begin{satz}[CIS, Version I]
Sei $D\subseteq\MdC$ offen, $f\in H(D)$ und $\gamma_0,\gamma_1:\ [0,1]\to\MdC$ seien Wege mit $\Tr(\gamma_0),\Tr(\gamma_1)\subseteq D$, $\gamma_0(0)=\gamma_1(0),\gamma_0(1)=\gamma_1(1)$.\\
Sind $\gamma_0$ und $\gamma_1$ in D homotop, so gilt:
\[\int_{\gamma_0}f(z)dz=\int_{\gamma_1}f(z)dz\]
\end{satz}
\begin{beweis}
Ohne Beweis\end{beweis}

\begin{satz}[CIS, Version II]
Sei $D\subseteq\MdC$ offen, $f\in H(D)$ und $\gamma$ sei ein geschlossener Weg mit Tr($\gamma$)
$\subseteq D$ . \\
Ist $\gamma$ nullhomotop in D, so gilt
\[\wegint f(z)dz=0\] 
\end{satz}
\begin{beweis}
21.1\end{beweis}

\begin{satz} [CIS, Version III]
$G\subseteq\MdC$ sei ein einfach zusammenhängendes Gebiet, es sei $f\in H(G)$ und $\gamma$ ein geschlossener Weg mit $\Tr(\gamma) \subseteq G$. Dann
\[\wegint f(z)dz=0\]
\end{satz}
\begin{beweis}
21.2\end{beweis}

\begin{satz} [Charakterisierung von Elementargebieten, II]
Sei G ein Gebiet in $\MdC$.\\
G ist ein Elementargebiet $\Leftrightarrow$ G ist einfach zusammenhängend
\end{satz}
\begin{beweis}
"`$\Rightarrow$"' Fall 1: $G=\MdC$ $\Rightarrow$ G konvex, also einfach zusammenhängend (siehe 20.4)\\
Fall 2: $G\neq\MdC$ $\stackrel{19.1}{\Rightarrow}\exists f\in H(G):\ f(G)=\MdD$ und f ist auf G injektiv.\\
Sei $\gamma$ ein geschlossener Weg mit $\Tr(\gamma)\subseteq G$. $z_o:=$Anfangspunkt von $\gamma$.\\
Zu zeigen: $\gamma$ und $\gamma_{z_0}$ sind in G homotop\\
~\\
$\Gamma:=f\circ\gamma$. $\Gamma$ ist ein geschlossener Weg in $\MdD$. $\MdD$ ist konvex $\stackrel{10.4}{\Rightarrow}\MdD$ ist einfach zusammenhängend.\\
Also existiert eine Homotopie $\tilde{H}$ von $\Gamma$ nach $\gamma_{f(z_0)}$ in $\MdD$. $H:=f^{-1}\circ\tilde{H}$ ist eine Homotopie von $\gamma$ nach $\gamma_{z_0}$ in G\\
"`$\Leftarrow$"' Zu zeigen: $\forall f\in H(G)\exists F\in H(G):\ F'=f$ auf G\\
Sei $f\in H(G)$. Sei $z_0\in G$ (fest).\\
Für $z\in G$ sei $\gamma^{(z)}$ ein Weg mit $\Tr(\gamma^{(z)})\subseteq G$, $\gamma^{(z)}(0)=z_0, \gamma^{(z)}(1)=z$. (Parameterintervall von $\gamma^{(z)}$ sei [0,1])\\
\[F(z):=\int_{\gamma^{(z)}}f(w)dw\ (z\in G)\]
Voraussetzung + 21.1, 21.3 $\Rightarrow$ diese Definition ist unabhängig von der Wahl von $\gamma^{(z)}$.\\
Fast wörtlich wie im Beweis von 9.2.: $F\in H(G), F'=f$ auf G.
\end{beweis}
\begin{satz}[Charakterisierung von Elementargebieten, III]
Sei $G\subseteq\MdC$ ein Gebiet. Dann sind die folgenden Aussagen äquivalent.
\begin{liste}
\item G ist ein Elementargebiet
\item G ist einfach zusammenhängend
\item $\wegint f(z)dz=0\ \forall f\in H(G)$ und für jeden geschlossenen Weg $\gamma$ mit $\Tr(\gamma)\subseteq G$
\item $\forall f\in H(G)$ mit $Z(f)=\emptyset\ \exists g\in H(G):\ e^g=f$ auf G
\item $\forall f\in H(G)$ mit $Z(f)=\emptyset\ \exists g\in H(G):\ g^2=f$ auf G
\item $G=\MdC$ oder $G\sim \MdD$
\end{liste}
\end{satz}
\begin{beweis}
(1)$\Leftrightarrow$(2): 21.4\\
(3)$\Rightarrow$(1): wie im Beweisteil "`$\Leftarrow$"' von 21.4\\
(3)$\Rightarrow$(4),(5): 11.4\\
(4)$\Rightarrow$(5): siehe Beweis von 11.4\\
(5)$\Rightarrow$(6): 19.6\\
(6)$\Rightarrow$(1): wie im Beweisteil "`$\Rightarrow$"'von 21.4\\
(2)$\Rightarrow$(3): 21.2 
\end{beweis}
\begin{definition}
Sei $A\subseteq\widehat{\MdC}$. A heißt in $\widehat{\MdC}$ zusammenhängend $:\Leftrightarrow$ jede lokal konstante Funktion $f:\ A\to\MdC$ ist auf A konstant.
\end{definition}
\begin{satz} [Charakterisierung von Elementargebieten, IV]
Sei $G\subseteq\MdC$ ein Gebiet. Dann sind äquivalent: 
\begin{liste}
\item G ist einfach zusammenhängend
\item $\widehat{\MdC}\backslash G$ ist zusammenhängend in $\widehat{\MdC}$
\item Aus $\MdC\backslash G=A\cup K$, $A\subseteq\MdC$ abgeschlossen, $K\subseteq\MdC$ kompakt und $A\cap K=\emptyset$ folgt: $K=\emptyset$
\end{liste}
\end{satz}
\begin{beweis}
Ohne Beweis.\end{beweis}

\chapter{Cauchyscher Integralsatz (Homologieversionen)} 
In diesem Paragraphen sei $G \subseteq \MdC$ stets ein \underline{Gebiet}.
\begin{definition}
Sei $\gamma$ ein geschlossener Weg in $\MdC$. 
\begin{liste}
\item Int($\gamma$) $:= \{ z \in \MdC \backslash \text{Tr}(\gamma): n(\gamma,z)
\neq 0\}$ (''Inneres'' von $\gamma$)\\
Ext($\gamma$) $:= \{ z \in \MdC \backslash \text{Tr}(\gamma): n(\gamma,z)
= 0\}$ (''Äußeres'' von $\gamma$)
\item Sei Tr($\gamma$) $ \subseteq G$. $\gamma$ heißt in $G$
\begriff{nullhomolog} $:\equizu$ $n(\gamma,z) = 0$ $\forall z \in \MdC \backslash G$ \\
($\equizu$ Int$(\gamma) \subseteq G$)
\end{liste}
\end{definition}

\begin{beispiele}
\begin{liste}
\item Jeder geschlossene Weg in $\MdC$ ist in $\MdC$ nullhomolog.
\item $G := \MdC \backslash \{0\}$, $\gamma(t) = e^{\ie t}$ $(t\in [0,2\pi])$,
$n(\gamma,0) = 1 \neq 0;$ $\gamma$ ist in $G$ nicht nullhomolog.
\end{liste}
\end{beispiele}

\begin{satz}
Sei $\gamma$ ein geschlossener Weg mit Tr($\gamma$) $\subseteq G$.
\begin{liste}
\item Ist $\gamma$ nullhomotop in $G$ $\Rightarrow$ $\gamma$ ist nullhomolog in
$G$. 
\item Ist $G$ einfach zusammenhängend, so ist $\gamma$ in $G$ nullhomolog.
\end{liste}
\end{satz}
\begin{beweis}
\begin{liste}
\item Sei $z_0 \in \MdC \backslash G$. Dann ist $f(z) = \frac{1}{z-z_0}$
holomorph auf $G$. \\ $\stackrel{\text{21.2}}{\Rightarrow} \underbrace{\wegint
f(z) dz}_{= 2 \pi \ie \text{ } n(\gamma, z_0)}=0$ $\Rightarrow$ $n(\gamma, z_0) =
0$
\item folgt aus (1)
\end{liste}
\end{beweis}
\begin{satz}
Sei $f \in H(G)$ und $\gamma$ sei ein geschlossener Weg mit Tr($\gamma$)
$\subseteq G$ . \\
$\varphi: G \times G \to \MdC$ sei definiert durch: \\ \\
\centerline{$\varphi(w,z) := $
$\begin{cases}
\frac{f(w)-f(z)}{w-z}& , w \neq z \\
f'(z)  & , w = z
\end{cases}$} \\
\begin{liste}
\item $\varphi$ ist stetig.
\item Für $z \in G$ (fest) hat $w \mapsto \varphi(w, z)$ in $z$ eine hebbare
Singularität; $w \mapsto \varphi(w, z)$ ist also holomorph auf $G$ \\
Für $w \in G$ (fest) hat $z \mapsto \varphi(w, z)$ in $w$ eine hebbare
Singularität; $z \mapsto \varphi(w, z)$ ist also holomorph auf $G$
\item $h(z) := \wegint \varphi(w,z) dw$ $(z\in G)$. Ist $\gamma$ nullhomolog in
$G$, so ist $h \equiv 0$ auf G.
\end{liste} 
\end{satz}
\begin{beweis}
\begin{liste}
\item 11.9
\item 13.1
\item (A) Es ist $h \in C(G)$. Sei $z_0 \in G$ und $(z_n)$ eine Folge in G mit
$z_n \to z_0$. $g_n(w) := \varphi(w, z_n)$, $g(w) := \varphi(w, z_0)$ $(w \in
G)$. Sei $\Gamma$ der stückweise glatte Ersatzweg für $\gamma$ (wie in §20). \\
Übung: $(g_n)$ konvergiert auf $\Gamma$ gleichmäßig gegen $g$. \\
$\stackrel{\text{8.4}}{\Rightarrow} \int_{\Gamma}$ $g_n(w) dw$ $\to
\int_{\Gamma}$ $g(w) dw = \int_{\Gamma}$ $\varphi(w,z_0) dw = \wegint
\varphi(w,z_0) dw = h(z_0)$ \\
Also: $h(z_n) \to h(z_0)$ \\ \\
(B) Es ist $h \in H(G)$. Sei $\Delta \subseteq G$ ein Dreieck. Wegen 9.7 genügt
es zu zeigen: $\int_{\partial \Delta} h(z) dz = 0$ \\ 
9.1 und (2) $\Rightarrow$ $\int_{\partial \Delta} \varphi(w,z) dz = 0$ $\forall
w \in G$ \\ $\Rightarrow$ $\int_{\partial \Delta} h(z) dz = \int_{\partial
\Delta} (\wegint \varphi(w,z) dw) dz$ $\stackrel{\text{Fubini}}{=}  \wegint ( \underbrace{\int_{\partial
\Delta} \varphi(w,z) dz}_{= 0}) dw = 0$ \\ \\
(C) $\MdC = \underbrace{\text{Int}(\gamma)}_{\subseteq G} \cup \text{Ext}(\gamma)
\cup \underbrace{\text{Tr}(\gamma)}_{\subseteq G}$ $= G \cup \text{Ext}(\gamma)$
\\ Sei $z_0 \in \text{Ext}(\gamma)$. Sei $C$ die Komponente von $\MdC \backslash
\Tr(\gamma)$: $z_0 \in C$. \\ $\stackrel{16.2}{\Rightarrow}$ $n(\gamma, z) =
n(\gamma, z_0) = 0$ $\forall z \in C$ \\ $\Rightarrow$ $C \subseteq
\text{Ext}(\gamma)$. $\stackrel{16.1/2}{\Rightarrow}$ $C$ ist offen.
$\Rightarrow$ $\exists \delta > 0 :$ $U_\delta(z_0) \subseteq C \subseteq
\text{Ext}(\gamma)$. \\Also ist $\text{Ext}(\gamma)$ offen. [Analog: $\text{Int}(\gamma)$ offen]\\
$g(z) := \wegint \frac{f(w)}{w-z} dw$ $(z \not \in \Tr(\gamma))$
$\stackrel{9.5}{\Rightarrow}$ $g \in H(\MdC \backslash \Tr(\gamma))$,
insbesondere gilt $g \in H(\text{Ext}(\gamma))$. \\
Sei $z \in G \cap \Ext$: $h(z) = \wegint \frac{f(w)-f(z)}{w-z} dw = \wegint
\frac{f(w)}{w-z} dw -  f(z) \wegint \frac{1}{w-z} dw$ $= g(z) - f(z) 2 \pi \ie$
$\underbrace{n(\gamma, z)}_{=0} = g(z)$. Also: $ h = g$ auf $G \cap \Ext$. Dann
ist \\
$F(z) = \begin{cases}
  h(z) &, z \in G \\
        g(z)&, z \in \Ext
	\end{cases}
$
eine ganze Funktion.\\ Übung: $F(z) \to 0$ $(|z| \to \infty)$. 10.2 $\Rightarrow$
$F \equiv 0$ $\Rightarrow$ $h \equiv 0$
\end{liste}
\end{beweis}

\begin{satz} [Allgemeine Cauchysche Integralformel]
Sei $\gamma$ ein geschlossener Weg mit $\Tr(\gamma) \subseteq G$  und $\gamma$
sei nullhomolog in $G$. Dann:
\\  
\centerline{ $n(\gamma,z) f(z) = \frac{1}{2 \pi \ie} \wegint \frac{f(w)}{w-z}
dw$ $\forall f \in H(G)$ $\forall z \in G \backslash \Tr(\gamma)$ }
\end{satz}
\begin{beweis}
Sei $f \in H(G)$ und $z \in G \backslash \Tr(\gamma)$.
$\stackrel{22.2(3)}{\Rightarrow} $ $0 = \frac{1}{2 \pi \ie} \wegint
\frac{f(w)-f(z)}{w-z} dw = \frac{1}{2 \pi \ie} \wegint \frac{f(w)}{w-z} dw -
f(z)  \underbrace{\frac{1}{2 \pi \ie} \wegint \frac{dw}{w-z}}_{=n(\gamma,z)}$
\end{beweis}
\begin{satz} [CIS, Homolgieversion I]
Sei $\gamma$ ein gechlossener Weg mit $\Tr(\gamma) \subseteq G$.\\ Dann: \\ 
\centerline{$\wegint f(z) dz = 0$ $\forall f \in H(G)$ $\equizu$ $\gamma$ ist in
$G$ nullhomolog}
\end{satz}
\begin{beweis}
$''\Rightarrow''$: Sei $z_0 \in \MdC \backslash G$; $f(z) := \frac{1}{2 \pi \ie}
\frac{1}{z-z_0} $ $\Rightarrow$ $f \in H(G)$
$\stackrel{\text{Vor.}}{\Rightarrow} \underbrace{\wegint f(z) dz}_{= n(\gamma, z_0
)} = 0$ \\
$''\Leftarrow''$: Sei $f \in H(G)$ und $z_0 \in G \backslash \Tr(\gamma)$;
$g(z) = (z-z_0)f(z)$; $g \in H(G)$. \\ Wende 22.3 auf $g$ an : \\
\centerline{$n(\gamma,z_0) \underbrace{g(z_0)}_{=0} = \frac{1}{2 \pi \ie} \wegint
\underbrace{\frac{g(w)}{w-z_0}}_{=f(w)} dw$ $\Rightarrow \wegint f(w) dw = 0.$} 
\end{beweis}
\begin{satz}
$G$ ist einfach zusammenhängend $\equizu$ jeder geschlossene Weg $\gamma$ mit
$\Tr(\gamma) \subseteq G$ ist in $G$ nullhomolog.
\end{satz}
\begin{beweis}
$''\Rightarrow''$ 22.1(2) \\
$''\Leftarrow''$ Sei $\gamma$ ein geschlossener Weg mit $\Tr(\gamma) \subseteq
G$ und $f \in H(G)$ \\
Vorraussetzungen $\Rightarrow$ $\gamma$ ist in $G$ nullhomolg. 22.4
$\Rightarrow$ $\wegint f(z) dz = 0$. 21.5 $\Rightarrow$ $G$ ist einfach zusammenhängend.
\end{beweis}
\begin{definition}
Seien $\gamma_1$ und $\gamma_2$ geschlossene Wege mit $\Tr(\gamma_1),
\Tr(\gamma_2) \subseteq G.$ $\gamma_1, \gamma_2$ heißen \begriff{in $G$ homolog}
$:\equizu$ $n(\gamma_1,z) = n(\gamma_2, z)$ $\forall z \in \MdC \backslash G.$
\end{definition}
\begin{satz} [CIS, Homologieversion II]
$\gamma_1, \gamma_2$ seien wie in obiger Definition und in $G$ homolog. \\
Dann: \\ 
\centerline{$\int_{\gamma_1} f(z) dz = \int_{\gamma_2} f(z) dz$ $\forall f \in
H(G)$}
\end{satz}
\begin{beweis}
Sei $f \in H(G)$ und $z_j :=$ Anfangspunkt von $\gamma_j$ ($j=1,2$). \\
$\stackrel{\text{3.4}}{\Rightarrow}$ $\exists$ Weg $\gamma: [0,1] \to \MdC$:
$\Tr(\gamma) \subseteq G$, $\gamma(0) = z_1$, $\gamma(1) = z_2$ \\
$\Gamma := \gamma_1 \oplus \gamma \oplus \gamma_2^- \oplus \gamma^-$.
$\Gamma$ ist ein geschlossener Weg mit $\Tr(\gamma) \subseteq G$ \\
Sei $z_0 \in \MdC \backslash G$: $n(\Gamma, z_0) = n(\gamma_1, z_0)+ n(\gamma,
z_0) - n(\gamma_2,z_0) - n(\gamma,z_0) = 0$ \\
D.h.: $\Gamma$ ist in $G$ nullhomolog. 22.4 $\Rightarrow$ $0 = \int_\Gamma f(z)
dz = \int_{\gamma_1}+\int_{\gamma}-\int_{\gamma_2}-\int_{\gamma} = \int_{\gamma_1}-\int_{\gamma_2}$
\end{beweis}

\appendix
\chapter{Satz um Satz (hüpft der Has)}
\listtheorems{satz,wichtigedefinition}

\renewcommand{\indexname}{Stichwortverzeichnis}
\addcontentsline{toc}{chapter}{Stichwortverzeichnis}
\printindex

\end{document}
