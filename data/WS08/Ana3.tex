\documentclass[a4paper]{scrreprt}
\usepackage{amsmath}
\usepackage{amsfonts}
\usepackage{amssymb}
\usepackage{amsthm}
\usepackage[T1]{fontenc} 
\usepackage[utf8]{inputenc}
\usepackage[ngerman]{babel}
\usepackage[normalem]{ulem}

\usepackage{latexki}
\lecturer{Prof. Dr. R. Schnaubelt}
\semester{Wintersemester 08/09}
\scriptstate{complete}

\usepackage[pdftex,plainpages=false]{hyperref}
\hypersetup{
    pdfborder={0 0 0},
    colorlinks=true,
    linkcolor=blue
}

\parindent 0pt
% \parskip 4pt


\newcommand{\PowerSet}{\mathcal{P}}
\newcommand{\doubleOne}{\textbf{1}}
\newcommand{\R}{\mathbb{R}}
\newcommand{\Rq}{\overline{\R}}
\newcommand{\N}{\mathbb{N}}
\newcommand{\Q}{\mathbb{Q}}
\newcommand{\Z}{\mathbb{Z}}
\newcommand{\Borel}{\mathcal{B}}
\newcommand{\Borelq}{\overline{\Borel}}
\newcommand{\toInf}{\rightarrow \infty}
\newcommand{\limToInf}[1]{\lim_{#1 \toInf}}
\newcommand{\Bd}{\Borel_d}
\newcommand{\Leb}{\mathcal{L}}
\newcommand{\Fd}{\mathcal{F}_d}
\newcommand{\Jd}{\mathcal{J}_d}
\newcommand{\dcup}{\dot{\cup}}
\newcommand{\bigdcup}{\biguplus}
\newcommand{\jemph}{\uline}
\newcommand{\overunderset}[3]{\overset{#1}{\underset{#3}{#2}}}

\newenvironment{jsmallmatrix}{\left( \begin{smallmatrix}}{\end{smallmatrix} \right)}


\newcommand{\jlabel}[1]{\label{j_#1}}
\newcommand{\jterm}[1]{\jlabel{#1}\uline{#1}}

\newcommand{\jshortlink}[1]{\jhyperref{#1}{\text{#1}}}
% \newcommand{\jshorteqlink}[1]{\hyperref[eq#1]{\text{#1}}}

\newcommand{\jhyperref}[2]{\hyperref[j_#1]{#2}}

\newcommand{\jlink}[1]{\jhyperref{#1}{#1}}
% \newcommand{\jeqlink}[1]{\hyperref[eq#1]{#1}}

\newcommand{\fu}{\text{\textit{(f.ü.)}}}
\newcommand{\fa}{\text{\textit{(f.a.)}}}
\newcommand{\BeppoLevi}{\text{\jhyperref{Thm 2.19}{Beppo Levi}}}

\newcommand{\mb}[2]{\jhyperref{messbar}{#1-#2-messbar}}
\newcommand{\calMb}[2]{\mb{$\mathcal{#1}$}{$\mathcal{#2}$}}

\newcommand{\jabb}[3]{ #1: #2 \rightarrow #3 }
\newcommand{\jsigalg}{$\sigma$-Algebra}

\newcommand{\jlinkFubini}{\jhyperref{Fubini}{Fubini}}
\newcommand{\jlinkFubiniA}{\jhyperref{FubiniA}{Fubini a)}}
\newcommand{\jlinkFubiniB}{\jhyperref{FubiniB}{Fubini b)}}
\newcommand{\jshortlinkFubini}{\jhyperref{Fubini}{\text{Fub}}}
\newcommand{\jshortlinkFubiniA}{\jhyperref{FubiniA}{\text{Fub a)}}}
\newcommand{\jshortlinkFubiniB}{\jhyperref{FubiniB}{\text{Fub b)}}}

\newcommand{\jspace}{\vspace{8pt}}
\newcommand{\jspacesmall}{\vspace{4pt}}

\newcommand{\jdate}[1]{\jspacesmall\begin{center}\jlabel{#1}\tiny{Ana III, #1}\end{center}}

% \numberwithin{equation}{section}

\newcommand{\supp}{\mathop{\mathrm{{supp}}}}
\newcommand{\Spur}{\mathop{\mathrm{{Spur}}}}
\renewcommand{\div}{\mathop{\mathrm{{div}}}}
\newcommand{\rot}{\mathop{\mathrm{{rot}}}}
\newcommand{\esssup}{\mathop{\mathrm{ess\ sup}}\limits}
%\newcommand{\esssup}{\text{ess sup}}

\renewcommand{\labelenumi}{\alph{enumi})}
\renewcommand{\labelenumii}{\roman{enumii})}

\theoremstyle{plain}
\newtheorem{thm}{Theorem}[chapter]
\newtheorem{lem}[thm]{Lemma}
\newtheorem{satz}[thm]{Satz}
\newtheorem{kor}[thm]{Korollar}
\newtheorem*{satz*}{Satz}

\theoremstyle{definition}
\newtheorem{defn}[thm]{Definition}
\newtheorem{expl}[thm]{Beispiel}
\newtheorem{bem}[thm]{Bemerkung}

\newtheorem*{defn*}{Definition}
\newtheorem*{expl*}{Beispiel}
\newtheorem*{bem*}{Bemerkung}

\newtheorem*{auswahlaxiom}{Auswahlaxiom}
\newtheorem*{gebrauchsanweisungFubini}{Gebrauchsanweisung für \jlink{Fubini}}

\title{Ana III}
\subtitle{Karlsruhe\\Ein inoffizieller Mitschrieb der Vorlesung von Prof. Dr. Roland Schnaubelt}
\author{\LaTeX-Code von Johannes Ernesti}
\date{Wintersemeseter 2008/2009}


\begin{document}
\pagenumbering{alph}
\maketitle

\begin{abstract}
    Diese Datei ist der Abschrieb meines Mitschriebes - dementsprechend wahrscheinlich haben sich einige Tippfehler eingeschlichen. (Wenn man zudem noch beachtet, wie schnell in dieser Vorlesung angeschrieben wird, ist sogar meine Vorlage mit hoher Wahrscheinlichkeit fehlerhaft.)
    
    \jspace
    
    Neben allem, was in der Vorlesung angeschrieben wurde, habe ich auch vor, am Ende des Dokuments alle nützlichen Sätze und Beispiele aufzuführen, die nur in der Übung oder auf Übungsblättern bewiesen bzw. vorgeführt wurden. Falls ihr noch gute Ideen habt, könnt ihr sie mir gerne mitteilen.
    
    \jspace
    
    Sollten euch irgendwelche Fehler/Unklarheiten auffallen oder sonstige Verbesserungsvorschläge in den Sinn kommen, zögert bitte nicht, mir eine E-Mail zu schreiben.\\
    Meine Adresse lautet: uni@johannes.lpe-media.de
    
    \jspace
    
    Ansonsten viel Spaß beim Lernen mit diesem Skript.
    
    \jspace
    
    Johannes
\end{abstract}

\clearpage\pagenumbering{arabic}
\tableofcontents

\newpage

\section{Satz-, Vorlesungs- und Beispielverzeichnis}

\newcommand{\jtablelink}[1]{\jlink{#1} & \jhyperref{#1}{\pageref{j_#1}}}

\subsection{Sätze und Beispiele}

Hier sind Verweise zu allen nummerierten Sätzen, Lemmata, Korollaren, Theoremen, Definitionen und Beispielen aufgeführt.

\begin{center}
% use packages: array
    \begin{tabular}{lc|lc}
        \textbf{Eintrag}       & \textbf{Seite}         & 
        \textbf{Eintrag}       & \textbf{Seite}         \\
        \hline
        \jtablelink{Def 1.1}   & \jtablelink{Bsp 1.2}   \\
        \jtablelink{Lem 1.3}   & \jtablelink{Lem 1.4}   \\
        \jtablelink{Def 1.5}   & \jtablelink{Lem 1.6}   \\
        \jtablelink{Bsp 1.7}   & \jtablelink{Def 1.8}   \\
        \jtablelink{Satz 1.9}  & \jtablelink{Lem 1.10}  \\
        \jtablelink{Kor 1.11}  & \jtablelink{Def 1.12}  \\
        \jtablelink{Bsp 1.13}  & \jtablelink{Satz 1.14} \\
        \jtablelink{Lem 1.15}  & \jtablelink{Lem 1.16}  \\
        \jtablelink{Satz 1.17} & \jtablelink{Thm 1.18}  \\
        \jtablelink{Thm 1.19}  & \jtablelink{Thm 1.20}  \\
        \jtablelink{Bem 1.21}  & \jtablelink{Thm 1.25}  \\
        \jtablelink{Satz 1.26} & \jtablelink{Def 2.1}   \\
        \jtablelink{Bem 2.2}   & \jtablelink{Satz 2.3}  \\
        \jtablelink{Def 2.4}   & \jtablelink{Satz 2.5}  \\
        \jtablelink{Lem 2.6}   & \jtablelink{Satz 2.7}  \\
        \jtablelink{Satz 2.8}  & \jtablelink{Def 2.9}   \\
        \jtablelink{Bem 2.10}  & \jtablelink{Satz 2.11} \\
        \jtablelink{Kor 2.12}  & \jtablelink{Def 2.13}  \\
        \jtablelink{Lem 2.14}  & \jtablelink{Lem 2.15}  \\
        \jtablelink{Def 2.16}  & \jtablelink{Lem 2.17}  \\
        \jtablelink{Lem 2.18}  & \jtablelink{Thm 2.19}  \\
        \jtablelink{Kor 2.20}  & \jtablelink{Lem 2.21}  \\
        \jtablelink{Def 2.22}  & \jtablelink{Satz 2.23} \\
        \jtablelink{Kor 2.24}  & \jtablelink{Satz 2.25} \\
        \jtablelink{Bem 2.26}  & \jtablelink{Def 3.1}   \\
        \jtablelink{Bem 3.2}   & \jtablelink{Def 3.3}   \\
        \jtablelink{Bsp 3.4}   & \jtablelink{Lem 3.5}   \\
        \jtablelink{Def 3.6}   & \jtablelink{Lem 3.7}   \\
        \jtablelink{Bem 3.8}   & \jtablelink{Thm 3.9}   \\
        \jtablelink{Thm 3.10}  & \jtablelink{Bem 3.11}  \\
        \jtablelink{Kor 3.12}  & \jtablelink{Kor 3.13}  \\
        \jtablelink{Thm 3.14}  & \jtablelink{Kor 3.15}  \\
        \jtablelink{Thm 3.16}  & \jtablelink{Lem 3.17}  \\
        \jtablelink{Lem 3.18}  & \jtablelink{Lem 3.19}  \\
        \jtablelink{Lem 3.20}  & \jtablelink{Satz 3.21} \\
        \jtablelink{Kor 3.22}  & \jtablelink{Bem 3.23}  \\
        \jtablelink{Bem 3.24}  & \jtablelink{Bsp 3.25}  \\
        \jtablelink{Bsp 3.26}  & \jtablelink{Thm 3.27}  \\
        \jtablelink{Bem 3.28}  & \jtablelink{Bem 3.29}  \\
        \jtablelink{Thm 3.30}  & \jtablelink{Lem 3.31}  \\
        \jtablelink{Lem 3.32}  & \jtablelink{Lem 3.33}  \\
        \jtablelink{Bsp 3.34}  & \jtablelink{Bsp 3.35}  \\
        \jtablelink{Bsp 3.36}  & \jtablelink{Satz 3.37} \\
    \end{tabular}
\end{center}

\begin{center}
    \begin{tabular}{lc|lc}
        \textbf{Eintrag}       & \textbf{Seite}         & 
        \textbf{Eintrag}       & \textbf{Seite}         \\
        \hline
        \jtablelink{Bsp 3.38}  & \jtablelink{Def 4.1}   \\
        \jtablelink{Bem 4.2}   & \jtablelink{Bsp 4.3}   \\
        \jtablelink{Def 4.4}   & \jtablelink{Bem 4.5}   \\
        \jtablelink{Bsp 4.6}   & \jtablelink{Lem 4.7}   \\
        \jtablelink{Def 4.8}   & \jtablelink{Bsp 4.9}   \\
        \jtablelink{Bsp 4.10}  & \jtablelink{Lem 4.11}  \\
        \jtablelink{Lem 4.12}  & \jtablelink{Def 4.13}  \\
        \jtablelink{Satz 4.14} & \jtablelink{Bsp 4.15}  \\
        \jtablelink{Bem 4.16}  & \jtablelink{Satz 4.17} \\
        \jtablelink{Thm 4.18}  & \jtablelink{Kor 4.19}  \\
        \jtablelink{Bsp 4.20}  & \jtablelink{Bsp 4.21}  \\
        \jtablelink{Thm 4.22}  & \jtablelink{Bsp 4.23}  \\
%         \jtablelink{} & \jtablelink{} \\
    \end{tabular}
\end{center}

\subsection{Vorlesungstermine}
Nachfolgend sind alle Vorlesungstermine mit der dazugehörigen Seitenzahl in diesem Skriptum aufgelistet.

\begin{center}
% use packages: array
    \begin{tabular}{lc|lc|lc}
        \textbf{Datum}       & \textbf{Seite}         & 
        \textbf{Datum}       & \textbf{Seite}         & 
        \textbf{Datum}       & \textbf{Seite}         \\
        \hline
        \jtablelink{20.10.2008} & \jtablelink{24.10.2008} & \jtablelink{27.10.2008} \\
        \jtablelink{31.10.2008} & \jtablelink{03.11.2008} & \jtablelink{07.11.2008} \\
        \jtablelink{10.11.2008} & \jtablelink{14.11.2008} & \jtablelink{17.11.2008} \\
        \jtablelink{21.11.2008} & \jtablelink{24.11.2008} & \jtablelink{28.11.2008} \\
        \jtablelink{01.12.2008} & \jtablelink{05.12.2008} & \jtablelink{08.12.2008} \\
        \jtablelink{12.12.2008} & \jtablelink{15.12.2008} & \jtablelink{19.12.2008} \\
        \jtablelink{22.12.2008} & \jtablelink{09.01.2009} & \jtablelink{12.01.2009} \\
        \jtablelink{16.01.2009} & \jtablelink{19.01.2009} & \jtablelink{23.01.2009} \\
        \jtablelink{26.01.2009} & \jtablelink{30.01.2009}
    \end{tabular}
\end{center}

\subsection{Nummerierte Gleichungen}

Hier folgt eine Liste der nummerierten Gleichungen.

\jspacesmall

\begin{center}
% use packages: array
    \begin{tabular}{lc|lc}
        \textbf{Eintrag}       & \textbf{Seite}         & 
        \textbf{Eintrag}       & \textbf{Seite}         \\
        \hline
        \jtablelink{(1.1)}  & \jtablelink{(1.2)} \\
        \jtablelink{(1.3)}  & \jtablelink{(1.4)} \\
        \jtablelink{(1.5)}  & \jtablelink{(1.6)} \\
        \jtablelink{(1.7)}  & \jtablelink{(1.8)} \\
        \jtablelink{(1.9)}  & \jtablelink{(2.1)} \\
        \jtablelink{(2.2)}  & \jtablelink{(3.1)} \\
        \jtablelink{(3.2)}  & \jtablelink{(3.3)} \\
        \jtablelink{(3.4)}  & \jtablelink{(3.5)} \\
        \jtablelink{(3.6)}  & \jtablelink{(3.7)} \\
        \jtablelink{(3.8)}  & \jtablelink{(3.9)} \\
        \jtablelink{(3.10)} & \jtablelink{(3.11)} \\
        \jtablelink{(3.12)} & \jtablelink{(4.1)} \\
        \jtablelink{(4.2)}  & \jtablelink{(4.3)} \\
        \jtablelink{(4.4)}  & \jtablelink{(4.5)} \\        
        \jtablelink{(4.6)}  & \jtablelink{(4.7)} \\
        \jtablelink{(4.8)}  & \jtablelink{(4.9)} \\
        \jtablelink{(4.10)} & \jtablelink{(4.11)}
    \end{tabular}
\end{center}

\newpage

\jdate{20.10.2008}

\section{Das Volumenproblem}
Das Elementavolumen eines Quaders $Q=I_1 \times \dots \times I_d$ für Intervalle $I_j \subset \R$ mit Länge $l_j$ ist: $\operatorname{vol}(Q) = l_1\cdots l_d$.

\vspace{12pt}

\uline{Ziel}: Setze dies sinnvoll auf $\PowerSet(\R^d) := \{A : A \subset \R^d\}$ fort, d.h.: Wir suchen eine Abbildung $\mu:\PowerSet(\R^d) \rightarrow [0,\infty]$ mit:
\begin{itemize}
 \item $\mu(\emptyset) = 0$
 \item für disjunkte $A_1, \dots, A_n$ gilt: $\mu(A_1 \dcup \dots \dcup A_n) = \mu(A_1) + \dots + \mu(A_n)$
\end{itemize}
Daraus folgt: Für $A,B \subset \R^d$ gilt:
\begin{itemize}
    \item $\mu(A \cup B) \le \mu(A) + \mu(B)$
    \item $A \subset B \Rightarrow \mu(A) \le \mu(B)$
\end{itemize}
Ferner soll gelten $\mu(Q) = \operatorname{vol}(Q)$ für alle Quader $Q$, sowie $\mu(T(A)) = \mu(A)$ für jede Bewegung $Tx = a + Ux \ (a \in \R^d, U \ \text{othogonale Matrix})$.

\vspace{12pt}

\uline{Inhaltsproblem}: Gibt es so ein $\mu$? \uline{Antwort:} Nein! (für $d \ge 3$)

\vspace{12pt}

\uline{Banach-Tarski-Paradoxon (1924)}\\
Es gibt 5 Mengen $A_1, \dots, A_5 \subset \overline{B}(0,1) =: K$ mit $A_i \cap A_j = \emptyset \ \forall i \ne j$ und Bewegungen $T_1,\dots, T_5$ mit $T_1(A_1) \cup T_2(A_2) \cup T_3(A_3) = K$ und $T_4(A_4) \cup T_5(A_5) = K$.
Das heißt: Wenn es ein solches wie oben $\mu$ gäbe, dann würde gelten:
\begin{displaymath}
    \begin{split}
        \mu(K) &= \sum_{k=1}^5 \mu(A_k) = \sum_{k=1}^5 \mu(T_k A_k) \ge \mu(T_1 A_1 \cup T_2 A_2 \cup T_3 A_3) + \mu(T_4 A_4 \cup T_5 A_5) \\
        &= 2 \mu(K) 
    \end{split}
\end{displaymath} 
Das ist ein Widerspruch, denn $\mu(K) \ge \mu(Q) > 0$ für jeden echten Quader $Q \subset K$.

\chapter{Das Lebesgue-Maß}
    % Etwas Maßtheorie
\section{Etwas Maßtheorie}

Sei stets $X$ eine nichtleere Menge mit Potzenzmenge $\PowerSet(X) := \{A : A \subset X\}$.


\begin{defn}
\jlabel{Def 1.1}
    Ein nichtleeres Mengensystem $\mathcal{A} \subset \PowerSet(X)$ heißt \uline{$\sigma$-Algebra}, wenn:
    \begin{quote}
    \begin{itemize}
        \item[(A1)] $X \in \mathcal{A}$
        \item[(A2)] Wenn $A \in \mathcal{A}$, dann auch $A^c := X \backslash A \in \mathcal{A}$
        \item[(A3)] Wenn $A_j \in \mathcal{A}, (j \in \N)$, dann auch $\bigcup_{j \in \N} A_j \in \mathcal{A}$
    \end{itemize}
    \end{quote}
\end{defn}


\jlabel{Bsp 1.2}
\begin{expl}
    \begin{enumerate}
        \item $\PowerSet(X)$ und $\{\emptyset, X\}$ sind $\sigma$-Algebren
        \item Sei $\emptyset \ne A \subset X$. Dann ist $\{\emptyset, A, A^c, X\}$ eine $\sigma$-Algebra
        \item $\mathcal{A} := \{A \subset X : A$ oder $A^c$ ist abzählbar$\}$ ist $\sigma$-Algebra
            \begin{proof} 
                \begin{enumerate}
                    \item[(A1)] $X^c = \emptyset$ ist abzählbar, also $X \in \mathcal{A}$.
                    \item[(A2)] gilt per Definition.
                    \item[(A3)] Seien $A_j \in \mathcal{A} \ (j \in \N)$.
                        \begin{enumerate}
                            \item
                                Seien alle $A_j$ abzählbar. Dann ist $\bigcup_{j \in \N} A_j$ abzählbar, denn: Es gilt $A_j = \{a_{j1}, a_{j2}, \dots\}$      für jedes $j \in \N$ und gewisse $a_{jk} \in X$. Schreibe:
                                
                                \vspace{12pt}
                                
                                TODO: Grafik
                                
                                \vspace{12pt}
                            
                                Nach Streichen mehrfach auftretender $a_{jk}$ liefert der Streckenzug eine Abzählung von $\bigcup_{j \in \N} A_j \Rightarrow \bigcup_{j\in\N}A_j \in \mathcal{A}$.
                            \item
                                Wenn ein $A_n$ nicht abzählbar ist, dann ist $A_n^c$ abzählbar. Somit gilt:\\
                                $\left( \bigcup_{j=1}^\infty A_j \right)^c = \bigcap_{j=1}^\infty A_j^c \subset A_k^c \Rightarrow \left( \bigcup_{j=1}^\infty A_j \right)^c$ abzählbar. Damit folgt $\bigcup_{j \in\N} A_j \in \mathcal{A}$.
                        \end{enumerate}
                \end{enumerate}
            \end{proof}

    \end{enumerate}

\end{expl}

\begin{lem}
\jlabel{Lem 1.3}
    Sei $\mathcal{A}$ eine $\sigma$-Algebra auf $X$ und $A_j \in \mathcal{A} \ (j \in \N)$. Dann:
    \begin{enumerate}
        \item $\emptyset = X^c \in \mathcal{A}$
        \item $A_1 \cup \dots \cup A_n \in \mathcal{A} \ (\forall n \in \N)$
        \item $\bigcap_{j=1}^\infty A_j \in \mathcal{A}$
        \item $A_1 \backslash A_2 := A_1 \cap A_2^c \in \mathcal{A}$
    \end{enumerate}
    \uline{Fazit:} Abzählbare Mengenoperationen bleiben in der $\sigma$-Algebra.
    \begin{proof}
        \begin{enumerate}
            \item Klar mit (A1) und (A2).
            \item Folgt aus (A3) und a), da $A_1 \cup \dots \cup A_n = A_1 \cup \dots \cup A_n \cup \emptyset \cup \emptyset \cup \dots$
            \item Nach (A2) und (A3): $\bigcup_{j=1}^\infty A_j^c \in \mathcal{A} \overset{(A2)}{\Rightarrow} \bigcap_{j=1}^\infty A_j = \left( \bigcup_{j=1}^\infty A_j^c \right)^c \in \mathcal{A}$
            \item Folgt aus c), (A1) und (A3), da $A_1 \cap A_2^c = A_1 \cap A_2^c \cap X \cap X$.
        \end{enumerate}

    \end{proof}
\end{lem}

\begin{lem}
\jlabel{Lem 1.4}
    Sei $\mathcal{F}$ eine nichtleere Famile von $\sigma$-Algebren $\mathcal{A}$ auf $X$.\\
    Dann ist 
    \begin{displaymath}
        \mathcal{A}_0 := \bigcap \{\mathcal{A} : \mathcal{A} \in \mathcal{F}\} := \{A \subset X: A \in \mathcal{A} \ (\forall \mathcal{A} \in \mathcal{F})\}
    \end{displaymath}
    eine $\sigma$-Algebra.
    \begin{proof}
        \begin{enumerate}
            \item[(A1)] $X \in \mathcal{A} \ (\forall \mathcal{A} \in \mathcal{F}) \Rightarrow X \in \mathcal{A}_0$.
            \item[(A2)] Sei $A \in \mathcal{A}_0 \overset{(A2)}{\Rightarrow} A^c \in \mathcal{A} \ (\forall \mathcal{A} \in \mathcal{F}) \Rightarrow A^c \in \mathcal{A}_0$.
            \item[(A3)] Sei $ A_j \in \mathcal{A}_0 \ (j\in\N) \overset{(A3)}{\Rightarrow} \bigcup_{j\in\N} A_j \in \mathcal{A} \ (\forall \mathcal{A} \in \mathcal{F}) \Rightarrow \bigcup_{j\in\N} A_j \in \mathcal{A}_0$.
        \end{enumerate}
    \end{proof}
\end{lem}

\jdate{24.10.2008}

\begin{defn}
\jlabel{Def 1.5}
    Sei $\mathcal{E} \subset \PowerSet(X)$ nicht leer. Dann heißt
    \begin{displaymath}
        \sigma(\mathcal{E}) := \bigcap\{\mathcal{A} \subset \PowerSet(X) : \mathcal{E} \subset \mathcal{A} \ \text{ist} \ \sigma\text{-Algebra}\}
    \end{displaymath}
    die von $\mathcal{E}$ erzeugte $\sigma$-Algebra.
    
    \vspace{12pt}
    
    \uline{Bemerkung:} Da $\PowerSet(X)$ eine $\sigma$-Algebra ist, ist $\sigma(\mathcal{E})$ nicht leer und nach \jlink{Lem 1.4} ist $\sigma(\mathcal{E})$ eine $\sigma$-Algebra.
\end{defn}

\begin{lem}
\jlabel{Lem 1.6}
    Sei $\emptyset \ne \mathcal{E} \subset \PowerSet(X)$. Dann gelten:
    \begin{enumerate}
        \item Wenn $\mathcal{A}$ eine $\sigma$-Algebra ist und $\mathcal{E} \subset \mathcal{A}$, dann $\mathcal{E} \subset \sigma(\mathcal{E}) \subset \mathcal{A}$.
        \item $\sigma(\mathcal{E})$ ist die einzige $\sigma$-Algebra, die a) erfüllt, d.h. $\sigma(\mathcal{E})$ ist die kleinste $\mathcal{E}$ enthaltende $\sigma$-Algebra auf $X$.
        \item Wenn $\mathcal{E}$ eine $\sigma$-Algebra ist, dann ist $\sigma(\mathcal{E}) = \mathcal{E}$.
        \item Wenn $\mathcal{E} \subset \overline{\mathcal{E}} \subset \PowerSet(X)$, dann gilt $\sigma(\mathcal{E}) \subset \sigma(\mathcal{\overline{E}})$.
    \end{enumerate}
    \begin{proof}
        \begin{enumerate}
            \item folgt direkt aus \jlink{Def 1.5}.
            \item
                Sei $\mathcal{A}_0$ eine $\sigma$-Algebra mit $\mathcal{E} \subset \mathcal{A}_0 \subset \mathcal{A}$ für jede $\sigma$-Algebra $\mathcal{A}$ mit $\mathcal{E} \subset \mathcal{A}$. Wähle $\mathcal{A} = \sigma(\mathcal{E}) \Rightarrow \mathcal{A}_0 \subset \sigma(\mathcal{E})$. Nach a) gilt mit $\mathcal{A} = \mathcal{A}_0: \sigma(\mathcal{E}) \subset \mathcal{A}_0$.
            \item folgt aus a) mit $\mathcal{A} = \mathcal{E}$.
            \item folgt aus a) mit $\mathcal{A} = \sigma(\overline{\mathcal{E}})$.
        \end{enumerate}
    \end{proof}
\end{lem}


\jlabel{Bsp 1.7}
\begin{expl}
    \begin{enumerate}
        \item 
            Sei $\mathcal{E} = \{A\}$ für ein nicht leeres $A \subset X$. Jede $\sigma$-Algebra $\mathcal{A}$ mit $\mathcal{E} \subset \mathcal{A}$ umfasst $\{X, \emptyset, A, A^c\}$ nach (A1) und (A2).\\
            Nach Beispiel 1.2b) ist dies eine $\sigma$-Algebra.\\
            \jlink{Lem 1.6} $\Rightarrow \sigma(\mathcal{E}) = \{\emptyset, A, A^c, X\}$.
        \item
            Sei $X = \{1,2,3,4,5\},\ \mathcal{E} = \{\{1\}, \{1,2\}\}$. Die $\sigma$-Algebra $\sigma(\mathcal{E})$ enthält folgende Elemente:\\
            $\{1\}, \{2\}= \{1,2\} \backslash \{1\}$ und somit auch $\{1\}^c = \{2,3,4,5\}$ und $\{1,2\}^c = \{3,4,5\}$.\\
            Ferner $\emptyset, X \in \sigma(\mathcal{E})$.\\
            \uline{Prüfe:} $\mathcal{A} = \{\emptyset, \{1\}, \{2\}, \{1,2\}, \{3,4,5\}, \{1,3,4,5\}, \{2,3,4,5\}, X\}$ ist eine $\sigma$-Algebra. Aus \jlink{Lem 1.6} folgt $\sigma(\mathcal{E})=\mathcal{A}$.
    \end{enumerate}
\end{expl}


\begin{defn}
\jlabel{Def 1.8}
    Sei $X \subset \R^d$ nicht leer und $\mathcal{O}(X)$ das System der in X offenen Mengen. Dann heißt 
    \begin{displaymath}
        \Borel(X) := \sigma(\mathcal{O}(X))
    \end{displaymath}
    die Borel $\sigma$-Algebra von X. Setze $\Bd := \Borel(\R^d)$.
    
    \vspace{12pt}
    
    \uline{Bemerkung:} $\Bd$ enthält alle offenen und alle abgeschlossenen Teilmengen des $\R^d$, alle abzählbaren Vereinigungen und Schnitte offener und abgeschlossener Menge, \uline{usw}.
\end{defn}
\uline{Intervalle} in $\R^d$ sind Mengen der Form $I = I_1 \times \dots \times I_d$, wobei $I_1, \dots, I_d \subset \R$ Intervalle in $\R$ sind. Für $a, b \in \R^d$ mit $a \le b$ (d.h: $a_1 \le b_1, \dots, a_d \le b_d$) schreibe:
\begin{displaymath}
    \begin{split}
        (a,b) := (a_1, b_1) \times \dots \times (a_d, b_d),\\
        (a,b] := (a_1, b_1] \times \dots \times (a_d, b_d].
    \end{split}
\end{displaymath}
Für $\alpha \in \R$ und $k \in \{1,\dots,d\}$ schreibe $H_k^-(\alpha) := \{x \in \R^d: x_k \le \alpha\}$.

{\begin{satz}
\jlabel{Satz 1.9}
    Es gilt:
    \begin{displaymath}
        \begin{split}{}
            \Bd & = \sigma(\{(a,b): a,b \in \Q^d, a \le b\}) =: A_1 \\
                & = \sigma(\{(a,b]: a,b \in \Q^d, a \le b\}) =: A_2 \\
                & = \sigma(\{H_k^-(\alpha): \alpha \in \Q, k \in \{1, \dots, d\}\}) =: A_3
        \end{split}
    \end{displaymath}
    \begin{proof}
        \begin{enumerate}
            \item Es gilt:
            \begin{displaymath}
                \begin{split}
                    &(a,b] = \bigcap_{k=1}^d H_k^-(b_k) \cap H_k^-(a_k)^c \Rightarrow (a,b] \in A_3\\
                    &\overset{\jshortlink{Lem 1.6}}{\Rightarrow} A_2 = \sigma(\{(a,b], a,b \in \Q^d, a \le b\}) \subset A_3.
                \end{split}                
            \end{displaymath}
            \item $H_k^-(\alpha)$ ist abgeschlossen, also $H_k^-(\alpha) \in \Bd$. \jlink{Lem 1.6} $\Rightarrow A_3 \subset \Bd$.
            \item Wenn ein $a_k = b_k$, dann $(a,b)=\emptyset \in A_2$. Anderenfalls
            \begin{displaymath}
                (a,b) = \bigcup_{n \ge n_0} (a, b -\left(\frac{1}{n}, \dots, \frac{1}{n}\right)^T],
            \end{displaymath}
            wobei $n_0\in\N$, sodass $a_k + \frac{1}{n} \le b_k \forall n \ge n_0, k=1,\dots, d$. Dann folgt: $(a,b) \in A_2$. Mit \jlink{Lem 1.6} folgt $\mathcal{A}_1 = \sigma(\{(a,b) : a,b \in \Q^d, a \le b\}) \subset A_2$.
            
            \vspace{12pt}
            
            \uline{Bislang wurde gezeigt:} $A_1 \subset A_2 \subset A_3 \subset \Bd$.
            \item
                Sei $O \subset \R^d$ offen, $J := \{(a,b) \subset O : a,b \in \Q^d, a\le b\}$
                
                \jspacesmall
                
                \uline{Zeige:} $\bigcup \{I : I \in J\} = O$\\
                Da die Vereinigung abzählbar ist, folgt $O \in A_1$. Damit $\Bd \subset A_1$ nach \jlink{Lem 1.6} und $\Bd = \sigma(\mathcal{O}(\R^d))$. Damit folgt die Behauptung.
                
                \jspacesmall
                
                \uline{Zu $\supset$:} Sei $y \in O$. Dann $\exists \epsilon > 0, \epsilon \in \Q$, sodass die $\lVert \cdot \rVert_\infty$-Kugel
                \begin{displaymath}
                    I_0 = (y_1-\epsilon, y_1 + \epsilon) \times \dots \times (y_d-\epsilon, y_d + \epsilon) \subset O.
                \end{displaymath}
                Durch Verschiebung von $y$ zu einem $z\in \Q^d$ nahe bei $y$ erhält man ein $I\in J$ der Kantenlänge $\epsilon$ mit $y \in I \Rightarrow O \subset \bigcap\{I: I\in J\}$. Damit gilt die Gleichheit.
        \end{enumerate}
    \end{proof}
\end{satz}

Für $Y\subset X, Y \ne \emptyset$ und $M \subset \PowerSet(X)$ definiert man die \uline{Spur}:

\jlabel{(1.1)}
\begin{equation}
    M_y = M \cap Y := \{A \subset Y: A = \overline{M} \cap Y \ \text{für ein} \ \overline{M} \in M\}
\end{equation}

{\begin{lem}
\jlabel{Lem 1.10}
    Sei $\emptyset \ne Y \subset X$. Dann gelten:
    \begin{enumerate}
        \item Wenn $\mathcal{A}$ eine $\sigma$-Algebra ist, dann ist auch $\mathcal{A}_y$ eine $\sigma$-Algebra auf Y. Ferner gilt $\mathcal{A}_y \subset \mathcal{A} \Leftrightarrow Y \in \mathcal{A}$.
        \item Sei $\emptyset \ne \mathcal{E} \subset \PowerSet(X)$. Dann $\sigma(\mathcal{E} \cap Y) = \sigma(\mathcal{E}) \cap Y$. \\
            (Beides sind $\sigma$-Algebren auf $Y$.)
    \end{enumerate}
    \begin{proof}
        \begin{enumerate}
            \item 
                \uline{Zu (A1):} $Y = X \cap Y \in \mathcal{A}_y$, da $X \in \mathcal{A}$.\\
                \uline{Zu (A2), (A3):} Seien $B_j = A_j \cap Y \in \mathcal{A}_y$, d.h. $A_j \in \mathcal{A} \ (j \in \N)$. Dann folgt
                \begin{displaymath}
                    \begin{split}
                        Y \backslash B_1 &= Y \cap \underbrace{(X \backslash A_1)}_{\in \mathcal{A}} \in \mathcal{A}_y,\\
                        \bigcup_{j\in \N} B_j &= \underbrace{( \bigcup_{j \in \N} A_j )}_{\in \mathcal{A}} \cap Y \in \mathcal{A}_y.
                    \end{split}
                \end{displaymath}

                Also ist $\mathcal{A}_y$ eine $\sigma$-Algebra.\\
                \uline{Zweite Behauptung:} $Y\in \mathcal{A} \Rightarrow M \cap Y \in \mathcal{A} \ (\forall M \in \mathcal{A})$, also $\mathcal{A}_y \subset \mathcal{A}$. Wenn $\mathcal{A}_y \subset \mathcal{A}$, folgt also $Y \in \mathcal{A}$.
                
            \item
                $\mathcal{E} \cap Y \subset \sigma(\mathcal{E}) \cap Y$ ist nach a) $\sigma$-Algebra.
                
                \jspacesmall
                
                \uline{Zu $\supset$ :} \uline{Prinzip der guten Mengen}\\
                Setze $\mathcal{C} := \{A \subset X : A \cap Y \in \sigma(\mathcal{E} \cap Y)\}$
                
                \jspacesmall
                
                \uline{$(*)$ Behauptung:} $\mathcal{C}$ ist eine $\sigma$-Algebra. Ferner gilt $\mathcal{E} \subset \mathcal{C}$, da $E \cap Y \in \sigma(\mathcal{E} \cap Y)$ für alle $E \in \mathcal{E} \overset{\jshortlink{Lem 1.6}}{\Rightarrow} \sigma(\mathcal{E}) \subset \mathcal{C}$.\\
                Aus der Definition von $\mathcal{C}$ folgt: $\sigma(\mathcal{E}) \cap Y \subset \sigma(\mathcal{E} \cap Y)$.
                
                \jspacesmall
                
                \uline{Beweis von $(*)$:}
                $Y = X \cap Y \in \sigma(\mathcal{E} \cap Y) \Rightarrow X \in \mathcal{C}$. Damit erfüllt $\mathcal{C}$ (A1).\\
                Zu (A2) und (A3): Seien $A_j \in \mathcal{C} \ (j \in \N) \Rightarrow A_j \cap Y \in \sigma(\mathcal{E} \cap Y)$. Dann gelten:
                \begin{itemize}
                    \item $(X\backslash A_1) \cap Y = Y \backslash \underbrace{(A_1 \cap Y)}_{\in \sigma(\mathcal{E} \cap Y)} \overset{(A2)}{\in} \sigma(\mathcal{E} \cap Y) \Rightarrow X \backslash A_1 \in \mathcal{C}$.
                    \item $(\bigcup_{j=1}^{\infty} A_j) \cap Y = \bigcup_{j=1}^\infty \underbrace{A_j \cap Y}_{\in \sigma(\mathcal{E} \cap Y)} \overset{(A3)}{\in} \sigma(\mathcal{E} \cap Y) \Rightarrow \bigcup_{j=1}^\infty A_j \in \mathcal{C}$
                \end{itemize}
                $\Rightarrow \mathcal{C}$ ist $\sigma$-Algebra auf X.
        \end{enumerate}

    \end{proof}
\end{lem}

\jdate{27.10.2008}

\begin{kor}
\jlabel{Kor 1.11}
    Sei $X \subset \R^d$. Dann gilt $\Borel(X) = \Bd \cap X = \{B \cap X : B \in \Bd\}$.
    Wenn $X \in \Bd$, dann $\Bd = \{A\in \Bd : A \subset X\}$
    \begin{proof}
        Folgt aus \jlink{Lem 1.10} mit $\mathcal{E} = \mathcal{O}(\R^d)$, da $\mathcal{O}(X) = \mathcal{O}(\R^d) \cap X$.\\
        (Wobei $X$ in \jlink{Lem 1.10} $\R^d$ in \jlink{Kor 1.11} entspricht und $Y$ in \jlink{Lem 1.10} $X$ in \jlink{Kor 1.11}.)
    \end{proof}
\end{kor}

\begin{expl*}
    \begin{enumerate}
        \item $\Q = \bigcup_{n\in\N} \{q_n\} \in \Bd$, da $\{q_n\}$ abgeschlossen ist, wobei $\Q = \{q_1, q_2, \dots\}$.
        \item Die Menge 
        \begin{displaymath}
            \begin{split}
                A &= \{(x,y) \in \R^2 : x^2 + y^2 \le 1\text{ für }x \le 0\text{ oder }x^2 + y^2 < 1\text{ für }x > 0\}\\
                &= B(0,1) \cup (\partial B(0,1) \cap \{x\le 0\})
            \end{split}
        \end{displaymath}
        ist abgeschlossen. Damit folgt $A\in \Borel_2$.
    \end{enumerate}
\end{expl*}


Sei $[0, +\infty] = \R_+ \cup \{+\infty\}$ (wobei: $+\infty = \infty$) versehen mit den Rechenregeln:
\begin{itemize}
    \item $\pm a +\infty = \infty \pm a = \infty \ \forall a\in \R_+$
    \item $\infty + \infty = \infty$
    \item \uline{Verboten:} $\infty - \infty$!
\end{itemize}
Ordnung: $a < \infty \forall a \in \R_+$

\jspacesmall

Konvergenz: $x_n \xrightarrow{n\rightarrow \infty} \infty \Leftrightarrow \forall c>0\  \exists N_c \in \N$ mit $x_n \ge c\ \forall n \ge N_c$.

\jspacesmall

Für $a_j \in [0,\infty] \ (j \in \N)$ gilt $\sum_{j=1}^\infty a_j = \infty$, falls (mindestens) ein $a_j = \infty$ ist, oder falls die Reihe in $\R$ divergiert.\\
Da $a_j\ge 0 \ \forall j\in\N$, kann die Reihe umgeordnet werden.

\begin{defn*}
    Ein Mengensystem $\mathcal{M} \subset \PowerSet(X)$ heißt (paarweise) disjunkt, wenn $\overline{M} \cap N \ne \emptyset$ für alle $\overline{M}, N \in \mathcal{M}$ mit $\overline{M} \ne N$. Für disjunkte Mengenvereinigung schreibe $\dcup$ und $\bigdcup$.
\end{defn*}

\begin{defn}
\jlabel{Def 1.12}
    Sei $\mathcal{A}$ eine $\sigma$-Algebra auf $X$. Eine Abbildung $\mu: \mathcal{A} \rightarrow [0, \infty]$ heißt \uline{Maß (auf $\mathcal{A}$)}, wenn gelten:
    \begin{itemize}
        \item[(M1)] $\mu(\emptyset) = 0$
        \item[(M2)] Für jede disjunkte Folge $A_j, j\in\N$ mit $A_j \in \mathcal{A} \ (\forall j\in\N)$ gilt
        \begin{displaymath}
            \mu \left(\bigdcup_{j \in\N}^\infty A_j \right) = \sum_{j=1}^\infty \mu(A_j)\ \ \  (\sigma\text{-Additivität}).
        \end{displaymath}
    \end{itemize}
    Erfüllt $\mu$ (M1) und (M2), dann heißt $(X, \mathcal{A}, \mu)$ ein \uline{Maßraum}.\\
    Wenn $\mu(X) < \infty$, dann heißt $\mu$ \uline{endlich}. Gilt $\mu(X) = 1$, dann heißt $\mu$ \\ \uline{Wahrscheinlichkeitsmaß}.
    
    \jspace
    
    \uline{Bemerkung:} Wenn $A_1, \dots, A_n \in \mathcal{A}$ disjunkt, dann gilt
    \begin{displaymath}
        \begin{split}
            \mu(A_1 \dcup \dots \dcup A_n) &= \mu(A_1 \dcup \dots \cup A_n \dcup \emptyset \cup \emptyset \dcup \dots)\\
            &\overset{(M2)}{=} \mu(A_1) + \dots \mu(A_n) + \mu(\emptyset) + \mu(\emptyset) + \dots\\
            &\overset{(M1)}{=} \mu(A_1) + \dots + \mu(A_n).
        \end{split}
    \end{displaymath}
\end{defn}


\jlabel{Bsp 1.13}
\begin{expl}
    \begin{enumerate}
        \item Sei $\mathcal{A} = \PowerSet(X)$ und $x \in X$ fest. Für $A \subset X$ definiere:
            \begin{displaymath}
                \delta_x(A) := \begin{cases} 1, & \mbox{für } x \in A \\ 0, & \mbox{für } x \notin A \end{cases}.
            \end{displaymath}
            Dann heißt $\delta_x$ \uline{Punktmaß} (Dirac-Maß).\\
            (M1) gilt offensichtlich.\\
            (M2): Seien $A_j \subset X$ disjunkt für $j \in \N$. Dann gilt $x \in \bigdcup_{j \in \N} A_j \Leftrightarrow \exists! k\ \in \N: x\in A_k$.
            \begin{displaymath}
                \begin{split}
                    \Rightarrow \delta_x(\bigdcup_{j \in \N} A_j) &\overset{\text{Def}}{=} \begin{cases} 0, x \notin \bigdcup_{j \in \N} A_j \\ 1, x\in \bigdcup_{j \in \N} A_j \end{cases} = \begin{cases} 0, x \notin A_k\\ 1, x\in A_k \end{cases} \\
                    &= \begin{cases}0, x\notin A_k\\ \delta_x(A_k), x \in A_k\end{cases} = \sum_{j=1}^\infty \delta_x(A_j)
                \end{split}
            \end{displaymath}
            $\Rightarrow \delta_x$ ist Maß.
            
        \item
            Sei $X = \N, \mathcal{A} = \PowerSet(\N)$. Seien $p_k \in [0,\infty]$ für $k\in\N$ gegeben. Setze 
            \begin{displaymath}
                \mu(A) := \sum_{k \in A} p_k\text{ für }A \subset X. 
            \end{displaymath}
            Klar $\mu(\emptyset) = 0$. Seien $A_j \subset \N, j\in\N$ disjunkt. Dann gilt
            \begin{displaymath}
                \mu \left (\bigdcup_{j\in\N} A_j \right) \overset{\text{Def}}{=} \sum_{k \in \bigdcup_{j\in\N} A_j} p_k = \sum_{j=1}^\infty \sum_{k\in A_j} p_k \overset{\text{Def}}{=} \sum_{j=1}^\infty \mu(A_j).
            \end{displaymath}
            $\Rightarrow \mu$ ist Maß. $\mu$ heißt \uline{Zählmaß}, wenn $p_k =1 \ \forall k\in \N$. (Dann $\mu(A) = |A|$)

        \item
            Seien $(X, \mathcal{A}, \mu)$ ein Maßraum, $X_0 \subset X$ und $\mathcal{A}_0$ $\sigma$-Algebra auf $X_0$ mit $\mathcal{A}_0 \subset \mathcal{A}$. Dann definiert $\mu_0(A) := \mu(A)$ (für alle $A\in \mathcal{A}_0$) ein Maß auf $\mathcal{A}_0$.\\
            Für $X_0 \in \mathcal{A}$ setze $\mathcal{A}_0 := \mathcal{A}_{X_0} := \{A \in \mathcal{A}: A\subset X_0\}$ (vgl. \jlink{Lem 1.10}). Dann ist $\mu|_{X_0}$ definiert durch $\mu|_{X_0}(A) = \mu(A)$ für $A\in \mathcal{A}_{X_0}$ ein Maß auf $\mathcal{A}_{X_0}$. $\mu|_{X_0}: \mathcal{A}_{X_0} \rightarrow [0,\infty]$ heißt Einschränkung von $\mu$.
    \end{enumerate}
\end{expl}


\begin{satz}
\jlabel{Satz 1.14}
    Seien $(X, \mathcal{A}, \mu)$ ein Maßraum und $A,B,A_j \in \mathcal{A}$ für $j \in \N$. Dann gelten:
    \begin{enumerate}
        \item Aus $A\subset B$ folgt $\mu(A) \le \mu(B)$. (Monotonie)\\
            Wenn zusätzlich $\mu(A) < \infty$, dann gilt: $\mu(A\backslash B) = \mu(A) - \mu(B)$.\\
            (Speziell: $\mu(A) < \infty \Rightarrow \mu(A^c) = \mu(X) - \mu(A)$)
        \item $\mu(\bigcup_{j=1}^\infty A_j) \le \sum_{j=1}^\infty \mu(A_j)$ ($\sigma$-Subadditivität)
        \item Wenn $A_1 \subset A_2 \subset \dots$, dann gilt
        \begin{displaymath}
            \mu(A_j) \xrightarrow{j\rightarrow\infty} \mu \left(\bigcup_{k \in\N} A_k \right).
        \end{displaymath}

        \item Wenn $A_1 \supset A_2 \supset \dots \text{, und } \mu(A_1) < \infty$, dann gilt
        \begin{displaymath}
             \mu(A_j) \xrightarrow{j\rightarrow \infty} \mu \left(\bigcap_{k \in\N} A_k \right).
        \end{displaymath}
    \end{enumerate}
    \begin{proof}
        \begin{enumerate}
            \item 
                Es gilt: $B= A\dcup B\backslash A$ (beachte: $A\subset B$). Dann folgt 
                \begin{displaymath}
                    \mu(B) = \mu(A) + \mu(B\backslash A) \geq \mu(A).
                \end{displaymath}
            \item
                Setze $B_1 := A_1, \ B_k := A_k \backslash \bigcup_{j=1}^{k-1} A_j$ für $k \geq 2, \ k \in \N$. Dann folgt $B_k \cap B_j = \emptyset \ \forall j < k \Rightarrow \{B_j,\ j\in\N\}$ ist disjunkt.\\
                Ferner gilt $\bigcup_{k=1}^\infty B_k = \bigcup_{k=1}^\infty A_k$, da $B_k \subset A_k$ und jedes $x \in A_k$ in einem $B_j, \ j\in\N$, enthalten ist. Somit gilt
                \begin{displaymath}
                    \mu(\bigcup_{k=1}^\infty A_k) = \mu(\bigdcup_{k\in\N} B_k) \overset{\text{(M2)}}{=} \sum_{k=1}^\infty \mu(B_k) \overset{\text{a)}}{\underset{B_k \subset A_k}{\le}} \sum_{k=1}^\infty \mu(A_k).
                \end{displaymath}
            \item
                Nach Voraussetzung gilt nun in a), dass $B_k = A_k \backslash A_{k-1}, \ k \ge 2$.\\
                Ferner gilt $A_n = \bigdcup_{k=1}^n B_k$. Wie in b) folgt dann
                \begin{displaymath}
                    \begin{split}
                        \mu(\bigcup_{k=1}^\infty A_k) &= \sum_{k=1}^\infty \mu(B_k) = \limToInf{n} \sum_{k=1}^n \mu(B_k)\\
                        &\overset{\text{(M2)}}{=} \limToInf{n} \mu(\bigdcup_{k=1}^n B_k) = \limToInf{n} \mu(A_n).
                    \end{split}
                \end{displaymath}
                Somit folgt c).
        \end{enumerate}
    \end{proof}
\end{satz}
    % Das Lebesgue-Maß
\section{Das Lebesgue-Maß}

\uline{Ansatz:} Für $I=(a,b] \subset \R^d$ setze:
\jlabel{(1.2)}
\begin{equation}
    \lambda(I) := \lambda_d(I) := (b_1-a_1) \cdots (b_d-a_d)
\end{equation}
Setze ferner $\Jd := \{(a,b] \subset \R^d: a \le b\}$. Beachte: $\sigma(\Jd) = \Bd$ (\jlink{Satz 1.9}).

\jspacesmall

\uline{Ziel:} Setze $\lambda_d$ von $\Jd$ auf $\Bd$ fort.

\jdate{31.10.2008}

\subsection*{1. Schritt}

Die Menge der Figuren ist
\begin{displaymath}
    \Fd := \left \{A = \bigcup_{j=1}^\infty I_j \ \text{mit} \ n\in\N, \ I_1, \dots, I_n \in  \Jd \right \}.
\end{displaymath}
Beachte: $\Jd \subset \Fd \subset \sigma(\Jd) = \Bd$. Mit \jlink{Lem 1.6} folgt dann $\sigma(\Fd) = \Bd$.

{\begin{lem}
\jlabel{Lem 1.15}
    Seien $I, I' \in \Jd$. Dann gelten
    \begin{enumerate}
        \item $I \cap I' \in \Jd$.
        \item $I\backslash I'$ ist eine endliche Vereinigung disjunkter Intervalle aus $\Jd$ $\Rightarrow I \backslash I' \in \Fd$.
        \item Jedes $A\in \Fd$ ist eine endliche Vereinigung disjunkter Intervalle aus $\Jd$.
        \item 
            $\Fd$ ist ein Ring, d.h. es gilt für alle $A,B\in \Fd$
            \begin{itemize}
                \item[(R1)] $\emptyset \in \Fd$
                \item[(R2)] $B\backslash A \in \Fd$
                \item[(R3)] $A \cup B \in \Fd$.
            \end{itemize}
    \end{enumerate}
    \begin{proof}
        \begin{enumerate}
            \item 
                Sei $I = (\alpha_1, \beta_1] \times \dots \times (\alpha_d, \beta_d], \ I' = (\alpha_1', \beta_1'] \times \dots \times (\alpha_d', \beta_d']$. Dann folgt $I \cap I' = (\overline{\alpha_1}, \overline{\beta_1}] \times \dots \times (\overline{\alpha_d}, \overline{\beta_d}]$ mit:\\
                $\overline{\alpha_k} = \max\{\alpha_k, \alpha_k'\}, \ \overline{\beta} = \min\{\beta_k, \beta_k'\}$, wobei $I\cap I' = \emptyset$, wenn ein $\overline{\alpha_k} \ge \overline{\beta_k}$. Also $I \cap I' \in \Jd$.
            \item
                (IA): Die Behauptung ist klar für $d=1$.\\
                (IV): Die Behauptung gelte für ein $d \ge 2$.\\
                (IS): Seien $I,I' \in \mathcal{J}_{d+1}$. Dann gibt es $I_1, I_1' \in \mathcal{J}_1$ und $I_2, I_2' \in \Jd$ mit
                \begin{displaymath}
                    \begin{split}
                        &I = I_1 \times I_2, \ I' = I_1' \times I_2'\\
                        &\Rightarrow I\backslash I' = ((I_1\backslash I_1')\times I_2)\cup((I_1 \cap I_1')\times(I_2\backslash I_2')).
                    \end{split}                    
                \end{displaymath}
                Nach (IV) ist dies eine disjunkte Vereinigung $\hat{I}_k \in \mathcal{J}_{d+1}$.
            \item
                (IA): Die Behauptung ist klar, wenn $A=I_1$ für ein $I_1\in \Jd$.\\
                (IV): Für ein $n \in\N$ gelte die Behauptung für alle $A = \bigcup_{j=1}^n I_j$ mit beliebigen $I_j \in \Jd$.\\
                (IS): Sei nun $A = \bigcup_{j=1}^{n+1} I_j$ für beliebige $I_j \in \Jd $\\
                $\overset{\text{(IV)}}{\Rightarrow}$ Es existieren disjunkte $I_1', \dots, I_n' \in \Jd$ mit $\bigcup_{j=1}^n I_j = \bigdcup_{k=1}^m I_k'$.
                \begin{displaymath}
                     \Rightarrow A = I_{n+1} \cup \ \bigdcup_{k=1}^m I_k' = I_{n+1} \cup \ \bigdcup_{k=1}^m \underbrace{(I_k'\backslash I_{n+1})}_{\genfrac{}{}{0pt}{}{\overset{\text{b)}}{=}\text{disjunkte, endliche}}{\text{Vereinigung von $I$ in $\Jd$}}}.
                \end{displaymath}
            \item
                (R1) gilt, da $\emptyset = (a, a] \in \Fd$.\\
                (R3) gilt nach Definition von $\Fd$.\\
                \uline{Zu (R2):} Seien $A,B \in \Fd$, also $A=\bigcup_{j=1}^n I_j, B= \bigcup_{k=1}^m I_k'$ für beliebige $I_j, I_k' \in \Fd, \ n,m \in \N$. Sei $m$ fest aber beliebig. Induktion über n:\\
                (IA): Sei $n=1$. Dann gilt $B\backslash A = \bigcup_{k=1}^m \underbrace{I_k'\backslash I_1}_{\in \Fd \text{nach b)}}$.\\
                (IV): Für ein $n \in\N$ gelte $B\backslash A \in \Fd$ für alle obigen $A$ und $B$.\\
                (IS): Sei nun $A' = \bigcup_{j=1}^{n+1} I_j = A \cup I_{n+1}$ (für $I_j \in \Fd$). Dann gilt6
                \begin{displaymath}
                    B\backslash A' = B\backslash (A \cup I_{n+1}) = \underbrace{(B \backslash A)}_{\in \Fd \text{nach (IV)}} \backslash I_{n+1} \Rightarrow B \backslash A' \in \Fd.
                \end{displaymath}
        \end{enumerate}
    \end{proof}
\end{lem}

\subsection*{Schritt 2: Fortsetzung von $\lambda_d$ aus \jlink{(1.2)} auf $\Fd$}

\uline{Idee:} TODO BILD

\begin{lem}
\jlabel{Lem 1.16}
    Seien $A = \bigdcup_{j=1}^n I_j = \bigdcup_{k=1}^m I_k'$ für disjunkte $I_j \in \Jd \ (j=1, \dots, n)$ und disjunkte $I_k' \in \Jd \ (k=1,\dots,m)$. Dann gilt
    \begin{displaymath}
        \sum_{j=1}^n \lambda_d(I_j) = \sum_{k=1}^m \lambda_d(I_k').
    \end{displaymath}
    \begin{proof}
        1) Sei $d=2, \ I = (a,b] \times (c,d], \ \alpha \in (a,b]$. Dann folgt $I=((a,\alpha] \times (c,d])\cup ((\alpha,b] \times (c,d]) = I' \cup I''$.\\
        Ferner $\lambda(I) \overset{\jlink{(1.2)}}{=} (b-a)\cdot (d-c) = ((b-\alpha) + (\alpha - a))\cdot (d-c) \overset{\jlink{(1.2)}}{=} \lambda(I') + \lambda(I'')$.\\
        Genauso: Dies gilt auch für $d \ge 3$ und für Zerlegungen in der der $k$-ten Koordinate. Per Induktion folgt: Wenn man ein $I \in \Jd$ mit endlich vielen Zwischenstellen $\alpha_1,\dots,\alpha_n$ in Intervalle $\tilde{I}_1,\dots, \tilde{I}_l$ zerlegt, dann gilt: $\lambda_d(I) = \lambda_d(\tilde{I}_1) + \dots + \lambda_d(\tilde{I}_l) $.
        
        \jspace
        
        TODO: BILD
        
        \jspace
        
        2) Setze $I_{jk}'' = I_j \cap I_k' \in \Jd \ (j=1,\dots,n, \ k=1,\dots,m)$. Die $I_{jk}''$ sind per Definition disjunkt und $I_j = \bigcup_{k=1}^m I_{jk}'',\ I_k' = \bigcup_{j=1}^n I_{jk}''$. $(*)$\\
        Zerlege alle $I_{jk}''$ weiter durch Schneiden mit allen Hyperebenen, auf denen Seiten eines der $I_{jk}''$ liegen.\\
        Erhalte dabei disjunkte $\hat{I}_1, \dots, \hat{I}_l \in \Jd$, wobei jedes $\hat{I}_i$ in genau einem $I_{jk}''$ und damit in genau einem $I_j$ und genau einem $I_k'$ liegt. Weiter werden alle $I_j$ und alle $I_k'$ durch die jeweils in ihnen liegenden $\hat{I}_k$ wie in 1) zerlegt. Damit gilt:
        \begin{displaymath}
            \sum_{j=1}^n \lambda(I_j) \overset{\text{1)}}{=} \sum_{j=1}^n \sum_{i: \hat{I}_i \subset I_j} \lambda(\hat{I}_j) = \sum_{i=1}^l \lambda(\hat{I}_i) = \sum_{k=1}^m \sum_{j: \hat{I}_j \subset I_k} \lambda(\hat{I}_j) \overset{1)}{=} \sum_{k=1}^m \lambda(I_k')
        \end{displaymath}
    \end{proof}
\end{lem}

Für $A\in\Fd$ setze
\jlabel{(1.3)}
\begin{equation}
    \lambda(A) := \lambda_d(A) := \sum_{j=1}^n \lambda_d(I_j),
\end{equation} 
wobei $A = \bigcup_{j=1}^n I_j$ für disjunkte $I_1, \dots, I_n \in \Jd$. Nach \jlink{Lem 1.16} definiert dies eine Abbildung $\lambda_d : \Fd \rightarrow \R_+$. Seien $A=\bigcup_{j=1}^n I_j, \ B = \bigcup_{k=1}^m I_k'$ für disjunkte $I_j, I_k' \in \Jd$ und es sei $A\cap B = \emptyset$. Setze
\begin{displaymath}
    I_i'' := \begin{cases} I_i, & i=1,\dots,n\\ I_i', &i = n+1,\dots,n+m\end{cases}.
\end{displaymath}
Dann sind die $I_j''$ disjunkt und es folgt
\begin{displaymath}
    \lambda_d(A \cup B) \overset{\jlink{(1.3)}}{=} \sum_{i=1}^{n+m} \lambda_d(I_j'') = \sum_{j=1}^n \lambda_d(I_j) + \sum_{k=1}^m \lambda_d(I_k') \overset{\jlink{(1.3)}}{=} \lambda_d(A) + \lambda_d(B).
\end{displaymath}

Per Induktion folgt für disjunkte $A_1,\dots,A_n \in \Fd$, dass
\jlabel{(1.4)}
\begin{equation}
    \lambda_d(A \dcup \dots \dcup A_n) = \lambda_d(A_1) + \dots + \lambda_d(A_n).
\end{equation}
Weiter gilt nach \jlink{Lem 1.15} für $A,B\in\Fd$ und ein $I \in J_d$ mit $A,B \subset I$, dass
\jlabel{(1.5)}
\begin{equation}
    \begin{split}
        A\cap B &= I\cap((I^c\cup A)\cap (I^c\cup B)) = I\cap ((I\cap A^c)\cup(I\cap B^c)^c \\
                &= I\backslash ((I\backslash A) \cup (I\backslash B)) \in \Fd
    \end{split}
\end{equation}
Wenn $A\subset B$, dann gilt
\jlabel{(1.6)}
\begin{equation}
    \lambda_d(A) \le \lambda_d(B).
\end{equation}
(Beweis genau wie in 1.14a)) \\
Außerdem gilt
\jlabel{(1.7)}
\begin{equation}
    \begin{split}
        \lambda_d(A\cup B) &= \lambda_d(A\cup B \backslash A)\\
        &\overset{\jshortlink{Satz 1.14}}{=} \lambda_d(A) + \lambda_d(B\backslash A) \overset{\jlink{(1.6)}}{\le} \lambda_d(A) + \lambda_d(B).
    \end{split}
\end{equation}

\begin{satz}
\jlabel{Satz 1.17}
    Die Abbildung $\lambda_d: \Fd \rightarrow \R_+$ ist ein \uline{Prämaß} auf dem Ring $\Fd$, d.h es gelten:
    \begin{itemize}
        \item[(M1)] $\lambda_d(\emptyset) = 0$
        \item[(M2*)] Für disjunkte $A_j\in\Fd, \ j \in N$ mit $A:=\bigdcup_{j=1}^\infty A_j \in \Fd$ gilt
            \begin{displaymath}
                \lambda_d(A) = \sum_{j=1}^\infty \lambda_d(A_j).
            \end{displaymath}
    \end{itemize}
    \begin{proof}
        (M1) folgt aus \jlink{(1.2)}, da $\emptyset = (a,a]$.
        \begin{itemize}
            \item[1)]
                \uline{Beh1}: Seien $B_n \in \Fd$ mit $B_{n+1} \subset B_n \ (n\in\N)$ und $\bigcap_{n \in\N} B_n = \emptyset$. Dann gilt $\lambda_d(B_n) \xrightarrow{n \rightarrow \infty} 0$.\\
                (Übung: 2.1, Beh1, \jlink{Lem 1.15} und \jlink{(1.4)} $\Rightarrow$ \jlink{Satz 1.17})
            \item[2)]
                \uline{Beweis von Beh1}:\\
                Sei $\epsilon > 0$. Dann gilt $\forall n \in \N \ \exists C_n \in\Fd$ mit $\overline{C_n} \subset B_n \subset B_1$ und
                \begin{displaymath}
                    \lambda_d(\underbrace{B_n \backslash C_n}_{\in \Fd}) \le 2^{-n}\cdot\epsilon.
                \end{displaymath}
                (Ersetze in allen Teilintervallen von $B_n$ der Form $(a_1,b_1]\times \dots \times (a_d, b_d]$ $a_j$ durch $a_j + \delta_n(\epsilon)$ für ein genügend kleines $\delta_n(\epsilon) \ge 0$ und $\delta_n(\epsilon) = 0$ falls ein $a_j=b_j$)
                
                Da weiterhin $\bigcap_{n\in\N} \overline{C_n} = \emptyset$, gilt $\bigcup_{n=1}^\infty \overline{C_n}^c =  \R^d \Rightarrow \{\overline{C_n}^c,\ n\in\N\}$ ist eine offene Überdeckung von $\overline{B_1}$, wobei $\overline{B_1}$ beschränkt und abgeschlossen ist. Dann folgt mit Heine-Borel: $\exists n_1 < \dots < n_m$ mit $\overline{B_1} \subset \overline{C_{n_1}}^c \cup \dots \cup \overline{C_{n_m}}^c \Rightarrow \overline{C_{n_1}} \cap \dots \cap \overline{C_{n_m}} \subset \overline{B_1}^c$. Mit $C_{n_j} \subset B_1$ folgt dann $\overline{C_{n_1}} \cap \dots \cap \overline{C_{n_m}} = \emptyset \Rightarrow \bigcap_{j=1}^n \overline{C_j} = \emptyset \ \forall n\ge n_m =: N_\epsilon \ (*)$\\
                Setze $D_n := \bigcap_{j=1}^n C_j \in \Fd$ (nach \jlink{(1.5)}), $n\in\N$.
                
                \vspace{12pt}
                
                \uline{Beh2}: $\lambda_d(B_n\backslash D_n) \le (1-2^{-n})\cdot \epsilon \ (\forall n\in\N)$\\
                Nach $(*)$ gilt: $D_n = \emptyset$ für $n\ge N_\epsilon$.  Beh2 zeigt: $\lambda_d(B_n) = \lambda_d(B_n\backslash D_n) \le (1-2^{n})\cdot \epsilon < \epsilon \ \forall n \ge N_\epsilon$. Damit ist der Beweis von \jlink{Satz 1.17} erbracht.
            
            \item[3)] Beweis von Beh2:\\
                (IA): Beh2 gilt für $n=1$ nach der Ungleichung zu Beginn von 2).\\
                (IV): Beh2 gelte für ein $n\in\N$.\\
                (IS): Es gilt mit $D_{n+1} = D_n\cap C_{n+1}$
                \begin{displaymath}
                    \begin{split}
                        \lambda_d(B_{n+1}\backslash D_{n+1}) &= \lambda_d(B_{n+1}\backslash (D_n \cap C_{n+1})\\
                        &= \lambda((B_{n+1} \backslash D_n) \cup (B_{n+1}\backslash C_{n+1}))\\
                        &\overset{\jlink{(1.7)}}{\le} \lambda(B_{n+1} \backslash D_n) + \lambda(B_{n+1} \backslash C_{n+1})\\ 
                        &\overset{\jlink{(1.6)}}{\le} \lambda(B_n\backslash D_n) + \lambda(B_{n+1} \backslash C_{n+1})\\
                        &\overset{\text{(IV)}}{\le} (1-2^{-n})\cdot \epsilon + 2^{-(n+1)}\epsilon = (1-2^{-(n+1)})\cdot \epsilon.
                    \end{split}
                \end{displaymath}                
        \end{itemize}
    \end{proof}
\end{satz}

\jdate{03.11.2008}

\subsection*{Schritt 3: Fortsetzung von $\lambda_d$ auf $\Bd$}

\begin{thm}[Caratheodory, Fortsetzungssatz, 1914]
\jlabel{Thm 1.18}
    Sei $\mathcal{R} \subset \PowerSet(X)$ ein Ring und $\mu: \mathcal{R} \rightarrow [0,\infty]$ ein Prämaß.\\
    Dann existieren eine $\sigma$-Algebra $\mathcal{A}(\mu)$ auf $X$ und ein Maß $\overline{\mu}$ aud $\mathcal{A}(\mu)$, sodass $\sigma(\mathcal{R}) \subset \mathcal{A}(\mu)$ und $\mu(A) = \overline{\mu}(A)$ für alle $A\in \mathcal{R}$ gelten. Also ist $\overline{\mu}$ ein Maß auf $\sigma(\mathcal{R})$ (vgl. Beispiel 1.13c)).
\end{thm}

\begin{thm}[Eindeutigkeitssatz]
\jlabel{Thm 1.19}
    Seien $\mathcal{E} \subset \PowerSet(X), \ \mathcal{A} = \sigma(\mathcal{E})$ und $\mu,\nu$ Maße auf $\mathcal{A}$ mit $\mu(E)=\nu(E) \ \forall E \in \mathcal{E}$. Weiter gelte:
    \begin{itemize}
        \item[A)] $E,F \in \mathcal{E} \Rightarrow E\cap F \in \mathcal{E}$ ($\cap$-stabil)
        \item[B)] $\exists E_n \in \mathcal{E}$ mit $\mu(E_n) < \infty,\ E_n \subset E_{n+1} \ \forall n\in\N$, und $\bigcup_{n=1}^\infty E_n = X$
    \end{itemize}
    Dann gilt $\mu = \nu$ (auf $\mathcal{A}$).
\end{thm}

\begin{bem*}
    \begin{enumerate}
        \item
            B) ist nötig.\\
            \uline{Bsp}: Seien $\mu,\nu$ Maße auf X mit $\mu(X)=1,\ \nu(X) = 0$, $\mathcal{E} = \{\emptyset\} \Rightarrow \sigma(\mathcal{E}) = \{\emptyset, X\} \Rightarrow \mu \neq \nu$, aber $\mu(\emptyset) =  \nu(\emptyset)$, d.h A) gilt.
        \item
            A) ist nötig.\\
            \uline{Bsp}: Seien $X=\{a,b,c,d\}, \mathcal{A}=\PowerSet(X)=\sigma(\mathcal{E}),$\\
            $\mathcal{E}=\{X, \{a,b\},\{a,c\},\{b,d\}\}$ und $\mu,\nu$ auf $\PowerSet(X)$ gegeben durch:
            \begin{quote}
                $\mu(\{a\})=\mu(\{d\})=\nu(\{b\})=\nu(\{c\})=1$\\
                $\mu(\{b\})=\mu(\{c\})=\mu(\{a\})=\mu(\{d\})=2$
            \end{quote}
            $\Rightarrow \mu\ne\nu$, aber $\mu(X)=\nu(X)=6,\ \mu(E)=\nu(E)\ \forall E\in \mathcal{E} \backslash \{X\}$, d.h. B) gilt ohne Monotonie.
    \end{enumerate}
\end{bem*}


\begin{thm}
\jlabel{Thm 1.20}
    Es gibt genau eine Fortsetzung von $\lambda_d$ aus \jlink{(1.2)} auf $\Bd$. Man schreibt $\lambda_d$ (oder $\lambda$) für diese Fortsetzung und nennt sie Lebesgue-Maß.
    \begin{proof}
        Aus \jlink{Lem 1.15} und \jlink{Satz 1.17} folgt: $\lambda_d$ aus \jlink{(1.2)} hat eine Fortsetzung zu einem Prämaß $\lambda_d$ auf dem Ring $\Fd$. Da $\Jd \subset \Fd \subset \Bd$, liefern \jlink{Satz 1.9} und \jlink{Lem 1.6}, dass $\sigma(\Fd)=\sigma(\Jd)=\Bd$. Aus \jlink{Thm 1.18} folgt dann die Existenz der Fortsetzung von $\lambda_d$ auf $\Bd$. Ferner folgt aus \jlink{(1.5)}, dass $\Fd$ $\cap$-stabil ist. Da die Folge $E_n := (-n,n]^d$ B) aus \jlink{Thm 1.19} erfüllt (wegen \jlink{(1.2)}), liefert \jlink{Thm 1.19} die Eindeutigkeit der Fortsetzung.
    \end{proof}
\end{thm}


\jlabel{Bem 1.21}
\begin{bem}
    \begin{enumerate}
        \item Sei $\emptyset \neq X \in \Bd$. Gemäß Beispiel 1.13 und Korollar 1.11 definiert die Einschränkung von $\lambda_d$ auf $\Borel(X)=\{\mathcal{A} \subset X: \mathcal{A} \in \Bd \}\subset \Bd$ ein Maß, das wir auch mit $\lambda_d$ bezeichnen und \uline{Lebesque-Maß} nennen.
        \item $\lambda_1([a,b]) = \lambda_1(\bigcap_{n=1}^\infty (a-\frac{1}{n},b]) \overset{\jshortlink{Satz 1.14}}{=} \limToInf{n} \lambda_1(\underbrace{(a-\frac{1}{n}, b]}_{\overset{\text{1.2}}{=} b-a+\frac{1}{n}}) = b-a$\\
        (Entsprechend für $d\ge2$ und andere Intervalltypen.)\\
        Sei $\Q = \{q_n : n\in\N\}$\\$\Rightarrow \lambda_1(\Q) = \lambda_1(\bigcup_{n\in\N} \{q_n\}) \overset{\jshortlink{Def 1.12}}{=} \sum_{n=1}^\infty \underbrace{\lambda_1([q_n, q_n])}_{=0} = 0$
        \item Sei $H := \{x \in \R^d : x_d = 0\} \Rightarrow H$ ist abgeschlossen, also $H\in \Bd$. Da $H=\bigcup_{n=1}^\infty ([-n,n]^{d-1} \times \{0\})$ und $\lambda_d([-n,n]^{d-1} \times \{0\}) = 0$ (vgl. b)), gilt $\lambda_d(H) = \lambda_d(\bigcup_{n=1}^\infty ([-n,n]^{d-1} \times \{0\}) \overset{\jshortlink{Satz 1.14}}{=} \limToInf{n} \lambda_d([-n,n]^{d-1} \times \{0\}) = 0$
    \end{enumerate}
\end{bem}
\uline{Zum Fortsetzungssatz}\\
Sei $\mu: \mathcal{R} \rightarrow [0,\infty]$ ein Prämaß auf dem Ring $\mathcal{R}$ und $A\subset X$. Setze
\jlabel{(1.8)}
\begin{equation}
    \mu^*(A) := \inf \left \{\sum_{k=1}^\infty \mu(B_k) : B_k \in \mathcal{R} \text{ für } n\in\N,\ A \subset \bigcup_{k=1}^\infty B_k \right \}    
\end{equation}
(Dabei ist $\inf \emptyset := \infty$.)\\
\uline{Ferner}:
\jlabel{(1.9)}
\begin{equation}
    \mathcal{A}(\mu) := \{A \subset X: \forall B \subset X \text{ gilt }\mu^*(B)\ge \mu^*(A\cap B) + \mu^*(A^c\cap B) \}
\end{equation}

\begin{lem}
\jlabel{Lem 1.22}
    $\mu^*$ ist ein \uline{äußeres Maß}, d.h.:
    \begin{enumerate}
        \item $\mu^*(\emptyset) = 0$
        \item $A\subset B \subset X \Rightarrow \mu^*(A) \le \mu^*(B)$
        \item $A_j \subset X, \ j\in\N \Rightarrow \mu^*\left(\bigcup_{j=1}^\infty A_j \right) \le \sum_{j=1}^\infty \mu^*(A_j)$
    \end{enumerate}
    \begin{proof}
        \begin{enumerate}
            \item folgt mit $B_1 = B_2 = \dots = \emptyset$.
            \item gilt, da in \jlink{(1.8)} die $B_k$ für $B$ auch für $A\ (\subset B)$ genommen werden können.
            \item Die Behauptung gilt, wenn ein $\mu^*(A_j)=\infty$. Andernfalls wähle $\epsilon>0$. Dann folgt mit \jlink{(1.8)}: 
            \begin{displaymath}
                \exists B_{jk} \in \mathcal{R} \ (j,k \in \N)\text{ mit }A_j \subset \bigcup_{k=1}^\infty B_{jk},
            \end{displaymath}
            \begin{displaymath}
                \begin{split}
                    \mu^*(A_j) \ge \sum_{k=1}^\infty \mu(B_{jk}) -2^{-j}\cdot \epsilon \Rightarrow \bigcup_{j=1}^\infty A_j \subset \bigcup_{j,k=1}^\infty B_{jk}
                \end{split}
            \end{displaymath}
            und
            \begin{displaymath}
                \mu^*(\bigcup_{j=1}^\infty A_j) \overset{\jlink{(1.8)}}{\le} \sum_{j,k=1}^\infty \mu^*(B_{jk}) \le \sum_{j=1}^\infty (\mu^*(A_j) + 2^{-j}\cdot \epsilon) = \sum_{j=1}^\infty \mu^*(A_j) + \epsilon.
            \end{displaymath}
            Grenzwertbildung für $\epsilon \rightarrow 0$ lieft die Behauptung.
        \end{enumerate}
    \end{proof}
\end{lem}

\begin{lem}
\jlabel{Lem 1.23}
    $\mathcal{A}(\mu)$ ist eine $\sigma$-Algebra und die Einschränkung $\overline{\mu}$ von $\mu^*$ auf $\mathcal{A}(\mu)$ ist ein Maß.
\end{lem}

\begin{proof}[Beweis von \jlink{Thm 1.18}]
        Sei $\mathcal{A} \in \mathcal{R}$.
        \begin{itemize}
            \item[1)]
                Da $A\subset A\cup \emptyset \cup \emptyset \ \dots$, gilt $\mu^*(A) \le \mu(A)$.\\
                Wenn $\mu^*(A)=\infty$, dann gilt $\mu^*(A) = \mu(A)$. Sei also $\mu^*(A) < \infty$.\\
                Wähle $\epsilon > 0$. Dann existieren $A_j\in \mathcal{R}\ (j\in\N)$ mit
                \begin{displaymath}
                    A\subset \bigcup_{j=1}^\infty A_j\text{ und }\sum_{j=1}^\infty \mu(A_j) \le \mu^*(A) + \epsilon.
                \end{displaymath}

                Ferner gilt
                \begin{displaymath}
                    A = A \cap \bigcup_{j=1}^\infty A_j = \bigcup_{j=1}^\infty (\underbrace{A\cap A_j}_{\in \mathcal{R} \text{ nach \jshortlink{Def 1.5}}}).
                \end{displaymath}
                Damit folgt
                \begin{displaymath}
                    \mu(A) \overset{\text{wie \jshortlink{Satz 1.14}}}{\le} \sum_{j=1}^\infty \mu(A\cap A_j) \overset{\text{wie \jlink{Satz 1.14}}}{\le} \sum_{j=1}^\infty \mu(A_j) \le \mu^*(A) + \epsilon.
                \end{displaymath}
                Mit $\epsilon \rightarrow 0$ folgt dann $\mu(A) = \mu^*(A) \ \forall A\in \mathcal{R}$.
            \item[2)]
                \uline{Zeige}: $A\in \mathcal{A}(\mu)$.\\
                Denn dann folgt mit \jlink{Lem 1.6} $\sigma(\mathcal{R}) \subset \mathcal{A}(\mu)$.
                
                \jspacesmall
                
                Sei $B \subset X$. Wenn $\mu^*(B)=\infty$, erfüllen $A$ und $B$ die Ungleichung in \jlink{(1.9)}. Sei also $\mu^*(B) < \infty$. Wähle $\epsilon >0 \Rightarrow \exists A_j \in \mathcal{R}\ (j \in \N)$ mit 
                \begin{displaymath}
                    B \subset \bigcup_{j=1}^\infty A_j\text{ und }\sum_{j=1}^\infty \mu(A_j) \le \mu^*(B) + \epsilon.
                \end{displaymath}
                Daraus folgt $B\cap A \subset \bigcup_{j=1}^\infty \underbrace{A_j \cap A}_{\in \mathcal{R}}$. Nun gilt außerdem
                \begin{displaymath}
                    B \cap A^c \subset \bigcup_{j=1}^\infty \underbrace{A_j \cap A^c}_{\in \mathcal{R}}.\ \ \ (*)
                \end{displaymath}
                Daraus folgt
                \begin{displaymath}
                    \begin{split}
                        \epsilon + \mu^*(B) &\ge \sum_{j=1}^\infty \mu(A_j) = \sum_{j=1}^\infty (\mu(A_j \cap A) + \mu(A_j \cap A^c))\\
                        &\overset{(*)}{\underset{\jlink{(1.8)}}{\ge}} \mu^*(B\cap A) + \mu^*(B\cap A^c).
                    \end{split}
                \end{displaymath}
                 Mit $\epsilon \rightarrow 0$ folgt \jlink{(1.9)}, also $A \in \mathcal{A}(\mu) \ (\forall A\in \mathcal{R})$.
        \end{itemize}
\end{proof}


\begin{satz}
\jlabel{Satz 1.24}
    Sei $x \in \R^d$ und $A\in \Bd$. Dann gelten:
    \begin{enumerate}
        \item $x + A \in \Bd$
        \item $\lambda_d(A) = \lambda_d(x + A)$
        \item 
            Wenn $\mu$ ein Maß auf $\Bd$ ist, dass b) erfüllt, dann:\\
            $\mu(B) = \mu((0,1]^d)\cdot \lambda_d(B) \ \forall B \in \Bd$
    \end{enumerate}
    \begin{proof}
        \begin{enumerate}
            \item
                Seien $x \in \R^d, \ A\in \Bd$ fest. Setze $\mathcal{A} := \{B\in \Bd \text{ mit } x + B \in \Bd\}$.\\
                \uline{Zeige}: $A\in\mathcal{A}$. \uline{Klar}: $\Jd \subset \mathcal{A},\ \R^d \in \Bd$.\\
                Wenn $B\in\mathcal{A}$, dann $x+B^c = \{y\in \R^d : y = x+d \text{ für ein } d \notin B\} \in \Bd \Rightarrow \bigcup_{j=1}^\infty B_j \in \mathcal{A} \Rightarrow \mathcal{A}$ ist $\sigma$-Algebra.\\
                Wenn $B_j \in \mathcal{A} \ (j\in\N)$, dann $x+B_j \in \Bd \Rightarrow \bigcup_{j=1}^\infty (x+B_j) = x + \bigcup_{j=1}^\infty B_j \in \Bd \Rightarrow \bigcup_{j=1}^\infty \in \mathcal{A} \Rightarrow \mathcal{A}$ ist eine $\sigma$-Algebra.\\
                \jlink{Lem 1.6} sagt uns: $\Bd = \sigma(\Jd)\subset \mathcal{A}$. Damit folgt $A\in \mathcal{A}$.
                
            \item
                Sei $x \in \R^d$ fest. Setze $\mu(B) = \lambda_d(x+B) \ \forall B\in \Bd \Rightarrow \mu(\emptyset) = \lambda_d(\emptyset) = 0 \Rightarrow$ (M1).\\
                Seien $B_j \in \Bd$ disjunkt $(j\in\N)$. Dann gilt:\\
                $\mu(\bigcup_{j\in\N} B_j) = \lambda_d(x + \bigcup_{j\in\N} B_j = \lambda_d(\bigcup_{j\in\N} (x + B_j)) $\\
                $\overset{\text{$\lambda_d$ ist Maß}}{=} \sum_{j=1}^\infty \lambda_d(x + B_j) = \sum_{j=1}^\infty \mu(B_j) \Rightarrow$ (M2) gilt für $\mu$.\\
                Sei $I\in \Jd \Rightarrow \mu(I) = \lambda_d(x+I) \overset{\jlink{(1.2)}}{=} \lambda_d(I) \Rightarrow \mu(I) = \lambda_d(I) \ \forall I \in \Jd$. Da $\Jd$ A), B) in \jlink{Thm 1.19} erfüllt und $\Bd = \sigma(\Jd)$, folgt mit \jlink{Thm 1.19}, dass $\mu = \lambda_d$ auf $\Bd$ gilt.
                
            \item
                (Skizze für d=1). Sei $\mu$ wie in Behauptung c) und $c:= \mu((0,1])\in[0,\infty)$. Dann gilt $c = \mu((0, \frac{1}{2}]) + \mu((\frac{1}{2},1]) \overset{\text{nach Vor.}}{=} 2\cdot \mu((0,\frac{1}{2}])$\\
                $\Rightarrow \mu((0,\frac{1}{2}]) = c\cdot\lambda_1((0,\frac{1}{2}])$\\
                Induktiv zeigt man: $\mu((0,2^{-n}]) = c\cdot\lambda_1((1, 2^{-n}])$ für alle $n\in\N$.\\
                Durch Verschieben und disjunkte Vereinigungen folgt $\mu(I) = c\cdot \lambda_1(I)$ für alle Intervalle der Form $I=(a,b]$ mit $a,b = m\cdot 2^n$ für gewisse $m,n \in \Z$\\
                Das System dieser Intervalle erzeugt $\Borel_1$ (Beweis von \jlink{Satz 1.9}) und erfüllt A), B) in \jlink{Thm 1.19}. Damit folgt die Behauptung.
        \end{enumerate}
    \end{proof}
\end{satz}

\jdate{07.11.2008}

\begin{thm}
\jlabel{Thm 1.25}
    $\lambda_d$ ist \uline{regulär}, d.h.: $\forall A \in \Bd$ gelten:
    \begin{enumerate}
        \item $\lambda_d(A) = \inf \{ \lambda_d(O) : O \text{ offen, } A\subset  O\}$
        \item $\lambda_d(A) = \sup \{ \lambda_d(K) : K \text{ kompakt, } K\subset A\}$
    \end{enumerate}
    \begin{proof}
         \begin{enumerate}
             \item 
                ``$\le$`` folgt aus der Monotonie von $\lambda_d$\\
                ```$=$` klar. wenn $\lambda_d(A) = \infty$. Sei also $\lambda_d < \infty$. Wähle $\epsilon > 0$. Nach \jlink{(1.8)} gilt:\\
                $\exists I_j\in \Jd\ (j\in\N)$ mit $A \subset \bigcup_{j=1}^\infty I_j,\ \sum_{j=1}^\infty \lambda_d(I_j) \le \underbrace{\lambda_d(A)}_{= \lambda_d^*(A)} + \epsilon \ (*)$\\
                Wie im Beweis von \jlink{Satz 1.17} findet man offene $O_j$ mit $I_j \subset O_j$ und $\lambda_d(O_j) \le \lambda_d(I_j) + 2^{-j} \cdot \epsilon \ (\forall j\in \N) \ (**)$\\
                $\Rightarrow O := \bigcup_{j=1}^\infty O_j$ ist offen und $A \overset{(*)}{\subset} \bigcup_{j=1}^\infty O_j$.\\
                $\Rightarrow \lambda_d(O) \overset{\jshortlink{Satz 1.14}}{\le} \sum_{j=1}^\infty \lambda_d(O_j) \overset{(**)}{\le} \sum_{j=1}^\infty \lambda_d(I_j) + \underbrace{\sum_{j=1}^\infty 2^{-j}}_{= 2} \cdot \epsilon \overset{(*)}{\le} \lambda_d(A) + 2 \cdot \epsilon$.\\
                Mit $\epsilon \rightarrow 0$ folgt a).
                
            \item
                ``$\ge$`` folgt aus der Monotonie von $\lambda_d$. Sei A $\in \Bd$.
                    \begin{itemize}
                        \item[1)]
                            Sei zuerst $A\subset \overline{B}(0,r) =: B$ für ein $r>0$. Sei $\epsilon > 0$. Nach a) für $B\backslash A$:\\ 
                            $\exists$ offenes $O$ mit $B\backslash A \subset O$ und $\lambda_d(0) \le \lambda_d(B\backslash A) + \epsilon \overset{\jshortlink{Satz 1.14}}{=} \lambda_d(B) - \lambda_d(B) + \epsilon \ \ (+)$\\
                            Daraus folgen:
                            \begin{itemize}
                                \item $K := B\backslash O = B \cap O^c$ ist abgeschlossen und beschränkt, also kompakt.
                                \item $K \subset B \cap (B\backslash A)^c = B \cap (B\cap A^c)^c = A$
                                \item $\lambda_d(B) \overset{B \subset K\cup B}{\le} \lambda_d(K \cup O) \overset{\jshortlink{Satz 1.14}}{\le} \lambda_d(K) + \lambda_d(O) \overset{(+)}{\le} \lambda_d(K) + \lambda_d(B) - \lambda_d(A) + \epsilon$
                            \end{itemize}
                            $\overset{\text{alles in } \R}{\Rightarrow} \lambda_d(A) \le \lambda_d(K) + \epsilon$. Damt folgt b) für beschränkte A.
                        
                        \item[2)]
                            Sei $A\in \Bd$ beliebig. Setze $A_n := A\cap \overline{B}(0,n) \ (n\in\N)$\\
                            $\Rightarrow A_n \subset A_{n+1} \ \ (\forall n\in\N),\ \bigcup_{j=1}^\infty A_n = A$. Mit 1) folgt:\\
                            $\exists K_n$ kompakt, mit $K_n\subset A_n$ und $\lambda_d(A_n) \le \lambda_d(K_n) + \frac{1}{n}$.\\
                            Durch Grenzwertbildung für $n\rightarrow \infty$ folgt mit \jlink{Satz 1.14}:\\
                            $\lambda_d(K_n) \xrightarrow{n \rightarrow \infty} \lambda_d(A)$, weiter $K_n \subset A$.
                    \end{itemize}
         \end{enumerate}
    \end{proof}
\end{thm}

\begin{bem*}
    Der Beweis von \jlink{Thm 1.25}a) zeigt, dass man $O$ als eine Vereinigung offener Intervalle nehmen darf.
\end{bem*}

\begin{auswahlaxiom}
    Sei $M$ eine nichtleeres System nichtleerer Mengen $A\subset X$.\\
    Dann gibt es eine Abbildung $\phi: M \rightarrow \bigcup_{A\in M} A \subset X$ mit $\phi(A) \in A \ \forall A\in M$.
\end{auswahlaxiom}

\begin{satz}
\jlabel{Satz 1.26}
    $\exists \Omega \in \PowerSet(\R^d) \backslash \Bd$.
    \begin{proof}
        Betrachte auf $(0,1]^d$ die Äquivalenzrelation gegeben durch $X\sim Y :\Leftrightarrow x-y \in \Q^d$. Sei $\Omega := \{\phi(A) : A \in M\}$, wobei $M$ die Menge der Äquivalenzklasse zu $\sim$ ist und $\phi$ aus dem Auswahlaxiom. Damit folgt: $\Omega \subset (0,1]^d$.\\
        Sei $\{q_1, q_2, \dots\} := \Q^d \cap [-1,1]^d$. Dann folgt
        \begin{displaymath}
            (0,1]^d \subset \bigcup_{n=1}^\infty (q_n + \Omega) \subset [-1,2]^d. \ \ \ (*)
        \end{displaymath}
        Diese Vereinigung ist disjunkt, da jedes $x\in (0,1]^d$ in genau einer Äquivalenzklasse liegt.
        
        \jspacesmall
        
        \uline{Annahme}: $\Omega \in \Bd$.\\
            Dann $3^d = \lambda_d([-1,2]^d) \overset{\sigma\text{-Subadd.}}{\ge} \lambda_d(\bigcup_{n\in\N} q_n + \Omega) \overset{\text{(M2)}}{=} \sum_{n=1}^\infty \lambda_d(q_n + \Omega) \overset{\jshortlink{Satz 1.24}}{=} \sum_{n=1}^\infty \lambda_d(\Omega) \Rightarrow \lambda_d(\Omega) = 0$
            
        \jspacesmall
            
        \uline{Aber}: $1 = \lambda_d((0,1]^d) \overset{(*)}{\le} \lambda_d(\bigcup_{n=1}^\infty (q_n + \Omega)) \overset{\text{wie oben}}{=} \sum_{n=1}^\infty \lambda_d(\Omega) = 0$, was ein Widerspruch ist.
    \end{proof}
\end{satz}


\chapter{Messbare Funktionen und das Lebesgue-Integral}

    TODO: Einleitung mit Bildern

    % Messbare Funktionen
\section{Messbare Funktionen}

\begin{defn}
\jlabel{Def 2.1}
\jlabel{messbar}
    Sei $\mathcal{A}$ eine $\sigma$-Algebra auf $X\ne\emptyset$ und $\Borel$ eine $\sigma$-Algebra auf $Y\ne \emptyset$. Eine Abbildung $f:X \rightarrow Y$ heißt \uline{\calMb{A}{B}}, wenn $f^{-1}(B) \in \mathcal{A}\ \forall B\in \Borel$.
\end{defn}

\jlabel{Bem 2.2}
\begin{bem}
    \begin{enumerate}
        \item
            Sei $f: X \rightarrow Y$ \calMb{A}{B}. Dann ist $f$ \calMb{A'}{B'} für jede $\sigma$-Algebra $\mathcal{A}'$ auf $X$ mit $A\subset \mathcal{A'}$ und $\mathcal{B'}$ auf $Y$ mit $\mathcal{B'} \subset \Borel$.\\
            Weiter ist die Einschränkung $\jabb{f|_{X_0}}{X_0}{Y}$ für jedes $X_0 \in \mathcal{A}$ \mb{$\mathcal{A}_{X_0}$}{$\Borel$} (vgl. \jlink{(1.1)}).
            
        \item Wenn $A = \PowerSet(X)$ oder $\Borel = \{\emptyset, X\}$, dann ist $\jabb{f}{X}{Y}$ \calMb{A}{B}.
        \item Sei $A\subset X$. Setze $\doubleOne_A(x) := \begin{cases} 1 &, x\in A \\ 0 &, x \notin A \end{cases}$. Sei $B \in \Bd$. Dann gilt:
        \begin{displaymath}
            \doubleOne_A^{-1}(B) = 
            \begin{cases}
                A &, 1 \in B \text{ und } 0 \notin B, \\
                A^c &, 1 \notin B \text{ und } 0 \in B, \\
                X &, 1 \in B \text{ und } 0 \in B, \\
                \emptyset &, 1 \notin B \text{ und } 0 \notin B
            \end{cases}
        \end{displaymath}
        \item Sei $\mathcal{A}$ eine \jsigalg \ auf X $\Rightarrow \doubleOne_A$ ist \mb{$\mathcal{A}$}{$\Borel_1$} $\Leftrightarrow A \in \mathcal{A}$.\\
        $\doubleOne_{\Omega}$ ist nicht \mb{$\Borel_1$}{$\Borel_1$}.
        
        
    \end{enumerate}
\end{bem}

\jdate{10.11.2008}

\begin{satz}
\jlabel{Satz 2.3}
    Seien $\mathcal{A}, \Borel, \mathcal{C}$ $\sigma$-Algebren auf $X,Y,Z \ne \emptyset$. Dann gelten:
    \begin{enumerate}
        \item Wenn $\jabb{f}{X}{Y}$ \calMb{A}{B} und $\jabb{g}{Y}{Z}$ \calMb{B}{C}, dann ist $\jabb{h:= g \circ f}{X}{Z}$ \calMb{A}{C}.
        \item Seien $\emptyset \ne \mathcal{E} \subset \PowerSet(Y), \ B = \sigma(\mathcal{E}),\ \jabb{f}{X}{Y}$. Dann gilt:\\
        $f$ \calMb{A}{B} $\Leftrightarrow f^{-1}(E) \in \mathcal{A} \ \forall E \in \mathcal{E}$.
    \end{enumerate}
    \begin{proof}
        \begin{enumerate}
            \item 
                Sei $C\in \mathcal{C}$. Dann folgt, weil $g$ \jlink{messbar} ist, dass $g^{-1}(C)\in \Borel$ gilt. Da auch $f$ \jlink{messbar} ist, gilt auch $f^{-1}(g^{-1}(C))\in \mathcal{A}$.
            \item
                ``$\Rightarrow$'' ist klar, denn $\mathcal{E}\subset \sigma(\mathcal{E}) = \Borel$.\\
                ``$\Leftarrow$'' zeigen wir mit dem Prinzip der guten Mengen:\\
                $f_*(\mathcal{A}) = \{C\subset Y : f^{-1}(B) \in \mathcal{A}\}$ ist eine \jsigalg{} auf $Y$ (siehe Übung). Nach Voraussetzung gilt $\mathcal{E} \subset f_*(\mathcal{A})$. Mit \jlink{Lem 1.6} folgt $\sigma(f_*(\mathcal{A})) = f_*(\mathcal{A})$, d.h., $f^{-1}(B)\in \mathcal{A}$ $\forall B\in \Borel$.
        \end{enumerate}
    \end{proof}
\end{satz}

\begin{defn}
\jlabel{Def 2.4}
    Sei $X \subset \R^d$ nichtleer, $X\in \Bd$. Die Funktion $\jabb{f}{X}{\R^k}$ heißt \uline{Borel-messbar}, wenn sie \mb{$\Borel(X)$}{$\Borel_k$} ist.
\end{defn}
Ab jetzt sei stets $\emptyset \ne X \in \Bd$ und ''messbar'' heiße stets Borel-messbar.

\begin{satz}[Eigenschaften Borel-messbarer Funktionen]
\jlabel{Satz 2.5}
    Seien $X\in \Bd, \ \jabb{f,g}{X}{\R^k}$, wobei $f = (f_1, \dots, f_k)^T$. Dann gelten:
    \begin{enumerate}
        \item $f$ stetig $\Rightarrow f$ messbar
        \item $f$ messbar $\Leftrightarrow \jabb{f_1,\dots, f_k}{X}{\R}$ messbar
        \item $f,g$ messbar, $\alpha, \beta \in \R \Rightarrow \jabb{\alpha \cdot f + \beta \cdot g}{X}{\R^k}$ messbar
        \item $\jabb{f,g}{X}{\R}$ messbar $\Rightarrow \jabb{f\cdot g}{X}{\R}$ und falls $f(x)\ne 0 \ \forall x\in X$, dann ist $\jabb{\frac{1}{f}}{X}{\R}$ messbar
        \item $\jabb{f,g}{X}{\R}$ messbar $\Rightarrow \{x\in X: f(x) \ge g(x)\} \in \Borel(X)$. (Analog für ''$>$'')
    \end{enumerate}
    \begin{proof}
        \begin{enumerate}
            \item 
                $U \subset \R^k$ offen $\overset{\text{f stetig}}{\Rightarrow} f^{-1}(U) \subset \mathcal{O}(X) \subset \Borel(X)$. Da $\mathcal{O}(X)$ Erzeuger von $\Borel(X)$ ist, folgt die Behauptung aus \jlink{Satz 2.3}b).
            
            \item 
                ``$\Rightarrow$'': Die Projektionen $\jabb{p_j}{\R^k}{\R}, p_j(x) = x_j$, sind stetig und damit nach a) messbar. Damit $f_j = p_j \circ f$ messbar nach \jlink{Satz 2.3}a).\\
                ``$\Leftarrow$'': Seien $a,b\in \Q^d,\ a\le b$ (Erzeuger).\\
                $f(x) \in (a,b] \Leftrightarrow f_j(x)\in(a_j,b_j] \ \forall j\in \{1,\dots,k\}$.\\
                $f^{-1}((a,b]) = \bigcap_{j=1}^k \underbrace{f_j^{-1}((a_j,b_j])}_{\in \Borel(X) \text{ nach Vor.}} \in \Borel(X)$, also ist $f$ messbar nach \\
                \jlink{Satz 2.3}b).
            
            \item 
                Nach b) gilt: $\jabb{h=(f, g)^T}{X}{\R^{k+k}}$ messbar. Ferner ist $\jabb{\varphi}{\R^{2k}}{\R}, \varphi(x,y) = \alpha\cdot x + \beta\cdot y$ stetig und nach a) \jlink{messbar}.\\
                $\overset{\jshortlink{Satz 2.3}a)}{\Rightarrow} \alpha\cdot f + \beta\ \cdot g = \varphi \circ h$ messbar.
            
            \item Wie c) durch Stetigkeit der Multiplikation und Inversion.
            
            \item 
                Nach c) ist $h=f-g$ messbar.
                \begin{displaymath}
                    \begin{split}
                        \{x\in X : f(x) \ge g(x)\} &= \{x\in X: h(x) \ge 0\} \\
                        &= h^{-1}(\underbrace{\{y \in \R : y \ge 0\}}_{\in \Borel_1}) \in \Borel(X).
                    \end{split}
                \end{displaymath}
        \end{enumerate}
    \end{proof}
\end{satz}

\begin{expl*}
    \begin{enumerate}
        \item
            Sei $\jabb{f}{X}{\R^k}$ messbar, $p\in [1,\infty]$. Dann ist $\jabb{g}{X}{\R}, g(x) = |f(x)|_p$, messbar, denn es gilt $g = |\cdot|_p \circ f$ und $|\cdot|_p$ ist stetig.
        \item
            Sei $X = A\cup B$, mit $A,B\in \Bd$ diskunkt und $\jabb{f}{A}{\R^k}, \jabb{g}{B}{\R^k}$ messbar. Dann ist $\jabb{h}{X}{\R^k}, h(x) = \begin{cases} f(x), & x\in A\\ g(x), & x\in B \end{cases}$.
            \begin{proof}
                Seien $a,b \in \Q^n,\ a\le b$ (Erzeuger). Dann gilt:
                \begin{displaymath}
                    \begin{split}
                    h^{-1}((a,b]) &= \{x\in X : h(x) \in (a,b]\} \\
                                  &= \{x\in A: f(x) \in (a,b]\} \cup \{x\in B : g(x) \in (a,b]\}\\
                                  &= \underbrace{f^{-1}((a,b])}_{\in \Borel(A) \overset{(*)}{\subset} \Borel(X)} \cup \underbrace{g^{-1}((a,b])}_{\in \Borel(B) \overset{(*)}{\subset} \Borel(X)} \in \Borel(X)
                    \end{split}
                \end{displaymath}
                Dabei gilt $(*)$ nach Korollar 1.11:
                \begin{displaymath}
                    \Borel(A) = \{C\in \Bd : C\subset A\} \subset \{C \in \Bd : C\subset X\} = \Borel(X)
                \end{displaymath}
                Mit \jlink{Satz 2.3}b) ist damit der Beweis erbracht.
            \end{proof}
    \end{enumerate}
\end{expl*}

\begin{expl*}
    $X = \R^2, \ \jabb{h}{\R^2}{\R},$\\
    $h(x,y) = \begin{cases} \frac{\sin(y)}{x}=:f(x,y), & (x,y)^T \in \R^2\backslash (\{0\}\times \R) =: A \\ c=:g(x,y), & (x,y)^T \in \{0\} \times \R =: B \end{cases}$,
    wobei $c\in \R$ beliebig ist und $f,g$ stetig auf $A$ bzw. $B$ sind. Da $\R^2 = A\dcup B$ und $A,B$ disjunkt, folgt mit b) aus dem obigen Beispiel, dass $h$ messbar ist.
\end{expl*}
Um für Funktionenfolgen $\jabb{f_j}{X}{\R}\ j\in\N$, immer $\sup_{n\in\N} f_j(x), \inf_{n\in\N} f_j(x)$ bilden zu können, setzt man $\Rq = [-\infty, \infty] := \R \cup \{-\infty, \infty\}$.

\vspace{12pt}

\uline{Rechenregeln}: Sei $a\in \R$.
\begin{itemize}
    \item $\pm \infty + (\pm \infty) = \pm \infty, \ \pm\infty + a = a \pm \infty = \pm \infty$
    \item $a\cdot (\pm\infty) = (\pm \infty)\cdot a = \begin{cases} \pm \infty, & a\in (0, \infty] \\ \mp \infty, &a\in [-\infty, 0) \end{cases}$
    \item Verboten bleiben: $+\infty + (-\infty),\ \frac{0}{0}, \ \frac{\pm\infty}{\pm\infty}, \frac{a}{0}$ usw.
\end{itemize}

\jspacesmall

\uline{Ordnung}: $-\infty < a < +\infty \ (\forall a\in \R)$.

\jspacesmall

\uline{Konvergenz}: Für $(x_n)_{n\in\N}\subset \Rq$ schreibe: $x_n \xrightarrow{n\rightarrow \infty} +\infty$,\\
falls $\forall C\in\R \ \exists N_c \in \N: x_n \ge c \ \forall n\ge N_c$\\
(Konvergenz gegen $-\infty$ entsprechend mit ``$\le$'')

\begin{expl*}
    $\jabb{f}{\R}{\Rq}, f(x) = \begin{cases} \frac{1}{x}, & x\ne 0 \\ +\infty, &x = 0 \end{cases}$
\end{expl*}

\uline{Notation}: Setze für $\jabb{f,g}{X}{\Rq}, \ a\in \Rq$ 
\begin{displaymath}
    \begin{split}
        \{f=g\} &:= \{x\in X: f(x) = g(x)\},\\
        \{f=a\} &:= \{x\in X: f(x) = a\}.
    \end{split} 
\end{displaymath}
(Analog für $\le, <, =, >, \ge, \dots$)

\vspace{12pt}

\uline{Erinnerung}: $f\le g \Leftrightarrow f(x) \le g(x) \ \forall x\in X$

\begin{defn*}
    Definiere auf $\Rq$ die Borel'sche $\sigma$-Algebra $\Borelq_1$ durch
    \jlabel{(2.1)}
    \begin{equation}
        \Borelq_1 = \{B\cup E : B\in \Borel_1, \ E\subset \{-\infty, +\infty\}\}.
    \end{equation}
    Man prüft leicht nach, dass $\Borelq_1$ wirklich eine $\sigma$-Algebra auf $\Rq$ ist.\\
    Offensichtlich gilt $\Borel_1 \subset \Borelq_1$.\\
    Funktionen $\jabb{f}{X}{\Rq}$, die \mb{$\Borel(X)$}{$\Borelq_1$} sind, heißen ebenfalls (Borel-) messbar.
\end{defn*}

\jlabel{Lem 2.6}
\begin{lem}
    \begin{enumerate}
        \item 
            \begin{displaymath}
                \begin{split}
                    \Borelq_1 &= \sigma(\{[-\infty, a]: a\in \Q\}) =: A_1\\
                                             &= \sigma(\{(a, \infty]: a\in \Q\}) =: A_2\\
                                             &= \sigma(\{[a, \infty]: a\in \Q\}) =: A_3\\
                                             &= \sigma(\{[-\infty, a): a\in \Q\}) =: A_4
                \end{split}
            \end{displaymath}
            
        \item $\jabb{f}{X}{\Rq}$ ist messbar
            \begin{displaymath}
                \begin{split}
                    &\Leftrightarrow \{f \le a\} \in \Borel(x) \ \forall a\in \Q\\
                    &\Leftrightarrow \{f < a\} \in \Borel(x) \ \forall a\in \Q\\
                    &\Leftrightarrow \{f > a\} \in \Borel(x) \ \forall a\in \Q\\
                    &\Leftrightarrow \{f \ge a\} \in \Borel(x) \ \forall a\in \Q\\
                \end{split}
            \end{displaymath}
    \end{enumerate}
    Als Spezialfall gelten die entsprechenden Äquivalenzen für Funktionen $\jabb{f}{X}{\R}$, denn so ein $f$ ist \mb{$\Borel(X)$}{$\overline{{\Borel_1}}$} genau dann, wenn es \mb{$\Borel(X)$}{$\Borel_1$} ist.
    \begin{proof}
        \begin{enumerate}
            \item $A_1\subset A_2$ folgt aus $[-\infty, a] = (a,\infty]^c \in A_2$ und \jlink{Lem 1.6}. Genauso $A_3 \subset A_4$.\\
            $A_2 \subset A_3$ wegen $(a,\infty] = \bigcap_{n=1}^\infty [a+\frac{1}{n},\infty] \in A_3$ und \jlink{Lem 1.6}.\\
            $A_4 \subset \Borelq_1$ wegen $[-\infty,a) = \{-\infty\}\cup (-\infty, a) \in \overline{{\Borel_1}}$ und \jlink{Lem 1.6}.\\
            \uline{Es bleibt zu zeigen}: $\Borelq_1 \subset A_1$\\
            Es gilt $\{-\infty\} = \bigcap_{n=1}^\infty [-\infty, -n] \in A_1 \Rightarrow (-\infty,a] = [-\infty,a]\backslash \{-\infty\}$\\
            $\overset{\jshortlink{Lem 1.6}}{\underset{\jshortlink{Satz 1.9}}{\Rightarrow}} \Borel_1 \subset(A_1)$. Ebenso $\{+\infty\} \in A_1 \Rightarrow \Borelq_1 \subset A_1$.
            
            \item folgt aus a) und \jlink{Satz 2.3}b).
        \end{enumerate}
        Spezialfall folgt aus $f^{-1}(B\cup E) = f^{-1}(B)$ für $B\in \Borel(X)$ und $E \subset \{-\infty, +\infty\}$.
    \end{proof}
\end{lem}

\jdate{14.11.2008}

\begin{defn*}
    Sei $\jabb{f_n}{X}{\Rq}$ messbar, $n\in\N$.\\
    Definiere $\jabb{\sup_{n\in\N} f_n}{X}{\Rq}$ durch 
    \begin{displaymath}
            \left( \sup_{n\in\N} f_n \right)(x) := \sup_{n\in\N} f_n(x), \ x\in X
    \end{displaymath}
    (Analog definert man $\inf_{n\in\N} f_n, \ \varliminf_{n\rightarrow \infty} f_n, \ \varlimsup_{n\rightarrow \infty} f_n$)
    
    \vspace{12pt}
    
    \uline{Falls}: $\limToInf{n} f_n(x)$ in $\Rq$ für alle $x\in X$ existiert, setzt man
    \begin{displaymath}
        \left (\limToInf{n} f_n \right)(x) := \limToInf{n} f_n(x) \in \Rq \ (\forall x\in X).
    \end{displaymath}
    Dabei gilt
    \begin{displaymath}
        \max_{1\le n\le N}\{f_1,\dots,f_N\} = \sup\{f_1,\dots,f_N, f_N,\dots\}.
    \end{displaymath}
    ($\min$ analog).
\end{defn*}


\begin{satz}
\jlabel{Satz 2.7}
    Seien $\jabb{f_n}{X}{\Rq}$ für jedes $n\in\N$ messbar. Dann sind die Funktionen $\sup_{n\in\N}f_n$, $\inf_{n\in\N} f_n$, $\varliminf_{n\rightarrow \infty} f_n$, $\varlimsup_{n\rightarrow \infty} f_n$ und (falls $\forall x \in X$ existent) $\limToInf{n} f_n$ messbar.
    \begin{proof}
        Sei $a\in \R$. Dann gilt $\{(\sup_{n\in\N} f_n) \le a\} = \bigcap_{n\in\N} \{f_n \le a\} \in \Borel(X)$ und $\{x\in X : \inf_{n\in \N} f_n(x) \ge a\} = \bigcap_{n\in\N} \{x: f_n(x) \ge a\} \in \Borel(X)$.\\
        Mit \jlink{Lem 2.6} folgt dann, dass $\sup_{n\in\N} f_n$ und $\inf_{n\in\N} f_n$ messbar sind. Damit sind auch $\varlimsup_{n\rightarrow \infty} f_n = \inf_{j\in \N} \sup_{n\ge j} f_n$ und $\varliminf_{n\rightarrow \infty} f_n = \sup_{j\in\N}\inf_{n\ge j} f_n$ messbar.\\
        Wenn existent für alle $x\in X$, dann ist somit auch $\limToInf{n} f_n = \varlimsup_{n\rightarrow \infty} f_n$ messbar.
    \end{proof}
\end{satz}

\begin{bem*}
    \jlink{Satz 2.7} ist falsch für überabzählbare Suprema, denn:\\
    Sei $\Omega \in \Borel_1$ aus \jlink{Satz 1.26}, $f_x := \doubleOne_{\{x\}}, \ x \in \Omega$. Dann sind alle $f_x$ messbar, aber $\sup_{x\in\Omega} \doubleOne_{\{x\}} = \doubleOne_\Omega$ ist nicht messbar, da $\Omega \notin \Borel_1$.
\end{bem*}

\begin{satz}
\jlabel{Satz 2.8}
    Seien $\jabb{f,g}{X}{\Rq}$. Dann gelten:
    \begin{enumerate}
        \item 
            Seien $f,g$ messbar und $\alpha, \beta \in \R$. Wenn $\alpha\cdot f(x) + \beta\cdot g(x)$ für alle $x\in X$ definiert ist, dann ist $\jabb{\alpha\cdot f + \beta\cdot g}{X}{\Rq}$ messbar. Wenn $f(x)\cdot g(x)$ für alle $x\in X$ definiert ist, dann ist $\jabb{f\cdot g}{X}{\Rq}$ messbar.
        \item
            $f$ messbar $\Leftrightarrow f_+ := \max\{f,0\}$ und $f_- := \max\{-f,0\}$ messbar $\Rightarrow |f|$ messbar.
    \end{enumerate}
    \uline{Bemerkung zu b)}: $f = \doubleOne_\Omega - \doubleOne_{\Omega^c}$ ist mit $\Omega$ aus \jlink{Satz 1.26} nicht messbar, aber $|f| = \doubleOne_{\R^d}$ ist messbar.
    \begin{proof}
        \begin{enumerate}
            \item 
                Betrachte $f_n(x) = \max \{-n, \min\{n, f(x)\}\}$ für $n\in\N, \ x\in X$. Genauso für $g$. Da konstante Funktionen immer messbar sind, sind nach \jlink{Satz 2.7} $f_n, g_n \ \forall n\in\N$ messbar.\\
                Es gilt: $f_n(x) \rightarrow f(x), \ g_n(x) \rightarrow g(x) \ (n\rightarrow \infty) \ \forall x\in X$ (auch dann, wenn $f(x),g(x) \notin \R$).\\\
                Sei $\alpha\cdot f(x) + \beta \cdot g(x)$ definiert. Dann gilt:\\
                $\alpha\cdot f_n(x) + \beta\cdot g(x)\xrightarrow{n\rightarrow\infty} \alpha\cdot f(x) + \beta \cdot g(x)$.\\
                (Das ist klar, wenn $f(x), g(x) \in \R$. Sei deshalb etwa $\alpha=\beta=1, f(x)=\infty, g(x)\in\R$. Sei $n>|g(x)|$. Dann $\alpha\cdot f_n(x) + \beta\cdot g_n(x) = n + g(x) \xrightarrow{n\rightarrow \infty} \infty = f(x)+g(x)$.)\\
                Mit \jlink{Satz 2.7} folgt dann, dass $\alpha\cdot f + \beta\cdot g$ messbar ist.\\
                (Ähnlicher Beweis für $f\cdot g$. Beachte dabei: Falls $f(x)=0,g(x)=\infty$ folgt: $f_n(x)\cdot g_n(x) = 0\cdot n = 0 \xrightarrow{n\rightarrow \infty} 0 = f(x)\cdot g(x)$.)
            \item
                Beide ``$\Rightarrow$'' folgen sofort aus \jlink{Satz 2.7}.\\
                Erstes ``$\Leftarrow$'' folgt aus a) und $f=f_+-f_-$, $|f| = f_++f_-$.\\
                Beachte: $f_+$ und $f_-$ sind nur einzeln gleich $0$.
        \end{enumerate}
    \end{proof}
\end{satz}

\begin{bem*}
    \jlink{Satz 2.7} und \jlink{Satz 2.8} gelten genauso für $\R$-wertige Funktionen.
\end{bem*}

\begin{expl*}
    Seien $\jabb{f_j}{X}{[0,\infty]}$ messbar für $j\in\N$. Dann existiert $g_n(x):= \sum_{j=1}^n f_j(x) \in [0,\infty]$ für alle $x\in X$ und $g_n$ ist nach \jlink{Satz 2.8}a) messbar für alle $n\in\N$.\\
    Mit \jlink{Satz 2.7} ist also $g := \sum_{j=1}^\infty f_j= \sup_{n\in\N}g_n = \limToInf{n} g_n$ ebenfalls messbar.
\end{expl*}

\begin{defn}
\jlabel{Def 2.9}
    Eine messbare Funktion $\jabb{f}{X}{\R}$ heißt \uline{einfach}, wenn sie endlich viele Werte annimmt, d.h. $|f(X)| < \infty$. Seien $y_1,\dots,y_n \in \R$ alle verschiedenen Funktionswerte von $f$. Setze $A_j = f^{-1}(\{y_j\})$. Da $f$ messbar ist, folgt nach Definition der Messbarkeit, dass $A_j\in\Borel(X)$ für alle $j\in\N$. Dann heißt
    \begin{displaymath}
        f = \sum_{j=1}^n y_j \cdot \doubleOne_{A_j}
    \end{displaymath}
    die \uline{Normalform} von $f$.\\
    Beachte: Die Vereinigung $X = A_1 \dcup \dots \dcup A_n$ ist disjunkt. 
\end{defn}

\begin{bem}
\jlabel{Bem 2.10}
    Linearkombinationen, Produkte, endliche Minima und Maxima einfacher Funktionen sind wieder einfach.
\end{bem}

\begin{satz}
\jlabel{Satz 2.11}
    Sei $\jabb{f}{X}{\R}$ messbar. Dann gelten:
    \begin{enumerate}
        \item Es existieren einfache Funktionen $f_n$ mit $f_n \xrightarrow{n\rightarrow\infty} f$ (punktweise).
        \item Ist $f$ beschränkt, so gilt a) mit gleichmäßiger Konvergenz.
        \item Sei $f\ge 0$. Dann gilt a) mit $f_n$, die $f_n \le f_{n+1} \ (\forall n\in\N)$ erfüllen.
    \end{enumerate}

\end{satz}

\begin{kor}
\jlabel{Kor 2.12}
    $\jabb{f}{X}{\Rq}$ messbar $\Leftrightarrow$ es existieren einfache $\jabb{f_n}{X}{\Rq}$ mit $f_n \xrightarrow{n\rightarrow \infty} f$ (punktweise).
    \begin{proof}
        \jlink{Satz 2.11} und \jlink{Satz 2.7}.
    \end{proof}
\end{kor}

\begin{proof}[Beweis von \jlink{Satz 2.11}]
    \begin{itemize}
        \item[c)] Sei $f\ge 0, \ n\in\N$.Setze
        \begin{displaymath}
            \begin{split}
                B_{jn} &:= \begin{cases} [j\cdot 2^{-n}, (j+1)\cdot 2^{-n}), \ j=0,\dots, n\cdot 2^{n-1} \\ [n, \infty), \ j = n\cdot 2^n \end{cases},\\
                A_{jn} &:= f^{-1}(B_{jn}) \in \Borel(X) \text{ (da $f$ messbar)}
            \end{split}
        \end{displaymath}
        für alle $j=0,\dots,n\cdot 2^n,\ n\in\N$. Dann folgt, dass die Vereinigung $X = \bigcup_{j=0,\dots,n\cdot 2^n} A_{jn}$ für jedes $n\in\N$ disjunkt ist. Setze außerdem für $n\in\N$
        \begin{displaymath}
            f_n := \sum_{j=0}^{n\cdot 2^n} \underbrace{j\cdot 2^{-n}}_{= \min B_{jn}} \cdot \doubleOne_{A_{jn}}.
        \end{displaymath}
        Dann ist $f_n$ einfach und für $x\in A_{jn}$ gilt: $f_n(x) = j\cdot 2^{-n} \le f(x)$, also $f_n \le f \ \forall n\in\N$.
        
        \vspace{12pt}
        TODO: BILD
        \vspace{12pt}
        
        Ferner gilt
        \begin{displaymath}
            A_{jn} =
                \begin{cases}
                    A_{2j,n+1} \dot{\cup} A_{2j+1,n+1}, &j=0,\dots,n\cdot 2^n-1\\
                    \bigcup_{k=n\cdot 2^{n+1}}^{(n+1)\cdot 2^{n+1}} A_{k,n+1}, & j=n\cdot 2^n
                \end{cases}.
        \end{displaymath}
        Für $x \in A_{jn}$ gilt
        \begin{displaymath}
            f_n(x) = j\cdot 2^{-n}
            \begin{cases}
                = 2\cdot j \cdot 2^{-(n+1)} = f_{n+1}(x), & x\in A_{2j,n+1} \\
                \le (2\cdot j +1)\cdot 2^{-(n+1)} = f_{n+1}(x), & x\in A_{2j+1, n+1}
            \end{cases}
        \end{displaymath}
        Also gilt $f_n(x) \le f_{n+1}(x) \ \forall x\in A_{jn}$, falls $j < n\cdot 2^n$.\\
        Sei $x\in A_{n\cdot 2^n}$. Dann gilt $f_n(x) = n = n\cdot 2^{n+1}\cdot 2^{-(n+1)} \le k\cdot 2^{-(n+1)} = f_{n+1}(x)$ für alle $k\in\{n\cdot 2^{n+1}, \dots, (n+1)\cdot 2^{n+1}\}$.\\
        \uline{Also}: $f_n \le f_{n+1} \ (\forall n\in\N)$.
        \begin{itemize}
            \item[A)]
                Wenn $f(x) = \infty$, dann $x\in A_{n\cdot 2^n,n}$ für alle $n\in\N \Rightarrow f_n(x) = n \xrightarrow{n\rightarrow \infty} \infty = f(x)$.
            \item[B)]
                Wenn $f(x) < \infty$, dann liegt $x$ für alle $n\in\N$ mit $n>f(x)$ in einem $A_{j(n),n}$ mit $j(n) < n\cdot 2^{-n}$. Dann folgt
                \begin{displaymath}
                    f_n(x) = j(n)\cdot 2^{-n} \le f(x) \le f_n(x) + 2^{-n} \ \ \ (*).
                \end{displaymath}
                Und somit $|f(x)-f_n(x)| \le 2^{-n} \xrightarrow{n\rightarrow \infty} 0$, woraus Behauptung c) folgt.
        \end{itemize}
        \item[a)]
            Setze $f_n := (f_+)_n - (f_-)_n$ für $n\in\N$. Dann ist $f_n$ einfach. Nach c) gilt: $f_n\xrightarrow{n\rightarrow\infty} f_+ - f_- = f$.
        \item[b)]
            Wenn $f$ beschränkt ist, tritt für $n>\lVert f\rVert_\infty$ in c) stets B) ein. Für alle $n>\lVert f\rVert_\infty$ gilt dann $(*)$ $\forall x\in X$, also $f_n \xrightarrow{n\rightarrow \infty} f$ (gleichmäßig).
    \end{itemize}
\end{proof}
        	
    % Konstruktion des Lebesgue-Integrals
\section{Konstruktion des Lebesgue-Integrals}

Weiterhin sei $\emptyset \ne X \in \Bd$ versehen mit $\Borel(X)$ und $\lambda = \lambda_d$.
\begin{bem*}
    Alles in dem Abschnitt 2.2 geht entsprechend für beliebige Maßräume $(X,\mathcal{A}, \mu)$.
\end{bem*}

\subsubsection*{Vorgehen}
\begin{itemize}
    \item[A)] Integral für einfache $\jabb{f}{X}{\R_+}$.
    \item[B)] Integral für \uline{jedes} messbare $\jabb{f}{X}{[0,\infty]}$.
    \item[C)] Integral für gewisse messbare $\jabb{f}{X}{\Rq}$.
\end{itemize}


\jdate{17.11.2008}
 
\subsection*{Schritt A: Integral für einfache, positive Funktionen}

\begin{defn}
\jlabel{Def 2.13}
    Sei $f: X \rightarrow \R_+$ einfach mit Normalform $f = \sum_{k=1}^m y_k \doubleOne_{A_k}$. Dann setzt man:
    \begin{displaymath}
        \int f(x) dx := \int_X f(x) dx = \sum_{k=1}^m y_k \lambda(A_k) \in [0, \infty]
    \end{displaymath} 
    \uline{Beachte:} $0\cdot\infty = \infty\cdot 0 = 0, y_1, \dots, y_m \ge 0$ und $f(x) = y_k \Leftrightarrow x \in A_k$ \\
    \uline{Problem:} $f$ hat viele Darstellungen, z.B.: $\doubleOne_A = 2\cdot \doubleOne_A - \doubleOne_X + \doubleOne_{A^c} = \doubleOne_A + 0\cdot \doubleOne_{A^c}$ \\
    \uline{Frage:} Ist $\int_x f dx$ unabhängig von der Darstellung von $f$?
\end{defn}

\begin{lem}
\jlabel{Lem 2.14}
    Seien $B_j \in \Borel(X), j=1,\dots,n$ mit $\bigcup_{j=1}^{\infty} B_j = X$ und $z_j \in \R, j=1,\dots,n$ sowie $f = \sum_{j=1}^n z_j \doubleOne_{B_j}$. Dann gilt:
    \begin{displaymath}
        \int_X f(x) dx = \sum_{j=1}^n z_j \lambda(B_j)
    \end{displaymath} 
    \begin{proof}
        Durch iteratives Schneiden und Differenzmengenbilden erhält man disjunkte $C_i \in \Borel(X), i=1,\dots,l$ sowie Mengen $I(j) \subset \{1,\dots,l\}$ und $J(i) \subset \{1,\dots, n\}$ mit:
        $(*) B_j = \bigdcup_{i \in I(j)} C_i$ und $C_i \subset B_j,  j \in J(i)\ (\forall j=1,\dots,n, i=1, \dots,l)$. Dann folgt:
        \begin{displaymath}
            \sum_{j=1}^n z_j \cdot \lambda(B_j) = \sum_{j=1}^n z_j \sum_{i \in I(j)} \lambda(C_i) = \sum_{i=1}^l \lambda(C_i) \sum_{j \in J(i)} z_j=:S
        \end{displaymath}
        Setze für $i=1,\dots,l: w_i := \sum_{j \in J(i)} z_j = f(x)$, wenn $x \in C_i$.
        Vereinige die $C_i$ mit gleichem $w_i (i=1,\dots,l)$ zu einer Menge $A_k \in \Borel(X) \ (k=1,\dots,m)$.
        Sei $f(x) = y_k$ für $x \in A_k$, \uline{d.h.}: $A_k = f^{-1}(\{y_k\})$. Dabei sind $y_1,\dots,y_m$ die Funktionswerte von $f$, die paarweise verschieden sind.
        Da die $C_i$ disjunkt sind, gilt
        $S = \sum_{k=1}^m y_k \lambda(A_k)$ und $f = \sum_{k=1}^m y_k \doubleOne_{A_k}$ ist die Normalform.
        Mit \jlink{Def 2.13} folgt dann die Behauptung.
    \end{proof}
\end{lem}

\begin{lem}
\jlabel{Lem 2.15}
    Seien $f,g: X \rightarrow \R_+$ einfache Funktionen, $\alpha, \beta \in \R_{+}$, $A \in \Borel(X)$. Dann:
    \begin{enumerate}
        \item $\int_X \doubleOne_A dx = \lambda(A)$
        \item $\int_x(\alpha\cdot f + \beta\cdot g)(x)dx = \alpha \cdot \int_X f(x) dx + \beta \cdot \int_X g(x) dx$ (Beachte \jshortlink{Bem 2.10})
        \item $f \le g \Rightarrow \int_X f(x) dx \le \int_X g(x) dx$
    \end{enumerate}
    \begin{proof}
        \begin{itemize}
            \item[a):] Folgt aus \jlink{Def 2.13}.
            
            \item[b),c):]
                Es seien $f = \sum_{j=1}^n y_j \doubleOne_{A_j}, g = \sum_{k=1}^m z_k \doubleOne_{B_k}$ in Normalform.
                Seien $C_i,\ i = 1,\dots,l$ alle Schnitte der Form $A_j \cap B_k$, sodass $\{Ci : i = 1,...,l\}$ disjunkt ist.
                Seien weiter $\overline{y_i} \in \{y_1, \dots, y_n\}$ und $\overline{z_j} \in \{z_1, \dots, z_m\}$ die Funktionswerte von $f$ bzw. $g$ auf $C_i$. Dann folgt: $f = \sum_{i=1}^l \overline{y_i} \doubleOne_{C_i}, g = \sum_{i=1}^l \overline{z_i} \doubleOne_{C_i}$.
                
            \item[b):]
                Es gilt $\alpha\cdot f + \beta\cdot g = \sum_{i=1}^l (\alpha\cdot \overline{y_i} + \beta\cdot \overline{z_i})\doubleOne_{C_i}$.
                Daraus folgt
                \begin{displaymath}
                    \begin{split}
                        \int_X (\alpha\cdot f + \beta\cdot g)dx &\overset{\jshortlink{Lem 2.14}}{=} \sum_{i=1}^l(\alpha\cdot \overline{y_i} + \beta\cdot \overline{z_i})\lambda(C_i)\\
                        &= \alpha \sum_{i=1}^l \overline{y_i} \lambda(C_i) + \beta \sum_{i=1}^l \overline{z_i} \lambda(C_i)\\
                        &\overset{\jshortlink{Lem 2.14}}{=} \int_X f dx + \int_X g dx.
                    \end{split}
                \end{displaymath}
            \item[c):]
                Nach Voraussetzung gilt $\overline{y_i} \le \overline{z_i}$. Damit und mit b),c) folgt:
                
                $\int_X f dx = \sum_{i=1}^l \overline{y_i} \lambda(C_i) \le \sum_{i=1}^l \overline{z_i} \lambda(C_i) = \int_X g dx$.
        \end{itemize}
    \end{proof}
\end{lem}

\subsection*{Schritt B: Integral für messbare Funktionen $f: X \rightarrow [0, \infty]$}

Sei $f: X \rightarrow [0, \infty]$ messbar. Nach \jlink{Satz 2.11} gilt:
\jlabel{(2.2)}
\begin{equation}
    \exists \ \text{einfache} \ f_n: X \rightarrow \R_+ \text{mit} f_n \le f_{n+1} \ (\forall n \in \N), f_n \rightarrow f \  (\text{pw}, n \toInf)
\end{equation} 
Nach \jlink{Lem 2.15} gilt:

\begin{displaymath}
    \int f_n dx \le \int f_{n+1} dx \ \ (\forall n \in \N)
    \Rightarrow \exists \limToInf{n} \int f_n dx = \sup_{n \in \N} \int f_n dx \in [0, \infty]
\end{displaymath}

\begin{defn}
\jlabel{Def 2.16}
    Sei $f: X \rightarrow [0,\infty]$ messbar und $f_n, n \in \N$ wie in \jlink{(2.2)}. Dann setze:
    \begin{displaymath}
        \int f dx = \int_X f(x) dx = \limToInf{n} \int f_n(x) dx = \sup_{n \in \N} \int_X f_n(x) dx \in [0,\infty]
    \end{displaymath}
\end{defn}

\begin{lem}
\jlabel{Lem 2.17}
    Sei $f: X \rightarrow [0, \infty]$ messbar. Dann gilt:
    \begin{displaymath}
        \int_X f(x) dx = \sup \left(\left\{\int_X g(x) dx : g: X \rightarrow \R_+ \text{einfach}, 0 \le g \le f\right\}\right) =: S
    \end{displaymath}
    \begin{proof}
        Sei $f_n$ wie in \jlink{(2.2)}. Da $\int f dx = \sup_{n \in \N} \int f_n dx$, gilt $\int f dx \le S$. \\
        \uline{Zu $\ge$:} Sei $g$ einfach mit $0 \le g \le f$ und $g = \sum_{k=1}^m y_k \doubleOne_{A_k}$ (Normalform). Sei $\alpha > 1$ fest, aber beliebig, und $B_n = \{x \in X : \alpha f_n(x) \ge g(x)\} =: \{\alpha f_n \ge g\} \ (n \in \N)$ \\
        Aus \jlink{Satz 2.5} folgt dann: $B_n \in \Borel(X) \forall n \in \N$. Beachte dabei $\alpha \cdot f_n \ge \doubleOne_g (*)$ \\
        Sei $x \in X$. Wenn $f(x) = 0$, dann folgt wegen $0 \le g \le f$: \\
        $g(x) = 0 \Rightarrow x \in B_n \forall n \in \N$ \\
        Wenn $f(x) > 0 \Rightarrow f(x) > \frac{1}{\alpha} g(x)$. Da $f_n(x) \rightarrow f(x)$, folgt: \\
        $\exists n(x) \in \N: f_n(x) \ge \frac{1}{\alpha} g(x), \forall n \ge n(x) \Rightarrow x \in B_n \forall n \ge n(x)$ \\
        $\Rightarrow X = \bigcup_{n \in \N} B_n$. Ferner $B_n \subset B_{n+1}$, da $f_n \le f_{n+1}$ ($\forall n \in \N$) $(**)$ \\
        Damit gilt
        \begin{displaymath}
            \begin{split}
                \int g(x) dx &\overset{\jshortlink{Def 2.13}}{=} \sum_{k=1}^m y_k \cdot \lambda(A_k) \overset{\jshortlink{Satz 1.14}}{\underset{(**)}{=}} \limToInf{n} \sum_{k=1}^m y_k \cdot \lambda(A_k \cap B_n)\\
                &\overset{\jshortlink{Lem 2.14}}{=} \limToInf{n} \int g(x)\cdot \doubleOne_{B_n}(x) dx \overset{\jshortlink{Lem 2.15}}{\underset{(*)}{\le}} \limToInf{n} \int \alpha\cdot  f_n(x) dx\\
                &\overset{\jshortlink{Lem 2.15}}{=} \alpha\cdot \limToInf{n} \int f_n(x) dx \overset{\jshortlink{Def 2.16}}{=} \alpha\cdot \int f(x) dx.
            \end{split}
        \end{displaymath}
         Daraus folgt mit $\alpha \rightarrow 1$: $\int g dx \le \int f(x) dx \overset{\text{sup g}}{\Rightarrow} S \le \int f dx$.
    \end{proof}
\end{lem}

\begin{lem}
\jlabel{Lem 2.18}
    Seien $f,g: X \rightarrow [0,\infty]$ messbar und $\alpha, \beta \in \R_+$. Dann:
    \begin{enumerate}
        \item $\int_x(\alpha\cdot f + \beta\cdot g)(x)dx = \alpha \cdot \int_X f(x) dx + \beta \cdot \int_X g(x) dx$
        \item $f \le g \Rightarrow \int_X f(x) dx \le \int_X g(x) dx$
        \item $\int_X f(x) dx = 0 \Leftrightarrow \lambda(\{f > 0\}) = 0$
    \end{enumerate}
    \begin{proof}
        \begin{enumerate}
            \item Seien $f_n,g_n$ wie in \jlink{(2.2)}. Nach \jlink{Bem 2.10} und $\alpha, \beta \in \R_+$ erfüllen $\alpha f_n + \beta g_n$ \jlink{(2.2)} für $\alpha f + \beta g$.\\
            Damit gilt
            \begin{displaymath}
                \begin{split}
                    \int(\alpha f + \beta g)dx &\overset{\jshortlink{Def 2.16}}{=} \limToInf{n} \int (\alpha f_n + \beta g_n)dx \\
                    &\overset{\jshortlink{Lem 2.15}}{=} \limToInf{n}(\alpha \int f_n dx + \beta \int g dx)\\
                    &\overset{\jshortlink{Def 2.16}}{=} \alpha \int fdx + \beta \int g dx.
                \end{split}
            \end{displaymath}
            \item Sei $A = \{f > 0\} \in \Borel(X)$, seien $f_n$ wie in \jlink{(2.2)} für $f$.
                \begin{enumerate}
                    \item
                        \uline{Sei $\lambda(A) = 0$.} Da $0 \le f_n \le f$, gilt $f_n(x)=0$, wenn $x \notin A$. Dann folgt $f_n \le \doubleOne_A \lVert f_n \rVert_\infty \overset{\jshortlink{Lem 2.15}}{\Rightarrow} 0 \le \int f_n dx \le \int \lVert f_n \rVert_\infty \doubleOne_A dx = \lVert f \rVert_\infty \lambda(A) = 0 \Rightarrow 0 = \limToInf{n} \int f_n dx \overset{\jshortlink{Def 2.16}}{=} \int f dx$.
                    \item Es gilt
                        \begin{displaymath}
                            \begin{split}
                                \int f(x)dx &\overset{\jshortlink{Lem 2.17}}{=} \sup_{0\le u\le f, u \text{ einfach}} \int u(x)dx\\
                                &\overset{f\le g}{\le} \sup_{0\le u\le g, u \text{ einfach}} \int u(x)dx = \int g(x) dx.
                            \end{split}
                        \end{displaymath} 

                    \item
                        \uline{Sei $\int f dx = 0$.} Setze $A_n := \{f \ge \frac{1}{n}\}$ für $n \in \N$. Daraus folgt: $\bigcup_{n \in \N} A_n = A, f \ge \frac{1}{n} \doubleOne_{A_n}$. Damit $0 = \int f dx \overset{\text{b)}}{\ge} \int \frac{1}{n} \doubleOne_{A_n} dx \overset{\jshortlink{Lem 2.15}}{=} \frac{1}{n}\lambda(A_n) \ge 0 \Rightarrow \lambda(A_n) = 0, \forall n \in \N$.\\
                        $\Rightarrow \lambda(A) = \lambda(\bigcup_{n \ge 1} A_n) \overset{\jshortlink{Satz 1.14}}{\le} \sum_{n=1}^\infty \lambda(A_n) = 0$.
                \end{enumerate}
        \end{enumerate}
    \end{proof}
\end{lem}


\begin{thm}[Monotone Konvergenz, B. Levi]
\jlabel{Thm 2.19}
\jlabel{BL}
    Seien $f_n:X \rightarrow [0, \infty]$ messbar und $f_n \le f_{n+1} \ (\forall n \in \N)$. Sei $f= \limToInf{n} f_n = \sup_{n \in \N} f_n$. Dann:
    \begin{displaymath}
        \int_X f(x)dx = \int_X \limToInf{n} f_n(x)dx \overset{!}{=} \limToInf{n} \int_X f_n(x)dx = \sup_{n\in\N} \int f_n(x)dx
    \end{displaymath}
\end{thm}

\begin{bem*}
    \begin{enumerate}
        \item Konvergenzaussage ist ohne Monotonie falsch:\\
            \uline{Bsp}: $f_n = \frac{1}{n} \doubleOne_{[0,n]} \rightarrow f = 0$ (glm.), aber $\int f_n dx = 1 \overset{n \toInf}{\nrightarrow} 0 = \int f dx$.
            
\jdate{21.11.2008}
            
        \item Die Konvergenzaussage ist im allgemeinen falsch für fallende Folgen.\\
            \uline{Bsp}: $f_n:=\jabb{\doubleOne_{[n,\infty]}}{\R}{\R} \Rightarrow f_n\xrightarrow{n\rightarrow \infty} 0$ (punktweise), $f_n \ge f_{n+1}$, aber $\int_\R f_n dx \overset{\text{einfache}}{\underset{\text{Funktion}}{=}} \lambda_1([n,\infty]) = \infty$ und $\int_\R 0 dx = 0$.
        \item Die Konvergenzaussage ist im allgemeinen sinnlos fürs Riemannintegral.\\
            \uline{Bsp}: Sei $\Q = \{q_1, q_2, \dots\},\ A_n:= \{q_1,\dots,q_n\},\ f_n = \doubleOne_{A_n}$. Dann ist $f_n$ Riemannintegrierbar mit $f_n \le f_{n+1}$, aber $\sup_{n\in\N} f_n = \doubleOne_\Q$ ist \uline{nicht} Riemannintegrierbar, obwohl R-$\int_\R f_n dx = 0 \ \forall n\in\N$.
    \end{enumerate}
\end{bem*}

\begin{proof}[Beweis von \jlink{Thm 2.19}]
    Nach \jlink{Satz 2.7} ist $f = \limToInf{n} f_n$ messbar. Zweites ``$=$'' folgt aus $\int f_n dx \le \int f_{n+1} dx$ und \jlink{Lem 2.18}b). Nach \jlink{(2.2)} gibt es $\forall n\in\N$ einfache $\jabb{u_{nj}}{X}{\R_+}, \ j\in\N$ mit $u_{nj} \le u_{n,j+1} \le f_n \ (*)$ und $u_{nj} \xrightarrow{j\rightarrow \infty} f_n$ (punktweise).\\
    \uline{Ziel}: Konstruiere zu $f$ einfache $v_j$ wie in \jlink{(2.2)} mit $v_j \le f_j$.\\
    \uline{Setze}: $v_j = \max \{u_{1j}, u_{2j},\dots, u_{jj}\}, \ j\in\N$. Dann ist $v_j$ nach \jlink{Bem 2.10} einfach.
    \[
        \begin{matrix}
            u_{11} & u_{12} & \dots  & u_{1j} & \le    & u_{1j+1} \\
                   & u_{22} & \dots  & u_{2j} & \le    & u_{2j+2} \\
                   &        & \ddots & \vdots & \vdots & \vdots   \\
                   &        &        & u_{jj} & \le    & u_{jj+1} 
        \end{matrix}
    \]
    
    Wegen $(*)$: $v_j\le v_{j+1}$ und $v_j \le \max \{f_1, \dots, f_j\} \overset{\text{nach Vor.}}{\underset{\text{Monotonie}}{=}} f_j \le f \ (**)$.\\
    Ferner gilt für alle $j\in\N$ mit $j\le n \Rightarrow u_{nj} \le v_j$, damit:\\
    $f_n \overset{\text{nach Vor.}}{=} \sup_{j\in\N} u_{nj} \le \sup_{j\in\N} v_j \ (***)$.\\
    Es folgt $f = \sup_{n\in\N} f_n \overset{(***)}{\le} \sup_{n\in\N} (\sup_{j\in\N} v_j) \le f$, also $f=\sup_{j\in\N} v_j$, d.h., $v_j$ erfüllt \jlink{(2.2)} für $f$. Per Definition gilt dann
    \begin{displaymath}
        \int_X f dx \overset{\text{Def}}{=} \sup_{j\in\N} \int_X v_j dx \overset{(**)}{\underset{\jshortlink{Lem 2.18}}{\le}} \sup_{j\in\N} \int_X f_j dx \overunderset{(**)}{\le}{\jshortlink{Lem 2.18}} \int_X f dx.
    \end{displaymath}
\end{proof}


\jlabel{Kor 2.20}
\begin{kor}
    \begin{enumerate}
        \item
            Seien $\jabb{f_j}{X}{[0,\infty]}$ messbar. Dann gilt
            \begin{displaymath}
                \int_X \left( \sum_{j=1}^\infty f_j \right) dx = \sum_{j=1}^\infty \int_X f_j dx
            \end{displaymath}
        \item
            Sei $\jabb{\omega}{X}{[0,\infty]}$ messbar. Setze für $A\in\Borel(X)$
            \begin{displaymath}
                \mu(A) := \int_X \doubleOne_{A}(x)\cdot \omega(x) dx
            \end{displaymath}
            (Dann ist $\mu$ ein Maß auf $\Borel(X)$ und wird \uline{Gewicht} oder \uline{Dichte} genannt.)
    \end{enumerate}
    \begin{proof}
        \begin{enumerate}
            \item Es gilt
                \begin{displaymath}
                    \begin{split}
                        \uwave{\int_X \left( \sum_{j=1}^\infty f_j \right) dx} &= \int_X\left( \limToInf{n} \sum_{j=1}^n f_j \right) dx \overset{\jshortlink{Thm 2.19}}{=} \limToInf{n} \int_X \left( \sum_{j=1}^n f_j \right)dx\\
                        &\overset{\jshortlink{Lem 2.18}}{=} \limToInf{n} \sum_{j=1}^n \int_X f_j dx = \uwave{\sum_{j=1}^\infty \int_X f_j dx}.
                    \end{split}
                \end{displaymath}
            \item Zeige die Maßeigenschaft.
                \begin{itemize}
                    \item[(M1)]: $\mu(\emptyset) = \int_X \doubleOne_\emptyset(x) \omega(x) dx = \int_X 0 dx = 0$.
                    \item[(M2)]: Seien $A_n \in \Borel(X)$ disjunkt, $n\in\N$. Dann:
                    \begin{displaymath}
                        \begin{split}
                            \mu \left(\bigcup_{n\in\N} A_n \right) &= \int_X \doubleOne_{\bigcup_{n\in\N} A_n}(x)\cdot \omega(x) dx \\
                            &= \int_X \left(\sum_{n=1}^\infty \doubleOne_{A_n}\right)(x) \cdot \omega(x) dx\\
                            &\overset{\text{a)}}{=} \sum_{n=1}^\infty \int_X \doubleOne_{A_n}(x)\omega(x) dx \overset{\text{Def}}{=} \sum_{n=1}^\infty \mu(A_n)
                        \end{split}
                    \end{displaymath}
                \end{itemize}

        \end{enumerate}
    \end{proof}
\end{kor}

\begin{lem}
\jlabel{Lem 2.21}
    Seien $\jabb{f}{X}{[0,\infty]}$ messbar und $\emptyset \ne Y \in \Borel(X)$. Dann sind $\jabb{f|_Y}{Y}{[0,\infty]}$ auf $Y$ und $\jabb{\doubleOne_{Y}\cdot f}{X}{[0,\infty]}$ auf $X$ messbar und es gilt
    \begin{displaymath}
        \int_Y f dx = \int_X \doubleOne_Y\cdot f dx.
    \end{displaymath}
    \begin{proof}
        $f|_Y$ ist messbar wegen Bemerkung 2.2 und $\doubleOne_Y\cdot f$ ist messbar nach \jlink{Satz 2.5}d). Setze
        \begin{displaymath}
            g:= \jabb{\sum_{j=1}^n z_j \cdot \doubleOne_{B_j}}{X}{\R_+}, \ \ \ z_j\in \R_+,\ B_j \in \Borel(X).
        \end{displaymath}
        Dann ist $g$ einfach und es gelten $g|_Y = \sum_{j=1}^n z_j\cdot \doubleOne_{B_j \cap Y}$ und
        \begin{displaymath}
            \begin{split}
                \int_Y g dx &\overset{\text{Def}}{=} \sum_{j=1}^n z_j \cdot\lambda(B_j\cap Y) = \int_X \sum_{j=1}^n z_j \cdot \doubleOne_{B_j \cap Y} dx\\
                &= \int_X \sum_{j=1}^n z_j \cdot \doubleOne_{B_j}\cdot \doubleOne_Y dx = \int_X \doubleOne_Y \cdot g dx.
            \end{split}
        \end{displaymath}
        Damit gilt die Behauptung für einfache Funktionen.\\
        Für den allgemeinen Fall, in dem $\jabb{f}{X}{[0,\infty]}$ beliebig ist und messbar ist, gibt es nach \jlink{Satz 2.11} einfache $\jabb{f_n}{X}{[0,\infty]}$ mit $f_n \nearrow f$ $(n\rightarrow \infty)$. Dann gelten auch $f_n|_Y \nearrow f|_Y$ und $\doubleOne_Y \cdot f_n \nearrow \doubleOne_Y \cdot f$ $(x \rightarrow \infty)$, also
        \begin{displaymath}
            \int_Yf dx \overset{\text{Def}}{=} \limToInf{n} \int_Y f_n dx \overset{\text{$f_n$ einfach}}{=} \limToInf{n} \int_X \doubleOne_Y\cdot f_n dx \overset{\text{Def}}{=} \int_X \doubleOne_Y \cdot f dx
        \end{displaymath}
    \end{proof}
\end{lem}


\subsection*{Schritt C: Integral für $\Rq$-wertige Funktionen}

Sei $\jabb{f}{X}{\Rq}$ messbar. Nach \jlink{Satz 2.8} sind dann auch $\jabb{f_+,f_-}{X}{[0,\infty]}$ messbar.

\begin{defn}
\jlabel{Def 2.22}
    Sei $\emptyset \ne X\in \Bd$. Eine messbare Funktion $\jabb{f}{X}{\Rq}$ heißt \uline{(Lebesgue-) integrierbar}, wenn $\int_X f_+ dx, \int_X f_- dx < \infty$.\\
    In diesem Fall definiert man \uline{(Lebesgue-) Integral)} durch
    \begin{displaymath}
        \int_X f dx := \int_X f(x) dx := \int_X f_+(x) dx - \int_X f_-(x) dx \in \R.
    \end{displaymath}
    Hiervon ist $\jabb{f}{X}{\R}$ ein Spezialfall. Man setzt
    \begin{displaymath}
        \Leb^1(X) := \{\jabb{f}{X}{\R}\ |\ f \text{ \jlink{messbar} und integrierbar}\}.
    \end{displaymath}
\end{defn}

\begin{bem*}
    Sei $\jabb{f}{X}{[0,\infty]}$ messbar. Wegen $f_- = 0$ gilt
    \begin{displaymath}
        f \text{ integrierbar} \Leftrightarrow \int_X f(x) dx < \infty
    \end{displaymath}
\end{bem*}

\begin{bem*}
    Für einen Maßraum $(X, \mathcal{A}, \mu)$ definiert man das Integral $\int f dx$ völlig analog, indem man \mb{$\mathcal{A}$}{$\overline{\Borel}_1$}e Funktionen betrachtet und überall $\lambda(A)$ durch $\mu(A)$ ersetzt.
\end{bem*}

\begin{expl*}
    Sei $X =\N, \mathcal{A} = \PowerSet(\N)$ und $\mu$ das Zählmaß, d.h.: $\mu(A) := |A|$. Schreibe $\jabb{f}{\N}{\Rq}$ als $a_n = f(n)$. Dann $\int_\N f d\mu = \sum_{n=1}^\infty a_n$, falls existent.
\end{expl*}


\begin{satz}
\jlabel{Satz 2.23}
    Sei $\jabb{f}{X}{\R}$ messbar. Dann sind äquivalent:
    \begin{enumerate}
        \item $f$ ist integrierbar.
        \item Es existieren integrierbare $\jabb{u,v}{X}{[0,\infty]}$ mit $f=u-v$ (wobei $u$ und $v$ nie gleichzeitig $\infty$-wertig sind.).
        \item Es existiert ein integrierbares $\jabb{g}{X}{[0,\infty]}$ mit $|f| \le g$.
        \item Die messbare Funktion $\jabb{|f|}{X}{[0,\infty]}$ ist integrierbar.
    \end{enumerate}
    Wenn a)-d) gelten, dann $\int_X f dx = \int_X u dx - \int_X v dx$.\\
    Weiter folgt $\Leb^1(X) = \{\jabb{f}{X}{\R} \ | \ f \text{ messbar}, \ \int_X |f| dx < \infty\}$.
    \begin{proof}
        Wir zeigen die Äquivalenz von a),b),c) und d) durch einen Ringschluss.
        \begin{itemize}
            \item \uline{a) $\Rightarrow$ b)}: Wegen \jlink{Lem 2.21} gilt $u=f_+, \ v=f_-$ ($f_+,f_-$ nie gleichzeitig $\infty$).
            \item \uline{b) $\Rightarrow$ c)}: $g := u+v$ ist integrierbar nach \jlink{Lem 2.18}. $|f| = |u+v| \le u+v =g$.
            \item \uline{c) $\Rightarrow$ d)}: Aus \jlink{Lem 2.18}b) folgt $\int |f|dx \le \int g dx < \infty$.
            \item \uline{d) $\Rightarrow$ a)}: Es gilt $0 \le f_+,f_- \le |f| \Rightarrow f_+,f_-$ integrierbar $\overset{\jshortlink{Def 2.22}}{\Rightarrow} f$ ist integrierbar.
        \end{itemize}
        \uline{Letze Behauptung}: Nach b) gilt: $\exists u,v\ge 0: f=f_+-f_- = u-v \Rightarrow f_++v = f_-+u \Rightarrow \int_X f_+ + \int_X v dx = \int_X f_- dx + \int_X u dx \Rightarrow \int_X f dx = \int_X u dx - \int_X v dx$, da alle Integrale endlich sind.
    \end{proof}
\end{satz}

\jdate{24.11.2008}

{\begin{kor}
\jlabel{Kor 2.24}
    Sei $\jabb{f}{X}{\Rq}$ integrierbar. Dann gilt $\lambda(\{|f| = \infty\}) = 0$.
    \begin{proof}
        Betrachte $A:= \{|f| = \infty\} \in \Bd$. Es gilt $|f| \ge n\cdot \doubleOne_A \ (\forall n\in\N)$. Dann gilt $n\cdot \lambda(A) \overset{\jshortlink{Lem 2.15}}{=} \int_X n\cdot \doubleOne_A dx \overset{\jshortlink{Lem 2.18}}{\le} \int_X |f| dx =: C \overset{\jshortlink{Satz 2.23}}{<} \infty$\\
        $\Rightarrow 0 \le \lambda(A) \le \frac{C}{n} \ (\forall n\in\N)$.
    \end{proof}
\end{kor}

\begin{satz}
\jlabel{Satz 2.25}
    Seien $\jabb{f,g}{X}{\R}$ integrierbar und $\alpha \in \R$. Dann gelten:
    \begin{enumerate}
        \item 
            $\alpha\cdot f$ und (soweit überall definiert) $f+g$ sind integrierbar und es gelten:
            \begin{displaymath}
                \begin{split}
                    \int_X \alpha \cdot f(x) dx &= \alpha\cdot \int_X f(x) dx,\\
                    \int_X (f(x) + g(x)) dx &= \int_X f(x)dx + \int_X g(x) dx.
                \end{split}
            \end{displaymath}
            Somit ist $\Leb^1(X)$ ein Vektorraum und das Integral eine lineare Abbildung von $\Leb^1(X)$ nach $\R$.
        \item Die Funktionen $\max\{f,g\}$ und $\min\{f,g\}$ sind integrierbar.
        \item Wenn $f\le g$, dann $\int_X f(x)dx \le \int_X g(x) dx$. (Das Integral ist monoton.)
        \item $|\int_X f(x)dx| \le \int_X |f(x)| dx$.
        \item
            Sei $\emptyset \ne Y \in \Borel(X)$. Dann sind $f|_Y$ und $\doubleOne_Y\cdot f$ integrierbar und es gilt
            \begin{displaymath}
                \int_Y f|_Y(x) dx = \int_X \doubleOne_Y (x) \cdot f(x) dx.
            \end{displaymath}
        \item
            Seien $\lambda(X) < \infty$ und $\jabb{h}{X}{\R}$ messbar und beschränkt. Dann liegt $h$ in $\Leb^1(X)$ und $|\int_X h(x) dx| \le \lVert h \rVert_\infty \cdot \lambda(X)$.
    \end{enumerate}
    \begin{proof}
        \begin{enumerate}
            \item 
                Es gilt
                \begin{displaymath}
                    (\alpha\cdot f)_\pm = \begin{cases}
                                              \alpha \cdot f_\pm, &\alpha \ge 0\\
                                              [(-\alpha)\cdot (-f)]_\pm = (-\alpha)\cdot f_\mp, &\alpha \le 0.
                                          \end{cases}
                \end{displaymath}
                Ferner sind nach \jlink{Satz 2.23} und \jlink{Lem 2.18} die Funktionen $\alpha\cdot f_\pm$ ($\alpha \ge 0$) und $(-\alpha)\cdot f_\mp$ ($\alpha \le 0$) integrierbar. Somit folgt die Integrierbarkeit von $\alpha\cdot f$ und es gilt
                \begin{displaymath}
                    \begin{split}
                        \int \alpha\cdot f dx &\overset{\jshortlink{Def 2.22}}{=} \int (\alpha\cdot f)_+ dx - \int (\alpha\cdot f)_-dx\\
                        &= \begin{cases}
                               \int \alpha\cdot f_+ dx - \int \alpha \cdot f_- dx. &\alpha \ge 0\\
                               \int (-\alpha)\cdot f_-dx - \int (-\alpha)\cdot f_+dx. &\alpha \le 0
                           \end{cases}\\
                        &\overset{\jshortlink{Lem 2.18}}{\underset{\jshortlink{Def 2.22}}{=}} \alpha\cdot \int fdx. \hspace{20pt} (\text{Beachte $-\alpha > 0$ für $\alpha < 0$})
                    \end{split}
                \end{displaymath}
                Ferner gilt $f+g = \underbrace{f_+ + g_+}_{=:u} - \underbrace{(f_- + g_-)}_{=:v}$. Dabei sind $u,v$ nach \jlink{Lem 2.18} und \jlink{Satz 2.23} integrierbar und nie gleichzeitig $\infty$-wertig.
                
                \jspacesmall
                
                $\ulcorner$\jemph{Denn}: Sei z.B. $f(x)=\infty$. Dann gilt $f_+(x)=\infty, \ f_-(x)=0$. Dann ist $u(x) =\infty$. Ferner gilt $g(x)\ne \infty$ nach Voraussetzung, woraus $v(x) = g_-(x) \ne \infty$.$\lrcorner$
                
                \jspacesmall
                
                Aus \jlink{Satz 2.23} folgt, dass $f+g$ integrierbar ist und es gilt
                \begin{displaymath}
                    \begin{split}
                        \int(f+g)dx &= \int(f_+ + g_+)dx - \int(f_- + g_-)dx\\
                        &= \left(\int f_+dx + \int g_+dx \right) - \left(\int f_- + \int g_- dx \right)\\
                        &\overset{\jshortlink{Def 2.22}}{=} \int f dx + \int g dx.
                    \end{split}
                \end{displaymath}
                
            \item
                $\max\{f,g\}$ ist \jlink{messbar} nach \jlink{Satz 2.7}, Ferner gilt $0\le \max\{f,g\}\le |f|+|g|$, wobei $|f|+|g|$ nach a) und \jlink{Satz 2.23} integrierbar ist. Dann folgt mit \jlink{Satz 2.23}, dass $\max\{f,g\}$ integrierbar ist. Für das Minimum zeigt man es genauso.
                
            \item
                Sei $f\le g$. Dann gilt $f_+\le g_+$ und $f_- = (-f)_+ \ge (-g)_+ = g_-$. Damit folgt
                \begin{displaymath}
                    \int f dx = \int f_+ dx - \int f_- dx \overset{\jshortlink{Lem 2.18}}{\le} \int g_+ - \int g_- dx = \int g dx.
                \end{displaymath}
            
            \item
                Da $\pm f\le |f|$ gilt, liefern a) und c)
                \begin{displaymath}
                    \pm \int f dx = \int \pm f dx \le \int |f| dx.
                \end{displaymath}

            \item
                $f|_Y$ ist auf $Y$ und $\doubleOne_Y\cdot f$ ist nach \jlink{Bem 2.2}, \jlink{Satz 2.5} und \jlink{Satz 2.7} auf $X$ \jlink{messbar}. Klar ist, dass 
                \begin{displaymath}
                    \tag{$*$}
                    (f|_Y)_\pm = f_\pm|_Y \text{ und } (\doubleOne_Y\cdot f)_\pm = \doubleOne_Y\cdot f_\pm
                \end{displaymath}
                gelten. Nach \jlink{Lem 2.18} gilt $\int_X \doubleOne_Y \cdot f_\pm dx \le \int_X f_\pm dx \overset{\text{n. Vor.}}{<} \infty$.\\
                Mit $(*)$ und \jlink{Def 2.22} ist also $\doubleOne_Y\cdot f$ untegrierbar und es gilt
                \begin{displaymath}
                    \begin{split}
                        \int_X \doubleOne_Y\cdot f dx &\overset{(*)}{\underset{\jshortlink{Def 2.22}}{=}} \int_X \doubleOne_Y \cdot f_+ dx - \int_X \doubleOne_Y \cdot f_- dx\\
                        &\overset{\jshortlink{Lem 2.21}}{=} \int_Y (f_+)|_Y dx -\int_Y (f_-)|_Y dx \overset{(*)}{=} \int_Y f|_Y dx.
                    \end{split}
                \end{displaymath}
            
            \item
                Sei nun $\lambda(X) < \infty$. Da $|h| \le \lVert h \rVert_\infty \doubleOne_X$ integrierbar ist, ist nach \jlink{Satz 2.23} integrierbar und nach d) und c) folgt
                \begin{displaymath}
                    \left| \int_X h dx \right| \le \int_X |h| dx \le \lVert h \rVert_\infty \cdot \lambda(X).
                \end{displaymath}
        \end{enumerate}
    \end{proof}
\end{satz}

\begin{expl*}
    Sei $\jabb{f}{X}{\R}$ einfach mit Normalform $f = \sum_{k=1}^m y_k \cdot \doubleOne_{A_k}$, wobei $y_k=0$, falls $\lambda(A_k) = \infty$. Dann ist $f$ integrierbar und $\int_X f(x)dx = \sum_{k=1}^m y_k\cdot \lambda(A_k)$.
    \begin{proof}
        \jlink{Satz 2.25}a), da $\int \doubleOne_{A_k} = \lambda(A_k)$.
    \end{proof}
\end{expl*}


\jdate{28.11.2008}


\jlabel{Bem 2.26}
\begin{bem}
    \begin{enumerate}
        \item
            Sei $\jabb{f}{X}{\Rq}$ integrierbar und $X=A\dcup B$ für disjunkte $A,B\in \Bd$. Dann gilt
            \begin{displaymath}
                \int_X f dx = \int_X (\doubleOne_A + \doubleOne_B) \cdot f dx \overset{\jshortlink{Satz 2.25}a)}{=} \int_X \doubleOne_A \cdot f dx + \int_X \doubleOne_B \cdot f dx
            \end{displaymath}
        \item
            Sei $\jabb{f}{[a,b]}{\R}$ stetig und Riemannintegrierbar. Nach \jlink{Satz 2.25}f) ist $f$ Lebesgueintegrierbar.\\
            Weiter gilt $R-\int_a^b f(x)dx = \int_{[a,b]} f(x) dx$.\\
            Wir schreiben von nun an auch, $\int_a^bf(x)dx$ für das Lebesgueintegral. Der Hauptsatz der Differential- und Integralrechnung gilt auch für das Lebesgueintegral aus Ana I.
            \begin{proof}
                Es gilt 
                \begin{displaymath}
                    R-\int_a^b f(x) dx = \limToInf{n} \sum_{j=1}^n f(\underbrace{a + j\cdot \frac{b-a}{n}}_{=: t_{jn}}) = \limToInf{n} \int_{[a,b]} u_n dx,
                \end{displaymath}
                wobei
                \begin{displaymath}
                    u_n = \sum_{j=1}^n f(t_{jn})\cdot \doubleOne_{[t_{j-1,n},t_{j,n}]}.
                \end{displaymath}
                Ferner $\lVert f- u_n \rVert_\infty \xrightarrow{n\rightarrow \infty}0$ (da $f$ gleichmäßig stetig ist, vergleiche Ana1 §6). Damit gilt
                \begin{displaymath}
                    \begin{split}
                        |\int_{[a,b]} f(x)dx - \int_{[a,b]} u_n dx| &\overset{\jshortlink{Satz 2.25}}{\le} \int_{[a,b]} |f-u_n|dx \\
                        &\overset{\jshortlink{Satz 2.25}}{\le} \lVert f-u_n\rVert_\infty \cdot (b-a) \xrightarrow{n\rightarrow \infty} 0.
                    \end{split}
                \end{displaymath}
            \end{proof}
        \item
            \uline{Warnung}: Es gibt stetige, uneigentlich Riemannintegrierbare Funktionen, die nicht Lebesgueintegrierbar sind.
            \begin{expl*}
                Sei $X=[1,\infty], \ f(x) = \frac{\sin(x)}{x}$. Aus Ana I §6 wissen wir: $f$ ist uneigentlich Riemannintegrierbar und
                \begin{displaymath}
                    |f| \ge \sum_{n=1}^\infty \frac{c}{2n} \cdot \doubleOne_{[\pi\cdot n + \frac{\pi}{2}, \pi\cdot n + \frac{3}{4}\cdot \pi]} =: g
                \end{displaymath}
                für ein $c>0$. Damit folgt
                \begin{displaymath}
                    \int_X g(x) dx \overset{\text{Bem 2.10}}{=} \sum_{n=1}^\infty \frac{c}{2n} \cdot \int_X \doubleOne_{[\pi\cdot n + \frac{\pi}{2}, \pi\cdot n + \frac{3}{4}\cdot \pi]} = \frac{c\pi}{2} \cdot \sum_{n=1}^\infty \frac{1}{n} = \infty
                \end{displaymath}
                Damit folgt, dass $g$ nicht integrierbar ist und \jlink{Lem 2.18} liefert
                \begin{displaymath}
                    \int_X|f|dx \ge \int_X g(x)dx = \infty. 
                \end{displaymath}
                Also ist $f$ nach \jlink{Satz 2.23} nicht integrierbar.
            \end{expl*}

    \end{enumerate}
\end{bem}

\chapter{Vertiefung der Theorie}
    Weiterhin sei $\emptyset \ne X \in \Bd$.
    
    % Nullmengen
\section{Nullmengen}

Problem: $\Leb^1(X)$ ist Vektorraum, aber $\lVert f \rVert_1 = \int |f|dx$ ist keine Norm auf $\Leb^1(X)$, da $\int \doubleOne_N dx = 0$ für alle $N \in \Bd$ mit $\lambda(N) = 0$, z.B. $N=\Q, d=1$.

\begin{defn}
\jlabel{Def 3.1}
    Eine Menge $N\in \Bd$ mit $\lambda_d(N) = 0$ heißt ($d$-dimensionale, Borel-) \uline{Nullmenge} (NM).
\end{defn}


\jlabel{Bem 3.2}
\begin{bem}
    \begin{enumerate}
        \item 
            Wir haben bereits die eindimensionalen Nullmengen $\Q$ und die Cantormenge $C$, sowie Nullmengen in höheren Dimensionen wie Hyperebenen und Graphen stetiger Funktionen gesehen.
        \item
            Wenn $M,N\in \Bd, \ M\subset N$ und $N$ eine Nullmenge ist, dann ist auch $M$ eine Nullmenge.\\
            Wenn $N_j \in \Bd$ Nullmengen sind, dann ist $N= \bigcup_{j\in\N} N_j$ eine Nullmenge.
            \begin{proof}
                Dass $N = \bigcup_{j\in\N} N_j \in\Bd$ liegt, ist klar. Nach \jlink{Satz 1.14} gilt:
                \begin{displaymath}
                    0\le \lambda_d(N) \le \sum_{j=1}^\infty \lambda_d(N_j) = \sum_{j=1}^\infty 0 = 0 \Rightarrow \lambda_d(N) = 0
                \end{displaymath}
                Damit sind abzählbare Mengen Nullmengen. Ferner gilt:
                \begin{displaymath}
                    \Q \times \R^{d-1} = \bigcup_{n=1}^\infty \{q_n\} \times \R^{d-1}
                \end{displaymath}
                ist eine $d$-dimensionale Nullmenge, wobei $\Q = \{q_1, q_2, \dots\}$.\\
                Beachte: $\R := \bigcup_{x\in\R} \{x\}$ ist keine eindimensionale Nullmenge (Vereinigung nicht abzählbar).
            \end{proof}
        \item
            Sei $A\in\Bd$. Nach \jlink{Thm 1.25} gilt, dass $A$ genau dann eine Nullmenge ist, wenn offene Intervalle $I_j \ (j\in\N)$ existieren mit:
            \begin{displaymath}
                A\subset \bigcup_{j=1}^\infty I_j, \ \sum_{j=1}^\infty \lambda(I_j) \le \epsilon.
            \end{displaymath}
        \item
            Sei $N\in\Bd$ eine Borel-Nullmenge. Eine Teilmenge $M\subset N$ heißt dann Lebesque-Nullmenge. Es gibt ein $C\subset \R$ (Cantormenge) mit $C \notin \Borel_1$.\\
            $\Rightarrow$ Dieses $M$ ist \uline{keine} Borel-Nullmenge (AE 3. kor IX 5.30)\\
            Nach Aufgabe 3.1 ist 
            \begin{displaymath}
                \Leb_d = \{A \subset \R^d: A = B\cup N,\ B\in \Bd, N \text{ ist Lebesgue-Nullmenge}\}
            \end{displaymath}
            eine $\sigma$-Algebra und $\tilde{\lambda}_d(A) = \lambda_d(B)$ (wobei $A= B\cup N$ für $B\in \Bd$ und eine Nullmenge $N$) ist Maß auf $\Leb_d$. Ferner stimmt das Integral bezüglich $\tilde{\lambda}_d$ für Borelfunktionen $f$ mit unserem Integral dem bezüglich $\lambda_d$ überein.\\
            Es gibt in \jlink{(1.9)} $\Leb_d = \mathcal{A}(\lambda_d)$ (AE 3: Theorem IX. 5. 7+8)\\
            Ferner: Sei $\jabb{f}{[a,b]}{\R}$ Riemannintegrierbar. Man kann zeigen, dass $f$ außerhalb einer Lebesgueschen Nullmenge stetig ist. Da $f$ beschränkt ist, ist es folglich integrierbar bezüglich dem fortgesetzen Lebesguemaß $\tilde{\lambda}_d$ und Riemannintegral und Lebesgueintegral stimmen überein (Elstrodt, Satz IV 6.1).
    \end{enumerate}
\end{bem}

\begin{defn}
\jlabel{Def 3.3}
    Eine Eigenschaft $E$ besteht für fast alle \textit{(f.a.)} $x\in X$ oder fast überall $\fu$, wenn es eine Nullmenge $N$ gibt, sodass $E$ für alle $x\in X\backslash N$ gilt.
\end{defn}

\begin{expl}
\jlabel{Bsp 3.4}
    Sei $\jabb{f}{X}{\Rq}$ integrierbar. Nach Korollar 2.24 ist die Menge $N:= \{|f| = \infty\}$ eine Nullmenge, also: $f(x) \in \R$ für alle $x\in X\backslash N$, also ist $f$ fast überall endlich.
\end{expl}


\jlabel{Lem 3.5}
\begin{lem}
    \begin{itemize}
        \item 
            Sei $\jabb{f}{X}{\Rq}$ integrierbar und $\jabb{g}{X}{\Rq}$ messbar und sei $f=g \ \fu$. Dann ist $g$ integrierbar und $\int_X f dx = \int_X g dx$.\\
            Insbesondere kann man $f$ durch $\tilde{f} := \doubleOne_{\{|f| < \infty\}} \cdot f$ ersetzen (vgl. Beispiel 3.4) und es gilt $\int_X f dx = \int_X \tilde{f} dx$).
        \item 
            Wenn $\jabb{f,g}{X}{[0,\infty]}$ messbar und $f=g \ \fu$, dann gilt auch $\int_X f dx = \int_X g dx$.
    \end{itemize}
    \begin{proof}
        Nach Voraussetzung: $\exists \text{ NM } N$ mit $f(x) = g(x) \ \forall x\in X\backslash N$.\\
        Da $g$ messbar ist, existiert:
        \begin{displaymath}
            \begin{split}
                \int_X|g|dx &= \int_X \doubleOne_N |g| dx + \int_X \doubleOne_{X\backslash N} \underbrace{|g|}_{= |f|} dx = \int_X \doubleOne_N|f|dx \overset{\jshortlink{Lem 2.18}}{=} 0\\
                &\overset{\jshortlink{Bem 2.26}}{=} \int_X|f|dx < \infty
            \end{split}
        \end{displaymath}
        Nach Voraussetzung folgt mit \jlink{Satz 2.23}, dass $g$ integrierbar ist.\\
        Ferner liefert \jlink{Satz 2.25}:
        \begin{displaymath}
            \begin{split}
                &0 \le \left|\int_N g(x) dx\right| \le \int_N |g(x)| dx \overset{\jshortlink{Lem 2.18}}{=} 0\\
                &\Rightarrow \int_X g dx \overset{Bem 2.26}{=} \underbrace{\int_N g dx}_{ = 0= \int_N f dx} + \int_{X\backslash N} g dx = \int_X f dx
            \end{split}
        \end{displaymath}
        Zweite Behauptung folgt genauso.
    \end{proof}
\end{lem}

\begin{defn}
\jlabel{Def 3.6}
    Funktionen $\jabb{f_n}{X}{\Rq}$ messbar $(\forall n\in\N)$ sind fast überall konvergent, wenn $f_n(x)$ für $n\rightarrow \infty$ und fast alle $x\in \Rq$ konvergiert. Wenn $f_n(x) \xrightarrow{n\rightarrow \infty} f(x)$ für fast alle $x\in\Rq$, dann konvergiert $f_n$ fast überall gegen $f$.
\end{defn}

\begin{lem}
\jlabel{Lem 3.7}
    Seien $\jabb{f_n}{X}{\Rq}$ messbar für alle $n\in\N$ und fast überall konvergent. Dann existiert eine messbare Funktion $\jabb{f}{X}{\Rq}$, sodass $f_n\xrightarrow{n\rightarrow\infty} f \ \fu$.\\
    Jede andere messbare Funktion $\jabb{g}{X}{\Rq}$ mit $f_n \xrightarrow{n\rightarrow \infty} f \ \fu$ ist fast überall gleich f.\\
    \uline{Bemerkung}: Nicht jeder fast überall Limes messbarer Funktionen ist messbar.
    \begin{proof}
        Nach Voraussetzung existiert eine Nullmenge $N$, sodass\\
        $\exists \limToInf{n} f_n(x)\in\Rq, \ \forall x\in X\backslash N$.\\
        Nach \jlink{Satz 2.8} ist $\doubleOne_{X\backslash N} \cdot f_n$ messbar. Ferner konvergiert $\doubleOne_{X\backslash N}\cdot f_n$ für $n \rightarrow \infty$ punktweise gegen $\jabb{f}{X}{\Rq}$ mit: $f(x) = \begin{cases} \limToInf{n} f_n(x), & x\in X\backslash N\\ 0, & x\in N \end{cases}$\\
        Mit \jlink{Satz 2.7} folgt, dass $f$ messbar ist.\\
        Nach der Konstruktion gilt: $f_n \underset{\fu}{\xrightarrow{n \rightarrow \infty}} f$. Wenn $f_n(x) \xrightarrow{n\rightarrow \infty} g(x)$ fast überall für eine messbare Funktion $g$, dann existiert eine Nullmenge $N_1$, sodass $f_n(x) \xrightarrow{n\rightarrow \infty} g(x) \ (\forall x\in X\backslash N_1) \Rightarrow f_n(x) \rightarrow f(x)$ und $f_n(x) \rightarrow g(x) \ \forall x\notin N\cup N_1 =: N_2$ (Nullmenge). Mit der Eindeutigkeit des Limes folgt dann:\\
        $f(x) = g(x) \ \forall x\in X\backslash N_2$.
    \end{proof}
\end{lem}

\begin{expl*}
    Sei $M\notin \Borel_1$ die Lebesgue-Nullmenge aus Bemerkung 3.2d), wobei $M\subset C$. Dann konvergiert $f_n=0 \ \fu$ gegen $f = \doubleOne_M$, da $f_n(x) = f(x) = 0 \ \forall x \in \R\backslash C$. $C$ ist eine Nullmenge.\\
    \uline{Aber}: $f= \doubleOne_M$ ist nicht messbar.
\end{expl*}

\begin{bem}
\jlabel{Bem 3.8}
    Es gibt folgende Variante des Satzes von der Monotonen Konvergenz.\\
    Seien $\jabb{f_n}{X}{[0,\infty]}$ messbar $(\forall n\in\N)$, sodass für jedes $n\in \N$ $f_n \le f_{n+1} \ \fu$.\\
    Dann existiert eine messbare Funktion $\jabb{f}{X}{[0,\infty]}$ mit $f_n \underset{\fu}{\xrightarrow{n\rightarrow \infty}} f$ und $\limToInf{n} \int_X f_n dx = \int_X f dx$.
    \begin{proof}
        Nach Voraussetzung: $\forall n\in\N \ \exists$ eine Nullmenge $N_n$ mit $f_n(x) \le f_{n+1},$ $\forall x \in X\backslash N_n$.\\
        Die Menge $N=\bigcup_{n\in\N} N_n$ ist eine Nullmenge. Daraus folgt $f_n(x) \le f_{n+1}(x) \ \forall x\notin N, \ n\in\N$.\\
        Setze $\tilde{f}_n = \doubleOne_{X\backslash N} \cdot f_n \Rightarrow \tilde{f}_n = f_n \ \fu, \ \tilde{f}_n \le \tilde{f}_{n+1}, \ (\forall n\in\N)$.\\
        Setze $f := \sup_{n\in\N} \tilde{f}_n$ ist messbar.
        \begin{displaymath}
            \int f dx \overset{\jshortlink{Thm 2.19}}{=} \limToInf{n} \int \tilde{f}_n dx \overset{\jshortlink{Lem 3.5}}{=} \limToInf{n} \int f_n dx
        \end{displaymath}
    \end{proof}
\end{bem}
    
    % Der Lebesguesche Konvergenzsatz
\section{Der Lebesguesche Konvergenzsatz}

\begin{thm}[Lemma von Fatou]
\jlabel{Thm 3.9}
\jlabel{Fatou}
    Seien $\jabb{f_n}{X}{[0,\infty]}$ für jedes $n\in\N$ messbar. Dann gilt:
    \begin{displaymath}
        \int_X \varliminf_{n\rightarrow \infty} f_n(x) dx \le \varliminf_{n\rightarrow\infty} \int_X f_n(x) dx
    \end{displaymath}
    Speziell konvergiere $f_n$ fast überall gegen ein $\jabb{f}{X}{[0,\infty]}$. Dann folgt:
    \begin{displaymath}
        \int_X f(x) dx \le \lim_{n\rightarrow \infty} \int_X f_n(x) dx
    \end{displaymath}
    (Damit ist $f$ integrierbar, falls $\left (\int f_n(x) dx \right)_{n\in\N}$ beschränkt ist.)
    \begin{proof}
        Setze $g_j := \inf_{n\ge j} f_n$ für jedes $j\in\N$. Dann folgt $g_j \le g_{j+1}$ und für alle $j\in\N$ ist $g_j$ nach \jlink{Satz 2.7} messbar. Ferner gilt $\sup_{j\in\N} g_j = \varliminf_{n\rightarrow \infty} f_n$ und $g_j \le f_n \ (\forall n \ge j)$. Damit gilt:
        \begin{displaymath}
            \int_X \varliminf_{n\rightarrow \infty} f_n(x)dx = \int_X \sup_{j\in\N} g_j(x)dx \overset{\jshortlink{Def 2.9}}{=} \sup_{j\in \N} \int_X g_j(x)dx
        \end{displaymath}
        \uline{Ferner}: $g_j \le f_n \ \forall n\ge j$. Mit \jlink{Lem 2.18} folgt dann:
        \begin{displaymath}
            \begin{split}
                &\int g_j(x) dx \le \int f_n(x) dx \ (\forall n\ge j) \\
                &\Rightarrow \int g_j(x) dx \le \inf_{n\ge j} \int f_n(x) dx \\
                &\Rightarrow \int \varliminf_{n\rightarrow \infty} f_n(x) dx \le \sup_{j\in N} \inf_{n\ge j} \int f_n(x) dx = \varliminf_{n\rightarrow \infty} \int f_n(x) dx
            \end{split}
        \end{displaymath}
        Für die zweite Behauptung: Sei $N$ eine Nullmenge mit $f_n(x) \rightarrow f(x) \ \forall x\in X\backslash N$. Dann:
        \begin{displaymath}
            \begin{split}
                \int f(x) dx &\overset{\jshortlink{Lem 3.5}}{=} \int \doubleOne_{X\backslash N} \cdot f(x) dx = \int \lim_{n\rightarrow \infty} \doubleOne_{X\backslash N} \cdot f_n(x) dx \\
                &\overset{\text{s.o.}}{\le} \limToInf{n} \int \doubleOne_{X\backslash N} \cdot f_n(x) dx \overset{\jshortlink{Lem 3.5}}{=} \limToInf{n} \int f_n(x) dx.
            \end{split}
        \end{displaymath}
    \end{proof}
\end{thm}

\begin{thm}[Legesgue, majorisierte Konvergenz]
\jlabel{Thm 3.10}
    Seien $\jabb{f_n}{X}{\Rq}$ messbar für alle $n\in\N$ und $\jabb{g}{X}{[0,\infty]}$ integrierbar, sodass $f_n$ fast überall konvergiert für $n\rightarrow \infty$ und $|f_n| \le g \ \fu$ für alle $n\in\N$. Dann gibt es ein integrierbares $\jabb{f}{X}{\Rq}$, sodass $f_n \underset{\fu}{\xrightarrow{n\rightarrow \infty} f}$ und
    \begin{displaymath}
        \limToInf{n} \int_X f_n(x)dx = \int_X f(x) dx \text{ und } \int_X |f_n(x)-f(x)|dx \xrightarrow{n\rightarrow \infty} 0
    \end{displaymath}
    Diese Aussage gilt auch für jedes $\jabb{\tilde{f}}{X}{\Rq}$ anstelle von $f$, wenn $\tilde{f} = f \ \fu$.
\end{thm}

\begin{bem*}
    \begin{enumerate}
        \item Sei $\lambda(X) < \infty, \ |f_n(x)| \le M \ (\forall x, n) \Rightarrow g:= M\cdot \doubleOne_X$ integrierbar und $|f_n| \le g$ (einfache Majorante).
        \item Sei $\{q_1, q_2,\dots\} = \Q \cap [0,1]$, setze $f_n := \doubleOne_{\{q_1, \dots, q_n\}} \Rightarrow |f_n| \le \doubleOne_{[0,1]}$ und $f_n \rightarrow \doubleOne_{\Q \cap [0,1]} = f$\\
        Damit ist der Satz von Lebesgue anwendbar, aber $f$ ist nicht Riemannintegrierbar, also ist \jlink{Thm 3.10} für das Riemannintegral sinnlos.
    \end{enumerate}

\end{bem*}


\begin{bem}
\jlabel{Bem 3.11}
    Ohne Majorante kann die Aussage von \jlink{Thm 3.10} falsch sein. Beispiele für $X = \R$:
    \begin{enumerate}
        \item 
            $f_n = n\cdot \doubleOne_{(0, \frac{1}{n})} \rightarrow f = 0$ (p.w.), aber $\int_\R f_n(x) dx = 1$ und $\int_\R f(x)dx = 0$.
            
        \item
            $f_n = \doubleOne_{[m, \infty]} \rightarrow 0$ (p.w.). Hier gilt sogar $f_n \ge f_{n+1}$. Trotzdem ist: \\
            $\int f dx = 0 < \infty = \limToInf{n} \underbrace{\int f_n dx}_{=\infty}$.
    \end{enumerate}

\end{bem}


\jdate{01.12.2008}


\begin{proof}[Beweis von \jlink{Thm 3.10}]
    Nach \jlink{Lem 3.7} existiert ein integrierbares $\jabb{\hat{f}}{X}{\Rq}$ mit $f_n \xrightarrow{n\rightarrow \infty} \hat{f} \ \fu$.\\
    Wie im Beweis von Bemerkung 3.8. existiert eine Nullmenge $N$, sodass\\
    $|f(x)| \le g(x) \ (\forall x\notin N, n \in\N)$ und $f_n(x) \xrightarrow{n \rightarrow \infty} \hat{f}(x) \ (\forall x\in N)$\\
    $\Rightarrow |\hat{f}(x)| \le g(x) \ (\forall x\notin N)$.\\
    $\overset{\jshortlink{Satz 2.23}}{\Rightarrow} \doubleOne_{X\backslash N}\cdot f, \ \doubleOne_{X\backslash N}\cdot \hat{f}$ sind integrierbar $(\forall n\in\N) \overset{\jshortlink{Lem 3.5}}{\Rightarrow} f_n, \hat{f}$ sind integrierbar.\\
    Sei $N_1 = N \cup \{|\hat{f}| = \infty\} \cup \{g=\infty\}$. Nach Korollar 2.24 ist $N_1$ eine Nullmenge.\\
    Setze $g_n := |f| + \doubleOne_{X\backslash N_1} \cdot g - \doubleOne_{X\backslash N_1}\cdot |f-f_n|$ und $f=\jabb{\doubleOne_{X\backslash N_1}\cdot \hat{f}}{X}{\R}$ ist integrierbar. $\Rightarrow f_n \xrightarrow{n\rightarrow \infty} f \ \fu$.\\
    Es gilt: $g_n \xrightarrow{n\rightarrow \infty}|f|+g \ \fu$. Da $|f_n-f| \le |f_n| + |f| \le g + |f|$ (auf $X\backslash N_1$), ist $g_n \ge 0 \ (\forall n\in\N)$.\\
    \uline{Dann}:
    \begin{displaymath}
        \begin{split}
            \int (|f|+g) dx &\overset{\text{Fatou}}{\le} \liminf_{n\rightarrow \infty} \int g_n dx \\
            &\overset{\jshortlink{Satz 2.25}}{=} \liminf_{n\rightarrow \infty}(\int_{X\backslash N_1} |f| + g dx - \int_{X\backslash N_1} |f-f_n| dx) \\
            &\overset{\jshortlink{Lem 3.5}}{=} \underbrace{\int_X(|f| + g)dx}_{< \infty,\text{ da $f,g$ int'bar}} - \underbrace{\lim_{n\rightarrow\infty} \int_X|f-f_n|dx}_{\ge 0}
        \end{split}
    \end{displaymath}
    $\Rightarrow \limToInf{n} \int_X |f-f_n| dx = 0$. Damit folgt die Behauptung.\\
    (Beachte: $g_n$ ist messbar nach \jlink{Satz 2.8})
\end{proof}


\begin{kor}
\jlabel{Kor 3.12}
    Sei $\jabb{f}{X}{\overline{R}}$ messbar und $A_n \in \Borel(X)$ mit $A_n \subset A_{n+1} \ (\forall n\in\N)$ und $X = \bigcup_{n\in\N} A_n$.\\
    Weiter seien alle $f_n = \doubleOne_{A_n} \cdot f$ integrierbar und $\sup_{n\in\N} \int_{A_n} |f|dx < \infty$. Dann ist $f$ integrierbar und es gilt:
    \begin{displaymath}
        \int_{X}fdx = \limToInf{n} \int_{A_n} f dx.
    \end{displaymath}
    Falls zusätzlich $X\subset \R$ ein Intervall ist, sowie $f$ stetig und $|f|$ auf $X$ uneigentlich Riemannintegrierbar sind, dann ist $f$ integrierbar und das Riemann- und Lebesgueintegral stimmen überein.
    \begin{proof}
        Sei $f_n = \doubleOne_{A_n}\cdot f \ (\forall n\in\N)$. Nach Voraussetzung gilt: $f_n \xrightarrow{n\rightarrow \infty} f$ (pw). Ferner: $|f_n| = \doubleOne_{A_n}\cdot |f| \le \doubleOne_{A_{n+1}}|\cdot f = |f_{n+1}| \ (\forall n\in\N)$. Aus \jlink{Thm 2.19} folgt: 
        \begin{displaymath}
            \int_X f dx = \sup_{n\in\N} \int_X |f_n| dx = \sup_{n \in\N} \int_{A_n} |f| dx < \infty
        \end{displaymath}
        $\overset{\jshortlink{Satz 2.23}}{\Rightarrow} f$ ist integrierbar.\\
        Weiter gilt $|f_n| \le |f| \ (\forall n\in\N)$, also ist $|f|$ eine Majorante der $f_n$.\\
        Nach \jlink{Thm 3.10} gilt nun:
        \begin{displaymath}
            \int_{A_n} f dx = \int_X f_n dx \xrightarrow{n \rightarrow \infty} \int f dx
        \end{displaymath}
        Für die letzte Behauptung wähle $a_n+1 \le a_n < b_1 \le b_{n+1} \ (n \in\N)$ mit $X = \bigcup_{n\in\N} [a_n, b_n]$. Da $|f|$ uneigentlich riemannintegrierbar ist, konvergiert $\int_{a_n}^{b_n} |f| dx$, ist aber beschränkt. Betrachte $A_n = [a_n, b_n]$. Dann folgt die Behauptung aus dem ersten Beweisteil und $R-\int_{a_n}^{b_n} f dx = \int_{[a_n,b_n]} f dx$. (siehe Bemerkung 2.26)
    \end{proof}
\end{kor}

\begin{expl*}
    Sei $X = [1,\infty)$. Es gilt:\\
    $\int_1^\infty x^{-\frac{3}{2}}dx = \limToInf{b} \underbrace{\int_1^b x^{-\frac{3}{2}} dx}_{-2 \cdot \frac{1}{\sqrt{x}}} = \limToInf{b} \left( 2- \frac{2}{\sqrt{b}}\right) = 2$\\
    $\overset{\text{Kor 3.12}}{\Rightarrow} g(x) := x^{-\frac{3}{2}}$ ist integrierbar auf $X$.\\
    Setze $f_n(X) = x^{-\frac{3}{2}} \cdot \sin(\frac{x}{n})$ für $n\in\N, X \ge 1$. Dann folgt $f_n(x) \xrightarrow{n\rightarrow\infty} 0 \ (\forall x \ge 1)$. $|f_n| \le g \ (\forall n\in\N) \overset{\text{Thm 3.10}}{\Rightarrow} \int f_n dx \rightarrow 0, \int |f_n| dx \rightarrow 0 \ (n\rightarrow\infty)$.
\end{expl*}


\jlabel{Kor 3.13}
\begin{kor}
    \begin{enumerate}
        \item 
            Seien $\jabb{f_j,g}{X}{\R}$ integrierbar für jedes $j\in\N$. Sei $N$ eine Nullmenge, sodass $g_n(x)=\sum_{j=1}^n f_j(x)$ in $\Rq$ für $n\rightarrow\infty$ und alle $x\in X\backslash N$ konvergiert und sodass $|g_n(x)|\le g(x) \ (\forall n\in\N, \ x\in X\backslash N)$. Setze $\sum_{j=1}^\infty f_j(x) := 0$ für $x \in N$. Dann ist $\jabb{\sum_{j=1}^\infty f_j}{X}{\Rq}$ integrierbar und es gilt 
            \begin{displaymath}
                \int_X \sum_{j=1}^\infty f_j(x) dx = \sum_{j=1}^\infty \int_X f_j(x) dx.
            \end{displaymath}
        \item
            Sei $\jabb{f}{X}{R}$ integrierbar und $X = \bigcup_{j\in N} A_j$ für disjunkte $A_j \in \Borel(X)$. Dann gilt:
            \begin{displaymath}
                \int_X f(x) dx = \sum_{j=1}^\infty \int_{A_j} f(x) dx.
            \end{displaymath}
    \end{enumerate}
    \begin{proof}
        \begin{enumerate}
            \item
                Da $|g_n| \le g \ $\fu und $g_n \xrightarrow{n\rightarrow \infty} \sum_{j=1}^\infty f_j \ $(\fu, ist $\sum_{j=1}^\infty f_j$ integrierbar und 
                \begin{displaymath}
                    \begin{split}
                        \exists \int_X \underbrace{\sum_{j=1}^\infty f_j dx}_{=f} &= \int_X \limToInf{n} g_n dx \overset{\jshortlink{Thm 3.10}}{=} \limToInf{n} \int_X g_n dx \\
                        &\overset{\jshortlink{Satz 2.25}}{=} \limToInf{n} \sum_{j=1}^\infty \int_X f_j dx = \sum_{j=1}^\infty \int_X f_j dx.
                    \end{split}
                \end{displaymath}
                
            \item
                Setze $f_j := \doubleOne_{A_j} \cdot f, \ g := |f|$. Dann gilt $|\sum_{j=1}^n f_j| = |\doubleOne_{A_1 \cup \dots \cup A_n} \cdot f| \le |f|$ und $\sum_{j=1}^\infty f_j = f$. Also folgt b) aus a).
        \end{enumerate}
    \end{proof}
\end{kor}

\begin{thm}[Stetigkeitssatz]
\jlabel{Thm 3.14}
\jlabel{Stetigkeitssatz}
    Seien $U\subset \R^k$ offen, $t_0 \in U$ und $\jabb{f}{U\times X}{\R}$ mit den folgenden Eigenschaften gegeben.
    \begin{enumerate}
        \item Für jedes $t\in U$ ist die Funktion $x \mapsto f(t,x)$ von X nach $\R$ messbar.
        \item Es gibt ein integrierbares $\jabb{g}{X}{[0,\infty]}$ und Nullmengen $N_t$ für jedes $t \in U$, sodass $|f(t,x)| \le g(x)$ für alle $t\in U$ und alle $x\in X\backslash N_t$.
        \item Es gibt eine Nullmenge $N$, sodass die Funktion $t \mapsto f(t,x)$ von $U$ nach $\R$ für jedes $x \in X\backslash N$ bei $t_0$ stetig ist.
    \end{enumerate}
    Dann ist die Funktion $\jabb{F}{U}{\R}, F(t) = \int_X f(t,x)dx$, für alle $t\in U$ definiert und bei $t_0$ stetig. Das heißt:
    \begin{displaymath}
        \lim_{t\rightarrow t_0} F(t) = \lim_{t\rightarrow t_0} \int_X f(t,x)dx \overset{!}{=} \int_X f(t_0,x) = F(t_0).
    \end{displaymath}
    \begin{proof}
        Nach a) und b) ist $x \mapsto f(t,x)$ für jedes $t\in U$ integrierbar.\\
        Seien $t_n\in U$ mit $t_n \xrightarrow{n\rightarrow \infty} t_0$. Setze $g_n(x) := f(t_n,x)\ (x\in X, n\in \N)$.\\
        $\tilde{N} := \bigcup_{n=1}^\infty N_{t_1} \cup N$ ist eine Nullmenge.\\
        Nach c) gilt: $g_n(x) \xrightarrow{n\rightarrow \infty} f(t_0,x) \ (\forall x \notin \tilde{N})$ und nach b):\\
        $|g_n(x)| \le g(x) \ (\forall n\in \N, x \notin \tilde{N})$. Mit \jlink{Thm 3.10} folgt
        \begin{displaymath}
            \int_X g_n(x)dx \xrightarrow{n \rightarrow \infty} \int_X f(t_0,x) dx = F(t_0).
        \end{displaymath}
    \end{proof}
\end{thm}

\begin{kor}
\jlabel{Kor 3.15}
    Sei $I \subset \R$ ein Intervall, $\jabb{f}{I}{\Rq}$ integrierbar, $a=\inf I$. Dann ist $t\mapsto F(t) = \int_a^t f(s)ds$ auf I stetig und $F(t) \xrightarrow{t \rightarrow a} 0$.
    \begin{proof}
        Es gilt $F(t) = \int_I \underbrace{\doubleOne_{(a,t)}(x) \cdot f(x)}_{=h(t,x)} dx \Rightarrow |h(t,x)| \le |f(x)|, \ \forall t,x$ und $|f|$ ist integrierbar. Sei $t_0 \in I$ fest aber beliebig.\\
        \uline{Es gilt}: $h(t,x) = \begin{cases} f(x), & t > x, \\ 0, & t \le x \end{cases} \xrightarrow{t \rightarrow t_0} h(t_0,x)$ für jedes $x \ne t_0$.\\
        Nun liefert \jlink{Thm 3.14} die Behauptung mit $N=N(t_0)=\{t_0\}$ in c), denn:
        \begin{displaymath}
            F(t) = \int_I h(t,x) dx \xrightarrow{t\rightarrow t_0} \int_I h(t_0,x) = F(t_0).
        \end{displaymath}
    \end{proof}
\end{kor}

\begin{expl*}
    Sei $\jabb{f}{\R_+}{\R}$ messbar und beschränkt. Sei $t>0$ fest, aber beliebig. Wähle $\epsilon \in (0,t)$. Für $x>0$ gilt dann $|e^{-tx}\cdot f(x)| \le \lVert f \rVert_\infty \cdot e^{-\epsilon x}$ ist integrierbar auf $\R_+$.\\
    Genauso: Sei $t\ge t_0 > 0, \ \epsilon \in (0,t_0)$. Dann ist $g(x) = e^{-\epsilon x}\cdot \lVert f \rVert_\infty$ die Majorante in \jlink{Thm 3.14} und somit existiert die ``Laplacetransformation``\\
    $\hat{f}(t) = \int_0^\infty e^{-tx}\cdot f(x) dx$ und sie ist stetig für $t\ge 0$. Da $t_0$ beliebig war, gilt dies für alle $t>0$.
\end{expl*}


\jdate{05.12.2008}

\begin{thm}
\jlabel{Thm 3.16}
    Sei $U \subset \R^k$ offen, $\jabb{f}{U\times X}{\R}$ erfülle
    \begin{enumerate}
        \item $\forall t\in U: \ x\mapsto f(t,x), \ X \rightarrow \R$ ist integrierbar
        \item $\exists$ eine Nullmenge $N_1$, sodass $t\mapsto f(t,x), \ U\rightarrow \R$ partiell differenzierbar für alle $x\notin N_1$ und alle $t\in U$
        \item $\exists$ eine Nullmenge $N_2$ und ein integrierbares $\jabb{g}{X}{[0,\infty]}$, sodass
            \begin{displaymath}
                \left | \frac{\partial f}{\partial t_j}(t,x) \right | \le g(x), \ \forall x\notin N_2, \ j=1,\dots,k, \ t\in U
            \end{displaymath}
    \end{enumerate}
    Dann ist
    \begin{displaymath}
        F(t) := \int_X f(t,x)dx
    \end{displaymath}
    in $t\in U$ partiell differenzierbar und
    \begin{displaymath}
        \frac{\partial}{\partial t_j} \int_X f(t,x)dx = \int_X \frac{\partial}{\partial t_j} f(t,x) dx \ \ \ (\forall j \in \N, \ t\in U).
    \end{displaymath}
    \begin{proof}
        Sei $j\in \{1,\dots,k\}, \ t_0 \in U$ fest, aber beliebig. Sei $\tau_n \ne 0$ mit $\tau_n \xrightarrow{n \rightarrow \infty} 0$. Setze $s_1 := t_0 + \tau_n \cdot e_j$. Da $U$ offen ist, gibt es ein $r>0$ und ein $n_0 \in \N$, sodass $s_n \in B(t_0, r)\subset U$. Sei $N = N_1 \cup N_2$ eine Nullmenge.\\
        Setze $g_n(x) := \frac{1}{\tau_n}(f(s_n,x)-f(t_0,x))$ für $n\in N, \ x\in X$. Nach b) gilt dann $g_n(x) \xrightarrow{n \rightarrow \infty} \frac{\partial}{\partial t_j} f(t_0,x) \ \forall X\notin N$.\\
        Der Mittelwertsatz liefert: Es existieren $\sigma_n$ mit $|\sigma_n| \le |\tau_n|$ (abhängig von $x,t_0,j$), sodass
        \begin{displaymath}
            |g_n(x)| = \left | \frac{\partial f}{\partial t_j} (t_0+\sigma_1\cdot e_j, x) \right| \overset{c)}{\le} g(x) \ \ (\forall x\notin N, \ n\in \N).
        \end{displaymath}
        Dann folgt:
        \begin{displaymath}
            \begin{split}
                \int_X \frac{\partial f}{\partial t_j} (t_0, x)dx &\overset{\text{majorisierte}}{\underset{\text{Konvergenz}}{=}} \limToInf{n} \int_X g_n(x) dx = \limToInf{n} \frac{1}{\tau_n}(F(s_1) - F(t_0) \\
                &= \frac{\partial}{\partial t_j} \int_X f(t_0, x)dx.
            \end{split}
        \end{displaymath}
    \end{proof}
\end{thm}

\begin{expl*}
    Sei $\jabb{f}{\R_+}{\R}$ messbar und beschränkt. Wir haben schon gesehen, dass $t\mapsto \hat{f}(t) = \int_0^\infty e^{-tx}\cdot f(x)dx$ für $t>0$ existiert und stetig ist. Sei $\epsilon >0$ beliebig und $t\ge \epsilon$. Dann gilt
    \begin{displaymath}
        \left | \frac{\partial}{\partial t} e^{-tx} \cdot f(x) \right | = \left | -x e^{-tx}\cdot f(x) \right |\le xe^{-\frac{\epsilon}{2}x} e^{-\frac{\epsilon}{2}x} \cdot \lVert f \rVert_\infty \le \frac{2}{e\epsilon} \lVert f\rVert_\infty\cdot e^{-\frac{\epsilon}{2}x}.
    \end{displaymath}
    Und $\frac{2}{e\epsilon} \lVert f\rVert_\infty\cdot e^{-\frac{\epsilon}{2}x}$ ist integrierbar. Da $\epsilon>0$ beliebig war folgt mit \jlink{Thm 3.16}:
    \begin{displaymath}
        \exists \ \hat{f}'(t) = \int_0^\infty xe^{-tx}\cdot f(x) dx.
    \end{displaymath}
\end{expl*}
        
    % Iterierte Integrale
\section{Iterierte Integrale}
Schreibe $z\in \R^d$ als $z=(x,y) \in \R^k \times \R^l$ mit $d= k+l$. Sei $\jabb{f}{\R^d}{\Rq}$. Zeige
\begin{displaymath}
    \int_{\R^d} f(z) dz = \int_{\R^k} \left( \int_{\R^l} f(x,y) dy \right) dx = \int_{\R^l}\left( \int_{\R^k} f(x,y) dx \right) dy.
\end{displaymath}

\vspace{12pt}

\uline{Probleme:}
\begin{itemize}
    \item[1)] Sind $y\mapsto f(x,y), \ x\mapsto f(x,y)$ \jlink{messbar} und integrierbar?
    \item[2)] Sind $x \mapsto \int f(x,y)dy, \ y\mapsto \int f(x,y)dx$ \jlink{messbar} und integrierbar?
    \item[3)] Gilt die Formel?
\end{itemize}

Sei $p_1(x,y) = x, \ p_2(x,y)=y$. Dann folgt, dass $\jabb{p_1}{\R^d}{\R^k}$ und $ \jabb{p_2}{\R^d}{\R^l}$ stetig und damit \jlink{messbar} sind.

\begin{lem}
\jlabel{Lem 3.17}
    Wenn $A\in \Borel_k$ und $B \in \Borel_l$, dann gilt $A\times B \in \Bd$.
    \begin{proof}
        Es gilt $A\times \R^l = p_1^{-1}(A)\in \Bd$ und $\R^k \times B = p_2^{-1}(B) \in \Bd$. Damit folgt $A\times B = (A\times \R^l)\cap (\R^k\times B) \in \Bd$.
    \end{proof}
\end{lem}

\begin{defn*}
    Für $C\in\Bd$ definiere die Schnitte
    \begin{displaymath}
        \begin{split}
            C_y &:= \{x\in \R^k: (x,y) \in C\} \text{  (für jedes feste $y\in\R^l$)}\\
            C^x &:= \{y\in \R^l: (x,y) \in C\} \text{  (für jedes feste $x\in\R^k$)}
        \end{split}
    \end{displaymath}

    Sei $A\subset \R^k, \ B \in \R^l, \ x \in \R^k$. Dann gilt für $C = A\times B$:
    \jlabel{(3.1)}
    \begin{equation}
        C^x = \begin{cases} B, &x\in A\\ \emptyset, &x\notin A \end{cases}
    \end{equation}

    Setze:
    \jlabel{(3.2)}
    \begin{equation}
        \begin{split}
            &\jabb{j_y}{\R^k}{\R^d}, \ j_y(x) = (x,y) \text{  (für jedes feste $y\in \R^l$).}\\
            &\jabb{j^x}{\R^l}{\R^d}, \ j^x(y) = (x,y) \text{  (für jedes feste $x\in \R^k$).}\\
            &\Rightarrow j_y, j^x \text{ sind stetig und messbar.}
        \end{split}
    \end{equation}

    Dann gilt $C_y = j_y^{-1}(C), \ C^x = (j^x)^{-1}(C)$ für alle $x\in \R^k$ und alle $y\in \R^l$.\\
    Für $\jabb{f}{\R^d}{\Rq}$ definiere die Schnittfunktionen:
    \jlabel{(3.3)}
    \begin{equation}
        \begin{split}
            f_y &= \jabb{f\circ j_y}{\R^k}{\Rq}, \ f_y(x) = f(x,y) \text{  (für jedes feste $y\in \R^l$)}\\
            f^x &= \jabb{f\circ j^x}{\R^l}{\Rq}, \ f^x(x) = f(x,y) \text{  (für jedes feste $x\in \R^k$)}.
        \end{split}
    \end{equation}
\end{defn*}

\begin{lem}
\jlabel{Lem 3.18}
    Seien $C\int \Bd$, $ \jabb{f}{\R^d}{\Rq}$ messbar $x\in \R^k$ und $y\in \R^l$. Dann gelten:
    \begin{itemize}
        \item $C_y\in \Borel_k$ und $C^x \in \Borel_l$
        \item $\jabb{f_y}{\R^k}{\Rq}$ und $\jabb{f^x}{\R^l}{\Rq}$ sind messbar
    \end{itemize}
    \begin{proof}
        Folgt aus \jlink{(3.2)}, \jlink{(3.3)} und der Messbarkeit von $f_y$, $f^x$.
    \end{proof}
\end{lem}

\begin{defn*}
    Sei $C\in \Bd$. Dann definiere:
    \jlabel{(3.4)}
    \begin{equation}
        \begin{split}
            \varphi_C(x) &= \lambda_l(C^x) = \int_{\R^l} \doubleOne_{C^x}(y) dy = \int_{\R^l} \doubleOne_C (x,y) dx \ (\forall x\in \R^k)\\
            \psi_C(y) &= \lambda_k(C_y) = \int_{\R^k} \doubleOne_{C_y} (x) dx = \int_{\R^k} \doubleOne_C(x,y)dx \ (\forall y \in \R^l)
        \end{split}
    \end{equation}
    (Diese Definition ist sinnvoll wegen \jlink{Lem 3.18} und weil $\doubleOne_{C} > 0$)\\
    An dieser Stelle wird z.B. verwendet, dass
    \begin{displaymath}
        \doubleOne_{C^x}(y) = \begin{cases} 1, &y\in C^x\\ 0, &y\notin C^x \end{cases} = \begin{cases} 1, &(x,y)\in C\\ 0, &(x,y)\notin C \end{cases} = \doubleOne_C(x,y) \ \ \ (\forall x\in\R^k, y\in\R^l)
    \end{displaymath}
    gilt.
\end{defn*}

\begin{lem}
\jlabel{Lem 3.19}
    Sei $C\in\Bd$. Dann sind $\jabb{\varphi_C}{\R^k}{[0,\infty]}$, $\jabb{\psi_C}{\R^l}{[0,\infty]}$ messbar.
    \begin{proof}
        Es reicht $f_c$ zu betrachten. Sei dafür $I = I'\times I''$ mit $I'\in \mathcal{J}_k,\ I''\in \mathcal{J}_l$. Dann gilt
        \begin{displaymath}
            f_I(x) \overset{\jshortlink{Def 3.1}}{=} \begin{cases} \lambda_l(I''), & x\in I' \\ 0, &x\notin I'  \end{cases} = \lambda_l(I'')\cdot \doubleOne_{I'}(x) \ \ \ (\forall x\in \R^k)
        \end{displaymath}
        Damit folgt die \hyperref[j_messbar]{Messbarkeit} von $f_i$ $(+)$.\\
        Somit $\Jd \subset \mathcal{D} = \{C\in \Bd : \varphi_c \text{ messbar}\}$ $(*)$.\\
        Für $n\in\N$ setze $Q_n := (-n,n]^d$ und $\mathcal{D}_n := \{C \subset Q_n : C \in \mathcal{D}\}$. Wir schreiben $Q_n = Q_n' \times Q_n''$ mit $Q_n' = (-n,n]^k, \ Q_n'' = (-n,n]^l$.\\
        Damit ergeben sich folgenden Eigenschaften für $D_n$:
        \begin{itemize}
            \item[(A1)] Wegen $(+)$ gilt $Q_n \in \mathcal{D}_n$.
            \item[(A2)] 
                Da $\lambda_l(C^x) \le \lambda_l(Q_n'') < \infty$, sind für $C\in \mathcal{D}_n$ $\varphi_c,\varphi_{Q_n},\varphi_{Q_n\backslash C}$ $\R_+$-wertig.\\
                Weiter gilt $\doubleOne_{Q_n\backslash C} = \doubleOne_{Q_n} - \doubleOne_C$. Damit ist $\varphi_{Q_n\backslash C} \overset{\jlink{(3.4)}}{=} \varphi_{Q_n} - \varphi_C$ \jlink{messbar}, da $C\in \mathcal{D}$ und $(+)$ gilt $Q_n \backslash C \in \mathcal{D}_n$.
            \item[(A3')]
                Seien $\{C_j,\ j\in \N\} \subset\mathcal{D}_n$ disjunkt. Dann gilt $\bigdcup_{j\in \N}C_j \in \mathcal{D}_n$. Denn
                \begin{displaymath}
                    \begin{split}
                        \varphi_{\bigdcup_{j\in\N}}(x) &\overset{\jlink{(3.4)}}{=} \int_{\R^l} \doubleOne_{\bigdcup_{j\in\N}C_j}(x,y)dy \overset{C_j\text{ disjunkt}}{=} \int_{\R^l}\sum_{j=1}^\infty \doubleOne_{C_j}(x,y) dy \\
                        &\overset{\jshortlink{Kor 2.20}}{=} \sum_{j=1}^\infty \int_{\R^l} \doubleOne_{C_j}(x,y) dy \overset{\jshortlink{(3.4)}}{=} \sum_{j=1}^\infty \varphi_{C_j}(x) \ \ \ (\forall x\in \R^k).
                    \end{split}
                \end{displaymath}
                Nach Voraussetzung ist $\varphi_{C_i}$ messbar. Damit folgt, dass $\varphi_{\bigdcup_{j\in\N} C_j}$ als Reihe \hyperref[j_messbar]{messbarer}, positiver Funktionen \jlink{messbar}. Also gilt
                \begin{displaymath}
                    \bigdcup_{j\in\N} C_j \in \mathcal{D}_n.
                \end{displaymath}
                Somit gilt (A3').
        \end{itemize}
        Ferner gilt nach $\Jd \cap Q_n \overset{\jshortlink{Lem 1.10}}{=} \{F \in \Jd : I \subset Q_n\} \subset \mathcal{D}_n$.\\
        Mit \jlink{Lem 3.20} folgt dann $\mathcal{D}_n = \sigma(\Jd\cap Q_n) = \Borel(Q_n)$.
        
\jdate{08.12.2008}

        Sei $C\in \Bd$. Setze
        \begin{displaymath}
            \varphi_C(x) := \int_{\R^l} \doubleOne_C(x,y)dy \hspace{15pt} (\forall x\in\R^k)
        \end{displaymath}
        Dann ist $\varphi_C$ \jlink{messbar} für $C\in \Bd$ und $C \subset Q_n$ für ein $n\in\N$. Sei $C\in \Bd$ beliebig. Dann gilt $C\cap Q_n \subset Q_n$, $C\cap Q_n \in \Bd$. Somit ist auch $\varphi_{C\cap Q_n}$ messbar und es gilt $C\cap Q_n\subset C\cap Q_{n+1}$ $(\forall n\in\N)$.\\
        Es gilt $C = \bigcup_{n=1}^\infty C\cap Q_n$. Daraus folgt
        \begin{displaymath}
            \doubleOne_{C\cap Q_n} \le \doubleOne_{C\cap Q_{n+1}}, \ \doubleOne_{C \cap Q_n} (x,y) \xrightarrow{n\rightarrow\infty} \doubleOne_C (x,y) \hspace{15pt} \forall z=(x,y)\in\R^d = \R^k\times \R^l.
        \end{displaymath}
        Mit \BeppoLevi{} folgt dann
        \begin{displaymath}
            \varphi_{C\cap Q_n}(x) \overset{\jshortlink{(3.4)}}{=} \int_{\R^l} \doubleOne_{C\cap Q_n} (x,y)dy \xrightarrow{n\rightarrow\infty} \int_{\R^l} \doubleOne_C (x,y) dy \overset{\jshortlink{(3.4)}}{=} \varphi_C \hspace{10pt} (\forall x\in \R^k).
        \end{displaymath}
        Damit ist $\varphi_C$ \jlink{messbar} als punktweiser Limes \jhyperref{messbar}{messbarer} Funktionen.
    \end{proof}
\end{lem}

\begin{lem}
\jlabel{Lem 3.20}
    Sei $\emptyset \ne \mathcal{E} \subset \PowerSet(X), \ \mathcal{E} \subset \mathcal{D} \subset \sigma(\mathcal{E})$ und $\mathcal{D}$ erfülle (A1), (A2) und (A3'). Dann gilt $\mathcal{D} = \sigma(\mathcal{E})$.
    \begin{proof}
        Siehe Extravorlesung.
    \end{proof}
\end{lem}

\begin{satz}
\jlabel{Satz 3.21}
    Sei $C\in \Bd$. Dann gilt
    \begin{displaymath}
        \lambda_d(C) = \int_{\R^k} \lambda_l(C^x)dx = \int_{\R^l} \lambda_k(C_y)dy.
    \end{displaymath}
    \uline{Also gilt}
    \begin{displaymath}
        \int_{\R^d} \doubleOne_C (z)dz = \int_{\R^k} \left( \int_{\R^l} \doubleOne_C(x,y)dy \right)dx = \int_{\R^l} \left(\int_{\R^k} \doubleOne_C(x,y)dx \right) dy.
    \end{displaymath}
    (Vergleiche \jlink{(3.4)}.)
    \begin{proof}
        Nach \jlink{Lem 3.19} und da die Integranden positiv sind, existieren für alle $C\in \Bd$
        \begin{displaymath}
            \mu(C) = \int_{\R^k} \lambda_l(C^x)dx, \hspace{30pt} \nu(C) = \int_{\R^l} \lambda_k(C_y)dy.
        \end{displaymath}
        Sei $I \in \Jd$ Dann folgt $\exists I' \in \mathcal{J}_k, \ I'' \in \mathcal{J}_l$ mit $I=I'\times I''$. Dann gilt
        \begin{displaymath}
            \mu(I) \overset{\jshortlink{(3.1)}}{=} \int_{\R^k} \lambda_l(I'') \cdot \doubleOne_{I'}(x) dx = \lambda_l(I'')\cdot \lambda_k(I') \overset{\jshortlink{(1.2)}}{=} \lambda_d(I).
        \end{displaymath}
        Genauso zeigt man, dass $\mu(I) = \lambda_d(I)$ gilt.
        
        \jspacesmall
        
        \uline{Also gilt} $\lambda_d = \mu = \nu$ auf $\Jd$.
        
        \jspacesmall
        
        Es ist klar, dass $\lambda_d$ ein Maß ist und dass $\mu(\emptyset) = \nu(\emptyset) = 0$.\\
        Seien $C_j\in \Bd$ disjunkt $(j\in\N)$. Dann gilt
        \begin{displaymath}
            \begin{split}
                \mu\left( \bigdcup_{j\in\N} C_j \right) &= \int_{\R^k} \lambda_l \left (\left( \bigdcup_{j\in\N} C_j \right)^x \right) = \int_{\R^k} \lambda_l \left(\bigdcup_{j\in\N} C_j^x \right )dx\\
                &\overset{\text{(M2)}}{=} \int_{\R^k} \sum_{j=1}^\infty \underbrace{\lambda_l(C_j^x)}_{\ge 0}dx \overset{\jshortlink{Kor 2.20}}{=} \sum_{j=1}^\infty \int_{\R^k} \lambda_l(C_j^x)dx = \sum_{j=1}^\infty \mu(C_j^x).
            \end{split}
        \end{displaymath}
        Also ist $\mu$ ein Maß. Genauso zeigt man die Maßeigenschaft für $\nu$.
        
        \jspacesmall
        
        Im Beweis von \jlink{Thm 1.20} haben wir gesehen, dass $\Jd$ die Voraussetzungen A), B) von \jlink{Thm 1.19} (Eindeutigkeitssatz) erfüllt. Somit impliziert \jlink{Thm 1.19}, dass $\lambda_d = \mu = \nu$ auf $\Bd$ gilt.
        
    \end{proof}
\end{satz}

\begin{kor}
\jlabel{Kor 3.22}
    Für $N\in\Bd$ sind äquivalent:
    \begin{enumerate}
        \item $\lambda_d(N)=0$
        \item Für $\fa \ x\in\R^k$ gilt $\lambda_d(N^x) = 0$
        \item Für $\fa \ x\in\R^l$ gilt $\lambda_k(N^x) = 0$
    \end{enumerate}
    \begin{proof}
        Folgt direkt aus \jlink{Satz 3.21} und \jlink{Lem 2.18}c).
    \end{proof}
\end{kor}


\begin{expl*}
    Sei $M\subset \R^k$ eine Nullmenge. Setze $N:=M\times \R^l$. Dann folgt mit \jlink{Lem 3.17} $N\in \Bd$. Es gilt $N_y = M \ (\forall y\in \R^l)$ (vergleiche \jlink{(3.1)}). Schließlich folgt dann mit \jlink{Kor 3.22}, dass $N$ ebenfalls eine Nullmenge ist.
\end{expl*}


\begin{bem}
\jlabel{Bem 3.23}
    Es exisitert ein $M\subset [0,1]^2$, sodass $M$ in keiner Nullmenge aus $\Borel_2$ liegt (und insbesondere ist $M$ keine zweidimensionale Nullmenge), aber \uline{alle} $M^x,\ M_y$ höchstens ein Element haben. Damit folgt $\lambda(M^x) = \lambda(M_y) = 0 \ \forall x,y$.\\
    Also implizieren b) und c) in \jlink{Kor 3.22} nicht einmal, dass $M\in\Borel_2$.\\
    (Vergleiche Elstrodt Bsp V. 1.9)
\end{bem}


\begin{bem}
\jlabel{Bem 3.24}
    Sei $\emptyset \ne D\in \Bd$ und $\jabb{f}{D}{\Rq}$ messbar. Setze
    \begin{displaymath}
        \tilde{f}(x) := \begin{cases} f(x), &x\in D\\ 0, &x\in \R^d\backslash D \end{cases} \hspace{25pt} (\text{``0-Fortsetzung``})
    \end{displaymath}
    Dann ist $\jabb{\tilde{f}}{\R^d}{\Rq}$ messbar.
    \begin{proof}
        Sei $a\in \R$. Dann gilt
        \begin{displaymath}
            \{x\in \R^d : \tilde{f}(x) \le 0\} = \begin{cases} \{x\in D : f(x) \le a\} \cup D^c, &a\ge 0\\ \underbrace{\{x\in D: f(x)\le a\}}_{\in\Borel(C) \subset \Bd}, &a < 0 \end{cases} \in \Bd.
        \end{displaymath}
        Also ist $\tilde{f}$ messbar.
    \end{proof}
\end{bem}


\begin{expl}
\jlabel{Bsp 3.25}
    Sei $B:=B(0,r) = \{(x,y)\in\R^2 : |x| < r, \ |y| < \sqrt{r^2 - x^2}\} \in \Borel_2$. Damit folgt
    \begin{displaymath}
        B^x = \begin{cases} \emptyset, &|x| \ge r\\ (-\sqrt{r^2-x^2},+\sqrt{r^2-x^2}), &|x|<r. \end{cases}
    \end{displaymath}
    Und damit
    \begin{displaymath}
        \lambda_1(B^x) = \begin{cases} 0, &|x|\ge r\\ w\cdot \sqrt{r^2-x^2}, &|x| <r.\end{cases}
    \end{displaymath}
    Mit \jlink{Satz 3.21} folgt dann
    \begin{displaymath}
        \lambda_2(B) = \int_\R \lambda_1(B^x)dx = \int_{-r}^r 2\cdot \sqrt{r^2-x^2}dx \overset{\text{Ana1}}{\underset{\text{Bsp 6.14}}{=}} \pi\cdot r^2.
    \end{displaymath}
    Genauso zeigt man, dass $\lambda_2(\overline{B}) = \pi \cdot r^2$. Damit folgt für alle $A\in\Borel_2$ mit $B\subset A\subset \overline{B}:$ $\lambda_2(A) = \pi\cdot r^2$.
\end{expl}


\begin{expl}[Rotationskörper]
\jlabel{Bsp 3.26}
    Seien $I\subset \R$ ein Intervall und $\jabb{f}{I}{[0,\infty]}$ \jlink{messbar}. Setze $f$ wie in \jlink{Bem 3.24} \jlink{messbar} auf $\R$ fort. Definiere dann
    \begin{displaymath}
        V: =\{(x,y,z)\in\R^3 : z\in I,\ x^2+y^2 < f(z)^2\} = \{(\tilde{f}\circ p_z)^2-p_x^2-p_y^2 > 0\} \in \Borel_3.
    \end{displaymath}
    Setze weiter für $z\in I$ $V_2:= B(0,f(z))$ und für $z\notin I$ $V_2 := \emptyset$. Dann folgt mit \jlink{Satz 3.21}
    \begin{displaymath}
        \lambda_3(V) = \int_I \lambda_2(B(0, f(z)))dz \overset{\jshortlink{Bsp 3.25}}{=} \pi\cdot \int_I f(z)^2dz.
    \end{displaymath}
    \uline{Beispiel}:\\
    Sei $I= [1,\infty), \ f(z) = \frac{1}{z}, \ V = \{(x,y,z)\in \R^3: z\ge 1, \ x^2+y^2<\frac{1}{z^2}\}$. Dann folgt
    \begin{displaymath}
        \lambda_3(V) = \pi \cdot \int_1^\infty \frac{dz}{z^2} = \pi \cdot \limToInf{b} \int_1^b \frac{dz}{z^2} = \pi \cdot \limToInf{b} \left[-\frac{1}{z}\right]_1^b = \pi\cdot \limToInf{b} \left(1 - \frac{1}{b}\right) = \pi.
    \end{displaymath}
\end{expl}

Sei $\jabb{f}{\R^d}{[0,\infty]}$ \jlink{messbar}. Nach \jlink{Lem 3.18} existieren dann
\jlabel{(3.5)}
\begin{equation}
    \begin{split}
        F(x) &:= \int_{\R^l} f(x,y)dy = \int_{\R^l}f^x(y)dy \hspace{20pt} (\forall x\in \R^k)\\
        G(x) &:= \int_{\R^k} f(x,y)dx = \int_{\R^k} f_y(x)dx \hspace{20pt} (\forall y\in \R^l).
    \end{split}
\end{equation}


\begin{thm}[Fubini]
\jlabel{Thm 3.27}
\jlabel{Fubini}
    Sei $d=k+l,\ \R^d = \R^k\times \R^l$.
    \begin{enumerate}
        \item
            \jlabel{FubiniA}
            Sei $\jabb{f}{\R^d}{[0,\infty]}$ \jlink{messbar}. Dann sind $\jabb{F}{\R^k}{[0,\infty]}$ und  $\jabb{G}{\R^l}{[0,\infty]}$ messbar und es gilt
            \jlabel{(3.6)}
            \begin{equation}
                \int_{\R^d} f(z)dz = \int_{\R^k} \left( \int_{\R^l} f(x,y)dy \right)dx = \int_{\R^l} \left( \int_{\R^k} f(x,y)dx \right)dy.
            \end{equation}
        \item
            \jlabel{FubiniB}
            Sei $\jabb{f}{\R^d}{\Rq}$ integrierbar. Dann gibt es Nullmengen $M\in\R^k,\ N\in \R^k$, sodass $\jabb{f^x}{\R^l}{\Rq}$ integrierbar ist für alle $x\in\R^k\backslash M$ und $\jabb{f_y}{\R^k}{\Rq}$ integrierbar ist für alle $y\in \R^l\backslash N$.\\
            Definiere $F(x)$ für $x\in \R^k\backslash M$ und $G(x)$ für $y\in \R^l\backslash N$ wie in \jlink{(3.5)} und setze $F(x)=0$ für alle $x\in M$ und $G(y)=0$ für alle $y\in N$.\\
            Dann sind $\jabb{F}{R^k}{\Rq}$ und $\jabb{G}{\R^l}{\Rq}$ integrierbar und es gilt
            \begin{displaymath}
                \int_{\R^d} f(z) dz = \int_{\R^k} F(x)dx = \int_{\R^l} G(y) dy.
            \end{displaymath}
            Meist schreibt man dafür wieder \jlink{(3.6)}.
    \end{enumerate}
    \begin{proof}
        (Der Beweis erfolgt in den vier Schritten der Integraldefinition.)
        \begin{enumerate}
            \item 
                \begin{itemize}
                    \item[0)] Für $f=\doubleOne_C, \ C\in \Bd$ wurde a) schon in \jlink{Lem 3.19} und \jlink{Satz 3.21} gezeigt.
                    \item[1)] 
                        Sei $f:= \sum_{k=1}^m a_k\cdot \doubleOne_{C_k} \ge 0$ \jlink{messbar}. Dann ist $F \overset{\jshortlink{(3.4)}}{=} \sum_{k=1}^m a_k\cdot \varphi_{C_k}$ nach \jlink{Satz 2.8} \jlink{messbar} (verwende \jlink{Lem 3.19}). Ferner gilt
                        \begin{displaymath}
                            \begin{split}
                                \int_{\R^d} f(z)dz &= \sum_{k=1}^m a_k \cdot \int_{\R^d} \doubleOne_{C_k}(z) dz\\ 
                                                   &\overset{\text{0)}}{=} \sum_{k=1}^m a_k \cdot \int_{\R^k} \left( \int_{\R^l} \doubleOne_{C_k}(x,y)dy \right)dx\\
                                                   &= \int_{\R^k} \left( \int_{\R^l} \sum_{k=1}^m a_k \cdot \doubleOne_{C_k}(x,y)dy \right)dx\\
                                                   &=\int_{\R^k} \left( \int_{\R^l} f(x,y) \right)dx.
                            \end{split}
                        \end{displaymath}
                        (Die andere Gleichheit in \jlink{(3.6)} zeigt man genauso.)
                    
\jdate{12.12.2008}
                    
                    \item[2)]
                        Sei $\jabb{f}{\R^d}{[0.\infty]}$ \jlink{messbar}. Dann gibt es einfache $\jabb{f_n}{\R^d}{\R_+}$ mit $f_n(z)\le f_{n+1}(z)$, $f(z) = \sup_{n\in\N} f_n(z) \ (\forall n\in\N, z\in\R^d)$.\\
                        Setze $F_n(x):= \int_{\R^l} f_n(x,y)dy \le F_{n+1}(x) \ (\forall x\in\R^k, n\in\N)$. Mit 1) folgt dann die \jhyperref{messbar}{Messbarkeit} von $F_n$ $(\forall n\in\N)$. Mit \BeppoLevi{} für $(f_n)^x$ folgt
                        \begin{displaymath}
                            F_n(x) \xrightarrow{n\rightarrow \infty} \int_{\R^l}f(x,y)dy =: F(x) \hspace{15pt} (\forall x\in\R^k).
                        \end{displaymath}
                        Damit ist $\jabb{f}{\R^k}{[0,\infty]}$ als Grenzwert \jhyperref{messbar}{messbarer} Funktionen \jlink{messbar}.
                        Weiter gilt
                        \begin{displaymath}
                            \begin{split}
                                \int_{\R^d} f(z)dz &\underset{\text{für }f_n}{\overset{\text{\BeppoLevi}}{=}} \limToInf{n} \int_{R^d} f_n(z)dz \\
                                &\overset{\text{1)}}{=} \limToInf{n} \int_{\R^k} \underbrace{\left(\int_{\R^l} f_n(x,y)dy \right)}_{=F_n(x)}dx\\
                                &\underset{\text{für }F_n}{\overset{\text{\BeppoLevi}}{=}} \int_{\R^k} \left(\int_{\R^l} f(x,y) dy \right)dx.                                
                            \end{split}
                        \end{displaymath}
                        Die Andere Gleichheit in \jlink{(3.6)} folgt genauso. Damit ist a) gezeigt.
                \end{itemize}
                
            \item
                Sei $\jabb{f}{\R^d}{\Rq}$ integrierbar. Dann folgt, dass $\jabb{|f|}{\R^d}{[0,\infty]}$ integrierbar ist und dass $\jabb{f^x}{\R^l}{\Rq}$ nach \jlink{Lem 3.18} \jlink{messbar} ist. Dann gilt
                \begin{displaymath}
                    \tag{$+$}
                    \infty > \int_{\R^d} |f(z)| dz = \int_{\R^k} \underbrace{\left(\int_{\R^l} |f(x,y)|dy \right)}_{=: \Phi(x) = \int_{\R^l} |f^x|dy}dx.
                \end{displaymath}
                Dann folgt die Integrierbarkeit von $\jabb{\Phi}{\R^k}{[0,\infty]}$ ($\Phi$ ist \jlink{messbar} nach \jlink{Lem 3.19}). \jlink{Kor 2.24} impliziert, dass $M:= \{\Phi = \infty\} \subset \R^k$ eine Nullmenge ist. Damit ist nach \jlink{Kor 3.22} auch $M\times \R^l$ eine d-dimensionale Nullmenge. Mit $(+)$ folgt nun, dass $\jabb{f^x}{\R^l}{\Rq}$ für alle $x\in\R^k\backslash M$ integrierbar ist. Setze
                \begin{displaymath}
                    \tag{$*$}
                    F(x) := \begin{cases}
                                \int_{\R^l} f(x,y)dy, &x\in M\\
                                0, &x\in M
                            \end{cases}
                         = \int_{\R^l} \tilde{f}(x,y)dy,
                \end{displaymath}
                wobei $\tilde{f} = \doubleOne_{M^x\times \R^l} \cdot f$ ist. Also ist $\tilde{f}$ \jlink{messbar} ist.\\
                Da $|\tilde{f}| = \doubleOne_{M^c\times \R^l}\cdot |f^x|$ gilt,  folgt, dass $\jabb{\tilde{f}}{\R^l}{\Rq}$ ist für alle $x\in \R^k$ integrierbar ist.\\
                Ferner gilt
                \begin{displaymath}
                    F(x) = \int_{\R^l} \tilde{f}_+(x,y)dy - \int_{\R^l} \tilde{f}_(x,y)dy =: F^+(x) - F^-(x) \hspace{15pt} (\forall x\in \R^k),
                \end{displaymath}
                wobei $F^\pm(x) \in \R_+$ $(\forall x\in\R^k)$. Nach a) sind damit $F^\pm$ messbar und somit ist auch $F$ messbar. Außerdem gilt $|F| \le \Phi$, welches integrierbar ist. Mit \jlink{Satz 2.23} ist dann $F$ integrierbar. Dann gilt
                \begin{displaymath}
                    \begin{split}
                        &\underbrace{\int_{\R^l\backslash M} \left(\int_{\R^l} f(x,y)dy \right)dx}_{(**)} = \int_{\R^k} F(x) dx\\
                        &\overset{\jshortlink{Satz 2.23}}{=} \int_{\R^k} F^+(x) dx - \int_{\R^k} F^-(x)dx\\
                        &= \int_{\R^k} \left(\int_{\R^l} \tilde{f}_+(x,y)dy \right)dx - \int_{\R^k}\left(\int_{\R^l} \tilde{f}_-(x,y) \right)dx\\
                        &\overset{a)}{=} \int_{\R^d} \tilde{f}_+dz - \int_{\R^d} \tilde{f}_-(z)dz \underset{\text{Integrals}}{\overset{\text{Def. des}}{=}} \underbrace{\int_{\R^d} \tilde{f}(z)dz}_{(++)}\\
                        &\overset{f=\tilde{f}\ \fu}{=} \int_{\R^d}f(z)dz.
                    \end{split}
                \end{displaymath}
                Die andere Gleichheit in \jlink{(3.6)} folgt analog.
        \end{enumerate}
    \end{proof}
\end{thm}

\begin{bem*}
    In \jlink{Thm 3.27}b) gilt \jlink{(3.6)}, wenn man die iterierten Integrale wie in $(**)$ und $(++)$ modifiziert.
\end{bem*}

\jlabel{Bem 3.28}
\begin{bem}
    \begin{enumerate}
        \item Man kann \jlink{Fubini} auf endlich oft iterierte Integrale verallgemeinern. Es existiert also eine Variante für $f(z) = f(x_1, \dots, x_m)$.
        \item 
            Nach \jlink{Bem 3.23} existiert $\jabb{f = \doubleOne_M}{\R^2}{\R_+}$, sodass die iterierten Integrale existieren und gleich sind (es gilt $F=0,\ G=0$), aber $f$ ist nicht in $(x,y)\in\R^2$ \jlink{messbar}. Also muss man die  \jhyperref{messbar}{Messbarkeit} in \jlink{Fubini} vorausgesetzt werden.
        \item 
            Wenn $f$ weder integrierbar noch positiv ist, kann es passieren, dass die iterierten Integrale in \jlink{(3.6)} existieren und ungleich sind.
        
            \jspacesmall
            
            \jemph{Beispiel (Cauchy)}:\\
            Sei $\jabb{f}{\R^2}{\R}$ mit
            \begin{displaymath}
                f(x,y) = \begin{cases} 
                            \frac{x^2-y^2}{(x^2+y^2)^2} , &(x,y)\in (0,1)^2\\
                            0, &(x,y)\in\R^2\backslash(0,1)^2.                         
                        \end{cases}
            \end{displaymath}
            Dann gilt für $(x,y)\in (0,1)^2$
            \begin{displaymath}
                f(x,y) = \frac{\partial\partial}{\partial y \partial x} \arctan\left(\frac{x}{y} \right) = \frac{\partial\partial}{\partial x \partial y} \arctan\left(\frac{x}{y} \right).
            \end{displaymath}
            Sei $x>0$. Dann existiert
            \begin{displaymath}
                \int_0^1 \frac{x^2-y^2}{(x^2+y^2)^2}dx = \int_0^1 \frac{\partial}{\partial y} \cdot \underbrace{\frac{\partial}{\partial x}\arctan\left(\frac{x}{y}\right)}_{=\frac{y}{x^2+y^2}} = \left[\frac{y}{x^2+y^2}\right]_{y=0}^{y=1} = \frac{1}{1+x^2}.
            \end{displaymath}
            Dann folgt die Existenz von
            \begin{displaymath}
                \int_\R\left(\int_\R f(x,y)dy\right)dx = \int_0^1 \frac{dx}{1+x^2} = \left[\arctan(x) \right]_0^1 = \frac{\pi}{4}.
            \end{displaymath}
            Entsprechend exsitiert auch
            \begin{displaymath}
                \int_\R\left(\int_\R f(x,y)dx\right)dy = \left[\arctan(x) \right]_0^1 = -\frac{\pi}{4},
            \end{displaymath}
            aber es gilt
            \begin{displaymath}
                \int_\R\left(\int_\R f(x,y)dy\right)dx = \frac{\pi}{4} \ne -\frac{\pi}{4} = \int_\R\left(\int_\R f(x,y)dx\right)dy.
            \end{displaymath}
        
        \item
            Selbst wenn $f$ \jlink{messbar} (und nicht positiv) ist, folgt im Allgemeinen aus der Existienz und Gleichheit der iterierten Integrale in \jlink{(3.6)} \jemph{nicht} die Integrierbarkeit von $\jabb{f}{\R^d}{\R}$.
            
            \jspacesmall
            
            \jemph{Beispiel}:
            \begin{displaymath}
                f(x,y) = \begin{cases}
                             \frac{xy}{(x^2+y^2)^2}, &(x,y) \in [-1,1]\backslash \{0,0\}\\
                             0, & \text{sonst}.
                         \end{cases}
            \end{displaymath}
    \end{enumerate}
\end{bem}

\begin{gebrauchsanweisungFubini}
    Seien $\jabb{f}{D}{\Rq}$ und $D\in \Bd$.
    \begin{enumerate}
        \item Prüfe, ob $f$ in $(x,y)$ \jlink{messbar} ist.
        \item Setze $f$ \jlink{messbar} fort zu $\jabb{\tilde{f}}{\R^d}{\Rq}$ (etwa mit $0$, vergleiche \jlink{Bem 3.24}). Dann folgt die \jlink{messbar}{Messbarkeit} von $\doubleOne_D\cdot \tilde{f}$.
        \item
            Falls nötig, zeige mit a)
            \begin{displaymath}
                \int_{\R^d} \doubleOne_D\cdot |\tilde{f}| dz = \int \int |\doubleOne_D \cdot \tilde{f}| dxdy = \int \int |\doubleOne_D \cdot \tilde{f}| dydx < \infty.
            \end{displaymath}
            Dann folgt $\doubleOne_D\cdot \tilde{f}$ ist integrierbar.
        \item
            \jlinkFubiniB liefert dann
            \begin{displaymath}
                \begin{split}
                    \int_{\R^d} \doubleOne_D\cdot \tilde{f}(z)dz &= \int_{\R^k}\int_{\R^l} \doubleOne_D(x,y) \cdot \tilde{f}(x,y)dydx\\
                    &= \int_{\R^l}\int_{\R^k} \doubleOne_D(x,y) \cdot \tilde{f}(x,y)dxdy.
                \end{split}
            \end{displaymath}
    \end{enumerate}
\end{gebrauchsanweisungFubini}

\begin{bem}
\jlabel{Bem 3.29}
    Sei $Q=X\times Y$ für $\emptyset \ne X \in \Borel_k, \ \emptyset \ne Y \in \Borel_l$. Sei $\jabb{f}{Q}{\Rq}$ integrierbar oder \jlink{messbar} und positiv. Sei $\tilde{f}(x,y) = \doubleOne_X(x)\cdot \doubleOne_Y(y)\cdot \tilde{f}(x,y)$. Dann folgt
    \begin{displaymath}
        \begin{split}
            \int_Q f(z)dz &= \int_{\R^d} \tilde{f}(z)dz \overset{\jshortlink{Fubini}}{=} \int_{\R^d} \int_{\R^d} \doubleOne_X(x)\cdot \doubleOne_Y(y) \cdot \tilde{f}(x,y)dydx\\
            & = \int_X\int_Y f(x,y)dydx \overset{\text{genauso}}{=} \int_Y\int_X f(x,y)dxdy.
        \end{split}
    \end{displaymath}
\end{bem}


\begin{expl*}
    \begin{enumerate}
        \item 
            Sei $D = \{(x,y) \in \R^2 : x \ge 1, \ 0\le y \le \frac{1}{x}\}$. Da $D$ abgeschlossen ist, gilt $D\in \Borel_2$. Seien $f(x,y) = \frac{1}{x}\cos(xy)$ für $(x,y) \in D$ und $\tilde{f}(x,y) = \frac{1}{x}\cos(xy)$, $(x,y)\in Q := (0,0)\times \R_+$. Dann sind $f$ und $\tilde{f}$ stetig und damit \jlink{messbar} und es gilt $\tilde{f}|_D = f$ (in $(x,y)$).\\
            Ferner gilt
            \begin{displaymath}
                \begin{split}
                    \int_D |f|d(x,y) &= \int_Q \doubleOne_D |\tilde{f}|d(x,y)\\
                    &\overset{\jshortlinkFubini}{=} \int_0^\infty \int_0^\infty \underbrace{\doubleOne_D(x,y)}_{= 1 \Leftrightarrow x\ge 1, 0 \le y \le \frac{1}{x}}\cdot \frac{1}{x} \cdot \underbrace{|\cos(xy)|}_{\le 1} dydx\\
                    &\le \int_1^\infty \int_0^{\frac{1}{x}} \frac{1}{x}dydx = \int_1^\infty \frac{1}{x^2} dx = 1.
                \end{split}
            \end{displaymath}
            Somit ist $f$ integrierbar. Nun folgt
            \begin{displaymath}
                \begin{split}
                    \int_D f(x,y) d(x,y) &\overset{\jshortlinkFubini}{=} \int_0^\infty \int_0^\infty \doubleOne_D(x,y) \cdot \frac{\cos(xy)}{x} dydx\\
                    &\overset{\text{s.o.}}{=} \int_1^\infty \int_0^\frac{1}{x} \frac{1}{x}dydx = \int_1^\infty \frac{1}{x} \left[\frac{1}{x} \cdot \sin(xy) \right]_{y=0}^{y=\frac{1}{x}} dx\\
                    &= \int_1^\infty \frac{\sin(1)}{x} dx = \sin(x).
                \end{split}
            \end{displaymath}

\jdate{15.12.2008}

        \item 
            Sei $\alpha \in (0,1)$, $\jabb{g}{(0,1)}{\Rq}$ integrierbar und $D:=\{(x,y)\in \R^2 : 0 \le y \le x \le 1\}$. Setze $\jabb{f}{D}{\Rq}, \ f(x,y) = (x-y)^{-\alpha}\cdot g(x)$. $\tilde{f}$ sei die Nullfortsetzung von $f$. Sei $G(x,y):=g(y) \ \forall (x,y)\in D$. Dann ist $G = g\circ p_2$ \jlink{messbar} auf $D$. Damit ist auch $f$ in $(x,y)$ als Produkt \jlink{messbar}er Funktionen \jlink{messbar}. Weiter gilt
            \begin{displaymath}
                \begin{split}
                    \int_D |f| dz &\overset{\jshortlinkFubiniA}{=} \int_0^1 \left( \int_0^1 \underbrace{\doubleOne_D(x,y)}_{=1 \Leftrightarrow y \le x \le 1 \Leftrightarrow x\in D_y} \cdot \tilde{f}(x,y) dx \right)dy\\
                    &= \int_0^1 \left ( \int_0^1 (x-y)^{-\alpha} \cdot |g(y)| dx \right) dy\\
                    &\underset{dt=dx}{\overset{t=x-y}{=}} \int_0^1 \underbrace{\int_0^{1-y} t^{-\alpha}dt}_{\left [\frac{1}{1-\alpha} t^{1-\alpha} \right]_0^{1-y} = \frac{1}{1-\alpha}(1-y)^{1-\alpha}} \cdot |g(y)| dy\\
                    &= \frac{1}{1-\alpha} \int_0^1 \underbrace{(1-y)^{1-\alpha}}_{\le 1} \cdot |g(y)| dy = \frac{1}{1-\alpha} \int_0^1 |g(y)|dy < \infty.
                \end{split}
            \end{displaymath}
            Also ist $f$ integierbar, also existiert das Integral und es gilt
            \begin{displaymath}
                \begin{split}
                    \int_D f(x)dz &\overset{\jshortlinkFubiniB}{=} \int_0^1\left ( \underbrace{\doubleOne_D(x,y)}_{=1 \Leftrightarrow 0 \le y \le x \Leftrightarrow y D^x}\cdot \tilde{f}(x,y)dy \right )dx\\
                    &=\int_0^1 \underbrace{\left( \int_0^x (x-y)^{-\alpha} \cdot g(y) dy\right)}_{=: F(x)}dx.
                \end{split}
            \end{displaymath}
            \underline{Beachte}: für $g(y) := |\frac{1}{2}-y|^{\alpha-1},\ y \in (0,1)$ (integrierbar) gilt aber
            \begin{displaymath}
                F(0.5) = \int_0^{\frac{1}{2}} \left(\frac{1}{2}-y \right)^{-\alpha} \cdot \left(\frac{1}{2}-y \right)^{\alpha-1} dy = \int_0^{\frac{1}{2}} \left(\frac{1}{2}-y \right)^{-1} dy = \infty.
            \end{displaymath}

        \item
            In Ana1 haben wir bereits die Existenz von folgendem Limes gezeigt
            \begin{displaymath}
                \limToInf{R} \underbrace{\int_0^R \frac{\sin(x)}{x}dx}_{=:I_R, \ R>0}.
            \end{displaymath}
            Für $x>0$ gilt 
            \begin{displaymath}
                \int_0^\infty e^{-xy}dy = \limToInf{b} \int_0^b e^{-xy}dy = \limToInf{b} \left[-\frac{1}{x}\cdot e^{-xy}\right]_0^b = \limToInf{b} \frac{1}{x} \cdot (1-e^{-bx}).
            \end{displaymath}
            Somit folgt
            \begin{displaymath}
                I_R = \int_0^R \int_0^\infty \underbrace{\sin(x)\cdot e^{-xy}}_{=:f(x,y)} dydx
            \end{displaymath}
            und $\jabb{f}{[0,R]\times \R_+}{\R}$ ist stetig. Ferner gilt
            \begin{displaymath}
                \int_D |f| dz \overset{\jshortlinkFubiniA}{=} \int_0^R\int_0^\infty |\sin(x)|\cdot e^{-xy} dydx = \int_0^R \frac{|\sin(x)|}{x}dx < \infty.
            \end{displaymath}
            Da der Integrand stetig ist, folgt, dass $\jabb{f}{D}{\R}$ integrierbar ist. Somit gilt
            \begin{displaymath}
                \begin{split}
                    I_R &\overset{\jshortlinkFubiniB}{=} \int_D f dz = \int_0^\infty \int_0^R \sin(x)\cdot e^{-xy} dxdy\\
                    &\overset{\text{2x PI}}{=} \int_0^\infty \frac{1}{1+y^2} \cdot \left(1-e^{-yR}\cdot(y\cdot \sin(R) + \cos(R)) \right)dy\\
                    &= \underbrace{\int_0^\infty \frac{dy}{1+y^2}}_{\left[\arctan(y)\right]_0^\infty =\frac{\pi}{2}} - \underbrace{\int_0^\infty \frac{1}{1+y^2} \cdot e^{-yR}\cdot (y\cdot \sin(R) + \cos(R)dy}_{\le \int_0^\infty \frac{1+y}{1+y^2} \cdot e^{-yR}dy \overset{(*)}{\le} 2\cdot \int_0^\infty e^{-yR}dy = \frac{2}{R} \xrightarrow{R\rightarrow\infty} 0}.
                \end{split}
            \end{displaymath}
            Dabei gilt $(*)$: $\frac{1+y}{1+y^2} \le 2$. Also folgt $\limToInf{R} \int_0^R \frac{\sin(x)}{x}dx = \frac{\pi}{2}$.
    \end{enumerate}
\end{expl*}
        
    % Der Transformationssatz
\section{Transformationssatz}
\uline{Wir kennen aus Ana1 bereits die Substitutionsregel}:\\
Sei $\jabb{f}{[a,b]}{\R}$ stetig, $\varphi \in C([a,b])$ mit $\varphi([a,b]) = [a,b]$. Dann gilt
\begin{displaymath}
    \int_a^b f(y) dy = \int_\alpha^\beta f(\varphi(x))\cdot \varphi'(x) dx.
\end{displaymath}
Unser Ziel ist es nun, dies auf Funktionen $\jabb{\phi}{\R^d}{\R^d}$ zu verallgemeinern.

\begin{defn*}
    \jlabel{Diffeomorphismus} \jlabel{diffeomorph}
    Schon in Ana2 haben wir folgendes definiert. Seien $X,Y\subset \R^d$ offen und nichtleer. Sei $\jabb{\phi}{X}{Y}$ bijektiv mit $\phi\in C^1(X,\R^d)$ und $\phi^{-1} \in C^1(Y,\R^d)$. Dann heißt $\phi$ ein \jlink{Diffeomorphismus}.
\end{defn*}

TODO: Wie schreibe ich das schön auf...?

\begin{bem*}
    Sei $\phi$ \jlink{diffeomorph}. Dann $x\in \phi'(\phi(x)) \Rightarrow I=(\phi)'(\phi(x))\phi'(x)$. Also ist $\phi'(x)$ invertierbar für alle $x$.
\end{bem*}


\begin{satz*}[Grundversion des Transformationssatzes]
    Seien $\emptyset \ne X,Y\subset \R^d$ offen und $\jabb{\phi}{X}{Y}$ \jlink{diffeomorph}. Sei $\jabb{f}{Y}{\Rq}$ integrierbar oder \jlink{messbar} und positiv. Dann ist $g(x):= f(\phi(x))\cdot |\det \phi'(x)|$ integrierbar oder \jlink{messbar} und positiv und es gilt
    \begin{displaymath}
        \int_Y f(y)dy = \int_X f(\phi(x)) \cdot |\det \phi'(x)|dx.
    \end{displaymath}
\end{satz*}

\begin{expl*}[Polarkoordinaten für $d=2$]
    Sei $\phi(r,\varphi) = \begin{pmatrix} r\cdot \cos (\varphi) \\ r\cdot \sin(\varphi)\end{pmatrix}$. Dann ist $\phi \in C^1(\R^2,\R^2)$. Aus Ana2 wissen wir, dass $\det \phi'(r,\varphi) = r > 0$ $(\forall r>0, \ \varphi \in \R)$ gilt und dass $\jabb{\phi}{(0,R)\times (0, 2\pi)}{B(0,R)\backslash (\R_+ \times \{0\})}$ für alle $R>0$ bijektiv ist.
    
    \jspace
    
    TODO: Blödes Bild... :-P
    
    \jspace
    
    Für $\alpha \in (0, 2\pi)$ und $R>0$ setze $V_\alpha:= \phi((0,R)\times (0, \alpha))$. Dann gilt
    \begin{displaymath}
        \begin{split}
            \lambda_2(V_\alpha) &= \int_{V_\alpha} 1 d(x,y) \overset{\text{Trafo}}{=} \int_{(0,R)\times(0,\alpha)} 1\cdot r d(r,\varphi) \overset{\jshortlinkFubini}{=} \int_0^\alpha \int_0^R r dr d\varphi\\
            &= \int_0^\alpha d\varphi \cdot \int_0^R r dr = \frac{\alpha R^2}{2}.
        \end{split}
    \end{displaymath}
    \uline{Problem}: $B(0,R) = \phi(\underbrace{[0,R)\times [0,2\pi)}_{=: Q})$. Dann ist $Q$ nicht offen und $\det \phi'(0,\varphi)$. Außerdem gilt $\phi(0,\alpha) = \phi(0, \beta) \ \forall \alpha, \beta \in [0,2\pi]$, also ist $\phi$ nicht injektiv.
\end{expl*}

\begin{defn*}
    Wir nennen weiter für eine beliebige Menge $A\subset \R^d$ 
    \begin{displaymath}
        A^0 = \{x\in A : \exists r>0: B(x,r) \subset A\}
    \end{displaymath}
    das Innere von $A$.
\end{defn*}

\begin{thm}[Transformationssatz]
    \jlabel{Thm 3.30}
    \jlabel{Trafo}
    Sei $U\subset \R^d$ offen, $\phi \in C^1(U, \R^d)$, $A\in \Bd, \ A\subset U$, sodass $X:= A^0 \ne \emptyset$ gilt und $A\backslash A^0$ eine Nullmenge ist. Ferner sei $B:=\phi(A)\in \Bd$, $\phi$ sei auf $X$ inkjektiv $\det \phi'(x) \ne 0 \ \forall x\in X$.\\
    Dann ist $Y = \phi(X)$ offen, $\jabb{\phi}{X}{Y}$ \jlink{diffeomorph}, $B\backslash Y$ ist eine Nullmenge. Weiter gelten
    \begin{enumerate}
        \item
            Sei $\jabb{f}{B}{[0,\infty]}$ messbar. Dann gilt
            \begin{equation}
                \jlabel{(3.7)}
                \int_B f(y) dy = \int_A \underbrace{f(\phi(x))\cdot |\det \phi'(x)|}_{:= g(x)} dx.
            \end{equation}
        \item
            Sei $\jabb{f}{B}{\Rq}$ \jlink{messbar}. Dann ist $f$ integrierbar auf $B$ äquivalent dazu, dass $g$ integrierbar auf $A$ ist. In diesem Fall gilt \jlink{(3.7)}.
    \end{enumerate}
    \begin{proof}
        Extra Vorlesung am 16.12.2008.
    \end{proof}
\end{thm}

\begin{bem*}
    \begin{enumerate}
        \item Grundversion folgt aus \jlink{Thm 3.30}, falls $A=X=U$ offen, nach der Vorbemerkung über Ana2.
        \item Die Funktion $g$ in \jlink{Thm 3.30} ist \jlink{messbar}, da $f$ \jlink{messbar} ist und $\phi \in C^1$ nach Kapitel 2.
        \item $A = \R\backslash \Q\in \Bd$, $\lambda_1(A)=\infty$, aber $A^0 = \emptyset$.
        \item Sei $A=[0,R)\times [0,2\pi)$. Dann ist $A^0 = (0,R)\times (0,2\pi)$, also ist $A\backslash A^0$ eine zweidimensionale Nullmenge. Sei $\phi$ die Polarkoordinatenabbildung. Dann gilt $\phi(A) = B(0,R)$ und $\jabb{\phi}{A^0}{B(0,R)\backslash (\R_+ \times \{0\})}$ \jlink{diffeomorph}. Mit dem \jlink{Trafo} folgt
        \begin{displaymath}
            \begin{split}
                \lambda(B(0,R)) &= \int_{B(0,R)} 1 dy = \int_A 1 \cdot r d(r\varphi) \overset{\jshortlinkFubini}{=} \int_0^{2\pi}\int_0^R r dr d\varphi \\
                &= \frac{2\pi R^2}{2} = \pi R^2.
            \end{split}
        \end{displaymath}
    \end{enumerate}
\end{bem*}

\begin{lem}
    \jlabel{Lem 3.31}
    Sei $T\in L(\R^d)$ invertierbar, $v \in \R^d$, $A\in \Bd$. Dann gelten $B=TA + v = \{y\in \R^d : \exists x \in A : y = Tx + v\} \in \Bd$ und $\lambda_d(TA+v) = |\det T| \lambda_d(A)$.\\
    Also gilt für jede Bewegung $T$ (d.h., $\det T = \pm 1$) $\lambda(TA + v) = \lambda(A)$. Somit ist $\lambda$ bewegungsinvariant.
    
\jdate{19.12.2008}

    \begin{proof}
        Sei $\phi(x) = Tx+v$. Dann gilt $B:=\phi(A) = (\phi^{-1})^{-1}(A)\in \Bd$, da $\phi^{-1}$ stetig und damit \jlink{messbar}. \jlink{Satz 1.24} sagt dann $\lambda(TA+v) = \lambda(TA)$. Für $A\in \Bd$ setze $\mu(A):=\lambda(TA)$. Dann gilt sofort $\mu(\emptyset) = \lambda(\emptyset) = 0$.\\
        Seien $A_j\in \Bd \ (j\in \N)$ disjunkt. Da $T$ injektiv ist, sind auch die $TA_j$ ($j\in\N$) disjunkt.\\
        Ferner gilt $T\bigdcup_{j\in\N}A_j = \{y\in \R^d : \exists! x \in A_j \text{ mit } y=Tx\} = \bigdcup_{j\in\N}TA_j$. Damit gilt
        \begin{displaymath}
            \mu\left( \bigdcup_{j\in\N} A_j \right) = \lambda \left(T \bigdcup_{j\in\N} A_j \right) = \lambda \left(\bigdcup_{j\in\N} A_j \right) = \sum_{j=1}^\infty \lambda(TA_j) = \sum_{j=1}^\infty \mu(A_j).
        \end{displaymath}
        Also ist $\mu$ ein Maß.\\
        Sei $x\in\R^d$. Dann gilt $\mu(A+x)= \lambda(TA+Tx) \overset{\jshortlink{Satz 1.24}}{=} \lambda(TA) = \mu(A)$, womit $\mu$ Translationsinvariant ist. Mit \jlink{Satz 1.24} folgt dann
        \begin{displaymath}
            \tag{$*$}
            \mu(A) = c(T)\cdot \lambda(A),
        \end{displaymath}
        wobei $c(T) = \mu([q_1]^d) = \lambda(T[q_1]^d)$ gilt. Demnach müssen wir $c(T) = |\det T|$ zeigen. Dies erledigen wir in drei Schritten:
        \begin{itemize}
            \item[1)] 
                Sei speziell $T =  \left(
                                        \begin{smallmatrix}
                                             \lambda_1 &        &  0       \\
                                                       & \ddots &          \\
                                             0         &        & \lambda_d
                                        \end{smallmatrix}
                                  \right)$ mit $\lambda_j \in \R\backslash \{0\}$. Dann ist $T[0,1)^d$ ein Würfel mit $0^d$ als Ecke und den Kantenlängen $|\lambda_1|,\dots, |\lambda_d|$. Also gilt für sein Volumen $\lambda(T[0,1)^d) = |\lambda_1|\cdots|\lambda_d| = |\det T|$.
            
            \item[2)] 
                Sei speziell $T$ orthogonal (d.h. $\exists T^{-1} = T^T$). Dann gilt
                \begin{displaymath}
                    |Tx|_2^2 = (Tx|Tx) = (T^TT|x) = |x|_2^2.
                \end{displaymath}
                Genauso gilt
                \begin{displaymath}
                    \tag{$+$}
                    |T^{-1}x| = |x|_2 \ \Rightarrow \ TB(0,1) = B(0,1).
                \end{displaymath}
                Damit folgt
                \[
                    c(T)\cdot \lambda(B(0,1)) \overset{(*)}{=} \mu(B(0,1)) = \lambda(TB(0,1)) \overset{(+)}{=} \lambda(B(0,1)).
                \]
                Also gilt $c(T) = 1 = |\det T|$, da $T^{-1} = T^T$.
            
            \item[3)]
                Sei nun $T$ beliebig invertierbar.\\
                \uline{Beh}: $\exists$ orthogonale Matrizen $Q,S$ und eine Matrix $D = \left(
                                                                                        \begin{smallmatrix}
                                                                                        \lambda_1 &        &  0       \\
                                                                                                & \ddots &          \\
                                                                                        0         &        & \lambda_d
                                                                                        \end{smallmatrix}
                                                                                       \right)$, $\lambda_j \in \R\backslash \{0\}$, mit $T^{-1} = Q^{-1}DS$.\\
                \uline{Ist Beh gezeigt, dann folgt} $|\det T| = |\det Q^{-1}| \cdot |\det D| \cdot |\det S| = |\det D|$. Weiter gilt dann
                \[
                    \begin{split}
                        c(T) &= \lambda\left(T[0,1)^d\right) = \lambda\left(Q^{-1}DS[0,1)^d\right) \underset{(*)}{\overset{2)}{=}} \lambda\left(DS[0,1)^d\right) \overset{1)}{\underset{(*)}{=}} |\det D|\\
                        &=\lambda\left(S[0,1)^d\right) \overset{2)}{=} |\det D| = \lambda\left([0,1)^d\right) = 1.
                    \end{split}
                \]
                Das bedeutet, dass wir den Beweis erbracht haben, sobald \uline{Beh} gezeigt ist.
                \begin{proof}[Beweis von \uline{Beh}]
                    Die Matrix $T^TT$ ist symmetrisch, da $\left(T^TT \right)^T =$\\$T^T \left(T^T\right)^T = T^TT$, und positiv definit nach $(+)$. Aus der Linearen Algebra wissen wir, dass $Q^{-1} = Q^T$ und $D^2$ wie oben exisitieren, sodass
                    \[
                        \tag{$++$}
                        T^TT=Q^TD^2Q
                    \]
                    gilt. Setze $S:=D^{-1}QT$. Dann gelten $Q^{-1}DS = T$ und
                    \[
                        SS^T = D^{-1}QTT^TQ^TD^{-1} \overset{(++)}{=} D^{-1}\underbrace{QQ^T}_{=I}D^2 \underbrace{QQ^T}_{=I} D^{-1} = I.
                    \]
                \end{proof}
                Damit ist das Lemma gezeigt.
        \end{itemize}
    \end{proof}
\end{lem}

\begin{lem}
    \jlabel{Lem 3.32}
    Seien $\emptyset \ne X,Y \subset \R^d$ offen, $\jabb{\phi}{X}{Y}$ \jlink{diffeomorph}, $A\in \Borel(X)$. Dann ist $\phi(A)\in \Bd$ und es gilt
    \[
        \lambda_d(\phi(A)) = \int_A|\det \phi'(x)|dx.
    \]
    \begin{proof}
        Extra Vorlesung.
    \end{proof}
\end{lem}

\begin{lem}
    \jlabel{Lem 3.33}
    Sei $U\subset \R^d$ offen, $F\in C^1(U,\R^d)$ und $N\subset U$ eine d-dim. Nullmenge. Dann ist $F(N)$ auch eine d-dimensionale Nullmenge.
\end{lem}

\begin{proof}[Beweis von \jlink{Thm 3.30}]
    \uline{Vorberkung}: Nach Voraussetzung gilt $B\backslash Y \in \Bd$ und $A\backslash X$ ist eine Nullmenge. Ferner gilt $B\backslash Y = \phi(A)\backslash \phi(X) \subset \phi(A\backslash X) \overset{\jshortlink{Lem 3.33}}{\subset}$ Nullmenge. Damit ist $B\backslash Y$ eine Nullmenge.
    \begin{itemize}
        \item[a)] Sei $f\ge 0$. Dann gilt $\int_{B\backslash Y} f dx = 0 = \int_{A\backslash X} g dx$. Daraus folgt
            \[
                \int_B f dy = \int_Y f dy, \hspace{15pt} \int_A g dx = \int_X g dx.
            \]
        \item[b)] $f = \doubleOne_Y\cdot f$ \fu, $g = \doubleOne_X\cdot g$ \fu. \jlink{Lem 3.5} zeigt
        \[
            \begin{split}
                f\text{ integrierbar auf } B \ \Leftrightarrow \ &f \text{ integrierbar auf } Y \text{ und dann gilt }\\
                                                             &\int_Y f dx = \int_B f dy.
            \end{split}
        \]
    \end{itemize}
    Entsprechendes gilt für $g$. \uline{Fazit}: Das Theorem muss nur für $A=X$ und $B=Y$ gezeigt werden.\\
    Nach Ana2 ist $\jabb{\phi}{X}{Y}$ \jlink{diffeomorph} und $Y$ ist offen.
    \begin{enumerate}
        \item 
            \begin{itemize}
                \item[1)] 
                    Sei $f\ge 0$ einfach mit $f=\sum_{k=1}^m z_k\cdot \doubleOne_{B_k}$, $B_k\in \Borel(Y)$. Da $\phi$ stetig ist, folgt $\phi^{-1}(B_k) =: A_k \in \Borel(X) \ \forall k\in \N$. Weiter ist $B_k = \phi(A_k)$ und es gilt für alle $x\in X, k=1,\dots,m$
                    \[
                        \doubleOne_{A_k}(x) = \begin{cases}
                                                1, & x\in A_k\\
                                                0, & x\notin A_k
                                            \end{cases}
                                            = \begin{cases}
                                                1, &\phi(x) \in \Borel_k\\
                                                0, &\phi(x) \notin \Borel_k
                                            \end{cases}
                                            = \doubleOne_{B_k}(\phi(x)).
                    \]
                    Damit gilt
                    \[
                        \begin{split}
                            \int_Y f dy &= \sum_{k=1}^m z_k\cdot \int_Y \doubleOne_{B_k} dy \overset{\text{s.o.}}{=} \sum_{k=1}^m z_k\cdot \lambda(\phi(A_k))\\
                                        &\overset{\jshortlink{Lem 3.32}}{=} \sum_{k=1}^m z_k \cdot \int_{A_k} |\det \phi'(x)|dx\\
                                        &= \int_X \sum_{k=1}^m z_k\cdot \underbrace{\doubleOne_{A_k}(x)}_{=\doubleOne_{B_k}(\phi(x))}\cdot |\det \phi'(x)|dx\\
                                        &= \int_X f(\phi(x))\cdot |\det \phi(x)|dx.
                        \end{split}
                    \]
                
                \item[2)]
                    Sei $\jabb{f}{Y}{[0,\infty]}$ \jlink{messbar}. Dann existieren einfache $\jabb{f_n}{Y}{\R_+}$ mit $f_n \le f_{n+1}$ ($\forall n\in\N$) und $f_n \xrightarrow{n\rightarrow\infty} f$. Dann gilt
                    \[
                        \begin{split}
                            \int_Y f dy &\overset{\jshortlink{BL}}{=} \limToInf{n} \int_Y f_n dy \overset{1)}{=} \limToInf{n} \int_X \underbrace{f_n(\phi(x))\cdot |\det \phi'(x)|}_{=:g_n(x) \le g_{n+1} \xrightarrow{n\rightarrow\infty} g(x)} dx\\
                            &\overset{\jshortlink{BL}}{=} \int_Xf(\phi(x))\cdot |\det \phi'(x)|dx.
                        \end{split}
                    \]
            \end{itemize}
            Damit ist a) gezeigt.
            
        \item
            \begin{itemize}
                \item[3)] Sei $\jabb{f}{Y}{\Rq}$ \jlink{messbar}. Dann ist $g_\pm(x) = f_\pm(\phi(x))\cdot |\det \phi'(x)|$ für alle $x\in X$. Aus 2) folgt, dass \jlink{(3.7)} für $f_\pm$ und $g_\pm$ gilt. Damit gelten
                \[
                    f \text{ integrierbar} \Leftrightarrow f_\pm \text{ integrierbar} \overset{a)}{\Leftrightarrow} g_\pm \text{ integrierbar} \Leftrightarrow g \text{ integrierbar}
                \]
                und
                \[
                    \int_Y f dy = \int_Y f_+ dy - \int_Y f_- dy \overset{\jshortlink{(3.7)}}{=} \int_X g_+ dx - \int_X g_- dx = \int_X g dx.
                \]
            \end{itemize}
            Damit ist b) gezeigt.
    \end{enumerate}

\end{proof}


\begin{expl}[Affiner Transformationssatz]
    \jlabel{Bsp 3.34}
    Sei $\phi(x) = Tx + v$ für festes $v\in \R^d$ und $T\in L(\R^d)$ mit $\det T \ne 0$. Sei $A\in \Bd$, sodass $A\backslash A^0$ eine Nullmenge ist, $A^0 \ne \emptyset$. Daraus folgt $B=TA+v\in \Bd$. Für $\jabb{f}{B}{\Rq}$ (integrierbar oder \jlink{messbar} und positiv) gilt
    \[
        \int_B f(y) dy = |\det T|\cdot \int_A f(Tx+v) dx.
    \]
\end{expl}

\begin{expl}[d-dimensionale Polarkoordinaten]
    \jlabel{Bsp 3.35}
    Sei $(r, \varphi, \Theta_1,\dots \Theta_{d-2}) =: (r,\varphi, \Theta) \in \R^d, \ d\ge 2$. Setze
    \begin{equation}
        \jlabel{(3.8)}
        \phi_d(r,\varphi,\Theta) := \begin{pmatrix}
                                            r\cdot \cos(\varphi) & \cos(\Theta_1)        & \cos(\Theta_2) & \cdots & \cos(\Theta_{d-2})\\
                                            r\cdot \sin(\varphi) & \cos(\Theta_1)        & \cos(\Theta_2) & \cdots & \cos(\Theta_{d-2})\\
                                                                 & r\cdot \sin(\Theta_1) & \cos(\Theta_2) & \cdots & \cos(\Theta_{d-2})\\
                                                                 &                       & \ddots         &        & \vdots \\
                                                                 &                       &                &        & r\cdot \sin(\Theta_{d-2})
                                        \end{pmatrix}.
    \end{equation}
    \uline{Klar}: $\phi_d \in C^\infty(\R^d, \R^d)$.\\
    Sei $H_d = \R_+\times \{0\} \times \R^{d-2}$, wobei $H_2 = \R_+\times \{0\}$. $W_d = (0,2\pi)\times(-\frac{\pi}{2}, \frac{\pi}{2})^{d-2}$, wobei $H_2 = (0,2\pi)$.\\
    \uline{Beh}: Seien $R>0,\ (r, \varphi, \Theta) \in \R^d$. Dann gelten $|\phi_d(r,\varphi, \Theta)| = r$ und
    \[
        \det \phi'(r,\varphi, \Theta) = r^{d-1}\cdot \cos(\Theta_1)\cdot \cos^2(\Theta_2)\cdots \cos^{d-2}(\Theta_{d-2}) = r^{d-1} \cdot TODO.
    \]
    $\jabb{\phi_d}{(0,\infty)\times W_d}{\R^d}$ ist injektiv, $\phi_d((0,\infty)\times W_d) = \R^d\backslash H_d$,\\
    $\phi_d(\R_+\times \overline{W}_d) = \R^d$, $\phi_d((0,R)\times W_d) = B(0,R), \ \phi_d([0,R)\times \overline{W}_d) = B(0,R)\backslash H_d$.
    \begin{proof}[Zum Beweis siehe]
        Aman/Escher III, Lemma X.8.8.
    \end{proof}
    Sei $A = [0,R)\times \overline{W}_d$ Dann folgt $A^0 = (0,R)\times \overline{W}_d \Leftarrow \lambda_d(A\backslash A^0) = 0$. Ferner ist $B(0,R) = \phi(A)$. Damit gilt
    \[
        \begin{split}
            &\lambda(B(0,R)) = \int_{B(0,R)} 1 dy \overset{\jshortlink{Trafo}}{=} \int_{(0,R)} |\det \phi| dx\\
            &\underset{\jshortlinkFubini}{\overset{\jshortlink{(3.8)}}{=}} \int_0^R\int_0^{2\pi}\int_{-\frac{\pi}{2}}^{\frac{\pi}{2}} \cdots \int_{-\frac{\pi}{2}}^{\frac{\pi}{2}} r^{d-1} \cos(\Theta_1) \cdots \cos^{d-1}(\Theta_{d-2})d\Theta_{d-2}\cdots d\Theta_1 d\varphi dr\\
            &= \underbrace{\int_0^R r^{d-1}dr}_{\frac{1}{d} R^d}\cdot \underbrace{\int_0^{2\pi} d\varphi}_{= 2\pi} \cdot \prod_{k=1}^{d-2}2\cdot \underbrace{\int_0^{\frac{\pi}{2}} \cos^k(t)dt}_{=:I_k} = \frac{2^{d-1}}{d} \cdots \pi \cdot I_1\cdot I_2\cdots I_{d-2}.
        \end{split}
    \]
    \uline{Dabei}: $I_k \overset{\text{PI}}{=} \frac{k-1}{k} \cdot I_{k-2}$ mit $I_1 = 1,\ I_2 = \pi$. Per Induktion folgt
    \[
        \lambda(B(0,R)) = \begin{cases}
                              \frac{\pi^{\frac{d}{2}}}{\frac{d}{2}!}\cdot R^d, &d \text{ gerade}\\
                              \frac{2\cdot(2\pi)^{\frac{d-1}{2}}}{d\cdot(d-2)\cdots 3 \cdot 1}\cdot R^d, &d \text{ ungerade}
                          \end{cases}
                        \overset{!}{=} \frac{2\pi^{\frac{d}{2}}}{d\cdot\Gamma(\frac{d}{2})}\cdot R^d.
    \]
    Setze
    \begin{equation}
        \jlabel{(3.9)}
        \kappa_d := \lambda_d(B(0,1)) = \frac{2\pi^{\frac{d}{2}}}{d\Gamma(\frac{d}{2})}, \text{ wobei } \kappa_1 = 2, \ \kappa_2 = \pi, \ \kappa_3 = \frac{4}{3}\pi \text{ gilt}.
    \end{equation}
    Dabei ist $\Gamma(x) = \int_0^\infty t^{x-1}\cdot e^{-t}dt$. Es gelten
    \begin{equation}
        \jlabel{(3.10)}
        \Gamma(1)=1, \ x\cdot \Gamma(x) = \Gamma(x+1).
    \end{equation}
    Also $\Gamma(n) = n!$ ($\forall n\in\N$). Und es gilt $\Gamma(\frac{1}{2}) = \sqrt{\pi}$, siehe Ana1.
\end{expl}

\jdate{22.12.2008}

\begin{expl}[rotationssymmetrische Funktion]
    \jlabel{Bsp 3.36}
    Sei $I\subset \R_+$ ein Intervall, $a=\inf I_+,\ b = \sup I, \ \jabb{\phi}{I}{\Rq}$ messbar. Setze $f(y):= \phi(|y|_2)$ für $y \in R = \{y\in\R^d : |y|_2 \in I\}$ in $\R^2$.
    
    \jspace
    
    TODO: Bild
    
    \jspace
    
    Mit \jlink{Bsp 3.35}: $\jabb{\phi_d}{A:= I\times \overline{W}_d}{R}$ surjektiv. $\jabb{\phi_d}{A^0}{R^0}$ ist \jlink{diffeomorph}, $A^0 = I^0\times W_d$. Damit ist $A\backslash A^0$ eine Nullmenge. Mit \jlink{Thm 3.30} folgt:
    \[
        f\text{ auf }\R\text{ integrierbar} \Leftrightarrow g = f \circ \phi_d(\det \phi_d')\text{ integrierbar auf } A,
    \]
    wobei $g(r,\varphi, \Theta) = \phi(r)\cdot r^{d-1}\cdot \underbrace{\cos(\Theta)\cdots \cos^{d-2}(\Theta_{d-2})}_{=: w(\Theta)}$ und $w>0$ stetig und beschränkt auf $W_d$ ist.\\
    \jlinkFubini sagt:
    \[
        f \text{ integrierbar} \Leftrightarrow g \text{ integrierbar} \Leftrightarrow r\mapsto r^{d-1}\phi(r) \text{ integrierbar auf } I
    \]
    und
    \begin{equation}
        \jlabel{(3.11)}
        \begin{split}
            \int_R f(y)dy &\overset{\jshortlink{Trafo}}{\underset{\jshortlinkFubini}{=}} \int_a^b \int_0^{2\pi} \int_\frac{-\pi}{2}^\frac{\pi}{2}\cdots\int_\frac{-\pi}{2}^\frac{\pi}{2} \phi(r)\cdot r^{d-1}\cdot w(\Theta)d\Theta d\varphi dr\\
                          &=\int_a^b r^{d-1}\phi(r)dr \cdot \underbrace{\int_0^{2\pi}d\varphi}_{=2\pi} \cdot \underbrace{\int_{W_d} w(\Theta) d\Theta}_{\overset{\jlink{(3.9)}}{=} \kappa_d \frac{d}{2\pi}}\\
                          &=d\cdot \kappa_d \cdot \int_a^b r^{d-1}\phi(r) dr = \frac{2\pi^{\frac{d}{2}}}{r(\frac{d}{2})}\cdot \int_a^br^{d-1}\phi(r)dr.
        \end{split}
    \end{equation}
    \jlink{(3.11)} gilt stets für $\phi \ge 0$. Wir setzen
    \begin{equation}
        \jlabel{(3.12)}
        w_d = \frac{2\pi^\frac{d}{2}}{r(\frac{d}{2})} = d\cdot \kappa_d.
    \end{equation}
    Speziell $w_2 = 2\pi,\ w_3 = 4\pi$.
    
    \begin{expl*}
        $d=3, \ \phi(r) = \frac{1}{r}, \ I=(0,R)$. Dann gilt
        \[
            \int_{B(0,R)} \frac{dx}{|x|_2} \overset{(3.11)}{=} w_B\cdot \int_0^R r^2\cdot \frac{1}{r}dr = 4\pi\cdot \int_0^R r dr = 2\pi\cdot R^2.
        \]
        Hier gilt $f(y) = \frac{1}{|y|_2}$.
    \end{expl*}
\end{expl}

\begin{satz}
    \jlabel{Satz 3.37}
    Sei $\jabb{f}{\R^d}{\R}$ \jlink{messbar}, es gebe Konstanten $c,\epsilon>0$ mit
    \[
        |f(x)| = \begin{cases}
                     c\cdot|x|_2^{-d+\epsilon}, & 0 \le |x|_2 \le 1\\
                     c\cdot|x|_2^{-d-\epsilon}, & |x|_2 \ge 1
                 \end{cases}.
    \]
    Dann ist $f$ integrierbar.
    \begin{proof}
        Sei $\phi(r) = c\cdot r^{-d+\epsilon}$ für $0<r \le 1,\ \phi(r)=c\cdot r^{-d-\epsilon}$ für $r\ge 1$. Sei $g(x) = \phi(|x|_2)$ für $x\in \R^d\backslash \{0\}$ und $g(0) = 0$. Dann ist $g\ge 0$ und \jlink{messbar}. Mit \jlink{Bsp 3.36} folgt nun
        \[
            \begin{split}
                \int_{\R^d} g(x) dx &= w_d \cdot \int_0^\infty r^{d-1}\phi(r)dr\\
                                    &= c\cdot \underbrace{\int_0^1 r^{d-r} r^{-d+\epsilon}}_{=r^{-1+\epsilon}}dr + c\cdot \int_1^\infty \underbrace{r^{d-1} r^{-d-\epsilon}}_{=r^{-1-\epsilon}}dr < \infty.
            \end{split}
        \]
        Somit ist $g$ integrierbar. Da $|f| \le g$, folgt mit \jlink{Satz 2.23}, dass auch $f$ integrierbar ist.
    \end{proof}
\end{satz}

\begin{expl}
    \jlabel{Bsp 3.38}
    \[
        J := \int_{\R^d} e^{-|x|_2^2}dx = \pi^\frac{d}{2}.
    \]
    \begin{proof}
        \[
            J \overset{\jshortlinkFubini}{=} \underbrace{\int_\R \cdots \int_\R}_{d\text{ mal}} e^{-x_1^2} \cdots e^{-x_d^2} dx_d \dots dx_1 = \left( \int_\R e^{-t^2} dt \right)^d.
        \]
        Im Falle $d=2$ gilt
        \[
            \begin{split}
                \left(\int_\R e^{-t^2} dt \right)^2 &= \int_\R e^{-x^2} dx \cdot \int_\R e^{-y^2} dy \overset{\jshortlinkFubini}{=} \int_{\R^2} e^{-(x^2+y^2)} d(x,y)\\
                &\overunderset{\jshortlink{Trafo}}{=}{r^2=x^2+y^2} \int_0^{2\pi}\int_0^\infty e^{-r^2} dr d\varphi = \int_0^{2\pi}d\varphi \int_0^\infty e^{-r^2} \cdot r dr\\
                &\overunderset{s=r^2=\phi(r)}{=}{\frac{ds}{dr}=\phi'(r)=2r} 2\pi^{\frac{1}{2}} \cdot \int_0^\infty e^{-s} ds = \pi.                
            \end{split}
        \]
        Also gilt
        \[
            \int_\R e^{-t^2} dt = \sqrt{\pi},
        \]
        woraus die Behauptung folgt.
    \end{proof}
\end{expl}

\uline{Folgerung}:
\[
    \Gamma \left(\frac{1}{2} \right) = \int_0^\infty \frac{1}{\sqrt{t}} \cdot e^{-t} dt \overunderset{t=s^2=\phi(s)}{=}{\frac{dt}{ds}=\phi'(s)=2s} \int_0^\infty \frac{1}{s}c^{-s^2} 2s \overset{\text{s.o.}}{=} \sqrt{\pi} = 2\cdot \int_0^\infty e^{-s^2} ds.
\]

\chapter{Integralsätze}
    TODO: Einleitung

    % Etwas Differentialgeometrie
\section{Etwas Diffentialgeometrie}

\begin{defn}
    \jlabel{Def 4.1}
    Eine beschränkte offene Menge $D\subset \R^d$ hat einen \uline{$C^1$-Rand $\partial D$}, wenn es für jedes $x\in \partial D$ eine offene beschränkte Umgebung $\tilde{V}\subset \R^d$, eine offene beschränkte Menge $\tilde{U}\subset \R^d$ und einen \jlink{Diffeomorphismus} $\jabb{\psi}{\tilde{V}}{\tilde{U}}$ gibt, sodass $\psi(D\cap \tilde{V}) \subset \R^{d-1}\times (-\infty,0)$ und $\psi(\partial D\cap \tilde{V}) \subset \R^{d-1}\times \{0\}$.\\
    Wir nehmen ferner an, dass $(\psi^{-1})'$ beschränkt ist. Dann heißt $(\psi, \tilde{V})$ \jlabel{Karte}\uline{Karte} und $V:= \tilde{V}\cap \partial D$ \jlabel{Kartengebiet}\uline{Kartengebiet}.\\
    Setze $U\times \{0\} := \tilde{U} \cap (\R^{d-1}\times \{0\})$ und $F(t) = \psi^{-1}(t,0)$ für $t\in U \subset \R^{d-1}$. Dann heißt $\jabb{F}{U}{V\subset \R^d}$ \jlabel{Parametrisierung}\uline{Parametrisierung} des Kartengebiets $V$. Wenn $\psi \in C^k(\tilde{V}, \R^d)$ für jedes $x\in \partial D$, dann heißt $\partial D$ \uline{$C^k$-Rand} ($k\in\N$). Man schreibt dann $\partial D \in C^k$ für $k\in\N$.
    
    \jspace
    
    TODO: Bild
    
    \jspace
\end{defn}

\jlabel{Bem 4.2}
\begin{bem}
    \begin{enumerate}
        \item Wenn in \jlink{Def 4.1} $\psi \in C^k(\tilde{V}, \R^d)$, dann gilt auch $\psi^{-1} \in C^k(\tilde{U}, \R^d)$. (Folgt aus $(\psi^{-1})'(y) = [\psi'(\psi^{-1}(y))]^{-1}$, siehe dazu Kettenregel und Umkehrsatz).
        \item Da $\partial D$ kompakt ist, kann man in \jlink{Def 4.1} $\partial D$ mit endlich vielen $\tilde{V}_1,\dots,\tilde{V}_n$ überdecken.
        \item Die Abblildung $\jabb{F}{U}{V}$ ist homöomorph (d.h., bijektiv, stetig, $F^{-1}$ stetig). Ferner sind die Spalten $\partial_kF(t) \ (t=1,\dots,d-1)$ der Jakobimatrix bei jedem $t\in U$ gleich $\partial \psi^{-1}(t,0)$ und somit linear unabhängig. Daraus folgt
        \begin{equation}
            \jlabel{(4.1)}
            \mathrm{Rang } F'(r) = d-1 \hspace{15pt} (\forall t\in U)
        \end{equation}
        \item $\partial D \subset \psi_1^{-1}(U_1 \times \{0\})\times \dots \times \psi_m^{-1}(U_m \times \{0\})$ ist eine Nullmenge nach \jlink{Lem 3.33}, da $\psi_k^{-1}$ ein \jlink{Diffeomorphismus} ist und $\lambda_d(U_ \times \{0\}) = 0$.
    \end{enumerate}
\end{bem}


\jlabel{Bsp 4.3}
\begin{expl}
    \begin{enumerate}
        \item (Sphäre) Sei $D=B(0,R),\ \partial D=S(0,R)$ in $\R^3$. Wähle 
        \[
            \begin{split}
                \tilde{V} &= \left\{x \in \R^3 : \frac{R}{2} < |x|_2 < \frac{3R}{2} \right\} \backslash (\R_+ \times \{0\} \times \R),\\
                \tilde{U} &= \left(-\frac{R}{2}, \frac{R}{2} \right) \times (0,2\pi) \times \left(-\frac{\pi}{2}, \frac{\pi}{2} \right)
            \end{split}
        \]
        und
        \[
            \psi^{-1}(r,\varphi, \Theta) = (r+R)\cdot \begin{pmatrix}
                                                          \cos(\varphi)\cdot \cos(\Theta)\\
                                                          \sin(\varphi)\cdot \cos(\Theta)\\
                                                          \sin(\Theta)
                                                      \end{pmatrix} : \tilde{U} \rightarrow \tilde{V}.
        \]
        Dann folgt, dass
        \[
            F(\varphi, \Theta) = \psi^{-1}(0, \varphi, \Theta) = R\cdot \begin{pmatrix}
                                                                            \cos(\varphi)\cdot \cos(\Theta)\\
                                                                            \sin(\varphi)\cdot \cos(\Theta)\\
                                                                            \sin(\Theta)
                                                                        \end{pmatrix}
        \]
        die geschlitze Sphäre $\partial B(0,R)\backslash (\R_+ \times \{0\} \times \R)$ \jhyperref{Parametrisierung}{parametrisiert}.\\
        Beachte dabei, dass für $-\frac{R}{2} < r < 0$ $\psi^{-1}(r,\varphi, \Theta) \in B(0,R)$ gilt.\\
        Für den Schlitz braucht man noch eine rotierte Variante der obigen \jlink{Karte}.
        
        
\jdate{09.01.2009}
        
        
        \item Sei $\partial D$ lokal ein Graph oberhalb von $D$, d.h., dass ein offenes beschränktes $U\subset \R^d$ und ein beschränktes $h\in C^1(U,\R)$ mit beschränkter Ableitung $\triangledown h$ existieren und $a < 0 < b$, sodass mit
        \[
            \tilde{U} = U \times (a,b), \hspace{15pt} \tilde{V} = \{(t,s) \subset U \times \R : s \in (a+h(t), b+h(t))\}
        \]
        gilt:
        \[
            \begin{split}
                \tilde{V} \cap D &= \{(t,s) \in U \times \R : s\in (a+h(t), h(t))\},\\
                \tilde{V} \cap \partial D &= \{(t,h(t)) : t \in U\}.
            \end{split}
        \]<
        Setze hier $\psi(t,s) := \begin{jsmallmatrix} t \\ s - h(t) \end{jsmallmatrix}$ für $(t,s) \in \tilde{V}$. Dann folgt $\jabb{\psi}{\tilde{V}}{\tilde{U}}$ ist bijektiv und $C^1$ mit inverser $\jabb{\psi^{-1}}{\tilde{U}}{\tilde{v}},\ \psi^{-1}(t,\tau) = \begin{jsmallmatrix} t \\ \tau + h(t) \end{jsmallmatrix}$. Hier gilt $\psi'(t,s) = \begin{jsmallmatrix}I_{d-1} & 0 \\ h'(t) & 1\end{jsmallmatrix}$, also gilt $\det \psi'(t,s)=1$. Damit sind $\psi$ und $\psi^{-1}$ \jlink{diffeomorph} mit beschränkter Ableitung.\\
        Ferner gilt $\psi(\tilde{V}\cap D) = U \times (a,0),\ \psi(\tilde{V} \cap \partial D) = U \times \{0\}$. Demnach ist $(\psi, \tilde{V})$ eine \jlink{Karte} (\jlink{Def 4.1}). Hier gilt $F(t) = \begin{jsmallmatrix}t \\ h(t)\end{jsmallmatrix}$ für $t\in U$.
        
        \item Seien $w \in \R^d\backslash \{0\},\ a\in \R$ fest. Setze $D := \{x \in \R^d: (x|w) < a\}$ (Halbraum). Dann ist $\partial D = \{x \in \R^d : (x|w) = a\}$. Wähle Basis $v_1,\dots, v_{d-1}$ von $\{w\}^\bot$ und $p = \frac{a}{|w|_2^2}w$. Daraus folgt $(p|w) = a$. Setze $\psi^{-1}(t,\tau) := \sum_{j=1}^{d-1} t_j\cdot v_j + \tau w + p$ für $t \in \R^{d-1},\ \tau \in \R$. Also ist $\jabb{\psi^{-1}}{\R^d}{\R^d}$ \jlink{diffeomorph}.\\
        Mit $x = \psi^{-1}(t,\tau)$ gilt $(x|w) = 0 + \tau |w|_2^2 + a \overunderset{<}{=}{>} a$ für $\tau \overunderset{<}{=}{>} a$. Damit gilt $\psi^{-1}(\R^{d-1} \times (-\infty,0)) = D, \ \psi^{-1}(\R^{d-1} \times \{0\}= \partial D$. Also ist $\psi$ eine ``Karte''. Hier ist $F(t) = t_1 v_1 + \dots + t_{d-1} v_{d-1} + p$.
    \end{enumerate}
\end{expl}

\subsubsection*{Zur Tangentialgleichung}

Sei $x\in \partial D$, $F$ sei eine \jlink{Parametrisierung} wie in \jlink{Def 4.1} mit $F(t) = x$. Sei $\phi \in C^1((-1,1), \R^{d-1})$ mit $\phi(0) = t, \ \phi'(0) = w \in \R^{d-1} \ \{0\}$. Dann folgt
\[
    \begin{split}
        \gamma &= F \circ \phi \in C^1((-1,1), \R^d)\\
        \gamma(0) &= F(t) = v, \ \gamma(\tau) \in \partial D \ \forall t \in (-1,1)\\
        \gamma'(0 &= F'(\phi(0))w = F'(t)w \in \R^d \backslash \{0\}.
    \end{split}
\]
TODO: Bild

\jspace

Betrachte $v$ als Tangentialrichtung an $\partial D$ bei $x$.

\begin{defn}
    \jlabel{Def 4.4}
    Sei $\partial D \subset C^1$. Der \jterm{Tangentialraum} $T_x\partial D$ an $\partial D$ bei $x \in \partial D$ ist der Bildraum $F'(t)(\R^{d-1})$ einer \jlink{Parametrisierung} (\jlink{Def 4.1}) mit $F(t) = x$.\\
    Die \jterm{Tangentialhyperebene} ist $x + T_x\partial D$. Das orthogonale Komplement von $T_x\partial D$ in $\R^d$ heißt \jterm{Normalenraum} $N_x\partial D$.\\
    Ein $w \in N_x\partial D$ heißt \uline{äußere Normale}, wenn ein $\delta > 0$ existiert, sodass $x + \tau w \in D$ für alle $t\in (-\delta, 0)$ und $x + \tau w \notin D$ für alle $t\in (0, \delta)$.
\end{defn}

\jlabel{Bem 4.5}
\begin{bem}
    \begin{enumerate}
        \item Die Spalten $\partial_1 F'(t), \dots, \partial_{d-1} F'(t)$ spannen $T_x\partial D$ auf. Mit \jlink{Def 4.1} folgt $\dim T_x\partial D = d-1$ und daraus $\dim N_x\partial D = 1$.
        
        \item Sei $\jabb{G}{W}{\R^d}$ eine weitere \jlink{Parametrisierung} von $\partial D$ bei $x = F(t)$ mit $G(s) =x$. Sei $(\psi, \tilde{V})$ die Karte bei $x$ zu $F$ (aus \jlink{Def 4.1}). Wähle $W$ so klein, dass $s\in W,\ G(W) \subset \tilde{V}$. Setze $\phi \in C^1(W, \R^{d-1})$ durch $\phi = P \circ \psi \circ G$ mit $P(t', \tau)=t'$ für $t' \in \R^{d-1}, \tau\in \R$.\\
        Dann folgt $F \circ \phi = G$ und $\phi(s)=t$. Da $G'(s) = F'(t)\phi'(s)$ und $G'(s)$ injektiv (siehe \jlink{(4.1)} für $G$), folgt $\phi'(s)$ ist injektiv. Aus der Linearen Algebra wissen wir, dass $\phi'(s)$ als lineare Abbildung damit auch bijektiv ist. Daraus folgt $G'(s)\R^{d-1} = F'(s)\R^{d-1}$.\\
        Also sind $T_x\partial D$ und $N_x\partial D$ unabhängig von der der Wahl der \jlink{Parametrisierung}.
    \end{enumerate}
\end{bem}

\jlabel{Bsp 4.6}
\begin{expl}[vergleiche \jlink{Bsp 4.3}]
    \begin{enumerate}
        \item
            Seien $D=B(0,R)$ und $d=3$. Die Menge $\partial D \backslash (\R_+ \times \{0\} \times \R)$ hat die \jlink{Parametrisierung}
            \[
                F(\varphi, \Theta) = R \cdot \begin{pmatrix}
                                                 \cos(\varphi) \cdot \cos(\Theta) \\
                                                 \sin(\varphi) \cdot \cos(\Theta) \\
                                                 \sin(\Theta)
                                             \end{pmatrix}
            \]
            für $(\varphi, \Theta)\in U = (0,2\pi)\times(-\frac{\pi}{2}, \frac{\pi}{2})$. Sei $x = F(\varphi, \Theta)$. Dann folgt
            \[
                F'(\varphi, \Theta) = R \cdot \begin{pmatrix}
                                                  -\sin(\varphi)\cdot\cos(\Theta) & -\cos(\varphi)\cdot\sin(\Theta) \\
                                                  \cos(\varphi)\cdot\cos(\Theta)  & -\sin(\varphi)\cdot\sin(\Theta) \\
                                                  0                               & \cos(\Theta)
                                              \end{pmatrix}.
            \]
            Damit wird $T_x\partial D$ von $v_1 = \begin{jsmallmatrix}
                                                            -\sin(\varphi)\cdot\cos(\Theta) \\
                                                            -\sin(\varphi)\cdot\cos(\Theta) \\
                                                            0
                                                        \end{jsmallmatrix}$ und 
                                                 $v_2 = \begin{jsmallmatrix}
                                                            -\cos(\varphi)\cdot\sin(\Theta) \\
                                                            -\sin(\varphi)\cdot\sin(\Theta) \\
                                                            \cos(\Theta)
                                                        \end{jsmallmatrix}$ aufgespannt.
            Eine äußere Normale ist $x = R\cdot\begin{jsmallmatrix}
                                                   \cos(\varphi) \cdot \cos(\Theta) \\
                                                   \sin(\varphi) \cdot \cos(\Theta)  \\
                                                   \sin(\Theta)
                                               \end{jsmallmatrix}$,
            denn es gelten $(v_2|x) = 0$ und $|x + \tau x|_2 = |1 + \tau|\cdot |x|_2 = |1 + \tau|\cdot R =: J$. Es gilt nun $J > R$, wenn $\tau > 0 $ und $J < R$, wenn $\tau \in (-1,0)$. Also gilt auch $x + \tau \cdot x \in D$ für $\tau >0$ und $x+\tau \notin D$ für $\tau \in (-1,0)$.
            
            \jspace
            
            TODO: Bild
        \item
            $\partial D$ liege bei $x_0$ unterhalb vom Graph von $h$. Dann folgt, dass es ein offenes $U\subset \R^{d-1}$, $t_0\in U$ und ein $h\in C^1(U,\R)$ gibt, sodass $F(t) = \begin{jsmallmatrix} t \\ h(t) \end{jsmallmatrix}$ und $F(t_0) = x_0$ gelten. Damit sind die Tangentialvektoren bei $x_0$ gerade $v_j = \begin{jsmallmatrix} e_j \\ \partial_j h(t_0) \end{jsmallmatrix}$, $j=1,\dots,d-1$, wobei $e_j\in \R^{d-1}$. Weiter ist $w = \begin{jsmallmatrix} -\triangledown h(t_0) \\ 1 \end{jsmallmatrix}$ eine äußere Normale, denn es gelten $(v_j|w)=0 \ \forall j=1,\dots, d-1$ und $x_0 + t w = \underbrace{\begin{jsmallmatrix} t_0 - \tau\cdot \triangledown h(t_0) \\ h(t_0)+\tau \end{jsmallmatrix}}_{=: (t,s)^T}$.\\
            Somit folgt die Existenz eines $\delta > 0$, sodass für $|\tau|<\delta$ gilt: $t\in U$ und 
            \[  
                \begin{split}
                    \psi_d(t) &= s - h(t) \\
                              &\overset{\text{Taylor}}{=} h(t_0) + \tau - (h(t_0) + \triangledown h(t_0)\underbrace{(t-t_0)}_{=-\tau \triangledown h(t_0)} + \sigma(|t-t_0|_2)\\
                              &= \text{TODO: Hier macht einiges in meinem Mitschrieb keinen Sinn...}
                \end{split}
            \]
            (siehe dazu \jlink{Bsp 4.3}).

        \item In \jlink{Bsp 4.3}c) ist $\partial D$ die \jlink{Tangentialhyperebene} für alle $x\in\partial D$ und $w$ ist eine äußere Normale.
    \end{enumerate}
\end{expl}

\begin{lem}
    \jlabel{Lem 4.7}
    Sei $\partial D \in C^1$ und $x_0\in \partial D$. Nach eventueller Rotation und Spiegelung des Koordinatensystems gibt es eine offene Umgebung $\tilde{V}$ von $x_0$, sodass $\partial D$ eine \jlink{Karte} $(\psi, \tilde{V})$ wie in \jlink{Bsp 4.3}b) hat. Insbesondere gibt es eine eindeutig bestimmte äußere Normale $\nu(x_0)$, die durch
    \[
        \nu(x_0) = \frac{1}{\sqrt{1 + |\triangledown h(t)|_2^2}} \begin{pmatrix}
                                                                     -\triangledown h(t) \\
                                                                     1
                                                                 \end{pmatrix}
    \]
    gegeben ist. Dabei ist $x = \begin{jsmallmatrix} t \\ h(t) \end{jsmallmatrix}$ für $t\in U$.
    \begin{proof}
        Die Eindeutigkeit folgt aus \jlink{Def 4.4} und $\dim N_{x_0} \partial D = 1$. Sei $(\psi, \tilde{V})$ eine \jlink{Karte} von $\partial D$ bei $x_0$ mit zugehöriger \jlink{Parametrisierung} $\jabb{F}{U_0}{\R^d}$ mit $F(t_0) = x_0$. Da $\text{Rang} F'(t_0) = d-1$, gibt es eine Rotation des $\R^d$, sodass $\triangledown F_1(t_0),\dots, \triangledown F_2(t)$ linear unabhängig sind. (Wir behalten auch nach der Rotation die alten Bezeichnungen.)\\
        Sei $f = (F_1,\dots, F_{d-1})^T$. Aus dem Umkehrsatz folgt: $\exists$ offenes $U_1\subset U_0,\ t_0 \in U_1$ mit $\jabb{f}{U_1}{f(U_1)} =: W$ offen und $f$ \jlink{diffeomorph}.\\
        Setze $h(s) := F_d(f^{-1}(s)), \ G(s) := (s, h(s))^T$. Damit ist $h\in C^1(W,\R)$ und (nach eventueller Verkleinerung von $U_1$) sind $h$ und $\triangledown h$ beschränkt und $G(s) = (f(t), F_d(t))^T = F(t)$, $t := f^{-1}(s), \ s\in U_1$. Also ist $G$ eine \jlink{Parametrisierung}.\\
        $V_1 = G(U_1) = F(W) \subset V$. Aus \jlink{Bsp 4.6}b) folgt: haben $(-\triangledown h(s_0), 1)^T$ (TODO: WTF?!). Nach weiterer Rotation wird dieser Vektor zu $\alpha\cdot e_d \in \R^d$ für ein $\alpha >0$. Demnach wird $T_{x_0} \partial D$ von $\{e_1,\dots,e_{d-1}\}$ aufgespannt und ebenfalls von $\partial_j \psi^{-1}(t_0,0) = \partial_j F(t_0)$ ($j=1,\dots, d-1$) (\jlink{Bem 4.5}). Da $(\psi^{-1})'(t_0,0)$ invertierbar ist, ist $\partial_d(\psi^{-1})(t_0,0)$ linear unabhängig zu $\partial_1 F(t_0),\dots,\partial_{d-1}F(t_0)$. Folgt gilt $\partial_d \psi^{-1}_d(t_0,0) \ne 0$.
        
\jdate{12.01.2009}
        
        Wir können annehmen, dass $(\partial_d \psi^{-1})_d(t_0,0)>0$. (Andernfalls ersetze $\tau$ durch $-\tau$, was einer Spiegelung an der Ebene $\{t_d=0\}$ entspricht.) Daraus folgt: $\exists U_2\subset U_1$, $U_2$ offen mit $t_0\in U_2$, $\delta, \eta>0$, sodass $(\partial_d \psi^{-1})_d(t,0) > 0$, $\forall t \in U_2$, $|\delta| \le \eta$.\\
        Sei $t\in U_2, \tau \in (-\eta, 0)$. Mit dem Mittelwertsatz folgt dann
        \[
            \exists \sigma \in (\tau,0) \text{ mit } (\psi^{-1})_d(t,\tau) - \underbrace{(\psi^{-1})_d(t,0)}_{=F(t)=h(f(t))} = (\partial_d \psi^{-1})_d(t,\sigma)\cdot \tau \le \delta \cdot \tau < 0.
        \]
        Damit liegt $D$ für diese Punkte unterhalb von $\partial D$. Mit \jlink{Bsp 4.6}b) und \jlink{Bem 4.5}b) folgt der Ausdruck für $\nu$.
    \end{proof}
\end{lem}

        
    % Das Oberflächenintegral

\section{Das Oberflächenintegral}

\uline{Idee}: Sei $d=3$, $\jabb{F}{U}{V}$ eine \jlink{Parametrisierung}, $U\subset \R^3$. Dann sind $\partial_1 F(s,t), \ \partial_2 F(s,t)$ linear unabhängig $\forall (s,t)\in U$.

\jspace

TODO: Bilder

\jspace

$T_{jk}$ ist ein Parallelogramm in ${T_x}_{jk}$, das von $v_{jk} = \partial_1 F(s_k,t_j)\Delta s_j$ und $v_{jk} = \partial_2 F(s_k,t_j)\Delta t_k$. Dann gilt

\[
    \begin{split}
        ``\text{vol}_2(V)`` &= \sum_{s,k} \text{vol}_2 (v_{jk}) \approx \sum_{j,k} |v_{jk} \times w_{jk}|_2\\
                            &= \sum_{j,k} |\partial_1 F(s_j,t_k) \times \partial_2 F(s_j,t_k)|_2 \Delta s_j \Delta t_k\\
                            ``&\xrightarrow{\Delta s_j, \Delta t_k \rightarrow 0}`` \int_U \underbrace{|\partial_1 F(s,t) \times \partial_2 F(s,t) |_2}_{=: \sqrt{g_F(s,t)}} ds dt.
    \end{split}
\]
Weiter gilt
\[
    \begin{split}
        g_F &\overset{\text{LA}}{=} |\partial_1 F|_2^2 \cdot |\partial_2 F|_2^2 - (\partial_1 F| \partial_2 F)^2 = \det \begin{pmatrix}
                                                                                                                           |\partial_1 F|_2^2| & (\partial_1 F| \partial_2 F) \\
                                                                                                                           (\partial_1 F| \partial_2 F) & |\partial_2 F|_2^2
                                                                                                                         \end{pmatrix}\\
            &=\det \underbrace{\left((F')^T \cdot F'\right)}_{\text{symm. } 2\times 2 \text{ Matrix}}.
    \end{split}                                                                                                                        
\]

\begin{defn}
    \jlabel{Def 4.8}
    Sei $\jabb{F}{U}{\R^d}$ eine \jlink{Parametrisierung}. Dann ist die \jterm{Gramsche Determinante} von $F$ durch
    \[
        g_F(t) = \det \left( F'(t)^T \cdot F'(t) \right)
    \]
    für alle $t\in U$ gegeben.\\
    Im Spezialfall $d=3$ gilt
    \[
        g_F(t) = |\partial_1 F(t) \times \partial_2 F(t)|_2^2 = \left|\begin{pmatrix}
                                                                        \partial_1 F_2(t)\cdot \partial_2 F_3(t) - \partial_1 F_3(t)\cdot \partial_2 F_2(t) \\
                                                                        \partial_1 F_3(t)\cdot \partial_2 F_1(t) - \partial_1 F_1(t)\cdot \partial_2 F_3(t) \\
                                                                        \partial_1 F_1(t)\cdot \partial_2 F_2(t) - \partial_1 F_2(t)\cdot \partial_2 F_1(t) \\
                                                                     \end{pmatrix} \right |_2^2.
    \]
\end{defn}

\begin{bem*}
    Für $d=3$ ist $\partial_1 F(t) \times \partial_2 F(t)$ eine Normale an $V$ bei $F(t)=x$, da sie senkrecht auf $\partial_1 F(t)$ und  $\partial_2 F(t)$ steht.
\end{bem*}

\jlabel{Bsp 4.9}
\begin{expl}
    \begin{enumerate}
        \item 
            \jlabel{Bsp 4.9a)}
            Sei $V = \partial B(0,R) \backslash (\R_+ \times \{0\} \times \R) \subset \R^3$ parameterisiert durch die sphärischen Koordinaten (siehe \jlink{Bsp 4.3}a), \jlink{Bsp 4.6}a)). Dann gilt für $\varphi \in (0,2\pi), \ \Theta \in (-\frac{\pi}{2},\frac{\pi}{2})$
            \[
                F'(\varphi, \Theta) = R \cdot \begin{pmatrix}
                                                 -\sin(\varphi)\cdot \cos(\Theta)  &  -\cos(\varphi)\cdot \sin(\Theta) \\
                                                 \cos(\varphi)\cdot \cos(\Theta)   &  -\sin(\varphi)\cdot \sin(\Theta) \\
                                                 0                                 &   \cos(\Theta)
                                             \end{pmatrix}
            \]
            und
            \[
                F'(\varphi,\Theta)^T\cdot F'(\varphi,\Theta) = R^2\cdot \begin{pmatrix}
                                                                            \cos^2(\Theta) & 0 \\
                                                                            0              & 1
                                                                        \end{pmatrix}.
            \]
            Damit folgt
            \[
                \sqrt{g_F(\varphi, \Theta)} = R^2 \cdot \sqrt{\det \begin{pmatrix}
                                                                        \cos^2(\Theta) & 0 \\
                                                                        0              & 1
                                                                    \end{pmatrix}} = R^2\cdot \cos(\Theta).
            \]
            
        \item
            \jlabel{Bsp 4.9b)}
            $V$ als Graph (\jlink{Bsp 4.6}b)). Hier gilt $F(t) = (t, h(t))^T$ für $t\in U \subset \R^{d-1}$, $h\in C^1(U,\R)$. Damit folgt
            \[
                G(t) := F'(t)^T\cdot F'(t) = \left[ I_{d-1} \ \triangledown h(t) \right] \cdot \begin{bmatrix}
                                                                                                    I_{d-1} \\
                                                                                                    \triangledown h(t)^T
                                                                                                \end{bmatrix} = I_{d-1} + \underbrace{\triangledown h(t) \cdot \triangledown h(t)^T}_{=: H(t)}
            \]
            und damit
            \[
                \begin{split}
                G(t)\triangledown h(t) &= \triangledown h(t) + \triangledown h(t)\cdot |\triangledown h(t)|_2^2 = (1 + |\triangledown h(t)|_2^2)\cdot \triangledown h(t)\\
                H(t) &= \triangledown h(t)\cdot (\triangledown h(t)|v), \hspace{10pt} \forall v\in\R^{d-1}.
                \end{split}
            \]
            Somit gilt
            \[
                \text{Rang } H(t) = \begin{cases}
                                       1, & \triangledown h(t) \ne 0 \\
                                       0, & \triangledown h(t) = 0
                                   \end{cases} \Rightarrow \dim \text{Kern } H(t) = \begin{cases}
                                                                                        d-2, & \triangledown h(t) \ne 0\\
                                                                                        d-1, & \triangledown h(t) = 0
                                                                                    \end{cases}.
            \]
            Also hat $G(t)$ die Eigenwerte $1 + |\triangledown h(t)|_2^2$ und $1$ ($d-2$ fach). Schließlich gilt dann
            \[
                \sqrt{g_F(t)} = \sqrt{\det G(t)} = \sqrt{1 + |\triangledown h(t)|_2^2}.
            \]

        \item
            \jlabel{Bsp 4.9c)}
            Die geschlitzte $d$-dimensionale Sphäre $\partial B(0,R)\backslash H_d$ (vgl. \jlink{Bsp 3.35}) wird durch
            \[
                \begin{split}
                F(\varphi, \Theta_1, \dots, \Theta_{d-2}) &= \phi_d(R, \phi, \Theta)\\
                 &= R \cdot
                    \begin{pmatrix}
                        \cos(\varphi) & \cos(\Theta_1) & \cdots & \cos(\Theta_{d-2})\\
                        \sin(\varphi) & \cos(\Theta_1) & \cdots & \cos(\Theta_{d-2})\\
                                      & \sin(\Theta_1) & \cdots & \cos(\Theta_{d-2})\\
                                      &                & \ddots & \vdots \\
                                      &                &        & \sin(\Theta_{d-2})
                    \end{pmatrix}
                \end{split}
            \]
            mit $(\varphi, \Theta) \in (0,2\pi) \times (-\frac{\pi}{2},\frac{\pi}{2}) =: U$ parameterisiert. Hierbei gilt
            \[
                \sqrt{g_F(\varphi, \Theta)} = R^{d-1}\cdot \cos^1(\Theta_1) \cdots \cos^{d-2}(\Theta_{d-2})
            \]
            (Ohne Beweis)
    \end{enumerate}
\end{expl}

Es seien stets $\partial D$ und die \jhyperref{Kartengebiet}{Kartengebiete} $V$ mit der Borel-$\sigma$-Algebra $\Borel(\partial D)$, $\Borel(V)\subset \Bd$ versehen. Da $\jabb{F}{U}{V},\ \jabb{F^{-1}}{V}{U}$ stetig sind, gilt
\begin{equation}
    \jlabel{(4.2)}
    f=\jabb{f\circ F \circ F^{-1}}{V}{\Rq} \text{ \jlink{messbar}} \Leftrightarrow \jabb{f \circ F}{U}{\Rq} \text{ \jlink{messbar}}
\end{equation}

Sei ferner $f$ positiv oder $f\circ F \cdot \sqrt{g_F}$ integrierbar, dann definieren wir das \uline{Oberflächenintegral} auf dem \jlink{Kartengebiet} $V$ durch
\begin{equation}
    \jlabel{(4.3)}
    \int_V f d\sigma = \int_V f(x)d\sigma(x) =: \int_M f(F(t))\cdot \sqrt{g_F(t)} dt,    
\end{equation}
wobei $\jabb{F}{U}{V}$ eine \jlink{Parametrisierung} ist.\\
Für $B\in \Borel(V)$ gilt $A:= F^{-1}(B) \subset \Borel(U)$ und wir definieren das \uline{Oberflächenmaß} auf $V$ durch
\begin{equation}
    \jlabel{(4.4)}
    \sigma(B) := \int_V \doubleOne_B d\sigma = \int_U \underbrace{\doubleOne_B(F(t))}_{=\doubleOne_A(t)}\cdot \sqrt{g_F(t)} dt.
\end{equation}

\jlabel{Bsp 4.10}
\begin{expl}
    \begin{enumerate}
        \item 
            \jlabel{Bsp 4.10a)}
            (Hintere Halbsphäre) Es sei
            \[
                M = \{x\in \partial B(0,R) \subset \R^3 : x_2 > 0\} = F\left((0,\pi) \times \left(-\frac{\pi}{2},\frac{\pi}{2}\right)\right).
            \]
            (sphärische Koordinaten). Aus \jlink{Bsp 4.9} folgt $\sqrt{g_F(\varphi,\Theta)} = R^2\cdot \cos(\Theta)$. Dann folgt
            \[
                \int_M f d\sigma \overunderset{\jshortlink{(4.3)}}{=}{\jshortlinkFubini} \int_0^\pi\int_{-\frac{\pi}{2}}^{\frac{\pi}{2}} f(F(\varphi,\Theta))\cdot R^2\cdot \cos(\Theta) d\Theta d\varphi.
            \]
            \begin{expl*}
                \begin{itemize}
                    \item Fall $f=1$.
                        \[
                            \sigma(M) = \int_M 1 d\sigma = R^2\cdot \int_0^\pi d\varphi \int_{-\frac{\pi}{2}}^{\frac{\pi}{2}} \cos(\Theta) d\Theta = 2\pi R^2.
                        \]
                    \item Fall $f(x) = |x_3| = R\cdot |\sin(\Theta)| = f(F(\varphi, \Theta))$. Dann gilt
                        \[
                            \begin{split}
                                \int_M f d\sigma &= \int_0^\pi d\varphi \int_{-\frac{\pi}{2}}^{\frac{\pi}{2}} R^3 |\sin(\Theta)| \cos(\Theta) d\Theta d\varphi\\
                                                 &= R^3\cdot \int_0^\pi d\varphi \cdot \underbrace{2\cdot \int_{-\frac{\pi}{2}}^{\frac{\pi}{2}}}_{=\frac{1}{2}\sin^2(\Theta)|_0^\frac{\pi}{2}} = \pi R^3.
                            \end{split}
                        \]
                \end{itemize}
            \end{expl*}
        
        \item
            \jlabel{Bsp 4.10b)}
            (Paraboloidoberfläche). Wir betrachten den Graph von 
            \[
                h(s,t) = b \left(1 - \frac{s^2}{R^2} - \frac{t^2}{R^2}\right)
            \]
            mit $(s,t)\in U := B(0,R) \subset \R^2$. Hier gilt $F(s,t) = \left(s,t,h(s,t)\right)^T$,\\
            $\sqrt{g_F(s,t)} = \sqrt{1 + |\triangledown h(s,t)|_2^2}$ mit $\triangledown h(s,t) = \left(-\frac{2bs}{R^2}, \frac{2bt}{R^2} \right)$. Damit folgt
            \[
                \begin{split}
                    \sigma(M) &= \int_M d\sigma = \int_{B(0,R)} \sqrt{1 + |\triangledown h(s,t)|_2^2} ds dt\\
                              &= \int_{B(0,R)} \left(1 + \frac{4b^2}{R^4}(s^2+t^2) \right)^\frac{1}{2} dsdt\\
                              &\overunderset{\text{Polarkoord.}}{=}{\jshortlinkFubini} \int_0^{2\pi} \int_0^R \sqrt{1 + \frac{4b^2}{R^4}r^2}\cdot r dr d\varphi\\
                              &=2\pi\frac{2b}{R^2}\cdot \int_0^R \sqrt{\frac{R^4}{4b^2} + r^2} \cdot r dr\\
                              &\overunderset{x = r^2}{=}{dx = 2r dr} \frac{4\pi b}{R^2} \int_0^R \sqrt{\frac{R^4}{4b^2} + x} \frac{dx}{2} = \left [\frac{2\pi b}{R^2} \cdot \frac{2}{3} \left(\frac{R^4}{4b^2} +x \right)^\frac{3}{2} \right]_0^{R^2}.\\
                              &= \frac{4\pi b}{3}\cdot R \cdot \left (\left(\frac{R^2}{4b^2} + 1 \right)^\frac{3}{2} - \left(\frac{R^2}{4b^2} \right)^\frac{3}{2} \right)
                \end{split}
            \]
        
        \item
            \jlabel{Bsp 4.10c)}
            Sei \uline{Spezialfall $d=2$}: $\jabb{F}{(a,b)}{V\subset \R^2}$. Dann ist $g_F(t) = F'(t)^T\cdot F'(t) = |F'(t)|_2^2$. Dann gilt
            \[
                \int_V f d \sigma = \int_a^b f(F(t))\cdot |F'(t)|_2 dt.
            \]
    \end{enumerate}
\end{expl}


Sei $\partial D \in C^1$, $D\subset \R^d$ offen und beschränkt. 
\begin{equation}
    \jlabel{(4.5)}
    \begin{split}
        &\text{Dann gibt es Karten } (\psi_k, \tilde{V}_k) \text{ mit \jhyperref{Parametrisierung}{Parametrisierungen}}\\ &\jabb{F_k}{U_k}{V_k},\ U_k \subset \R^{d-1}\ (k=1,\dots,m) \text{ mit } \partial D \subset V_1 \cup \dots \cup V_m.
    \end{split}
\end{equation}
Wir haben also auf $V_k$ Oberflächenmaße $\sigma_k$ wie in \jlink{(4.4)} mit \jhyperref{Gramsche Determinante}{Gramscher Determinante} $g_k = g_{F_k}$.

\begin{lem}
    \jlabel{Lem 4.11}
    Seien $V_1,\dots,V_n$ wie in \jlink{(4.5)}. Dann existieren \jhyperref{messbar}{messbare} $\jabb{\varphi_k}{\partial D}{\R}$ mit $0 \le \varphi_k \le 1$, $\varphi_k = 0$ auf $\partial D\backslash V_k$ $(k=1,\dots,m)$ und $\sum_{k=1}^m \varphi_k(x) = 1, \ \forall x \in \partial D$.
    
\jdate{16.01.2009}
    
    \begin{proof}
        Definiere $W_1 = V_1$, $W_k = V_k\backslash (V_1 \cup \dots \cup V_{k-1})$ für $k=2,\dots,m$. Dann gilt $W_k \in \Borel(\partial D)$. Setze $\varphi_k = \doubleOne_{W_k}$. Aus $\partial D = W_1 \dcup \dots \dcup W_m$ folgt die Behauptung.
    \end{proof}
\end{lem}

\begin{lem}
    \jlabel{Lem 4.12}
    Sei $\partial D \in C^1$, $(\psi_k, \tilde{V}_k)$, $F_k$ wie in \jlink{(4.5)} und $\varphi$ wie in \jlink{Lem 4.11}. Dann definiert
    \[
        \sigma(B) := \sum_{k=1}^m \int_{V_k} \doubleOne_B \cdot \varphi_k d\sigma_k := \sum_{k=1}^m \int_{U_k} \doubleOne_B(F_k(t))\cdot \varphi(F_k(t)) \cdot \sqrt{g_k(t)} dt
    \]
    für $B\in \Borel(\partial D)$ ein endliches Maß auf $\Borel(\partial D)$. Es heißt \uline{Oberflächenmaß} (O-Maß). Wenn $B\subset V_i$ für ein $i\in \{1,\dots,n\}$, dann gilt $\sigma(B) = \sigma_i(B)$ (mit dem Oberflächenmaß $\sigma_i$ auf $V_i$ aus \jlink{(4.4)}). Weiter hängt $\sigma$ nicht von der Wahl der Karten $(\psi_k, \tilde{V}_k)$ und der Wahl der $\varphi_k$ wie in \jlink{Lem 4.11} ab.
    
    \begin{proof}
        Es gelten $\sigma(\emptyset)=0$ und
        \[
            \begin{split}
                \sigma(\partial D) &= \sum_{k=1}^m \int_{U_k} \underbrace{\varphi_k(F_k(t))}_{\le 1} \cdot \sqrt{g_k(t)}dt \\
                                   &\le \sum_{k=1}^m \lambda_{d-1}(U_k)\cdot \max_{t\in U_k}(\det F_k'(t)^T \cdot \underbrace{F'(t)}_{=:A})^\frac{1}{2}
                                   < \infty.
            \end{split}
        \]
        Dabei ist $\lVert A \rVert \le c \in \R$ nach \jlink{Def 4.1}.\\
        Seien $B_j\in \Borel(\partial D)$ disjunkt für $j\in\N$ und $B:= \bigdcup_{j\in\N} B_j$. Dann gilt
        \[
            \begin{split}
                \sigma(B) &\overset{\text{Def}}{=} \sum_{k=1}^m \int_{U_k} \underbrace{\doubleOne_{\bigdcup_{j\in\N} B_j} (F_k(t))}_{=\sum_{j=1}^\infty\doubleOne_{B_j}(F_k(t))} \cdot \varphi_k(F_k(t)) \cdot \sqrt{g_k(t)} dt\\
                          &\overset{\jshortlink{Kor 2.20}}{=} \sum_{j=1}^\infty \sum_{k=1}^m \int_{U_k} \doubleOne_{B_j} (F_k(t))\cdot \varphi_k(F_k(t))\cdot \sqrt{g_k(t)} dt \overset{\text{Def}}{=} \sum_{j=1}^\infty \sigma(B_j).
            \end{split}
        \]
        Somit ist $\sigma$ eine endliches Maß. Seien $(\kappa_l, \tilde{D}_l)$ weitere Karten von $D$ mit $\partial D \subset D_1\cup \dots \cup D_n$, $D_l = \tilde{D}_l\cap \partial D$, $l=1,\dots,n$ und $\jabb{G_l}{W_l}{D_l}$ die zugehörigen \jhyperref{Parametrisierung}{Parametrisierungen} mit offenen $W_l \subset \R^{d-1}$, sowie $\tilde{\sigma}_l$ das zu $G_l$ gehörende Oberflächenmaß auf $D_l$, $l=1,\dots,n$. Weiter sollen $\chi_1,\dots,\chi_n$ \jlink{Lem 4.11} für $D_1, \dots, D_n$ erfüllen. Definiere dann $\tilde{\sigma}$ zu $\tilde{\sigma}_l$, $\chi_l$ wie in \jlink{Lem 4.12}. Sei $B\in \Borel(\partial D)$. Dann gilt
        \[
            \tag{$*$}
            \begin{split}
                \tilde{\sigma}(B) &= \sum_{l=1}^m \int_{D_l} \underbrace{\sum_{k=1}^m \varphi_k}_{=1} \cdot \chi_l \cdot \doubleOne_B d\tilde{\sigma}_l\\
                &\overset{\text{Def}}{=} \sum_{k=1}^m \sum_{l=1}^n \int_{W_l} \underbrace{\varphi_k(G_l(s))}_{=0\text{, da }\sigma_l(s)\notin V_k} \cdot \chi_l(G_l(s)) \cdot \doubleOne_B(G_l(s)) \cdot \sqrt{g_{G_l}(s)} ds
            \end{split}
        \]
        Betrachte ein Paar $k,l$, sodass $D_l\cap V_k = G_l(W_l)\cap F_k(U_k) \ne \emptyset$. Setze $W_{kl}:= P\kappa_l(D_l\cap V_k), \ D(t,0) = t$ für $t\in\R^{d-1}$ und $\phi_{kl} = \jabb{P_{\psi_k}G_l}{W_{kl}}{\R^{d-1}}$. Wie in \jlink{Bem 4.5} sieht man, dass $\phi_{kl}$ injektiv ist und $\phi_{kl}\in C^1(W_{kl}, \R^{d-1})$ als Komposition solcher Abbildungen. Ferner ist $\phi_{kl}'(s)$ invertierbar (vgl. \jlink{Bem 4.5}) und $F_k(\phi_{kl}(s)) = G_l(s)$ $(\forall s \in W_{kl})$. Da $W_{kl}$ offen in $\R^{d-1}$ ist, liefert der Umkehrsatz: $\jabb{\phi_{kl}}{W_{kl}}{\underbrace{\phi_{kl}(W_{kl})}_{\subset U_k}}$ ist \jlink{diffeomorph}. Damit gelten
        \[
            \begin{split}
                G_l'^T \cdot G_l' &\overunderset{\text{Ketten-}}{=}{\text{regel}} \left((F_k \circ \phi_{kl}) \phi_{kl}' \right) \cdot (F_k' \circ \phi_{kl}) \phi_{kl}'\\
                                  &=\phi_{kl}' (F_k\circ \phi_{kl})^T\cdot (F_k' \circ \phi_{kl}) \phi_{kl}'
            \end{split}
        \]
        und
        \[
            \begin{split}
                g_{G_l} &= \det G_l'^T \cdot G_l \overset{\text{LA}}{=} \det (\phi_{kl}'^T)\cdot \det \left((F_k' \circ \phi_{kl})^T \cdot (F_k' \circ \phi_{kl}) \right) \cdot \det(\phi_{kl}')\\
                        &\overset{\text{LA}}{=} \left(\det \phi_{kl} \right)^2 \cdot g_{F_k}\circ \phi_{kl}
            \end{split}
        \]
        auf $W_{kl}$. (Zur Übersicht wurde $s\in W_{kl}$ weggelassen). Dann gilt
        \[
            \begin{split}
                \tilde{\sigma}(B)
                &\begin{split}
                    \overset{(*)}{=}\sum_{k=1}^m \sum_{l=1}^n \int_{W_{kl}}&\underbrace{\varphi_k(F(_k(\phi_{kl}(s)))}_{=0,\ s\in W_l\backslash W_{kl}} \cdot \chi_l(F_k(\phi_{kl}(s))) \cdot \doubleOne_B(F_k(\phi_{kl}(s)))\\
                &\cdot |\det \phi_{kl}'(s)| \cdot \sqrt{g_{F_k}(\phi_{kl}(s))} ds 
                \end{split}\\
                &= \sum_{k=1}^m \sum_{l=1}^n \int_{\phi_{kl}(W_{kl})} \varphi_k(F_k(t)) \cdot \chi_l(F_k(t)) \cdot \doubleOne_B(F_k(t)) \cdot \sqrt{g_{F_k}(t)} dt\\
                &=\sum_{k=1} \int_{U_k} \varphi(F_k(t)) \cdot \underbrace{\sum_{l=1}^n \chi_l(F_k(t))}_{=1} \cdot \doubleOne_B(F_k(t)) \cdot \sqrt{g_{F_k}(t)} dt = \sigma(B).
            \end{split}
        \]
        Sei $B \subset V_i$ für ein $i\in \{1,\dots,k\}$. Im Beweis von \jlink{Lem 4.11} kann man erreichen, dass $\varphi_k = 0$ auf $V_i$, $\forall k\ne i$ und $\varphi_i = \doubleOne_{V_i}$. Damit gilt
        \[
            \sigma(B) \overset{\text{Def}}{=} \sum_{k=1}^m \int_{V_k} \varphi_k\cdot \doubleOne_B d \sigma_k = \int_{V_i} \underbrace{\varphi_i}_{=1} \cdot \doubleOne_B d\sigma = \sigma_i(B).
        \]
    \end{proof}
\end{lem}

\begin{defn}
    \jlabel{Def 4.13}
    Sei $\partial D \in C^1$ wie in \jlink{(4.5)} und $\varphi_k$ wie in \jlink{Lem 4.11}. Eine \jhyperref{messbar}{messbare} Funktion $\jabb{f}{\partial D}{\Rq}$ heißt integrierbar (bzgl. $\sigma$), wenn jede der Funktionen $\jabb{f\circ F_k\cdot \sqrt{g_k}}{U_k}{\Rq}$ ($k=1,\dots,m$) integrierbar ist. Dann definiert man das \jterm{Oberflächenintegral} (O-Integral) durch
    \[
        \int_{\partial D} f d\sigma := \int_{\partial D} f(x) d\sigma(x) := \sum_{k=1}^m \int_{U_k} \varphi_k(F_k(t))\cdot f(F_k(t))\cdot \sqrt{g_k(t)} dt.
    \]
    Dieses Integral ist ferner für alle \jhyperref{messbar}{messbaren} $\jabb{f}{\partial D}{[0,\infty]}$ definiert. Man schreibt weiter $\Leb^1(\partial D, \sigma) = \{\jabb{f}{\partial D}{\R} : f \text{ integrierbar}\}$. Für $A\in \Borel(\partial D)$ gilt
    \[
        \int_A f d\sigma = \int_{\partial D} \doubleOne_A f d\sigma.
    \]
\end{defn}

\begin{satz}
    \jlabel{Satz 4.14}
    Seien $\partial D \in C^1$, $\jabb{f,g}{\partial D}{\R}$ integrierbar oder \jlink{messbar} und positiv. Dann gelten:
    \begin{enumerate}
        \item
            \jlabel{Satz 4.14a)}
            Das Oberflächenintegral hängt nicht von der Wahl der $(\psi_k, \tilde{V}_k)$ in \jlink{(4.5)} und den $\varphi_k$ in \jlink{Lem 4.11} ab.
        \item
            \jlabel{Satz 4.14b)}
            Wenn $f=0$ auf $\partial D \backslash V_k$ für ein $k\in \{1,\dots,m\}$. Dann gilt
            \[
                \tag{vgl. \jlink{(4.3)}}
                \int_{\partial D} f d\sigma = \int_{V_k} f d \sigma_k.
            \]

        \item
            \jlabel{Satz 4.14c)}
            Wenn $f,g$ reellwertig sind und $\alpha,\beta \in \R$, dann ist $\alpha\cdot f + \beta \cdot g$ integrierbar und es gilt
            \[
                \int_{\partial D} (\alpha\cdot f + \beta\cdot g) d\sigma = \alpha\cdot \int_{\partial D} f d\sigma + \beta \cdot \int_{\partial D} g d\sigma.
            \]
            Diese Gleichheit gilt auch für \jhyperref{messbar}{messbare} $\jabb{f,g}{\partial}{[0,\infty]}$, $\alpha,\beta >0$.
        \item
            \jlabel{Satz 4.14d)}
            $f\le g \Rightarrow \int_{\partial D} f d\sigma \le \int_{\partial D} g d\sigma$.
        \item
            \jlabel{Satz 4.14e)}
            $\left | \int_{\partial D} f d\sigma \right | \le \int_{\partial D} |f|d\sigma$.
        \item
            \jlabel{Satz 4.14f)}
            Sei $\jabb{h}{\partial D}{\R}$ \jlink{messbar} und beschränkt. Dann ist $h$ integrierbar und
            \[
                \left | \int_{\partial D} h d\sigma \right | \le \lVert h \rVert_\infty \sigma(\partial D).
            \]

        \item
            \jlabel{Satz 4.14g)}
            Wenn $\partial D = A \dcup B$ für disjunkte $A,B \in \Borel(\partial D)$, dann gilt
            \[
                \int_{\partial D} f d\sigma = \int_A f d\sigma + \int_B f d\sigma.
            \]

        \item
            \jlabel{Satz 4.14h)}
            Wenn $B\subset F_k(N)$ für ein $k\in \{1,\dots,m\}$ und eine $(d-1)$-dimensionale Nullmenge $N\subset U_k$. Dann gilt $\int_B f d\sigma = 0$.
        \item
            \jlabel{Satz 4.14i)}
            Die Sätze von der monotonen Konvergenz (\jlink{Thm 2.19}) und der majorisierten Konvergenz (\jlink{Thm 3.10}) gelten für das Oberflächenintegral entsprechend.
    \end{enumerate}
    Entsprechende Aussagen gelten auch für das Integral in \jlink{(4.3)} für eine \jlink{Parametrisierung} $\jabb{F}{U}{V}$.
    \begin{proof}
        \begin{enumerate}
            \item Wie in \jlink{Lem 4.12}.
            \item Wie in \jlink{Lem 4.12}.
            \item
                Nach Voraussetzung sind $(f\circ F_k)\cdot \sqrt{g_k}$, $(g\circ F_k)\cdot \sqrt{g_k}$ integrierbar auf $U_k$, $k=1,\dots,m$, falls $f,g$ integrierbar sind. Mit \jlink{Satz 2.25} folgt dann $\jabb{(\alpha\cdot f + \beta\cdot g)\circ F_k\cdot \sqrt{g_k}}{U_k}{\Rq}$ ist integrierbar. Daraus folgt dann, dass $\jabb{\alpha\cdot f + \beta\cdot g}{\partial D}{\Rq}$ integrierbar ist. Ferner gilt
                \[
                    \begin{split}
                        \int_{\partial D} (\alpha\cdot f + \beta\cdot g) d\sigma &\overset{\text{Def}}{=} \sum_{k=1}^m \int_{U_k} \varphi_k\cdot(\alpha\cdot f + \beta\cdot g)\circ F_k \cdot \sqrt{g_k} dt\\
                        &\begin{split} 
                            \overset{\jshortlink{Satz 2.25}}{=}\sum_{k=1}^m (&\alpha\cdot \int_{U_k}(\varphi_k\cdot f)\circ F_k\cdot \sqrt{g_k}dt\\
                            + &\beta\cdot \int_{U_k}(\varphi_k\cdot g)\circ F_k\cdot \sqrt{g_k} dt  )
                        \end{split}\\
                        &=\alpha\cdot \int_{\partial D} f d\sigma + \beta \cdot \int_{\partial D} g d\sigma.
                    \end{split}
                \]
            \item Geht analog zu c) unter Verwendung von Kapitel 2.
            \item Geht analog zu c) unter Verwendung von Kapitel 2.
            
\jdate{19.01.2009}

            \item
                Sei $h$ \jlink{messbar} und beschränkt. Dann gilt $|h\circ F_k|\cdot \sqrt{g_k} \le \lVert h \rVert_\infty \cdot \sqrt{g_k}$, wobei wegen \jlink{Def 4.1} $\lVert h \rVert_\infty \cdot \sqrt{g_k}$ integrierbar ist, d.h. $\jabb{h}{\partial D}{\Rq}$ ist integrierbar. Ferner gilt
                \[
                    \left| \int_{\partial D} h d\sigma \right| \overset{\text{e)}}{\le} \int_{\partial D} \underbrace{|h|}_{\le \lVert h \rVert_\infty \cdot \doubleOne_{\partial D}} \overset{\text{d)}}{\le} \int_{\partial D} \lVert h \rVert_\infty d \sigma = \lVert h \rVert_\infty \cdot \sigma(\partial D).
                \]
                
            \item
                Sei $A \dcup B = \partial D$. Dann gilt
                \[
                    \begin{split}
                        \int_{\partial D} f d \sigma &= \int_{\partial D} \left( \doubleOne_A + \doubleOne_B \right)\cdot f d \sigma \overset{\text{c)}}{=} \int_{\partial D} \doubleOne_A \cdot f d\sigma + \int_{\partial D} \doubleOne_B \cdot f d\sigma\\
                        &\overset{\text{Def}}{=} \int_A f d\sigma + \int_B f d\sigma.
                    \end{split}
                \]

            \item
                Sei $N$ eine $(d-1)$-dimensionale Nullmenge mit $B\subset F_k(N)$ und $N\subset U_k$ für ein $k\in \{1,\dots,m\}$. Dann folgt $\doubleOne_B \le \doubleOne_{F_k(N)}$. Damit gilt
                \[
                    \left| \int_B f d\sigma \right| \overset{\text{b)}}{=} \left|\int_{V_k} \doubleOne_B\cdot f d\sigma \right| \overunderset{\text{e)}}{\le}{\text{d)}} \int_{U_k} \underbrace{\doubleOne_{F_k(N)}(F_k(t))}_{=\doubleOne_N(t)} \cdot |f(F_k(t))|\cdot \sqrt{g_k(t)} dt = 0,
                \]
                da der Integrand fast überall den Wert 0 hat ($t\in N$).
                
            \item
                Sei $f_n \xrightarrow{n\rightarrow \infty} f$ punktweise mit $|f_n| \le g$ ($\forall n\in\N$) für integrierbare $\jabb{f_n}{\partial D}{\Rq}$, $\jabb{g}{\partial D}{[0,\infty]}$. Setze $h_{n,k}:= (\varphi \circ F_k)\cdot(f_n \circ F_k)\cdot\sqrt{g_k}$. Dann ist $|h_{n,k}| \le |g \circ F_k| \sqrt{g_k} = g\circ F_k \cdot \sqrt{g_k}$ integrierbar. Außerdem gilt $h_{n,k} \xrightarrow{n\rightarrow \infty} h_k := (\varphi_k \circ F_k)\cdot(f \circ F_k) \cdot \sqrt{g_k}$ punktweise. Mit \jlink{Thm 3.10} folgt dann $\int_{U_k} h_{n,k} dt \xrightarrow{n\rightarrow\infty} \int_{U_k} h_k dt$. Summenbildung für $k=1,\dots,m$ liefert
                \[
                    \int_{\partial D} f_n d\sigma \xrightarrow{n \rightarrow \infty} \int_{\partial D} f d\sigma.
                \]
                Den Satz von der monotonen Konvergenz (\jlink{Thm 2.19}) zeigt man analog.
        \end{enumerate}
    \end{proof}
\end{satz}

\jlabel{Bsp 4.15}
\begin{expl}
    \begin{enumerate}
        \item
            \jlabel{Bsp 4.15a)}
            Sei $D=B(0,R) \subset \R^d$, $H_d = \R_+ \times \{0\} \times \R^{d-2}$ für $d\ge 2$. Mit \jlink{Bsp 4.9} folgt $\partial D\backslash H_d = F(W_d)$ mit $W_d = (0,2\pi)\times (-\frac{\pi}{2}, \frac{\pi}{2})^{d-2}$, wobei $F(\varphi, \Theta) = \phi_d(R,\varphi, \Theta)$ (Polarkoordinaten).\\
            Setze $\hat{F}(\varphi,\Theta) := Q\cdot F(\varphi,\Theta)$ für $(\varphi,\Theta) \in W_d$,\\ $Q(x_1,\dots,x_d) := (-x_1,x_d,x_3,\dots,x_{d-1},x_2)$ $(d\ge 3)$. Dann folgt $\hat{F}(W_d) = \partial B(0,R) \backslash (\R_-\times \R^d\times \{0\})$. Daraus folgt $\partial B(0,R) \subset F(W_d) \cup \hat{F}(W_d)$ und $\partial B(0,R)\cap H_d = \hat{F}(N)$ mit $N = \left(\frac{\pi}{2}, \frac{3\pi}{2} \right) \times \left(-\frac{\pi}{2},\frac{\pi}{2}  \right)^{d-3} \times \{0\}$, da $F(N) = \{x\in\R^d : x_1 < 0, x_d = 0\}$. Damit gilt
            \[
                \begin{split}
                    \sigma(\partial B(0,R)) &\overset{\sigma \text{ ist Maß}}{=} \sigma(\partial D(0,R)\backslash H_d) + \underbrace{\sigma(\partial B(0,R)\cap H_d)}_{=0}\\
                    &\overset{\jshortlink{Satz 4.14}}{=} \int_{W_d} 1 \cdot \sqrt{G_F(\varphi, \Theta)} d(\varphi, \Theta)\\
                    &\overunderset{\jshortlinkFubini}{=}{\jshortlink{Bsp 4.9}} \int_0^{2\pi} \int_{\left(-\frac{\pi}{2},\frac{\pi}{2} \right)^{d-2}} R^{d-1}\cdot \cos(\Theta_1)\cdots \cos(\Theta_{d-2}) d\Theta d\varphi\\
                    &\overset{\jshortlink{Bsp 3.35}}{=} R^{d-1} \frac{2\pi^\frac{d}{2}}{\Gamma\left(\frac{d}{2}\right)} \overset{\jshortlink{(3.12)}}{=} R^{d-1}\cdot \omega_d,
                 \end{split}
            \]
            wobei $\omega_2 = 2\pi,\ \omega_3 = 4\pi$.
            
        \item
            \jlabel{Bsp 4.15b)}
            Sei $f(x,y,z) = |xyz|$, $D=B(0,R)\subset \R^3$. Setze $J:=\int_{\partial D} f d\sigma$. Seien $O_j,\ j=1,\dots 8$ die offenen Oktanten, $H =$ Vereinigung der Koordinatenebenen. $\partial D = \bigdcup_{j=1}^8 (\partial D \cap O_j) \dcup (\partial D \cap H)$, wobei $O_1 = \{(x,y,z)\in \R^3 : x>0, y>0, z>0\} = F\left( (0, \frac{\pi}{2}) \times (0,\frac{\pi}{2}) \right)$, $\sigma(\partial D \cap H) = 0$, wie in a) mit \jlink{Satz 4.14h)}. Damit folgt
            \[
                \begin{split}
                    J &\overset{\jshortlink{Satz 4.14g)}}{=} \underbrace{\sum_{j=1}^8 \int_{O_j\cap \partial D} f d\sigma}_{\text{alle Integrale gleich}} + \underbrace{\int_{H \cap \partial D} f d\sigma}_{\overset{\jshortlink{Satz 4.14h)}}{=} 0} = 8\cdot \int_{O_1} xyz \ d\sigma(x,y,z)\\
                    &\overunderset{\text{Sphären-}}{=}{\text{koord.}} 8\cdot \int_0^\frac{\pi}{2} \int_0^\frac{\pi}{2} R \cos\varphi \cos\Theta \cdot R \sin\varphi \cos\Theta \cdot R \sin\Theta \cdot R^2 \cos\Theta d\Theta d\varphi\\
                    &= 8 R^5 \cdot \int_0^\frac{\pi}{2} \sin(\varphi)\cos(\varphi)d\varphi \cdot \int_0^\frac{\pi}{2}\cos^3(\Theta)\sin(\Theta)d\Theta\\
                    &= 8 R^5\cdot \frac{1}{2} \left[-\frac{1}{4} \cos^4(\Theta) \right]_0^\frac{\pi}{2} = R^5.
                \end{split}
            \]
    \end{enumerate}
\end{expl}
        
    % Die Sätze von Gauß und Stokes
\section{Die Sätze von Gauß und Stokes}

\begin{bem}
    \jlabel{Bem 4.16}
    Sei $r>0, \ x_0 \in \R^d$. Die Funktion
    \[
        \phi_{r,x_0} (x) = \begin{cases}
                               \exp\left(-\frac{1}{r^2 - |x-x_0|_2^2} \right), &x\in B(x_0,r)\\
                               0, & x\in \R^d\backslash B(x_0,r)
                           \end{cases}
    \]
    ist auf $\R^d$ beliebig oft differenzierbar (vgl. Ana1, Aufgabe 12.7). Ferner gilt
    \[
        \phi_{r,x_0}(x) > 0 \Leftrightarrow x\in B(x_0,r), \hspace{10pt} \phi_{r,x_0} =0 \Leftrightarrow x\notin B(x_0,r).
    \]
    Seien $X,Y$ normierte Vektorräume, $D\subset X$, $\jabb{f}{X}{Y}$. Der \jterm{Träger} (support) von $f$ ist
    \[
        \supp f := \{x\in X : f(x) \ne 0\}.
    \]
    Bild zu $f$ in $d=1$: TODO\\
    Sei $U\subset \R^d$ offen. Wir schreiben $f\in C^k(\overline{U}, \R^l)$, wenn $f\in C^k(U, \R^l)$ und alle partiellen Ableitungen von $f$ der Ordnung kleiner oder gleich $k$ eine stetige Fortsetzung auf $\partial U$ haben, wobei $k\in\N \cup \{\infty\}$. Wir schreiben etwa zum Beispiel
    \[
        \partial_j f(x) := \lim_{y\rightarrow x, y\in D} \partial_j f(x), \text{ in diesem Fall für } j=1,\dots,d,\ x\in \partial D.
    \]
\end{bem}

\begin{satz}
    \jlabel{Satz 4.17}
    Sei $D\subset \R^d$ offen und beschränkt und $\overline{D}\subset U_1\cup\dots\cup U_m$ für offene, beschränkte $U_k\subset \R^d$. Dann gibt es $\varphi_k \in C^\infty(\overline{D})$, sodas $0\le \phi_k \le 1$, $\supp \varphi_k \subset U_k$ und $\sum_{k=1}^m \varphi_k = 1$ ($\forall k=1,\dots,m, \ x\in \overline{D}$). (Man nennt $\{\varphi_k : k=1,\dots,m\}$ \uline{glatte Zerlegung der Eins auf $\overline{D}$ zu $U_1,\dots,U_m$})
    
    \jspace
    
    TODO: Bild in $d=1, D=(a,b)$.
     
     \begin{proof}
         $\forall x \in \overline{D}\ \exists r(x) >0, \ h\in \{1,\dots,m\}$ mit $x\in B(x,r(x))\subset \overline{B}(0,r(x))\subset U_k$. (Kugeln bezüglich der $\sup$-Norm.) Da $\overline{D}$ kompakt ist, existieren $x_j\in \overline{D}$ mit $B_j = B(x_j, r(x_j))$, $(j=1,\dots,n)$, sodass $\overline{D} \subset \bigcup_{j=1}^n B_j$ und $\forall j\ \exists: \overline{B}_j \subset U_k$. Setze
        \[
            \phi_k := \sum_{j_0 : B_j \subset U_k} \phi_{r(x_j),x_j}
        \]
        Damit gilt $\phi_k \in C^\infty(\R^d), \ k=1,\dots,m$ und $\supp \phi_k \subset \bigcup_{j: B_j \subset U_k} \overline{B}_j\subset U_k$ und $\phi_k > 0$.\\
        Ferner gilt $\forall x\in \overline{D} \ \exists B_j$ mit $x\in B_j$. Dann folgt $\phi_{r(x_j),x_j}(x) >0$. Dann folgt $\alpha(x) := \sum_{k=1}^m \phi_k(x) >0$. Damit setze $\varphi_k := \frac{1}{\alpha} \phi_k \in C^\infty(\overline{D})$. Dann gilt $\varphi_k \ge 0$, $\supp \varphi_k = \supp \phi_k \subset U_k$, $\sum_{k=1}^m \varphi_k(x) = \frac{1}{\alpha(x)} \sum_{k=1}^m \phi_k(x) = 1\ (\forall x\in \overline{D})$. Daraus folgt $\phi_k \in [0,1]$.
     \end{proof}
\end{satz}

Man setzt für offene $U\subset \R^d, \ k\in \N \cup \{0,\infty\}$
\[
    \begin{split}
        C_b^k(U,\R^l) := \{f \in C^k(U,\R^l) : &f \text{ und alle partiellen Ableitungen von } f\\
                                               &\text{bis Ordnung $k$ sind beschränkt}\}.
    \end{split}
\]

\begin{bem*}
    Sei $a<b$ in $\R$, $g\in C_b^1((a,b)\times U), \ U\subset \R^d$ offen, $h\in C^1(U, \R)$. Setze 
    \[
        G(t,x):= \int_a^b g(s,x)ds, \hspace{10pt} t\in(a,b), \ x\in U.
    \]
    Mit dem Diff-/Stetigkeitssatz und dem Hauptsatz folgt $G\in C^1((a,b)\times U, \R)$. Mit der Kettenregel gilt dann für alle $j=1,\dots,d, \ x\in U$
    \begin{equation}
        \jlabel{(4.6)}
        \begin{split}
            \exists \frac{\partial}{\partial x_j} \int_a^{h(x)} g(s,x)ds &= \frac{\partial}{\partial x_j} G(h(x),x)\\
            &= g(h(x),x)\partial_j h(x) + \int_a^{h(x)} \frac{\partial}{\partial x_j} g(s,x) ds
        \end{split}
    \end{equation}


\jdate{23.01.2009}


    Sei $g\in C([a,b])$ mit $g|_{(a,b)} \in C^1((a,b))$ und $g' \in C^1((a,b))$. Dann gilt
    \begin{equation}
        \jlabel{(4.7)}
        \int_a^b g'(t)dt \overset{\text{vgl. \jshortlink{Kor 3.15}}}{=} \lim_{\epsilon \rightarrow 0} \int_{a+\epsilon}^{b-\epsilon} g'(t) dt \overset{\text{HS}}{=} \lim_{\epsilon \rightarrow 0} \left(g(b-\epsilon)-g(a+\epsilon) \right).
    \end{equation}
    Seien $U\subset \R^d$ offen, $f,g \in C^1(U,\R^d)$. Die \jterm{Divergenz} von $f$ ist folgendermaßen definiert:
    \begin{equation}
        \jlabel{(4.8)}
        \div f = \partial_1f_1(x) + \partial_2 f_2(x) + \dots + \partial_d f_d(x) = \Spur(f'(x)), \ x\in U.
    \end{equation}
    Beachte dabei $\div (f+g) = \div f + \div g$.
\end{bem*}

\begin{thm}[Divergenzsatz von Gauß]
    \jlabel{Thm 4.18}
    \jlabel{Gauss}
    Sei $D\subset \R^d$ offen und beschränkt mit $\partial D \in C^1$, $f\in C(\overline{D}, \R^d)$ mit $f|_D \in C_b^1(D, \R^d)$. Dann gilt
    \[
        \int_D \div f(x) dx = \int_{\partial D} (f(x)|\nu(x)) d\sigma(x),
    \]
    wobei $\nu$ die äußere Einheitsnormale von $D$ ist (vgl. \jlink{Lem 4.7}).\\
    (Eine allgemeinere Version findet man im Königsberger Ana2, §12.4)
    
    \begin{proof}
        \begin{itemize}
            \item[1)]
                Aus \jlink{Lem 4.7} folgt: $\forall x\in \partial D \ \exists$ eine Karte $(\psi,\tilde{V})$, sodass (eventuell nach orthogonaler Transformation $y = Qx$) $D$ lokal bei $x$ unter einem Graph liegt. Da $\partial D$ kompakt ist, exisiteren offene $V_1,\dots,V_m\subset \R^d$ mit $\partial D\subset V_1\cup \dots \cup V_m$ (die Tilde wurde weggelassen), orthogonale Matrizen $Q_1,\dots,Q_m, \ a_1,\dots,a_m \in \R$ und offene $U_1,\dots,U_m\subset \R^{d-1}$, $h_1,\dots,h_m\in C^1(U_k, \R)$ mit beschränkter Ableitung, sodass
                \[
                    Q_k(V_k\cap D) = \{y = (\underbrace{y_1,\dots,y_{d-1}}_{=: y'}, y_d) \in \R^d : y'\in U_k, \ y_d\in (a_k, h_k(y'))\}
                \]
                und
                \[
                    Q_k(\partial D \cap V_k) = \{(y', y_d): y' \in U_k, \ y_d = h_k(y')\} \hspace{10pt} (k=1,\dots,m).
                \]
                Wähle $V_0\subset \R^d$ offen mit $\overline{V}_0 \subset D$ und $D\subset V_0 \cup \dots \cup V_m$.
                
                \jspace
                
                TODO: Bild
                
                \jspace
                
                Wähle $\varphi_k$ wie in \jlink{Lem 4.7} auf $\overline{D}$ zu $\{V_0,V_1,\dots,V_m\}$. Setze $f^k := \varphi_k f$. Dann gelten $f^k \in C_b^1(D,\R^d)\cap C(\overline{D}, \R^d)$, $\supp f^k \subset V_k\cap \overline{D}$ ($k=1,\dots,m$) und
                \[
                    \sum_{k=0}^m f^k(x) = \underbrace{\sum_{k=0}^m \varphi_k}_{=1} \cdot f(x) = f(x) \hspace{15pt} \forall x\in \overline{D}.
                \]
                Beachte dabei, dass $f^k(x) = 0 \ \forall x \in \partial V_k \cap D$, aber dass $f^k(x) \ne 0$ möglich ist, wenn $x\in\partial N_k\cap \partial D$.
                
                \jspace
                
                TODO: Bild
                
            \item[2)]
                Sei $W\subset \R^n$ offen, $g\in C^1(W,\R)$ mit $\supp g \subset W$, $\supp g$ kompakt. Sei $j\in \{1,\dots,n\}$. Setze $g$ mit $0$ zu $\tilde{g} \in C^1(\R^n,\R)$ fort. Dann gilt
                \[
                    \begin{split}
                        &\int_W \partial_j g(x) dx = \int_{\R^n} \partial_j \tilde{g}(x) dx =\\
                        &\overset{\jshortlinkFubini}{=} \int_{\R^{n-1}} \int_\R \partial_j \tilde{g}(x_1,\dots,x_j,\dots,x_n) dx_j d(x_1,\dots,x_{j-1},x_{j+1},\dots,x_n)\\
                        &=\int_{-r}^r \partial_j\tilde{g}(x_1,\dots,x_j,\dots,x_n) dx_j = \left[\tilde{g} \right]_{x_j=-r}^{x_j=r} = 0,
                    \end{split}
                \]
                wobei $r$ so gewählt ist, dass $\supp g \subset B(0,r)$. Also gilt
                \[
                    \int_W \partial_j g dx = 0.
                \]
                Speziell gilt für $k=0$
                \[
                    \int_D \partial_j f_j^0(x) dx = 0 \hspace{10pt} \forall j=1,\dots,n,
                \]
                denn $\supp f^0 \subset V_0 \subset D$. Daraus folgt
                \[
                    \int_D \div f^0(x) dx = 0 = \int_{\partial D} (\underbrace{f^0(x)}_{=0}|\nu(x))d\sigma(x).
                \]
        
            \item[3)]
                Sei $k\in \{1,\dots,m\}$. Setze $\tilde{f}^k(y) := Q_k f^k(Q_k^{-1}y)$ für alle $y = Q_k x, \ x\in D$. Sei $\tilde{\nu}$ die äußere Einheitsnormale von $Q_k D$. Es gilt
                \[
                    \begin{split}
                        \div \tilde{f}^k(y) &= \Spur \left[Q_k \cdot (f^k\cdot Q_k^{-1})'(y) \right] = \Spur \left[ Q_k (f^k)'(x) Q_k^{-1} \right]\\
                        &\overset{\text{LA}}{=} \Spur (f^k)'(x) = \div f^k(x) \hspace{10pt} (\forall x\in D).
                    \end{split}
                \]
                Damit gilt
                \[
                    \begin{split}
                        \int_D \div f^k(x) dx &\overunderset{\jshortlink{Trafo}}{=}{x = Q_k^{-1}y} \int_{Q_k D} \div \tilde{f}^k(y) \cdot \underbrace{|\det Q_k^{-1}|}_{=1} dy\\
                        &\overunderset{!}{=}{(+)} \int_{\partial Q_k D} \left(\tilde{f}^k(y) | \tilde{\nu}(y) \right) d\sigma(y)\\
                        &\overunderset{\jshortlink{Trafo}}{=}{y = Q_k x} \int_{\partial D} (\tilde{f}^k(Q_k x)| \underbrace{\tilde{\nu}(Q_k x)}_{=Q_k \nu(x)} ) \cdot \underbrace{|\det Q_k|}_{=1} d\sigma(x)\\
                        &= \int_{\partial D} (Q_k f^k(x)| Q_k \nu(x))d\sigma(x)\\
                        &\overset{Q_k^T = Q_k^{-1}}{=} \int_{\partial D} (f^k(x)| \nu(x))d\sigma(x).
                    \end{split}
                \]
                Zeige nun das Gleichheitszeichen $(+)$. Es gilt $f=\tilde{f}$ auf $Q_k D$. Schreibe dann $f^k$ statt $\tilde{f}^k$, $D$ statt $Q_k D$, $x$ statt $y$, $\nu$ statt $\tilde{\nu}$.
                \[
                    \tag{$*$}
                    \begin{split}
                        \int_D \partial_d \cdot f_d^k(x) dx &\overunderset{\jlinkFubini}{=}{\supp f^k \subset V_k \cap \overline{D}} \int_{U_k} \underbrace{\int_{a_k}^{h_k(x')} \partial_d f_d^k(x', x_d) dx_d dx'}_{\overset{(4.7)}{=} f_d^k(x',h_k(x')) \cdot \underbrace{f_d(x', a_k)}_{=0}}\\
                        &=\int_{U_k} f_d^k(x', h_k(x')) dx'.
                    \end{split}
                \]
                Sei $j\in \{1,\dots, d-1\}$. Dann gilt
                \[
                    \begin{split}
                        \partial_j \cdot \underbrace{\int_{a_k}^{h_k(x')} f_j^k(x', x_d) dx_d}_{=\varphi(x')} \overset{(4.6)}{=} &f_j^k(x',h_k(x')) \partial_j h_k(x')\\
                        &+ \int_{a_k}^{h_k(x')} \partial_j f_j^k(x',x_d) dx_d.
                    \end{split}
                \]
                Damit folgt dann
                \[
                    \tag{$**$}
                    \begin{split}
                        &\int_D \partial_j f_j^k(x) dx \overset{\jshortlinkFubini}{=} \int_{U_k}\int_{a_k}^{h_k(x')} \partial_j f_j^k(x',x_d) dx_d dx'\\
                        &\overset{\text{s.o.}}{=} \int_{U_k} f_j^k(x',h_k(x')) \partial_j h_k(x') dx' + \underbrace{\int_{U_k} \partial_j \varphi(x') dx'}_{\overset{2)}{=}0, \text{ da }\supp \varphi \subset U_k}
                    \end{split}
                \]
                Durch Aufsummieren über $j$ ergibt sich dann
                \[
                    \begin{split}
                        &\int_D \div f^k(x) dx\\
                        &\overunderset{(*)}{=}{(**)} \int_{U_k} \left(f^k(x', h_k(x')) | \begin{pmatrix}
                                                                                                                  -\triangledown h_k(x')\\
                                                                                                                  1
                                                                                                              \end{pmatrix} \right)
                            \cdot \frac{\sqrt{1 + |\triangledown h_k(x')|_2^2}}{\sqrt{1 + |\triangledown h_k(x')|_2^2}} dx'\\
                            &\overunderset{\jshortlink{Lem 4.7}}{=}{\jshortlink{Bsp 4.9}} \int_{\partial D} (f^k(x)|\nu(x)) d\sigma(x).
                    \end{split}
                \]
            
            \item[4)]
                \[
                    \begin{split}
                        \int_D \div f dx &= \int_D \div \left(\sum_{k=0}^m f^k \right) dx \overset{(4.8)}{=} \sum_{k=0}^m \int_D \div f^k dx\\
                                         &\overunderset{2)}{=}{3)} \sum_{k=0}^m \int_{\partial D} (f^k(x)|\nu(x)) d\sigma(x) = \int_{\partial D} (f(x)| \nu(x)) d\sigma(x).
                    \end{split}
                \]
        \end{itemize}
    \end{proof}
\end{thm}


\uline{Erinnerung}: für $u\in C^2(U,\R)$ ist der Laplace-Operator gegeben durch
\[
    \triangle u(x) = \partial_{11} u(x) + \dots + \partial_{dd} u(x) = \Spur \triangledown^2 u(x).
\]
Damit gilt
\begin{equation}
    \jlabel{(4.9)}
    \triangle u = \div \begin{pmatrix}
                           \partial_1 u\\
                           \vdots\\
                           \partial_d u
                       \end{pmatrix} = \div \triangledown u
\end{equation}


\begin{kor}[Greensche Formeln]
    \jlabel{Kor 4.19}
    \jlabel{Green}
    Sei $D\subset \R^d$ offen und beschränkt mit $\partial D\in C^1$.
    \begin{enumerate}
        \item 
            \jlabel{Kor 4.19.a)}
            Sei $f\in C_b^1(D,\R^d)\cap C(\overline{D}, \R^d)$, $u \in C_b^1(D,\R)\cap C(\overline{D},\R)$. Dann gilt
            \[
                \begin{split}
                    \div (u(x)\cdot f(x)) &= \sum_{j=1}^d \left((\partial_j u)(x)\cdot f_j(x) + u(x)\cdot \partial_j f_j(x) \right)\\
                                          &= \left(\triangledown u(x)|f(x) \right) + u(x)\cdot \div f(x).
                \end{split}
            \]
            Dann folgt mit \jlink{Gauss}
            \[
                \int_D (\triangledown u(x)|f(x)) dx = -\int_D u(x) \div f(x) dx + \int_{\partial_D} u(x)\cdot (f(x)|\nu(x))d\sigma(x).
            \]

        \item
            \jlabel{Kor 4.19.b)}
            Seien $u,v\in C_b^2(D,\R) \cap C^1(\overline{D},\R)$. Dann gilt
            \[
                \begin{split}
                    \int_D u\cdot\triangle v dx &\overunderset{a)}{=}{\jshortlink{(4.9)}} -\int_D (\triangledown u | \triangledown v) dx + \int_D u(x)\underbrace{(\triangledown v(x)| \nu(x))}_{=\partial_\nu v(x)} d\sigma(x)\\
                    &=\int_D v \triangle u dx + \int_{\partial D} u(x)\partial_\nu v(x) - v(x)\partial_\nu u(x) d\sigma(x).
                \end{split}
            \]
    \end{enumerate}
\end{kor}

\jdate{26.01.2009}

\subsubsection{Zur Intepretation vom Divergenzsatz von Gauß}
Seien $D,f$ wie in \jlink{Thm 4.18}. Dabei entspreche $f$ einem elektrischen Feld. Setze den Durchfluss von $f$ durch $\partial D$
\[
    \phi := \int_{\partial D} (f|\nu) d\sigma.
\]
Zu $\nu$: TODO: Bild\\
Mit \jlink{Gauss} folgt
\[
    \tag{$*$}
    \phi = \int_D \div f dx.
\]
Also entspricht $\div f$ einer Quellstärke.

\begin{expl}
    \jlabel{Bsp 4.20}
    Sei $d=3,\ q\in \R,\ p \in \R^3,\ D\subset \R^3$ offen und beschränkt mit $\partial D\in C^1$. Setze für $x\in\R^3\backslash \{p\}$
    \[
        f(x) := q\frac{1}{|x-p|_2^3}(x-p).
    \]
    Dann folgt für alle $x \ne p, \ k=1,2,3$
    \[
        \partial_k f_k(x) = q\left(\sum_{k=1}^3 (x_k - p_k)^2 \right)^\frac{3}{2} + q\cdot \left(-\frac{3}{2} \right)\cdot 2(x_k-p_k)\cdot |x-p|_2^{-5}(x_k-p_k).
    \]
    Damit ergibt sich für die \jlink{Divergenz} von $f$ für alle $x\ne p$
    \[
        \div f(x) = 3q|x-p|_2^{-3} - 3q|x-p|_2^{-5}\cdot |x-p|_2^{2} = 0.
    \]

    \uline{1. Fall: $p\notin \overline{D}$}. Dann gilt $f\in C^1(\overline{D}, \R^3)$. Mit $(*)$ folgt dann $\int_D\div f dx = 0$.\\
    \uline{2. Fall: $p \in D$}. Dann ist $f\notin C^1(D,\R^3)$. Wähle $r>0$ mit $\overline{B}(p,r)\subset D$. Setze $D_r = \partial D\backslash \overline{B}(p,r)$. Daraus folgt $\partial D = \partial D \dcup \partial B(p,r)$ und damit $f\in C^1(\overline{D}_r, \R^3)$. Somit gilt
    \[
        \phi_r := \int_{\partial D} (f|\nu)d\sigma \overset{\jshortlink{Gauss}}{=} \int_{D_r} \underbrace{\div f}_{=0}dx = 0.
    \]
    Die äußere Einheitsnormale $\nu_B$ von $B(p,r)$ ist $\nu_B = \frac{1}{r}(x-p)$ ($x\in \partial B(p,r)$).
    Daraus folgt, dass die äußere Einheitsnormale $\nu$ von $D_r$ bei $x \in \partial B(p,r)$ $\nu(x) = -\nu_B(x)$ ist. Damit folgt
    \[
        0 = \phi_r \int_{\partial D} (f|\nu)d\sigma = \int_{\partial B(p,r)} \left(f(x)|\frac{1}{r} (x-p) \right)d\sigma(x).
    \]
    Daraus folgt
    \[
        \phi = \frac{q}{r} \cdot \int_{\partial B(p,r)} \underbrace{|x-p|_2^{2-3}}_{= \frac{1}{r}} d\sigma(x) = \frac{q}{r^2} \cdot \int_{\partial B(p,r)} d\sigma = \frac{q}{r^2} \cdot 4\pi r^2 = 4\pi q.
    \]
\end{expl}


\begin{expl}
    \jlabel{Bsp 4.21}
    Sei $C\subset \R^3$ offen und beschränkt mit $\partial D\in C^1$. Sei $u(t,x)$ die Konzentration eines Stoffes bei $x\in D$ zur Zeit $t\ge 0$. Es sei $u\in C^1(\R_+ \times \overline{D})$. Dann existiert $\frac{\partial^2 u}{\partial x_k \partial x_l} \in C_b(\R_+\times D)$ $(k,l=1,2,3)$. Der Stoff diffundiere gemäß des ``Fickschen Gesetz'' mit konstanter Diffusionsrate $a>0$. Gegeben sei weiter eine Anfangskonzentration $u_0 \in C^1(\overline{D}) \cap C_b^2(D)$. Durch $\partial D$ fließe keine Substanz. Insbesondere sei $\partial_\nu u_0(x) = 0, \ x\in \partial D$. Wir betrachten die \jterm{Diffusionsgleichung}
    \begin{equation}
        \jlabel{(4.10)}
        \begin{cases}
            \partial_t u(t,x) = a\triangle_x u(t,x), &t\ge 0\\
            \partial_\nu u(t,x) = 0, &t\ge 0,\ x\in \partial D\\
            u(0,x) = u_0(x), &x\in D.
        \end{cases}
    \end{equation}
    \uline{Behauptung}:
    \[
        \int_D |u(t,x)|^2 dx \le \int_D |u_0(x)|^2, \hspace{10pt} \forall t\ge 0.
    \]
    Wenn die Behauptung gilt, gibt es höchstens eine Lösung von \jlink{(4.10)}, denn:\\
    Sei $v$ eine weitere Lösung. Setze $w=u-v \in C^1(\R_+\times \overline{D})$. Dann folgt $\frac{\partial^2 w}{\partial x_k \partial X_l} \in C_b(\R_+\times D)$ und $w$ erfüllt \jlink{(4.10)} mit $w(0) = 0$. Mit der Behauptung folgt dann
    \[
        \int_D |w(t,x)|^2 dx \le 0.
    \]
    Also ist $w(t,x)=0$ $\forall t\ge0, \fa x$. Da $w$ stetig ist, folgt $w(t,x)=0$ $\forall t,x$. Mit \jlink{Bem 5.9} folgt dann $u=v$.\\
    \uline{Beweis von der Behauptung}:
    \[
        \begin{split}
            &\frac{\partial}{\partial t} \int_D \frac{1}{2}|u(t,x)|^2 dx\\
            &\overset{\jshortlink{Thm 3.16}}{=} \int_D \frac{1}{2}\frac{\partial}{\partial_t} (u(t,x))^2 dx = \int_D u(t,x)\partial_t u(t,x) dx\\
            &\overset{\jshortlink{(4.10)}}{=} a\cdot \int_D u(t,x) \triangle_x u(t,x) dx\\
            &\overset{\jshortlink{Green}}{=} -a\cdot \int_D \underbrace{(\triangledown_x u(t,x)|\triangledown_x u(t,x))}_{=|\triangledown u(t,x)|_2^2}dx + a \cdot \int_{\partial D} u(t,x) \underbrace{\partial_\nu u(t,x)}_{\overset{\jlink{(4.10)}}{=}} d\sigma(x) \le 0
        \end{split}
    \]

\end{expl}

\subsubsection{Stokes}

Sei $d=3$, $\jabb{F}{U_0}{V_0} \subset \R^3$ eine \jlink{Parametrisierung} mit $F\in C^2(U_0,\R^3)$, $U\subset \overline{U} \subset U_0 \subset \R^2$. Dabei seien $U$ und $U_0$ offen und beschränkt. Setze außerdem $V:= F(U)$. Sei $\partial U\in C^1$ eine geschlossene $C^1$-Kurve mit \jlink{Parametrisierung} $\jabb{\gamma}{[a,b]}{\partial U}$, die im Gegenuhrzeigersinn läuft. Damit ist $\gamma'(t) \ne 0,\ \forall t$. Dann folgt, dass $\partial V$ eine Kurve mit \jlink{Parametrisierung} $\varphi = \jabb{F \circ \gamma}{[a,b]}{\partial V}$

\jspace

TODO: Bild

\jspace

Dabei ist $\nu(\tau) = \frac{1}{|\gamma'(\tau)|_2}\begin{jsmallmatrix}
                                                      \gamma_2'(\tau)\\ -\gamma_1'(\tau)
                                                  \end{jsmallmatrix}$ äußere Einheitsnormale von $\partial U$ ($\gamma'(\tau)$ um $\frac{\pi}{2}$ nach rechts gedreht und normiert. Vgl. Walter Ana2, §5.17)
Mit der Bemerkung nach \jlink{Def 4.8} folgt, dass die Normale an $\partial V$ gegeben ist durch
\begin{equation}
    \jlabel{(4.11)}
    n(F(t)) = \frac{1}{|\partial_1 F(t) \times \partial_2 F(t)|_2}\cdot \partial_1 F(t) \times \partial_2 F(t).
\end{equation}

\uline{Erinnerung}
\[
    \rot f(x) = \begin{pmatrix}
                    \partial_2 f_3(x) - \partial_3 f_2(x)\\
                    \partial_3 f_1(x) - \partial_1 f_3(x)\\
                    \partial_1 f_2(x) - \partial_2 f_1(x)
                \end{pmatrix} = \triangledown \times f(x).
\]

Das Kurvenintegral zweiter Art ist definiert durch
\[
    \int_{\partial V} f \bullet dx = \int_a^b (f(\varphi(\tau))|\varphi'(\tau)) d\tau.
\]

\begin{thm}[Stokes]
    \jlabel{Thm 4.22}
    \jlabel{Stokes}
    $V$ erfülle die obigen Voraussetzungen und es sei $f \in C^1(D,\R^3)$ für ein offenes $D\subset \R^3$ mit $\overline{V} \subset D$. Dann gilt
    \[
        \int_V (\rot f(x)|n(x))d\sigma(x) = \int_{\partial V} f \bullet dx.
    \]
    \begin{proof}[Zum Beweis]
        Der Beweis erfolgt duch direktes Nachrechnen unter Verwendung obiger Formeln, Kettenregel und Gauß (\jlink{Thm 4.18}) für $\partial U$ vgl. Walter Ana2, §8.12.
    \end{proof}
\end{thm}

\uline{Zur Interpretation}: Sei $f$ ein elektrisches Feld. Dann entspricht $\int_{\partial V} f \bullet dx$ der Zirkulation in der Leiterschleife $\partial V$. \jlink{Stokes} sagt dann
\[
    \int_{\partial V} f \bullet dx = \int_V (\rot f | u)d\sigma.
\]

\jlabel{Bsp 4.23}
\begin{expl}
    \begin{enumerate}
        \item 
            \jlabel{Bsp 4.23a)}
            Wenn $u \in C^2(D, \R)$ und $f = \triangledown u = (\partial_1 u, \partial_2 u, \partial_3 u)^T$, folgt mit Schwarz (Ana2), dass $\rot f = \rot \triangledown u = 0$. ($f$ ist ein Radialfeld). Dann folgt
            \[
                \int_{\partial V} f \bullet dx \overset{\jshortlink{Stokes}}{=} 0.
            \]
            \begin{expl*}
                Sei $u = |x|_2^\alpha$ für $x\in \R^3\backslash \{0\}$, $\alpha \in \R$. Dann gilt $f(x) = \triangledown u(x) = \alpha |x|_2^{\alpha-2}x$. Also gilt $\rot f = 0$.
            \end{expl*}
            
        \item 
            \jlabel{Bsp 4.23b)}
            Einfaches Wirbelfeld: $f(x_1, x_2, x_3) = (-x_2,x_1,0)^T$. Hier gilt $\partial_2 f_1 = -1$, $\partial_1 f_2 = 1$. Alle anderen partiellen Ableitungen sind $0$. Also gilt $\rot f(x) = (0,0,2)^T$. Dann folgt
            \[
                \tag{$*$}
                \int_{\partial V} f\bullet dx \overset{\jshortlink{Stokes}}{=} \int_V (\rot f(x)| n(x))d\sigma(x) = 2\cdot \int_V n_3(x) d\sigma(x)
            \]
            \begin{expl*}
                Eine Kugelkappe lässt sich mit dem Graph von $h(x_1,x_2,x_3) = \sqrt{R^2-x_1^2-x_2^2}$ für $(x_1,x_2)\in B(0,r) \subset \R^2$, wobei $0 < r < R$ fest seien.
            \end{expl*}
            Hier gelten $F(x_1,x_2) = \begin{jsmallmatrix}x_1\\ x_2 \\ h(x_1,x_2)\end{jsmallmatrix}$ und\\
            $V = \left\{\begin{jsmallmatrix}x_1\\x_2\\x_3\end{jsmallmatrix} : \begin{jsmallmatrix}x_1\\x_2\end{jsmallmatrix} \in B(0,r) \subset \R^2, \ x_3 = h(x_1,x_2) \right\}$. Damit gilt $\partial_1 F = \begin{jsmallmatrix}1\\0\\ \partial_1 h\end{jsmallmatrix}$, $\partial_2 F = \begin{jsmallmatrix}0\\1\\ \partial_2 h\end{jsmallmatrix}$. Dann gilt $(\partial_1 F \times \partial_2 F)_3 = 1 \overset{\jshortlink{(4.11)}}{=} n_3 |\partial_1 F \times \partial_2 F|_2$. Dann folgt
            \[
                \begin{split}
                    \int_{\partial V} f \bullet dx &\overunderset{(*)}{=}{\jshortlink{Def 4.8}} 2\cdot \int_{B(0,r)} 1\cdot \frac{1}{|\partial_1 F \times \partial_2 F|_2} \cdot \underbrace{|\partial_1 F \times \partial_2 F|_2}_{=\sqrt{g_F}} dx_1 dx_2\\
                    &= 2\cdot \int_{B(0,r)} dx_1 dx_2 = 2\pi r^2.
                \end{split}
            \]
    \end{enumerate}
\end{expl}
\chapter{Lebesguesche Räume und Fourier-Reihen}
    
    Sei stets $\emptyset \ne X \in \Bd$ versehen mit $\Borel(X)$ und $\lambda$.
    
    % Das Öberflächenintegral
\jdate{30.01.2009}

\section{Die $L^p$-Räume}

Für $p \in [1,\infty)$ setze
\[
    \begin{split}
        \Leb^p(X) &:= \left\{\jabb{f}{X}{\R}\ \jshortlink{messbar} : \int_X |f|^p dx < \infty \right\},\\
        \Leb^\infty(X) &:= \left\{\jabb{f}{X}{\R}\ \jshortlink{messbar}, \ \fa \text{ beschränkt} \right\},
    \end{split}
\]
sowie für \jlink{messbar}e $\jabb{f}{X}{\R}$
\[
    \begin{split}
        \lVert f \rVert_p &:= \left(\int_X |f(x)|^p dx \right)^\frac{1}{p}, \hspace{10pt} 1\le p < \infty,\\
        \lVert f \rVert_\infty &:= \esssup_{x\in X} |f(x)| := \inf \{c>0 : \exists \text{ NM } N_c \text{ mit } |f(x)| \le c \ \forall x\in X\backslash N_c\}.
    \end{split}
\]

\begin{bem*}
    Für stetige $\jabb{f}{X}{\R}$ gilt $\sup_{x\in X} |f(x)| = \esssup_{x\in X} |f(x)|$. Denn sei $N_c$ wie in der obigen Definition. Dann ist $N_c^0=\emptyset$ (anderenfalls existiert ein $B\subset N_c$ mit $\lambda(B) > 0$, was ein Widerspruch ist). Aus $|f(x)| \le c$ für alle $x\notin N_c$ folgt mit der Stetigkeit von $f$, dass $|f(x)|\le c$ $\forall x\in X$. Durch $\inf$-Bildung erhält man $\esssup_{x\in X} |f(x)| \le \sup_{x\in X} |f(x)|$. Die andere Abschätzung ist klar mit $N_c = \emptyset$.
\end{bem*}

Wenn $\lVert f_n - f \rVert_p \to 0$ ($n\to \infty,\ 1\le p < \infty$), dann sagt man ``$f_n$ gegen $f$ im $p$-ten Mittel''.

\jspace

TODO: Bild

\jspace

Interpretation der $1$-Norm in \jlink{Bsp 4.21}. Man kann $u(t,x) \ge 0$ als Konzentration eines Stoffes zur Zeit $t\ge 0$ am Ort $x\in X$ interpretieren. Dann folgt, dass
\[
    \int_X |u(t,x)| dx = \int_X u(t,x) dx
\]
die Gesamtmenge des Stoffes zur Zeit $t$ beschreibt.

\jspace

\uline{Beachte}: $\Leb^1(X)$ ist nach Kapitel 2 ein Vektorraum. Ebenso ist $\Leb^\infty(X)$ ein Vektorraum, denn wenn $|f_j(x)|\le c_j$ ($\forall x \notin N_j$), wobei $N_j$ Nullmengen sind, und $\alpha_j \in \R$ ($j=1,2$), dann gilt
\[
    \tag{$*$}
    |\alpha_1 f_1(x) + \alpha_2 f_2(x)| \le |\alpha_1|c_1 + |\alpha_2| c_2 =: c \hspace{10pt} \forall x \notin N := N_1\cup N_2 \text{ (NM)}.
\]
Dann folgt $\alpha_1 f_1 + \alpha_2 f_2 \in \Leb^\infty(X)$.

\jspacesmall

\uline{Zu $\Leb^p$}: Wenn $f \in \Leb^p(X),\ \alpha \in \R$, dann gilt $\alpha f \in \Leb^p(X),\ \lVert \alpha f \rVert_p = |\alpha|\cdot \lVert f \rVert_p$. (Folgt aus der Definition).

\jspacesmall

Setze wie in Ana2 $p' := \frac{p}{p-1}$, wenn $1<p<\infty,\ 1':= \infty,\ \infty' = 1$. Dann gilt $\frac{1}{p} + \frac{1}{p'} = 1$ $\forall p [1,\infty]$.
\[
    p' = p \Leftrightarrow p = 2, \hspace{10pt} p\in [1,2) \Leftrightarrow p'\in(2,\infty],\ p'' = p.
\]


\begin{satz}
    \jlabel{Satz 5.1}
    Sei $p\in[1,\infty]$. Dann gelten
    \begin{enumerate}
        \item
            \jlabel{Satz 5.1a)}
            \jlabel{Hoelder}
            \uline{Höder-Ungleichung}: Für $f\in \Leb^p(X),\ g\in \Leb^{p'}(X)$ gelten $fg \in \Leb^1(X)$ und 
            \[
                \lVert fg \rVert_1 = \int |fg| dx \le \lVert f \rVert_p \cdot \lVert g \rVert_{p'} \overset{p\in (1,\infty)}{=} \left(\int |f|^p dx \right)^\frac{1}{p} \cdot \left(\int |g|^{p'} dx \right)^\frac{1}{p'}.
            \]
            (Für $p=p' = 2$ ist dies die Cauchy-Schwarz-Ungleichung.)
        \item
            \jlabel{Satz 5.1b)}
            \uline{Minkowski-Ungleichung}: Für $f,g \in \Leb^p(X)$ gilt  $f+g\in \Leb^p(X)$ und
            \[
                \lVert f+g \rVert_p \le \lVert f \rVert_p + \lVert g \rVert_p.
            \]
            Ferner ist $\Leb^p(X)$ ein Vektorraum.
    \end{enumerate}

    \begin{proof}
        $fg,\ |f+g|^p$ sind \jlink{messbar} $(p<\infty)$.
        \begin{enumerate}
            \item 
                \uline{$p=1$}: Dann folgt $g\in \Leb^\infty(X) \Rightarrow \exists \text{ Nullmenge } N,\ c>0$ mit $|g(x)|\le c$ $(\forall x \notin N)$. Setze $\tilde{g} := \doubleOne_{X\backslash N} \cdot g$. Dann gilt
                \[
                    \int |fg| dx \overset{\jshortlink{Lem 3.5}}{=} \int |f| \cdot \underbrace{|\tilde{g}|}_{\le c} dx \le c\cdot \lVert f \rVert_1.
                \]
                Infimumbildung über alle $c$ liefert die Behauptung. Genauso für $p=\infty$.
                
                \jspacesmall
                
                \uline{$1 < p < \infty$}: Wenn $\lVert f \rVert_p = 0$. oder $\lVert g \rVert_{p'} = 0$, dann $|f|^p = 0$ \fu\ oder $|g|^{p'}=0$ \fu\ (\jlink{Lem 2.18}). Dann folgt $f=0$ \fu\ oder $g=0$ \fu. Also $fg = 0$ \fu, womit wir fertig sind.\\
                Anderenfalls liefert die Young'sche Ungleichung (Ana2 Beweis von Satz 1.19) für festes $x\in X$:
                \[
                    \frac{|f(x)|}{\lVert f \rVert_p} \cdot \frac{|g(x)|}{\lVert g(x) \rVert_{p'}} \le \frac{1}{p} \cdot \frac{|f(x)|^p}{\lVert f \rVert_p^p} + \frac{1}{p'}\cdot \frac{|g(x)|^{p'}}{\lVert g(x) \rVert_{p'}^{p'}}.
                \]
                Integralbildung auf beiden Seiten liefert
                \[
                    \int |f| \cdot |g| dx = \frac{1}{p} \cdot \frac{1}{\lVert f \rVert_p^p} \cdot \underbrace{\int |f(x)|^p dx}_{= \lVert f \rVert_p^p} + \frac{1}{p'} \cdot \frac{1}{\lVert g \rVert_{p'}^{p'}} \cdot \lVert g \rVert_{p'}^{p'} = \frac{1}{p} + \frac{1}{p'} = 1.
                \]
                Daraus folgt $fg \in \Leb^1(X),\ \lVert fg \rVert_1 \le \lVert f \rVert_p \cdot \lVert g \rVert_{p'}$.
                
            \item
                \uline{$p=1$}: Kapitel 2. $p=\infty$: Die Behauptung folgt mit Infimumbildung über $c_1,c_2$ in $(*)$ mit $\alpha_1 = \alpha_2 = 1$.
                
                \jspacesmall
                
                Sei \uline{$p\in(1,\infty)$}. Dann gilt
                \[
                    \begin{split}
                        \int |f+g|^p dx &= \lVert f + g \rVert_p^p = \int |f+g| \cdot |f+g|^{p-1} dx\\
                                        &\le \int |f|\cdot |f+g|^{p-1} dx + \int |g| \cdot |f+g|^{p-1} dx\\
                                        &\begin{split}
                                            \overset{\text{Hölder}}{\le} &\lVert f \rVert_p \cdot \left(\int |f+g|^{(p-1)p'} dx \right)^\frac{1}{p'}\\
                                            + &\lVert g \rVert_p \cdot \left(\int |f+g|^p dx \right)^\frac{p-1}{p}
                                         \end{split}\\
                                        &= (\lVert f \rVert_p + \lVert g \rVert_p)\cdot \lVert f+g \rVert_p^{p-1} 
                    \end{split}
                \]
                Damit folgt $\lVert f+g \rVert_p \le \lVert f \rVert_p + \lVert g \rVert_p$. Dass $f+g \in \Leb^p(X)$ gilt, folgt aus $|f+g|^p \le (|f| + |g|)^p \overset{\text{{Hölder}}}{\le} 2^p\cdot (|f|^p + |g|^p)$, was integrierbar ist.
        \end{enumerate}
    \end{proof}
\end{satz}

\begin{expl}
    \jlabel{Bsp 5.2}
    Sei $X = [1,\infty)$ und $f(x) = x^{-\alpha}$, $g(x)=x^{-\beta}$ für Konstanten $\alpha, \beta >0$. Dann $f\in \Leb^p(X) \Leftrightarrow \int_1^\infty x^{-\alpha p} dx < \infty \Leftrightarrow \alpha p > 1 \Leftrightarrow \alpha > \frac{1}{p}$, $g\in \Leb^{p'}(X) \Leftrightarrow \beta > \frac{1}{p'}$, $fg \in \Leb^1(X) \Leftrightarrow \alpha + \beta > 1$, wobei $p \in (1,\infty)$.
\end{expl}


\begin{kor}
    \jlabel{Kor 5.3}
    Sei $\lambda(X) < \infty$. Dann $\Leb^q(X) \subset \Leb^p(X)$ für alle $1\le p \le q \le  \infty$ und $\lVert f \rVert_p \le \lambda(X)^{\frac{1}{p}-\frac{1}{q}}\cdot \lVert f \rVert_q$ $(\forall f \in \Leb^q(X))$. Mit $p=1$ folgt
    \[
        \left(\frac{1}{\lambda(X)} \cdot \int_X |f|dx \right)^q \le \frac{1}{\lambda(X)}\cdot \int_X |f|^q dx.
    \]
    Also folgt aus $f_n \to f$ bezüglich der $q$-Norm, dass auch $f_n \to f$ bezüglich der $p$-Norm. (Ersetze $f$ durch $f_n-f$)
    
    \begin{proof}
        Für $q=p$ und $p=\infty$ ist die Aussage klar. Sei $p<q<\infty$, $f\in \Leb^q(X)$. Dann gilt für $r:=\frac{p}{q}\in (1,\infty) \Rightarrow r' = \frac{q}{q-p},\ \frac{1}{r'} = 1-\frac{p}{q}\ (*)$:
        \[
            \int_X |f|^p dx = \int_X 1\cdot |f|^p \overunderset{\jshortlink{Hoelder}}{\le}{\text{mit }r\ (*)} \left(\int_X \left(1^{r'} \right) dx \right)^\frac{1}{r'} \cdot \left(\int_X |f|^{pr} \right)^\frac{1}{r}.
        \]
        Damit folgt
        \[
            \int_X |f|^p dx \le \lambda(X)^{1-\frac{p}{q}} \cdot \left(\int_X |f|^q dx \right)^\frac{p}{q} \overunderset{\text{nach}}{<}{\text{Vor.}} \infty.
        \]
        Durch die Abschätzung mit der $p$-ten Wurzel folgt dann die Behauptung.
    \end{proof}
\end{kor}


\jlabel{Bsp 5.4}
\begin{expl}
    \begin{enumerate}
        \item
            \jlabel{Bsp 5.4a)}
            Sei $X=(0,1]$, $f(x)= x^{-\alpha}$ für eine Konstante $\alpha >0$. Dann gilt
            \[
                f\in \Leb^p((0,1]) \Leftrightarrow \int_0^1 x^{-\alpha p}dx < \infty \Leftrightarrow \alpha p < 1 \Leftrightarrow a < \frac{1}{p}.
            \]
            Damit gilt $f(x)=x^{-\frac{1}{p}}$ und mit $p<q$ liegt $f$ in $\Leb^p(X)$, aber nicht in $\Leb^q(X)$. Also gilt $\Leb^q \subsetneqq \Leb^p(X)$.
            
        \item
            \jlabel{Bsp 5.4b)}
            Wenn $\lambda(X)<\infty$, dann gibt es keine Inklusion zwischen $\Leb^p(X)$ und $\Leb^q(X)$ (bezüglich $\lambda$).
            \begin{expl*}
                $p=1,\ X=[1,\infty)$. Dann ist $f(x)=\frac{1}{x}$ in $\Leb^q(X)\ \forall q > 1$, aber $f\notin \Leb^1(X)$. Ferner liegt $g(x)=\doubleOne_{[1,2)}(x)\cdot(2-x)^{-\frac{1}{q}}$ nicht in $\Leb^q(X)$, aber in $\Leb^1(X)$.
            \end{expl*}
    \end{enumerate}
\end{expl}

\begin{satz}[Majorisierte Konvergenz]
    \jlabel{Satz 5.5}
    Seien $1\le p <\infty,\ f_n \in \Leb^p(X),\ \jabb{f}{X}{\R}$ \jlink{messbar}, $f_n\xrightarrow{n\to\infty}f$ \fu, $|f_n|^p \le g$ \fu\ für alle $n\in\N$ und ein \uline{$g\in \Leb^p(X)$}. Dann gelten $f\in \Leb^p(X)$ und $\lVert f- f_n \rVert_p \to 0$ ($n\to \infty$).
    \begin{proof}
        $p=1$: \jhyperref{Thm 3.10}{Satz von Lebesgue}. Sei also $p>1$. Dann gilt $|f|^p \le g$ \fu\ und $|f(x)-f_n(x)|^p \le (f(x)+g(x))^p \le (2\cdot g(x)^\frac{1}{p})^p = 2^p\cdot g(x)$ \fa\ $x$. Ferner gilt $|f-f_n|^p \xrightarrow{n\to \infty} 0$, \fu. \jhyperref{Thm 3.10}{Lebesgue} angewendet auf $|f-f_n|^p$ liefert $\lVert f-f_n \rVert_p^p = \int_X |f-f_n|^p dx \to 0$. Dass $f\in \Leb^p(X)$ gilt, folgt aus $|f|^p \le g$ \fu.
    \end{proof}
\end{satz}

\begin{expl*}
    Sei $f_n = n\cdot \doubleOne_{[0,\frac{1}{n})},\ X=\R,\ p\in [1,\infty)$. Dann folgt $f_n \in \Leb^p(X)$ und $f_n \to 0$ punktweise. Aber es gilt $\lVert f_n \rVert_p = n^{1-\frac{1}{p}} \nrightarrow 0.$ (Vergleiche \jlink{Bem 3.11})
\end{expl*}


\jdate{02.02.2009}


Ab jetzt sei stets $1\le p < \infty$.

\jspace
Es ergibt sich folgendes Problem:
\[
    \left(\int_X |f|^p \right)^\frac{1}{p} = \lVert f \rVert_p = 0 \Leftrightarrow |f|^p=0\ \fu \Leftrightarrow f = 0\ \fu.
\]
Also ist die Normeigenschaft (N1) verletzt ((N2) und (N3) gelten allerdings in $\Leb^p$). Damit ist $\lVert \cdot \rVert_p$ ist \uline{keine} Norm auf $\Leb^p$.

\jspace

\uline{Ausweg}: Definiere
\[
    \mathcal{N} := \{\jabb{f}{X}{\R}: f \text{ messbar},\ f=0 \ \fu\}.
\]
Dann ist $\mathcal{N}$ ein Untervektorraum von $\Leb^p$. Wir setzen
\begin{equation}
    \jlabel{(5.1)}
    L^p:= \Leb^p \diagup \mathcal{N} \text{TODO} = \{\hat{f} = f + \mathcal{N} : f \in \Leb^p(X) \}.
\end{equation}
Aus der Linearen Algebra wissen wir, dass auch $L^p$ ein Vektorraum ist (bezüglich der kanonischen Verknüpfungen.)
\jspacesmall
\uline{Beachte}:
\[
    \hat{f} = \hat{g} \Leftrightarrow f=g\ \fu\ \forall f\in\hat{f},\ g\in\hat{g}.
\]
Für $\hat{f}\in L^1$ definiere
\begin{equation}
    \jlabel{(5.2)}
    \int_X \hat{f} dx := \int_X f(x) dx
\end{equation}
für einen beliebigen Repräsentanten $f\in \hat{f}$. Mit \jlink{Lem 3.5} folgt, dass \jlink{(5.2)} repräsentantenunabhängig ist, denn sei $g$ ein weiterer Repräsentant von $\hat{f}$, d.h. $\hat{f} = \hat{g}$, dann gilt $f=g$ \fu.

\jspace

Für das Integral in \jlink{(5.2)} gelten die bekannten Regeln. Somit ist $\lVert \hat{f} \rVert_p := \lVert f \rVert_p$ für ein beliebiges $f \in \hat{f}$ wohldefiniert.
\jspacesmall
\uline{Vorsicht}: $\hat{f} \mapsto f(x)$ für einen Repräsentanten $f\in \hat{f}$ und ein $x\in X$ definiert keine Abbildung von $L^p(X)$ nach $\R$!.

\jspace

 \uline{Nun}: $\lVert \hat{f} \rVert_p = 0 \Rightarrow f \in \mathcal{N}$ für jeden Repräsentanten $f\in \hat{f} \Rightarrow \hat{f} =0$. Weiter seien $\hat{f}, \hat{g} \in L^p(X)$ mit Repräsentanten $f,g$, sowie $\alpha \in \R$. Dann gelten
\begin{itemize}
    \item $\lVert \alpha \hat{f}\rVert_p \overset{\text{Def.}}{=} \lVert \alpha f \rVert_p = |\alpha|\cdot \lVert f \rVert_p \overset{\text{Def.}}{=} |\alpha|\cdot \lVert \hat{f} \rVert_p$

    \item $\lVert f + g \rVert_p \overset{\text{Def.}}{=} \lVert f+g \rVert_p \overset{\jshortlink{Satz 5.1b)}}{\le} \lVert f \rVert_p + \lVert g \rVert_p \overset{\text{Def}}{=} \lVert \hat{f} \rVert_p + \lVert \hat{g} \rVert_p$.
\end{itemize}
Also definiert $\lVert \cdot \rVert_p$ eine Norm auf $L^p(X)$.

\jspace

Seien $\hat{f},\hat{g},\hat{h} \in L^2(X)$ und $\alpha,\beta \in \R$. Nach \jlink{Hoelder} existiert für beliebige Repräsentanten $f\in\hat{f},g\in\hat{g}$
\begin{equation}
    \jlabel{(5.3)}
    (\hat{f}|\hat{g}) = \int_X f(x)\cdot g(x) dx.
\end{equation}
Es gelten
\begin{equation}
    \jlabel{(5.4)}
    |(\hat{f}|\hat{g})| \le \int_X |fg| dx \overset{\jshortlink{Hoelder}}{\le} \lVert f \rVert_2 \cdot \lVert f \rVert_2 \overset{\text{Def.}}{=} \lVert \hat{f} \rVert_p \cdot \lVert \hat{g} \rVert_p,
\end{equation}
\[
    (\hat{f}|\hat{g}) = (\hat{g}|\hat{f}),
\]
\begin{equation}
    \jlabel{(5.5)}
    (\alpha \hat{f} + \beta \hat{h} | \hat{g}) = \alpha (\hat{f}|\hat{h}) + \beta (\hat{h}|\hat{g})
\end{equation}
und
\begin{equation}
    \jlabel{(5.6)}
    (\hat{f}|\hat{f}) = \int_X |f(x)|^2 dx = \lVert f \rVert_2^2 = \lVert \hat{f} \rVert_2^2.
\end{equation}
Damit ist $(\cdot|\cdot)$ ein Skalarprodukt auf dem reellen Vektorraum $L^2(X)$ mit zugehöriger Norm $\lVert \cdot \rVert_2 = \sqrt{(\cdot | \cdot)}$.

\begin{defn*}
    Ein Banachraum, dessen Norm wie in \jlink{(5.6)} von einem Skalarprodukt induziert wird, heißt \jterm{Hilbertraum}.
\end{defn*}

\begin{bem*}
    Setze $\infty^p := \infty$. Dann ist die Abbildung $\jabb{\varphi}{[0,\infty]}{[0,\infty]}, x\mapsto x^p$ \jlink{messbar}, da $\varphi
    ^{-1}([a,\infty]) = [a^\frac{1}{p},\infty]\in \overline{\Borel}_1$ $(\forall a \ge 0)$.
\end{bem*}

\begin{thm}[Riesz/Fischer]
    \jlabel{Thm 5.6}
    Sei $(f_n)_{n\in\N}\subset \Leb^p(X)$, $1\le p < \infty$ eine Cauchy-Folge bezüglich der $p$-Norm. Dann gibt es ein $f\in\Leb^p(X)$ und eine Teilfolge $(n_j)_{j\in\N}$, sodass $\lVert f_n - f\rVert_p \xrightarrow{n\to \infty} 0$ und $f_{n_j} \xrightarrow{j\to \infty} f$ \fu.
    \jspacesmall
    Ferner ist $L^p(X)$ ein Banachraum und $L^2(X)$ ist ein \jlink{Hilbertraum}.
    
    \begin{proof}
        \begin{itemize}
            \item[1)]
                Zweite Behauptung: Wenn $(\hat{f}_n)_{n\in\N}$ eine Cauchy-Folge in $L^p(X)$ ist, dann gilt für Repräsentanten $f_n\in \hat{f}_n$
                \[
                    \forall \epsilon > 0 \exists N_\epsilon \in \N: \lVert \hat{f}_n - \hat{f}_m \rVert_p = \lVert f_n - f_m \rVert_p \le \epsilon\ (\forall n,m\ge N_\epsilon).
                \]
                Damit ist $(f_n)_{n\in\N}$ eine Cauchy-Folge in $\Leb^p(X)$. Nach der ersten Behauptung existiert dann ein $f\in\Leb^p(X)$, sodass
                \[
                    \lVert \hat{f}_n - \hat{f} \rVert_p = \lVert f_n - f \rVert_p \xrightarrow{n\to \infty} 0.
                \]
                Also ist $L^p(X)$ ein Banachraum und $L^2(X)$ ist ein \jlink{Hilbertraum}.
                
            \item[2)]
                Sei nun $(f_n)_{n\in\N}$ eine Cauchy-Folge in $\Leb^p(X)$. Wähle (mittels $e_j := 2^{-j}$) eine Teilfolge $(n_j)_{j\in\N}$ mit
                \[
                    \tag{$*$}
                    \lVert f_l - f_{n_j} \rVert_p \le \epsilon = 2^{-j}, \hspace{5pt} \forall l\ge n_j
                \]
                Setze $g_j := f_{n_j+1} - f_{n_j}$ für $j\in\N$. Sei $N\in\N$. Dann gilt
                \[
                    \begin{split}
                        s_N &:= \left(\int_X \left( \sum_{j=1}^N |g_j(x)| \right)^p dx \right)^\frac{1}{p} = \left\lVert \sum_{j=1}^N |g_j| \right\rVert_p \overset{\jshortlink{Satz 5.1b)}}{\le} \sum_{j=1}^N \lVert g_j \rVert_p\\
                        & \overset{(*)}{\le} \sum_{j=1}^N 2^{-j} \le 1 \hspace{10pt} (\forall N\in\N).
                    \end{split}
                \]
                Damit gilt
                \[
                    \begin{split}
                        \int_X (\underbrace{\sum_{j=1}^\infty |g_j(X)|}_{=: g(x)\text{ \jshortlink{messbar}}})^p dx &= \int_X \limToInf{N} \left(\sum_{j=1}^N |g_j(x)|\right)^p dx\\
                        &\overset{\jshortlink{Fatou}}{\le} \varliminf_{N\to \infty} \int_X \underbrace{\left(\sum_{j=1}^N |g_j| \right)^p}_{= s_N^p} dx \le 1.
                    \end{split}
                \]
                Also liegt $g\in\Leb^p(X)$ und somit existiert eine Nullmenge $N$ mit $g(x)^p < \infty\ \forall x\notin N$ ($\Leftrightarrow g(x) < \infty\ \forall x \notin N$) (wegen \jlink{Kor 2.24}).
                \jspacesmall
                Mit unserem Wissen aus Ana1 folgt
                \[
                    \exists\ \sum_{j=1}^\infty g_j(x) \in \R \hspace{10pt} (\forall x \notin N).
                \]
                Weiter gilt
                \[
                    \tag{$**$}
                    \sum_{j=1}^{m-1} g_j = f_{n_m} - f_{n_1} \hspace{10pt} (\forall m \in \N).
                \]
                Daraus folgt
                \[
                    \exists \limToInf{m} f_{n_m}(x) =: f(x) \in \R,\ \forall x\notin N.
                \]
                Setze $f(x):= 0 \ \forall x\in N$. Dann ist $\jabb{f}{X}{\R}$ \jlink{messbar} und $f_{n_m} \xrightarrow{m\toInf}f\ \fu$. Ferner gilt 
                \[
                    |f_{n_m}| \overset{(**)}{\le} |f_{n_1}| + \sum_{j=1}^{m-1} |g_j| \le |f_{n_1}| + g =: h,
                \]
                wobei $h\in \Leb^p(X)$, $m\in\N$.
                
                \jspacesmall
                
                Mit \jlink{Satz 5.5} folgt dann $f\in\Leb^p(X)$ und $\lVert f_{n_m} - f \rVert_p \xrightarrow{m\toInf} 0$.\\
                Für $\epsilon > 0$ wähle $m$ mit $2^{-m} \le \epsilon$ und $\lVert f - f_{n_m} \rVert_p \le \epsilon$. Sei $l\ge n_m =: N_\epsilon$. Dann gilt
                \[
                    \lVert f_l - f \rVert_p \le \lVert f_l - f_{n_m} \rVert_p + \lVert f_{n_m} - f \rVert_p \overset{(*)}{\le} 2\epsilon.
                \]
        \end{itemize}
    \end{proof}
\end{thm}

\begin{expl}
    \jlabel{Bsp 5.7}
    Sei $X=[0,1]$, $I_n = [0,1], [0,\frac{1}{2}),\ [\frac{1}{2},1),\ [\frac{1}{4}, \frac{1}{2}),\dots$. Setze $f_n := \doubleOne_{I_n}$. Damit $\lVert f_n \lVert_p = \lambda(I_n)^\frac{1}{p} \xrightarrow{n\toInf} 0$.
    
    \jspacesmall
    
    \uline{Aber}: $\forall x\in [0,1] \exists $ eine Teilfolge $(n_j)_{j\in\N}$ mit $f_{n_j}(x) = 1 \nrightarrow 0$ ($j\toInf$). Also  gilt $f_n(x) \nrightarrow 0$ ($n\toInf$) für jedes $x\in [0,1]$.
    
    \jspacesmall
    
    Also folgt aus Konvergenz $f_n \xrightarrow{n\toInf} f$ in $\Leb^p(X)$ \uline{nicht} die punktweise Konvergenz $f_n \xrightarrow{n\toInf} f$ in $\R$.
\end{expl}

\begin{kor}
    \jlabel{Kor 5.8}
    Seien $\hat{f}_n \in L^p(X) \cap L^q(X)$, $1\le p,q<\infty$ und $\hat{f}_n \xrightarrow{n\toInf} \hat{f}$ in $L^p(X)$, $\hat{f}_n \xrightarrow{n\toInf} \hat{g}$ in $L^q(X)$. Dann gilt $f=g$ \fu\ für alle Repräsentanten $f\in \hat{f}$ und $g\in\hat{g}$, also $\hat{f} = \hat{g} \in L^p(X) \cap L^q(X)$.
    
    \begin{proof}
        Seien $f,g,h$ Repräsentanten von $\hat{f}, \hat{g}, \hat{h}$. Dann folgt mit \jlink{Thm 5.6}, dass Teilfolgen $(n_m),\ (n_{m_l})$ und Nullmengen $N_1,N_2$ existieren, sodass\\
        $f_{n_m}(x) \xrightarrow{m\toInf} f(x)\ \forall x\notin N_1$, $f_{n_{m_l}} \xrightarrow{l\toInf}\ \forall x\notin N_2$. Daraus folgt, dass $f(x)=g(x)\ \forall x\notin N_1\cup N_2$ gilt, wobei $N_1\cup N_2$ selbst auch eine Nullmenge ist.
    \end{proof}
\end{kor}

\begin{bem}
    \jlabel{Bem 5.9}
    Die Abbildung $\jabb{J}{\Leb^p(X) \cap C(X)}{L^p(X)},\ Jf = \hat{f}$ ist injektiv und linear. Wir identifizieren deshalb $\Leb^p(X)\cap C(X)$ mit dem neuen Teilraum $L^p(X)$.
    \begin{proof}
        Seien $f,g \in \Leb^p(X)\cap C(X)$ mit $\hat{f} = \hat{g}$. Dann folgt, dass eine Nullmenge $N$ exsitiert, sodass $f(x)=g(x)\ \forall x\notin N$. Sei $y\in N$. Dann existiert  $x_n \notin N$ mit $x_n \xrightarrow{n\toInf}y$ (da $N^0 = \emptyset$). Aus der Stetigkeit von $f$ und $g$ folgt $f(y) = g(y)$.
    \end{proof}
\end{bem}

Im Folgenden schreiben wir $f$ statt $\hat{f}$ und identifizieren $L^p(x)$ mit $\Leb^p(X)$.

\begin{bem}
    \jlabel{Bem 5.10}
    Stetig sind
    \begin{enumerate}
        \item
            \jlabel{Bem 5.10a)}
            $L^p(X) \to \R,\ f\mapsto \lVert f \rVert_p$. (Gilt in jedem normierten Vektorraum)

        \item
            \jlabel{Bem 5.10b)}
            $L^1(X) \to \R,\ f\mapsto \int_X f(x)dx$, denn, wenn $f_n \xrightarrow{n\toInf} f$ in $L^1$, dann gilt
            \[
                \left|\int_X f_n dx - \int_X f dx \right| \le \int_X |f_n -f|dx = \lVert f_n - f \rVert_1 \xrightarrow{n\toInf} 0.
            \]

        \item
            \jlabel{Bem 5.10c)}
            $L^2(X)\times L^2(X) \to \R, (f,g) \mapsto (f|g)$ (Beweis siehe Übung).
    \end{enumerate}

\end{bem}
        

% \chapter{Übung}
% 
% Hier sollen alle wichtigen Aussagen, die nur auf Übungsblättern oder in der Übung bewiesen wurden, aufgelistet werden.
% 
% \jspace
% 
% TODO

\end{document}
